%% LyX 2.0.4 created this file.  For more info, see http://www.lyx.org/.
%% Do not edit unless you really know what you are doing.
\documentclass[oneside,svgnames,x11names,x11names,HTML]{book}
\usepackage[utf8x]{inputenc}
\pagestyle{headings}
\setcounter{secnumdepth}{3}
\setcounter{tocdepth}{3}
\usepackage{float}
\usepackage{amstext}
\usepackage{amssymb}
\usepackage[cm]{fullpage}
\usepackage{layouts}
\printinunitsof{cm}
\makeatletter

%%%%%%%%%%%%%%%%%%%%%%%%%%%%%% LyX specific LaTeX commands.
\newcommand{\lyxmathsym}[1]{\ifmmode\begingroup\def\b@ld{bold}
  \text{\ifx\math@version\b@ld\bfseries\fi#1}\endgroup\else#1\fi}

%% Special footnote code from the package 'stblftnt.sty'
%% Author: Robin Fairbairns -- Last revised Dec 13 1996
\let\SF@@footnote\footnote
\def\footnote{\ifx\protect\@typeset@protect
    \expandafter\SF@@footnote
  \else
    \expandafter\SF@gobble@opt
  \fi
}
\expandafter\def\csname SF@gobble@opt \endcsname{\@ifnextchar[%]
  \SF@gobble@twobracket
  \@gobble
}
\edef\SF@gobble@opt{\noexpand\protect
  \expandafter\noexpand\csname SF@gobble@opt \endcsname}
\def\SF@gobble@twobracket[#1]#2{}
%% Because html converters don't know tabularnewline
\providecommand{\tabularnewline}{\\}

%%%%%%%%%%%%%%%%%%%%%%%%%%%%%% Textclass specific LaTeX commands.
\newenvironment{lyxlist}[1]
{\begin{list}{}
{\settowidth{\labelwidth}{#1}
 \setlength{\leftmargin}{\labelwidth}
 \addtolength{\leftmargin}{\labelsep}
 \renewcommand{\makelabel}[1]{##1\hfil}}}
{\end{list}}

%%%%%%%%%%%%%%%%%%%%%%%%%%%%%% User specified LaTeX commands.
\usepackage{marco}
\usepackage{lettrine}
%\usepackage{keyval}% http://ctan.org/pkg/keyval
%\usepackage{environ}% http://ctan.org/pkg/environ
\usetikzlibrary{calc,trees,positioning,arrows,chains,shapes.geometric,
decorations.pathreplacing,decorations.pathmorphing,shapes,%
matrix,shapes.symbols,plotmarks,decorations.markings,shadows}
\usepackage{theoremref} %refereciar teoremas Por ejemplo: \thlabel{foobar}
\usepackage{marginnote}
\usepackage{fancybox}
\usepackage{hhline}
\usepackage{multirow} 
\usepackage{colortbl}%
\usepackage{pifont} 
\usepackage{eurosym} %para el Euro
%\usepackage{xltxtra}%para logos de la familia TeX
%\usepackage{mathspec}
\usepackage{metalogo}
%\usepackage{tabular}
%\usepackage{amsmath}
\usepackage{lipsum-es}
%\usepackage{xparse}% para declarar comandos
\usepackage{amssymb}
%\usepackage[thref,thmmarks,framed, amsthm]{ntheorem}
\usepackage{comment}
\usepackage[explicit]{titlesec}
\usepackage{emptypage}%pagina en blanco al final de capitulo
%%%%%%%%%%%%%%%%%%%%%%%%%%%%%%
%\documentclass[openany,svgnames,x11names]{book}
\usepackage{titletoc}
\usepackage{fancyhdr}
\usepackage{pagecolor}
\usepackage[spanish]{layout}
\usepackage{ucs}%codificacion vieja
\usepackage[utf8x]{inputenc}
%\usepackage[latin1]{inputenc}
%\usepackage{showframe}% traza layout en cada pagina
%%%======================================revisar
\usepackage{pifont}
\usetikzlibrary{shapes,snakes,positioning}
\pgfdeclarelayer{background}
\pgfdeclarelayer{foreground}
\pgfsetlayers{background,main,foreground}
\usepackage{wallpaper}
\usetikzlibrary{calc}
\usetikzlibrary{arrows}
\usepackage{epstopdf}
\usepackage{floatflt}
\usepackage{pgfplots}
\usepackage{setspace}
%%%%%%%%%%%%%%%%
\usepackage{extarrows}
%\usepackage{fixltx2e}
\usepackage{everyshi}% http://ctan.org/pkg/everyshi
%%%%%%%%%%%%%%%%%
\usepackage{amsmath,amssymb}
\usepackage{float}
\makeatletter
\let\@tmp\@xfloat     
\usepackage{fixltx2e}
\let\@xfloat\@tmp                    
\makeatother
%\usepackage{booktabs}
\usepackage{courier}
\usepackage{units}
\usepackage{url}
\usepackage{mathpazo}
\usepackage{amsfonts}
\usepackage{fancyvrb}
\usepackage{enumerate}
\usepackage{ifthen}
\usepackage{cancel}
\usepackage{layout}
\usepackage{footnote}
\usepackage{etex}%si pgfplot tiene problemas con new dim
\usepackage[frame,letter,cam]{crop}
\usepackage{microtype,soul,filecontents}
\usepackage{bbding}
\usepackage{lettrine,caption,multicol}
\usepackage{soul}
\usepackage{palatino}
\usepackage{calligra}
\usepackage[T1]{fontenc}
\usepackage[listings,theorems]{tcolorbox}
\usepackage{filecontents,ragged2e}
%\usepackage{floatflt}
\usepackage[makeindex]{imakeidx}
\usepackage{lmodern}
\usepackage{etoolbox}
\usepackage{tabularx}
%\usepackage{minitoc}
\usepackage{etoc}
%%%%%%%%%%%%%%%%%%%%%%%%%%%%%
\usepackage[paperheight=27.9cm,%
paperwidth=20.6cm,%
%centering,%
%textheight=22.9cm,
left=2.5cm,%
right=2.5cm,%
top=2.5cm,%
bottom=2cm,%
headheight=1cm,%
headsep=20pt,%
footskip=1cm,%
marginparsep=20pt,%
pdftex=false,%
letterpaper%
]{geometry}
%\usepackage[paperheight=25cm,%
%paperwidth=17cm,%
%centering,%
%left=1.5cm,%
%right=2cm,%
%top=2.5cm,%
%bottom=1.5cm,%
%headheight=0.5cm,%
%headsep=10pt,%
%%footskip=1cm,%
%marginparsep=20pt,
%margin=2cm,
%pdftex=false
%]{geometry}
%\usepackage[frame,center,letter,pdflatex]{crop}
%\usepackage{amsthm}
%\usepackage[framed, amsthm]{ntheorem}
\usepackage{listings}
\definecolor{lightgrey}{rgb}{0.9,0.9,0.9}
\definecolor{darkgreen}{rgb}{0,0.6,0}
\usepackage{fourier-orns}
% % % % % % % % % % % % % % % % % % % % % % % % % % %
% % % %preamble de framed % % %
%\usepackage{etex}
\usepackage{lmodern}
\usepackage{textcomp}
\usepackage{array}
\usepackage{booktabs}
\usepackage{microtype}
\usepackage{subcaption}
\newcommand*{\mail}[1]{\href{mailto:#1}{\texttt{#1}}}
\newcommand*{\pkg}[1]{\textsf{#1}}
\newcommand*{\cs}[1]{\texttt{\textbackslash#1}}
\makeatletter
\newcommand*{\cmd}[1]{\cs{\expandafter\@gobble\string#1}}
\makeatother
\newcommand*{\env}[1]{\texttt{#1}}
\newcommand*{\opt}[1]{\texttt{#1}}
\newcommand*{\meta}[1]{\textlangle\textsl{#1}\textrangle}
\newcommand*{\marg}[1]{\texttt{\{}\meta{#1}\texttt{\}}}

% % % % % % % % % % % % % % % % %framed
\def\indexname{\'Indice}
\def\contentsname{ CONTENIDO}
\def\listfigurename{Tabla de figuras}
\def\bibname{Bibliograf\'{\i}a}
\def\tablename{Tabla}
\def\proofname{Demostraci\'on}
\def\appendixname{Ap\'endice}
\def\chaptername{ Cap\'{\i}tulo}
\def\figurename{Figura}
%%%%%%%%%%%%%%%%%%%%%%%%%%%%%%%%% definicion de colores ================)
\definecolor{est1}{RGB}{0,177,235}
\definecolor{est2}{RGB}{0,119,158}
\definecolor{est3}{RGB}{235,137,0}
\definecolor{est4}{RGB}{158,66,0}
\definecolor{est5}{RGB}{20,20,20}
\definecolor{est6}{RGB}{235,235,235}
\definecolor{naranja1}{rgb}{1,0.5,0}
\definecolor{naranja2}{RGB}{255,127,0}
\definecolor{naranja3}{cmyk}{0,0.5,1,0}
\definecolor{naranja4}{HTML}{FF7F00}
%rgb
\definecolor{rojo}{rgb}{1,0,0}
\definecolor{verde}{rgb}{0,1,0}
\definecolor{azul}{rgb}{0,0,1}
%cmyk
\definecolor{blanco}{cmyk}{0,0,0,0}
\definecolor{cian}{cmyk}{1,0,0,0}
\definecolor{magenta}{cmyk}{0,1,0,0}
\definecolor{amarillo}{cmyk}{0,0,1,0}
\definecolor{negro}{cmyk}{0,0,0,1}
\definecolor{theblue}{rgb}{0.02,0.04,0.48}
\definecolor{thered}{rgb}{0.65,0.04,0.07}
\definecolor{thegreen}{rgb}{0.06,0.44,0.08}
\definecolor{thegrey}{gray}{0.5}
\definecolor{theshade}{gray}{0.94}
\definecolor{theframe}{gray}{0.75}
\definecolor{burl}{rgb}{0.27,0.22,0.20}
\definecolor{caper}{rgb}{0.36,0.46,0.23}
\definecolor{rhodo}{rgb}{0.58,0.63,0.45}
\definecolor{wood}{rgb}{0.61,0.51,0.43}
\definecolor{mesh}{rgb}{0.97,0.93,0.81}
\definecolor{wood}{rgb}{0.61,0.51,0.43}
\definecolor{warningColor}{named}{Red3}
\definecolor{doc}{RGB}{0,60,110}
\definecolor{boxheadcol}{gray}{.6}
\definecolor{boxcol}{gray}{.9}
\definecolor[named]{PowderBlue}{HTML}{B0E0E6}
\definecolor[named]{MidnightBlue}{HTML}{191970}
\definecolor{bl}{rgb}{0,0.2,0.8}
\definecolor{shcolor}{HTML}{FDEDD0}
\definecolor[named]{GreenTea}{HTML}{CAE8A2}
\definecolor[named]{MilkTea}{HTML}{C5A16F}
\definecolor[named]{SaddleBrown}{HTML}{8B4513}
\definecolor{FrameColor}{rgb}{0.25,0.25,1.0}
\definecolor{TitleColor}{rgb}{1.0,1.0,1.0}
\definecolor{TFFrameColor}{HTML}{CAE8A2}
\definecolor{TFTitleColor}{HTML}{C5A16F}
\definecolor{secnum}{RGB}{13,151,225}
\definecolor{ptcbackground}{RGB}{212,237,252}
\definecolor{ptctitle}{RGB}{0,177,235}
\definecolor{shadecolor}{RGB}{212,237,252}
\definecolor{visgreen}{rgb}{0.733, 0.776, 0}
\definecolor{myBGcolor}{HTML}{F6F0D6}
\definecolor[named]{PowderBlue}{HTML}{B0E0E6}
\definecolor[named]{MidnightBlue}{HTML}{191970}
\definecolor{mybrown}{RGB}{128,64,0}
\definecolor{lightgrey}{rgb}{0.9,0.9,0.9}
\definecolor{darkgreen}{rgb}{0,0.6,0}
\definecolor{Tan}{cmyk}{0.14,0.42,0.56,0}
%%%%%%%%%%%%%%%%
%%%%% Definicion de listing===========
\usepackage{caption}
\DeclareCaptionFont{white}{\color{white}}
\DeclareCaptionFormat{listing}{\colorbox{gray}{\parbox{\dimexpr\textwidth-2\fboxsep\relax}{C\'odigo \thesection .\ 
\thesource\ #3}}}
\captionsetup[source]{format=listing,labelfont=white,textfont=white, singlelinecheck=false, margin=0pt, font={bf,footnotesize}}
\newcounter{source}[section]
\lstnewenvironment{source}[2][]
{\refstepcounter{source}
\captionsetup{options=source}
\lstset{%
basicstyle=\tiny\ttfamily\bf,language={[LaTeX]TeX},caption=#1,label=#2,  
numbersep=5mm, numbers=left, numberstyle=\tiny, % number style
breaklines=true,framexleftmargin=10mm, xleftmargin=10mm,
backgroundcolor=\color{ptcbackground!60},frameround=fttt,escapeinside=??,
rulecolor=\color{ptctitle},
morekeywords={% Give key words here                                         % keywords
    maketitle},
keywordstyle=\color[rgb]{0,0,1},                    % keywords
        commentstyle=\color[rgb]{0.133,0.545,0.133},    % comments
        stringstyle=\color[rgb]{0.627,0.126,0.941}  % strings
%columns=fullflexible   
}
        }
{}
%%%%================= definicion del capitulo ======================)
 \newcommand*\chapterlabel{}
\titleformat{\chapter}
  {\gdef\chapterlabel{}
   \normalfont\sffamily\Huge\bfseries\scshape}
  {\gdef\chapterlabel{\thechapter\ }}{0pt}
  {\begin{tikzpicture}[remember picture,overlay]
    \node[yshift=-3cm] at (current page.north west)
      {\begin{tikzpicture}[remember picture, overlay]
        \draw[fill=ptcbackground!60,draw=ptcbackground!60] (0,0) rectangle
          (\paperwidth,3cm);
          \draw[ultra thick,fill=ptctitle,draw=ptctitle](0,0) -- (current page.east |- 0,0 );
          \draw [ptctitle,fill=ptctitle, ultra thick] (0.5,0) circle [radius=0.1];
         \draw[ptctitle,fill=ptctitle, ultra thick] (21,0) circle [radius=0.1];
        \node[anchor=east,xshift=.9\paperwidth,rectangle,
              rounded corners=20pt,inner sep=11pt,
              fill=ptctitle,draw=ptctitle]
              {\color{white} \chapterlabel\protect#1};
%               \draw[fill=green] (current page.north west) rectangle (current page.south east);
       \end{tikzpicture}        
      }; 
      \begin{pgfonlayer}{background}
%          \path (-1.4cm,2.8cm) node (tl) {};
%          \path (2.3cm, -8.4cm) node (br) {};
          \path[fill=ptcbackground] (current page.north west) rectangle (current page.south east);
      \end{pgfonlayer}
   \end{tikzpicture}  
     \vspace{20pt}
  }
\titlespacing*{\chapter}{0pt}{50pt}{-60pt}

%%%======================================== TOC
\preto{\frontmatter}{\pagecolor{ptcbackground}}{}{}
\preto{\mainmatter}{\pagecolor{myBGcolor}}{}{}
\preto{\backmatter}{\pagestyle{empty}\pagecolor{myBGcolor}}{}{}{}
%\patchcmd{\backmatter}{\pagecolor{myBGcolor}\pagestyle{empty}}{}{}{}
%\preto{\tableofcontents}{\begin{snugshade*}}{}{}
%\appto{\tableofcontents}{\end{snugshade*}}{}{}
%\patchcmd{\tableofcontents}{\contentsname}{\color{ptctitle}\contentsname}{}{}


    %%%%%%%%%%%%%%%%%%%%%%%%%%%%%%%%%%
 \setcounter{tocdepth}{2}
    \titlecontents{subsection}
  [5.8em]{\sffamily}
  {\color{secnum}\contentslabel{2.3em}\normalcolor}{}
  {\titlerule*[1000pc]{.}\contentspage\\\hspace*{-5.8em}\vspace*{2pt}%
    \color{ptctitle}\rule{\dimexpr\textwidth-15.5pt\relax}{1pt}}

    %%%%%%%%%%%%%%%%%%%%%%%%%%%%%%%
    \titlecontents{section}
  [4em]{\sffamily}
  {\color{secnum}\contentslabel{2.3em}\normalcolor}{}
  {\titlerule*[1000pc]{.}\contentspage\\\hspace*{-3em}\vspace*{2pt}%
    \color{ptctitle}\rule{\dimexpr\textwidth-20pt\relax}{1pt}}

\titlecontents{lsection}
  [5.8em]{\sffamily}
  {\color{secnum}\contentslabel{2.3em}\normalcolor}{}
  {\titlerule*[1000pc]{.}\contentspage\\\hspace*{-5.8em}\vspace*{2pt}%
    \color{ptctitle}\rule{\dimexpr\textwidth-15.5pt\relax}{1pt}}

\makeatletter
%%%%%%%%%%%%%%%%%%%%%%%%%%%%%%%%%
\newcommand\HUGE{\@setfontsize\Huge{38}{47}}
\newcommand\HHUGE{\@setfontsize\HHUGE{58}{67}}
\newcommand\peque{\@setfontsize\peque{8}{9}}
%%%%%%%%%%%%%%%%%%%%%%%%%%%%%%
\renewcommand*\l@chapter[2]{%
\thispagestyle{empty}
  \ifnum \c@tocdepth >\m@ne
    \addpenalty{-\@highpenalty}%
    \vskip 1.0em \@plus\p@
    \setlength\@tempdima{1.5em}%
    \begingroup
      \parindent \z@ \rightskip \@pnumwidth
      \parfillskip -\@pnumwidth
      \leavevmode
      \advance\leftskip\@tempdima
      \hskip -\leftskip
      \colorbox{ptctitle}{\strut%
        \makebox[\dimexpr\textwidth-2\fboxsep-7pt\relax][l]{%
          \color{white}\bfseries\sffamily\protect#1%
          \nobreak\hfill\nobreak\hb@xt@\@pnumwidth{\hss #2}}}\par\smallskip
      \penalty\@highpenalty
    \endgroup
  \fi}
\makeatother
%%% crear toc por capitulo con etoc 
\newcommand*\chaptertoc{% 
  \setcounter{tocdepth}{2}% 
  \etocsettocstyle{\subsection*{\subtoc}}{}% 
  {\footnotesize \localtableofcontents }
} 
\def\subtoc{\colorbox{ptctitle}{
\renewcommand{\baselinestretch}{1}
     \parbox[t]{\dimexpr\textwidth-2\fboxsep\relax}{%
    \strut\color{white}\bfseries\sffamily \makebox[5em]{%
Contenido      }\hfill Cap\'{i}tulo~\thechapter\hfill P\'agina}
}}
%%% crear toc por capitulo con titlesec
\newcommand\PartialToC{%
\startcontents[chapters]%
\begin{mdframed}[backgroundcolor=ptcbackground,hidealllines=true]
\printcontents[chapters]{l}{1}{\colorbox{ptctitle}{%
  \parbox[t]{\dimexpr\textwidth-2\fboxsep\relax}{%
    \strut\color{white}\bfseries\sffamily \makebox[5em]{%
Contenido      }\hfill Cap\'{i}tulo~\thechapter\hfill P\'agina}}\vskip5pt}
\end{mdframed}%
}
% Define partial toc for part pages
%% Set the uniform width of the colour box
%% displaying the page number in footer
%% to the width of "99"
\newlength\pagenumwidth
\settowidth{\pagenumwidth}{99}

%% Define style of page number colour box
\tikzset{pagefooter/.style={
anchor=base,font=\sffamily\bfseries\small,
text=white,fill=ptctitle,text centered,
text depth=17mm,text width=\pagenumwidth}}

%% Concoct some colours of our own
\definecolor[named]{GreenTea}{HTML}{CAE8A2}
\definecolor[named]{MilkTea}{HTML}{C5A16F}
%%%%%%%%%%%%%%% Encabezado y pie de pagina
%%%%%%%%%%
%%% Re-define running headers on non-chapter pages
%%%%%%%%%%
\fancypagestyle{headings}{%
  \fancyhf{}   % Clear all headers and footers first
  %% Right headers on odd pages
  \fancyhead[RO]{%
    %% First draw the background rectangles
    \begin{tikzpicture}[remember picture,overlay]
    \fill[ptcbackground] (current page.north east) rectangle (current page.south west);
    \fill[white, rounded corners] ([xshift=-10mm,yshift=-20mm]current page.north east) rectangle ([xshift=15mm,yshift=17mm]current page.south west);
    \begin{pgfonlayer}{background}
    %          \path (-1.4cm,2.8cm) node (tl) {};
    %          \path (2.3cm, -8.4cm) node (br) {};
              \path[fill=brown!20] (current page.north west) rectangle (current page.south east);
          \end{pgfonlayer}
    \end{tikzpicture}
    %% Then the decorative line and the right mark
    \begin{tikzpicture}[xshift=-.75\baselineskip,yshift=.25\baselineskip,remember picture,    overlay,fill=ptctitle,draw=ptctitle]\fill circle(3pt);
    \draw[semithick](0,0) -- (current page.west |- 0,0);
        \end{tikzpicture} \sffamily\itshape\small\protect\nouppercase{\rightmark}
  }

  %% Left headers on even pages
  \fancyhead[LE]{%
    %% Background rectangles first
    \begin{tikzpicture}[remember picture,overlay]
     \fill[brown!20] (current page.north east) rectangle (current page.south west);
    \fill[ptcbackground] (current page.north east) rectangle (current page.south west);
    \fill[white, rounded corners] ([xshift=-15mm,yshift=-20mm]current page.north east) rectangle ([xshift=10mm,yshift=17mm]current page.south west);
           \end{tikzpicture}
    %% Then the right mark and the decorative line
    \sffamily\itshape\small\protect\nouppercase{\leftmark}\ 
    \begin{tikzpicture}[xshift=.5\baselineskip,yshift=.25\baselineskip,remember picture, overlay,fill=ptctitle,draw=ptctitle]
    \fill (0,0) circle (3pt); \draw[semithick](0,0) -- (current page.east |- 0,0 );
       \end{tikzpicture}
  }

  %% Right footers on odd pages and left footers on even pages,
  %% display the page number in a colour box
  \fancyfoot[RO,LE]{\tikz[baseline]\node[pagefooter]{\thepage};}
   \fancyfoot[CO,CE]{\tikz\node{\color{ptctitle}Antalcides Olivo};}
  \renewcommand{\headrulewidth}{0pt}
  \renewcommand{\footrulewidth}{0pt}
}

%%%%%%%%%%
%%% Re-define running headers on chapter pages
%%%%%%%%%%
\fancypagestyle{plain}{%
  %% Clear all headers and footers
  \fancyhf{}
  %% Right footers on odd pages and left footers on even pages,
  %% display the page number in a colour box
  \fancyfoot[RO,LE]{\tikz[baseline]\node[pagefooter]{\thepage};}
 
  
    %% First draw the background rectangles
     \fancyhead[LE]{\begin{tikzpicture}[remember picture,overlay]
    \fill[ptcbackground] (current page.north east) rectangle (current page.south west);
    \fill[white, rounded corners] ([xshift=-10mm,yshift=-20mm]current page.north east) rectangle ([xshift=15mm,yshift=17mm]current page.south west);
    \begin{pgfonlayer}{background}
    %          \path (-1.4cm,2.8cm) node (tl) {};
    %          \path (2.3cm, -8.4cm) node (br) {};
              \path[fill=ptcbackground] (current page.north west) rectangle (current page.south east);
          \end{pgfonlayer}
    \end{tikzpicture}
    \sffamily\itshape\small\protect\nouppercase{\rightmark}
  }
    \fancyhead[RO]{%
    %% First draw the background rectangles
    \begin{tikzpicture}[remember picture,overlay]
    \fill[ptcbackground] (current page.north east) rectangle (current page.south west);
    \fill[white, rounded corners] ([xshift=-10mm,yshift=-20mm]current page.north east) rectangle ([xshift=15mm,yshift=17mm]current page.south west);
    \begin{pgfonlayer}{background}
    %          \path (-1.4cm,2.8cm) node (tl) {};
    %          \path (2.3cm, -8.4cm) node (br) {};
              \path[fill=ptcbackground] (current page.north west) rectangle (current page.south east);
          \end{pgfonlayer}
    \end{tikzpicture}
    \sffamily\itshape\small\protect\nouppercase{\rightmark}
  }
  \renewcommand{\headrulewidth}{0pt}
  \renewcommand{\footrulewidth}{0pt}
}
%%%%% def de seccion
 \usetikzlibrary{shapes.symbols,shadows,calc}
% the tikz picture that will be used for the title formatting
% \SecTitle{<signal direction>}{<node anchor>}{<node horiz, shift>}{<node x position>}{#5}
% the fifth argument will be used by \titleformat to write the section title using #1
\newcommand\SecTitle[5]{%
\begin{tikzpicture}[overlay,every node/.style={signal, draw, text=white, signal to=nowhere}]
  \node[ptctitle,fill, signal to=#1, inner sep=1em, drop shadow,
    text=white,font=\huge\sffamily,anchor=#2,
    xshift=\the\dimexpr-\marginparwidth-\marginparsep-#3\relax] 
    at (#4,0) {#5};
\end{tikzpicture}%
}

\titleformat{name=\section,page=even}
{\normalfont}{}{20pt}
{\SecTitle{east}{west}{16pt}{5cm}{\thesection\ #1}}[\addvspace{20pt}]

\titleformat{name=\section,page=odd}
{\normalfont\sffamily}{}{0em}
{\SecTitle{west}{east}{16pt}{\paperwidth}{#1\  \thesection}}[\addvspace{20pt}]
  %%%%%8============ Final de tabla de contenido ===========================)
% % % % % % % % % % % %bibname url
\usepackage{url}

%% Define a new 'leo' style for the package that will use a smaller font.
\makeatletter
\AtBeginDocument{%
\let\ref\autoref
\renewcommand\equationautorefname{\@gobble}
}
\def\url@leostyle{%
  \@ifundefined{selectfont}{\def\UrlFont{\sf}}{\def\UrlFont{\small\ttfamily}}}
\makeatother
%% Now actually use the newly defined style.
\urlstyle{leo}
% % % % % % % % % % % % % % % %
\usepackage{bodegraph}

\usetikzlibrary{intersections}
\usetikzlibrary{calc}
\usetikzlibrary{positioning}
% Define the layers to draw the diagram
\pgfdeclarelayer{background}
\pgfdeclarelayer{foreground}
\pgfsetlayers{background,main,foreground}


% % % % % % % % % % % % % % % % % %
\newenvironment{lista}{
\begin{itemize}
 \renewcommand{\labelitemi}{{
 \colorbox{wood!70!black}{\color{white}{\ding{42}}}
 }}
}{\end{itemize}}
\newenvironment{figura}[3]{\begin{figure}[H]
\centering
                               #1
                              \caption{#2}
                              \label{#3}
                              \end{figure}
}{ \vskip 5pt }
\newcommand{\nota}{\colorbox{teal!20!white}{\color{black}{Nota:}}\ }
\newcommand{\prop}{\colorbox{teal!20!white}{\color{black}{Proposición:}}\ }
\newcommand{\dem}{\colorbox{teal!20!white}{\color{black}{Demostraci\'on:}}\ }
\newcommand{\notacion}{\colorbox{teal!20!white}{\color{black}{Notaci\'on:}}\ }
\newcommand{\solucion}{\colorbox{teal!20!white}{\color{black}{Soluci\'on:}}\ }
\newcommand{\resp}{\colorbox{teal!20!white}{\color{black}{Respuesta:}}\ }
\def\texto{Sean $f$ y $g$ dos funciones y sean $\alpha$ y $\beta$ dos n\'umeros reales. 
Entonces se verifican las siguientes propiedades:

 \[1.\quad \int (f(x)+g(x))\,dx = \int f(x)\,dx + \int g(x)\,dx  \]
 \[2.\quad \int \alpha f(x)\, dx =\alpha \int f(x)\,dx \]

 Estas dos propiedades se pueden englobar en una:
 \[ \int (\alpha f(x)+\beta g(x)) \, dx = \alpha\int f(x)\,dx+\beta\int
   g(x)\,dx \]
{\bf Ejemplo}:

 \[ \int (2x-3x^2)\, dx = 2\int x\,dx -3\int x^2\, dx \]}
 \def\Web#1{\href{#1}{%
     \tikz \node[fill=myBGcolor](0,0) {#1};%
   }}
   \def\Item{\colorbox{wood!70!black}{\color{white}{\ding{42}}}}
   \def\web#1{\Item\ \href{http://ctan.org/pkg/#1}{\textbf{#1.}}\par\vspace{10pt}}
   % % % % % %listing
   \lstset{%
   basicstyle=\small\ttfamily\bf,language={[LaTeX]TeX}, numbersep=5mm, numbers=left, numberstyle=\tiny, % number style
   breaklines=true,framexleftmargin=10mm, xleftmargin=10mm,
   backgroundcolor=\color{ptcbackground!60},frameround=fttt,escapeinside=??,
   rulecolor=\color{ptctitle},
   morekeywords={% Give key words here                                         % keywords
       maketitle},
   keywordstyle=\color[rgb]{0,0,1},                    % keywords
           commentstyle=\color[rgb]{0.133,0.545,0.133},    % comments
           stringstyle=\color[rgb]{0.627,0.126,0.941}  % strings
   %columns=fullflexible   
   }
   % % % % % %
%%%%%% Definicion de caja %%%%%%%%%%%%%%%%%%%%%
%  \newboxedtheorem[title= Teorema. \thesection.\thecaja ,labelbox= ,boxcolor=MilkTea,background = ptcbackground!60,titleboxcolor=black,titleboxcolor=MilkTea,titlebackground=ptctitle]{caja}{Teorema}
%  % % % % % % % %
%  \newboxedtheorem[title=Lemma.\ \thecaja ,labelbox= ,boxcolor=MilkTea,background = ptcbackground!60,titleboxcolor=black,titleboxcolor=MilkTea,titlebackground=ptctitle]{cajo}{Teorema}
  % % % % % % % % % % % % % %
  \nboxedtheorem[boxcolor=MilkTea,background = ptcbackground!60,titleboxcolor=black,titleboxcolor=MilkTea,titlebackground=ptctitle]{ncaja}{Postulado}
  % % % % % % % % % % % % % % %
 \tipptheorem[tipplogo=interrogacion,boxheadcol=MidnightBlue,boxcol=PowderBlue]{notas}{Nota}
 % % % % % % % % % % % %
 \notatheorem[tipplogo=pregunta,boxheadcol=MidnightBlue,boxcol=PowderBlue]{obs}{Observaci\'on}
 %%%%%%%%%%%%%%
 \frametheorem[]{ejemplo}{Ejemplo}
 %%%%%%%%%%%%%%
 \beamertheorem[]{beamercaja}{Estilo Beamer}
 %%%%%%%%%%%%%%
 \framedtheorem[]{frameth}{Fancy}
 %%%%%%%%%%%%%%%%%%%
 \xcolortheorem[background=mybrown!5 ,titlebackground=mybrown!40!black ,titleboxcolor=mybrown!40!black ,boxcolor=mybrown!40!black]{geo}{Ejemplo}
 %%%%%%%%%%%%%%
%  \warningtheorem[textcol=black, boxheadcol=gray!80, boxcol=ptctitle, tipplogo=icon-tipp, texttcolor=black ,labeltext=, size=0.8\textwidth, iconline=red ]{xcolorth}{Fancy}
  %%%%%%%%%%tcolorbox%%%
  \newcounter{postulado}
\newenvironment{postulado}[2]{\vskip 5pt
\refstepcounter{postulado}
    \begin{tcolorbox}[colback=mybrown!5,colframe=mybrown!40!black,title=Postulado.\thechapter.\thepostulado  \ \bf{#2}]
 #1
\end{tcolorbox}\index{Postulado!#2}            }{

                \vskip 5pt
 }
  %%%%%%%%%%%%%%%
  %%%<
\newcommand{\cdefault}[4][named]{\begin{tikzpicture}
\fill[#2,draw=negro] (0,0) rectangle ++(2,1);
\node[below] at (1,0) {#2};
\node[below=4mm] at (1,0) {\tiny #3 \{#4\}};
\node[below=6mm] at (1,0) {\tiny #1};
\end{tikzpicture}}
%%%%%%%%%%%%%>
%  \lstnewenvironment{javacode}[2]
%{\singlespacing\lstset{language=java, label=#1, caption=#2}}
%{}
%%%%%%%%%%%%%%%cambio de margen
\newenvironment{changemargin}[5]
{
\begin{list}{}
{
\global\setlength{\textheight}{#1}%
  \global\setlength{\textwidth}{#2}
\setlength{\topsep}{0pt}
\setlength{\evensidemargin}{0pt}%
\setlength{\oddsidemargin}{0pt}
\setlength{\leftmargin}{#3}%
\setlength{\rightmargin}{#4}%
\setlength{\listparindent}{\parindent}%
\setlength{\itemindent}{\parindent}%
\setlength{\parsep}{\parskip}%
\hoffset #5
}
\item[]
}
{\end{list}}
%%%%%%%%%cambiamargen%%%%%%%%%%%%%%%
\newenvironment{cambiamargen}[5]
{
\begin{list}{}
{
\global\setlength{\textheight}{#1}%
 \setlength{\topmargin}{#2}
\setlength{\evensidemargin}{0pt}%
\setlength{\oddsidemargin}{0pt}
\setlength{\leftmargin}{-}%
\setlength{\rightmargin}{#4}%
\setlength{\listparindent}{\parindent}%
\setlength{\itemindent}{\parindent}%
\setlength{\parsep}{\parskip}%
\hoffset #5
}
\item[]
}
{\end{list}}
\newenvironment{dems}[1]{ \dem
\it #1  }{\hfill$\square$\vspace*{5pt}}
\newcommand{\oper}[1]{\tikz\draw(0,0)  node[draw,circle,inner sep=0pt,minimum size=1pt]{#1} ;}

%%%%%%%%%%%%%%%%%%%%%%%%%%%%
%%%%%%%%%%%%%%%%%%%%%%%%%%%%%%%%%%%%%%%%%
%%%%%%%%%%%%%%%%%%%%%%%%%%%%%%%%%
%%%%%%%%%%%%%%%%%%%%%%%%%%%%%%%%%%%%%%%%%%
\tikzstyle{mybox} = [draw=red!40!black, fill=mybrown!10, very thick,
    rectangle, rounded corners, inner sep=10pt, inner ysep=20pt]
\tikzstyle{fancytitle} =[fill=red!40!black, text=white]
\NewEnviron{competencias}{\vskip 5pt
\begin{tikzpicture}

% First box
\node [mybox] (box1){%
    \begin{minipage}{0.92\textwidth}
      \BODY
    \end{minipage}
    };
    \node[fancytitle, rounded corners] at (box1.north) {\Large Objetivos};
\end{tikzpicture}
 }
%%%%%%%%%%%%%%%%%%%%%%
\NewEnviron{logros}{ \vskip 5pt
\begin{tikzpicture}

% First box
\node [mybox] (box1){%
    \begin{minipage}{0.92\textwidth}
      \BODY
    \end{minipage}
    };
    \node[fancytitle, rounded corners] at (box1.north) {\Large Indicadores de Logros
};
\end{tikzpicture}
 }
%%%%%%%%%%%%%%%%%%%%%%%%%%%%%%%%%%%%%%%%
%%%%%%%%%%%%%%%%%%%%%%%%%%%%%%%%
%\newcounter{ideas}
%%\newboxedtheorem[title=Definici\'on.\thesection . \thedefinicionn, labelbox=, boxcolor=caper!75!black  ,background=caper!5  ,titleboxcolor=caper!75!black  , titlebackground=caper!60 ]{definicionn}{Definici\'on}
%%\newboxedtheorem[title=T\'ermino.\thesection . \theideas, labelbox=, boxcolor=rhodo!75!black ,background=rhodo!5  ,titleboxcolor=rhodo!75!black, titlebackground=rhodo!60]{ideas}{T\'ermino}
%%%%%%%%%%%%%%%%%%%%%%%%%%%%%%%%
%%\newboxedtheorem[title=T\'ermino no defnido.\thesection . \thetndefinido, labelbox=, boxcolor=MilkTea ,background=ptcbackground!60  ,titleboxcolor=MilkTea , titlebackground=ptctitle]{tndefinido}{T\'ermno no definido}
%%%%%%%%%%%%%%%%%%%%%%%%%%%%%%
\newboxedtheorem[title=Definici\'on.\  \thedefinicionn , labelbox= ,boxcolor=caper!75!black,background =caper!5,titleboxcolor=caper!75!black,titlebackground=caper!60]{definicionn}%{Teorema}
  %%%%%%%%%%%%%%%
 \newboxedtheorem[title=T\'ermino.\ \thesection . \theideas, labelbox= ,boxcolor=doc!75!black,background =doc!5 ,titleboxcolor=caper!75!black,titlebackground=doc!60]{ideas}%{Teorema}
 %%%%%%%%%%%%%%
  % % % % % % % %
  \newboxedtheorem[title=T\'ermino no defnido. \thesection . \thetndefinido, labelbox= ,boxcolor=MilkTea,background = ptcbackground!60,titleboxcolor=MilkTea,titlebackground=ptctitle]{tndefinido}%{T\'erminos no definidos}
  % % % % % % % %
  \newboxedtheorem[title=Ejemplo. \thesection . \theejem, labelbox= ,boxcolor=ptctitle!75!black,background = ptcbackground!60,titleboxcolor=ptctitle!60!black,titlebackground=ptctitle]{ejem}%{T\'erminos no definidos}
  %%%%%%%%%%%%%%%%%%%
 
 \newboxedtheorem[title=Ejemplos. \thesection . \theejems, labelbox= ,boxcolor=ptctitle!75!black,background = ptcbackground!60,titleboxcolor=ptctitle!60!black,titlebackground=ptctitle]{ejems}%{T\'erminos no definidos}
 
%%%%%%%%%%%%%%%%%%%%%%%  
 
% \newcounter{axioma}
%\newenvironment{axioma}[2]{\vskip 5pt
%\refstepcounter{axioma}
%    \begin{tcolorbox}[colback=mybrown!5,colframe=mybrown!40!black,title=Axioma.\thechapter.\thepostulado  \ \bf{#2}]
% #1
%\end{tcolorbox}\index{Axioma!#2}            }{
%
%                \vskip 5pt
% }
 \newboxedtheorem[title=Axioma.\ \theaxioma, labelbox= ,boxcolor=mybrown!40!black,background =mybrown!5 ,titleboxcolor=mybrown!75!black,titlebackground=mybrown!60]{axioma}%{Teorema}
 \newboxedtheorem[title=Leyes de inferencia.\ \thesection . \theley, labelbox= ,boxcolor=doc!75!black,background =doc!5 ,titleboxcolor=caper!75!black,titlebackground=doc!60]{ley}%{Teorema}
 \newboxedtheorem[title=Lema.\ \thesection . \thelema, labelbox= ,boxcolor=doc!75!black,background =doc!5 ,titleboxcolor=caper!75!black,titlebackground=doc!60]{lema}%{Teorema}
 \newboxedtheorem[title=Teorema.\  \theteorema, labelbox= ,boxcolor=mybrown!40!black,background =mybrown!5 ,titleboxcolor=mybrown!75!black,titlebackground=mybrown!60]{teorema}%{Teorema}
  \newboxedtheorem[title=Operaci\'on\ \thesection . \theoperacion, labelbox= ,boxcolor=doc!75!black,background =doc!5 ,titleboxcolor=caper!75!black,titlebackground=doc!60]{operacion}%{Teorema}
  \newboxedtheorem[title=Proposición.\ \thesection . \theproposicion, labelbox= ,boxcolor=mybrown!40!black,background =mybrown!5 ,titleboxcolor=mybrown!75!black,titlebackground=mybrown!60]{proposicion}%{Teorema}
  %%%%%%%%%%%% conjuntos num\'ericos
  \newcommand{\CC}{\mathbb{C}}
%\newcommand{\NN}{\mathbb{N}}
\newcommand{\PP}{\mathbb{P}}
\newcommand{\HH}{\mathbb{H}}
\newcommand{\pp}{\mathbb{\overline{P}}}
\newcommand{\DD}{\mathbb{D}}
\newcommand{\QQ}{\mathbb{Q}}
\newcommand{\RR}{\mathbb{R}}
\newcommand{\ZZ}{\mathbb{Z}}
\newcommand{\EE}{\mathbb{E}}
\newcommand{\TT}{\mathbb{T}}
\newcommand{\XX}{\mathbb{X}}
\newcommand{\der}{\mathcal{D}}
\newcommand{\kk}{\mathcal{K}}
\newcommand{\mm}[1]{{\mathcal{M}\/}(#1)}
\newcommand{\nequiv}{{\equiv \hspace*{-3.7mm}/}}
\newcommand{\capa}[1]{\mbox{{\em Cap\/}}(#1)}
\newcommand{\gra}[1]{\mbox{{\em grad\/}}(#1)}
\newcommand{\sop}[1]{\mbox{{\em supp\/}}(#1)}
\newcommand{\esup}[1]{{\mbox{\rm ess\,sup\/}}#1}
\newcommand{\lqqd}{\hfill $\blacksquare$}
\newcommand{\II}{\mathbb{I}}
%\newcommand{\na}{I\! N}
%\newcommand{\re}{I\! R}
{\makeatletter   % dies ist lokal, damit man die Datei wahlweise
                 % mit \input oder mit \documentstyle[...] einlesen kann.
 
\gdef\re{\relax\ifmmode I\hskip -3\p@ R\else
    \hbox{$I\hskip -3\p@ R$}\fi} %Reales
\gdef\na{\relax\ifmmode I\hskip -2.7\p@ N\else
    \hbox{$I\hskip -2.7\p@ N$}\fi} % Naturales
\gdef\en{\relax\ifmmode Z\hskip -4.8\p@ Z\else
    \hbox{$Z\hskip -4.8\p@ Z$}\fi} %Enteros
\gdef\co{\relax\ifmmode C\hskip-4.8\p@\vrule \@height 5.8\p@ \@depth\z@
    \hskip 6.3\p@\else
    \hbox{$C\hskip-4.8\p@\vrule \@height 5.8\p@ \@depth\z@ \hskip 6.3\p@$}\fi}% Complejos
\gdef\ra{\relax\ifmmode Q\hskip-5.0\p@\vrule
          \@height 6.0\p@ \@depth \z@ \hskip 6\p@\else
    \hbox{$Q\hskip-5.0\p@\vrule \@height 6.0\p@ \@depth \z@ \hskip 6\p@$}\fi}% Racionales
\gdef\hz{\relax\ifmmode I\hskip -3\p@ H\else
    \hbox{$I\hskip -3\p@ H$}\fi} % Espacio de Hilbert
\gdef\kz{\relax\ifmmode I\hskip -3\p@ K\else
    \hbox{$I\hskip -3\p@ K$}\fi} % campo K
    \gdef\ir{\relax\ifmmode I\hskip -3\p@ I\else
    \hbox{$I\hskip -3\p@ K$}\fi} % Irracionales}
    %%%%%%%%%%%%%%%%%%%%%%%%%
   % \renewcommand\p@axioma{\thesection.\theaxioma}
   \makeatother
\usepackage{parskip}
%%%%%%%%%%%%%%%%%%%%%%%%%%%%%%%
% \makeatletter
%  \renewcommand\thesection{\arabic{chapter}.\arabic{section}}
%\renewcommand\thesubsection{\arabic{chapter}.\arabic{section}.\arabic{subsection}}
%\renewcommand\p@subsection{\thesection.}
%\renewcommand\p@section{\thechapter.}
%\gdef\label#1{\@bsphack
%  \protected@write\@auxout{}%
%         {\string\newlabel{sec@#1}{{\thechapter  .\@currentlabel}{\thepage}}}%
%  \@esphack}
   \newcommand*{\fullref}[1]{\ref{#1} en la p\'agina~\pageref{#1}}
   %%%%%%%%%%%%%%%%%%%%%%%%%%%%%%%\def\firstcircle{(0,0) circle (1.5cm)} 
\def\secondcircle{(45:2cm) circle (1.5cm)}
\def\thirdcircle{(0:2cm) circle (1.5cm)} 
\def\rectangulo{(-2,-2) rectangle (4,3.6)}
\def\fourcircle{(1.1,1) circle (1.5cm)}
\def\fivecircle{(-2,1) circle (1.5cm)}
\def\rect{(-4,-2) rectangle (4,3.6)} 
\def\sixcircle{(-2.3,1) circle (0.8cm)} 
\def\sevencircle{(1.1,1) circle (0.8cm)}
\def\myref#1{\thesection .\ref{#1}}
 \renewcommand\theaxioma{\arabic{chapter}.\arabic{section}.\arabic{axioma}}
\renewcommand\thedefinicionn{\arabic{chapter}.\arabic{section}.\arabic{definicionn}}
\renewcommand\theteorema{\arabic{chapter}.\arabic{section}.\arabic{teorema}}
  %%%%%%%%%%%%%%%%%%%%%%%%%%%%% Empieza el documento
 \makeatletter
%%%%%%%%%%%%%%%%%%%%%%%%
%original de ldesc2e.sty 
\newcommand{\manual}{Manual de \emph{\LaTeX{}}~\cite{manual}} 
\newcommand{\companion}{\emph{The \LaTeX{} Companion}~\cite{companion}} 
\newcommand{\guia}{\emph{Gu\'{\i}a Local}~\cite{local}}
\newcommand{\contrib}[3]{#1\quad$<$\texttt{#2}$>$%
{\small\\\quad\textit{#3}}\\[1ex]}
%
% Algunas instrucciones para ayudar a la creaci'on del 'indice de
% materias.
%
%\newcommand{\bs}{\symbol{'134}}%Print backslash
\ifx\bs\undefined
  \newcommand{\bs}{\symbol{92}}%Print backslash
\else
  \renewcommand{\bs}{\symbol{92}}%Print backslash
\fi
%\newcommand{\bs}{\ensuremath{\mathtt{\backslash}}}%Imprime barra invertida
% Entrada en el 'indice para una orden
%
% Entrada de s'imbolo para la tabla de s'imbolos matem'aticos
%
\newcommand{\X}[1]{$#1$&\texttt{\string#1}\hspace*{1ex}}
% Text normal.... 
\newcommand{\SC}[1]{#1&\texttt{\string#1}\hspace*{1ex}}
% para los acentos en modo texto
\newcommand{\A}[1]{#1&\texttt{\string#1}\hspace*{1ex}}
\newcommand{\B}[2]{#1#2&\texttt{\string#1{} #2}\hspace*{1ex}}

\newcommand{\W}[2]{$#1{#2}$&
  \texttt{\string#1}\texttt{\string{\string#2\string}}\hspace*{1ex}}
\newcommand{\Y}[1]{$\big#1$ &\texttt{\string#1}}  %
% Tabla de s'imbolos matem'aticos
\newsavebox{\symbbox}
\newenvironment{symbols}[1]%
{\par\vspace*{2ex}
\renewcommand{\arraystretch}{1.1}
\begin{lrbox}{\symbbox}
\hspace*{4ex}\begin{tabular}{@{}#1@{}}}%
{\end{tabular}\end{lrbox}\makebox[\textwidth]{\usebox{\symbbox}}\par\medskip}
%
% Preparaci'on especial para imprimir los s'imblos de la AMS
% Si no se encuentra AMS, deber'ia funcionar.
%

% No se tienen versiones PS de los tipos rsfs.
% Por ello, esto no no se puede hacer para pdf
\ifx\pdfoutput\undefined % No estamos corriendo pdftex
\IfFileExists{mathrsfs.sty}
  {\RequirePackage{mathrsfs}\let\MathRSFS\mathscr\let\mathscr\relax}{}
\fi
\IfFileExists{amssymb.sty}
  {\let\noAMS\relax \RequirePackage{amssymb}}
  {\def\noAMS{\endinput}\RequirePackage{latexsym}}

\IfFileExists{euscript.sty}
  {\RequirePackage{euscript}}{}
%\IfFileExists{eufrak.sty}
%  {\RequirePackage{eufrak}}{}


%
% Imprimir |--| para mostrar distancia
%
\newcommand{\demowidth}[1]{\rule{0.3pt}{1.3ex}\rule{#1}{0.3pt}\rule{0.3pt}{1.3ex}}
\renewcommand{\cleardoublepage}
    {\clearpage\if@twoside \ifodd\c@page\else
    \hbox{}\thispagestyle{empty}\newpage\if@twocolumn\hbox{}\newpage\fi\fi\fi}

\newcommand{\BibTeX}
     {\textsc{Bib}\TeX}
%%%%% hasta aqui original de ldesc2e.sty
%%%%%%%%%%%%%%%%%%%%%%%%%%%%%%%%%%%%%%%%%%%%%%
\xcolortheorem[background=Tan!5 ,titlebackground=thered!40!black ,titleboxcolor=mybrown!40!black ,boxcolor=mybrown!40!black]{out1}{Salida}
\DeclareCaptionFormat{ejer}{\colorbox{thered!40!black}{\parbox{\dimexpr\textwidth-2\fboxsep\relax}{\color{white}Ejercicio \thesection .\ 
\theejercicio\hfill #3}}}
\captionsetup[ejer]{format=ejer,labelfont=white,textfont=white, singlelinecheck=false, margin=0pt, font={bf,footnotesize}}
\newcounter{ejercicio}[section]
\newwrite\ejercicio@out
\newenvironment{ejercicio}%
 {\begingroup% Lets Keep the Changes Local
  \@bsphack
  \immediate\openout \ejercicio@out \jobname.exa
  \let\do\@makeother\dospecials\catcode`\^^M\active
  \def\verbatim@processline{%
    \immediate\write\ejercicio@out{\the\verbatim@line}}%
  \verbatim@start}%
 {\immediate\closeout\ejercicio@out\@esphack\endgroup%
%
% Y aqu'i lo que se ha a~nadido

 \setlength{\parindent}{0pt}
    %\setlength{\parskip}{1ex plus 0.5ex minus 0.7ex}
     \noindent
  %  \hspace*{+2ex}
%  \setlength{\parindent}{0pt}
    \setlength{\parskip}{1ex plus 0.3ex minus 0.7ex}
            \makebox[0.45\linewidth][t]{
            \hspace{+2ex}
            \setlength{\parindent}{0pt}
              \begin{minipage}[t]{0.45\linewidth}
              \vspace{-3.5ex}
      \refstepcounter{ejercicio}
   \captionsetup{options=ejer}
   \lstset{%
   caption=Entrada,basicstyle=\tiny\ttfamily\bf,language={[LaTeX]TeX}, numbersep=5mm, numbers=left, numberstyle=\tiny, % number style
   breaklines=true,framexleftmargin=10mm, xleftmargin=10mm,
   backgroundcolor=\color{Tan!5},frameround=fttt,escapeinside=??,
   rulecolor=\color{thered!40!black},
   morekeywords={% Give key words here                                         % keywords
       maketitle},
   keywordstyle=\color[rgb]{0,0,1},                    % keywords
           commentstyle=\color[rgb]{0.133,0.545,0.133},    % comments
           stringstyle=\color[rgb]{0.627,0.126,0.941}  % strings
   %columns=fullflexible   
   }
    \lstinputlisting[]{\jobname.exa}
         \end{minipage}}%
      \hspace{30pt} 
           \hfill
  %
   \setlength{\parindent}{0pt}
   \setlength{\parskip}{1ex plus 0.5ex minus 0.7ex}
%   
  \makebox[0.5\linewidth][t]{%
%  %\colorbox{ptcbackground!60}{
  \setlength{\parindent}{0pt}
    \setlength{\parskip}{1ex plus 0.3ex minus 0.7ex}
      \begin{minipage}[t]{0.40\linewidth}
         \setlength{\parindent}{0pt}
  \setlength{\parskip}{1ex plus 0.4ex minus 0.7ex}
  \vspace*{-3.5ex} 
        \begin{trivlist}
     \scriptsize\bf\ttfamily \item\begin{out1}{Pdf}\input{\jobname.exa}\end{out1}
    \end{trivlist}
      \end{minipage}%}
      }
    
      \par\addvspace{3ex plus 1ex}\vskip -\parskip
} 
%%%%%%%%%%%%%%%%%%%%%%%%%%%%%%%%%%%%%%%%%%%%%%%%%%%%
\DeclareCaptionFont{white}{\color{white}}
\newcounter{example}[section]
\DeclareCaptionFormat{out}{\colorbox{mybrown!40!black}{\parbox{\dimexpr\textwidth-2\fboxsep\relax}{\color{white}C\'odigo del ejemplo\, \thesection .\ 
\theexample\hfill #3}}}
\captionsetup[out]{format=out,labelfont=white,textfont=white, singlelinecheck=false, margin=0pt, font={bf,footnotesize}}
\xcolortheorem[background=mybrown!5 ,titlebackground=mybrown!40!black ,titleboxcolor=mybrown!40!black ,boxcolor=mybrown!40!black]{out}{Salida}
\newwrite\example@out
\newenvironment{example}%
 {\begingroup% Lets Keep the Changes Local
  \@bsphack
  \immediate\openout \example@out \jobname_ex.exa
  \let\do\@makeother\dospecials\catcode`\^^M\active
  \def\verbatim@processline{%
    \immediate\write\example@out{\the\verbatim@line}}%
  \verbatim@start}%
 {\immediate\closeout\example@out\@esphack\endgroup%
 % \par\small\addvspace{3ex plus 1ex}\vskip -\parskip
 \setlength{\parindent}{0pt}
    \setlength{\parskip}{1ex plus 0.5ex minus 0.7ex}
  \noindent
  \makebox[0.45\linewidth][t]{%
  \hspace*{+2ex}
  \setlength{\parindent}{0pt}
    \setlength{\parskip}{1ex plus 0.3ex minus 0.7ex}
  \begin{minipage}[t]{0.45\linewidth}
   % \vspace*{-1ex}
   \vspace*{-3.5ex}
   \refstepcounter{example}
   \captionsetup{options=out}
   \lstset{%
   caption=Entrada,basicstyle=\tiny\ttfamily\bf,language={[LaTeX]TeX}, numbersep=5mm, numbers=left, numberstyle=\tiny, % number style
   breaklines=true,framexleftmargin=10mm, xleftmargin=10mm,
   backgroundcolor=\color{mybrown!5},frameround=fttt,escapeinside=??,
   rulecolor=\color{mybrown!40!black},
   morekeywords={% Give key words here                                         % keywords
       maketitle},
   keywordstyle=\color[rgb]{0,0,1},                    % keywords
           commentstyle=\color[rgb]{0.133,0.545,0.133},    % comments
           stringstyle=\color[rgb]{0.627,0.126,0.941}  % strings
   %columns=fullflexible   
   }
    \lstinputlisting[]{\jobname_ex.exa}
         \end{minipage}}
  \hfill
  \hspace{30pt}%
   \setlength{\parindent}{0pt}
    \setlength{\parskip}{1ex plus 0.5ex minus 0.7ex}
  \makebox[0.5\linewidth][t]{%
  %\colorbox{ptcbackground!60}{
  \setlength{\parindent}{0pt}
    \setlength{\parskip}{1ex plus 0.3ex minus 0.7ex}
      \begin{minipage}[t]{0.4\linewidth}
   % \vspace*{0.5ex}
    \setlength{\parindent}{0pt}
      \noindent
  \setlength{\parskip}{1ex plus 0.4ex minus 0.7ex}
  \vspace*{-3.5ex} 
        \begin{trivlist}
             \scriptsize\bf\ttfamily \item \begin{out}
             {Pdf}\input{\jobname_ex.exa}\end{out}
    \end{trivlist}
      \end{minipage}%}
      }
      \par\addvspace{3ex plus 1ex}\vskip -\parskip
}
%%%%%%%%%%%%%%%%%%%%%%%%
\DeclareCaptionFormat{salid}{\colorbox{thered!40!black}{\parbox{\dimexpr\textwidth-2\fboxsep\relax}{\color{white}Ejercicio \thesection .\ 
\thesalida\hfill #3}}}
\captionsetup[salid]{format=salid,labelfont=white,textfont=white, singlelinecheck=false, margin=0pt, font={bf,footnotesize}}
\xcolortheorem[background=mybrown!5 ,titlebackground=mybrown!40!black ,titleboxcolor=mybrown!40!black ,boxcolor=mybrown!40!black]{salid}{Salida}
\newcounter{salida}[section]
\newwrite\salida@out
\newenvironment{salida}%
  {\begingroup% Lets Keep the Changes Local
  \@bsphack
  \immediate\openout \salida@out \jobname_eps.tex
  \let\do\@makeother\dospecials\catcode`\^^M\active
  \def\verbatim@processline{%
    \immediate\write\salida@out{\the\verbatim@line}}%
  \verbatim@start}%
 {\immediate\closeout\salida@out\@esphack\endgroup%
 \immediate\write18{pdflatex \jobname_eps.tex \jobname_eps.pdf }
 \immediate\write18{pdf2ps \jobname_eps.pdf \jobname_eps.eps }
    
  \setlength{\parindent}{0pt}
    \setlength{\parskip}{1ex plus 0.3ex minus 0.7ex}
     \begin{minipage}[t]{0.45\linewidth}
   % \vspace*{-1ex}
   %\vspace*{-3.5ex}
   \refstepcounter{salida}
   \captionsetup{options=salid}
   \lstset{%
   caption=Entrada,basicstyle=\tiny\ttfamily\bf,language={[LaTeX]TeX}, numbersep=5mm, numbers=left, numberstyle=\tiny, % number style
   breaklines=true,framexleftmargin=10mm, xleftmargin=10mm,
   backgroundcolor=\color{Tan!5},frameround=fttt,escapeinside=??,
   rulecolor=\color{thered!40!black},
   morekeywords={% Give key words here                                         % keywords
       maketitle},
   keywordstyle=\color[rgb]{0,0,1},                    % keywords
           commentstyle=\color[rgb]{0.133,0.545,0.133},    % comments
           stringstyle=\color[rgb]{0.627,0.126,0.941}  % strings
   %columns=fullflexible   
   }
    \lstinputlisting[]{\jobname_eps.tex}
         \end{minipage}
  %\hspace{30pt}%
   % \setlength{\parindent}{0pt}
    %\setlength{\parskip}{1ex plus 0.4ex minus 0.2ex}
    \hspace{20pt}
     \begin{minipage}[t]{0.40\linewidth}
   \begin{figure}[H]
\fcolorbox{thered!40!black}{thered!5}{\includegraphics[scale=0.4]{\jobname_eps.eps}}\end{figure}\end{minipage}
}
% }

  
%}
\DeclareCaptionFormat{prob}{\colorbox{thered!40!black}{\parbox{\dimexpr\textwidth-2\fboxsep\relax}{\color{white}Ejercicio \thesection .\ 
\theprob\hfill #3}}}
\captionsetup[prob]{format=prob,labelfont=white,textfont=white, singlelinecheck=false, margin=0pt, font={bf,footnotesize}}
\xcolortheorem[background=mybrown!5 ,titlebackground=mybrown!40!black ,titleboxcolor=mybrown!40!black ,boxcolor=mybrown!40!black]{prob}{Salida}
\newcounter{problema}[section]
\newwrite\problema@out
\newenvironment{problema}%
 {\begingroup% Lets Keep the Changes Local
  \@bsphack
  \immediate\openout \problema@out \jobname_pdf.tex
  \let\do\@makeother\dospecials\catcode`\^^M\active
  \def\verbatim@processline{%
    \immediate\write\problema@out{\the\verbatim@line}}%
  \verbatim@start}%
 {\immediate\closeout\problema@out\@esphack\endgroup%
 \immediate\write18{pdflatex \jobname_pdf.tex \jobname_pdf.pdf }
   \par\small\addvspace{3ex plus 1ex}\vskip -\parskip
\noindent
    \vspace*{-2ex}%
  \makebox[0.4\linewidth][l]{%
  \begin{minipage}[t]{0.9\linewidth}
  \captionsetup{options=prob}
   \lstset{%
   caption=Entrada,basicstyle=\tiny\ttfamily\bf,language={[LaTeX]TeX}, numbersep=5mm, numbers=left, numberstyle=\tiny, % number style
   breaklines=true,framexleftmargin=10mm, xleftmargin=10mm,
   backgroundcolor=\color{mybrown!5},frameround=fttt,escapeinside=??,
   rulecolor=\color{mybrown!40!black},
   morekeywords={% Give key words here                                         % keywords
       maketitle},
   keywordstyle=\color[rgb]{0,0,1},                    % keywords
           commentstyle=\color[rgb]{0.133,0.545,0.133},    % comments
           stringstyle=\color[rgb]{0.627,0.126,0.941}  % strings
   %columns=fullflexible   
   }
  
        \lstinputlisting[]{\jobname_pdf.tex}
        \end{minipage}
       }
 \vspace{20pt}

 \IfFileExists{\jobname_pdf.pdf}{% Si el fichero existe
 \begin{minipage}[t]{0.8\linewidth}
      \setlength{\parindent}{0pt}
    \setlength{\parskip}{1ex plus 0.4ex minus 0.2ex}
    \begin{figure}[H]
    \centering
    \caption{Salida}
  \fcolorbox{mybrown!40!black}{mybrown!5}{ \includegraphics[scale=0.4]{\jobname_pdf.pdf}}
\end{figure}\end{minipage}
}


  \par\addvspace{3ex plus 1ex}\vskip -\parskip
}

\newcommand{\figcaption}[1]{\def\@captype{figure}\caption{#1}}
 \renewcommand\@seccntformat[1]%
{\color{green}\csname the#1\endcsname.\quad}
\newcommand{\helv}{\fontfamily{phv}\fontsize{9}{11}\selectfont}
\newcommand{\helvi}{\fontfamily{phv}\fontseries{b}\fontsize{9}{11}\selectfont}
%\newcounter{notas}
% \newcommand{\notas }{\stepcounter{notas}\vskip 6pt \colorbox{red}{\thechapter.\thenotas.\color{blue}{Nota:}\vskip 6pt}
%  }
\newcommand{\margen }[1]{\marginpar{\parbox{4cm}{\small\emph{#1}}}}
\newenvironment{marnota}[1]{\begin{minipage}{4cm}\small\emph{#1}
}
{\end{minipage}
}
\newcommand{\pie}[1]{\begin{changemargin}{6cm}{5cm}{-0.1cm}{5cm}{2cm}#1\end{changemargin}}
%\renewenvironment{code}{\begin{quote}}{\end{quote}}
\newcommand{\cih}[1]{%
\index{instrucciones!#1@\texttt{\bs#1}}%
\index{#1@\texttt{\hspace*{-1.2ex}\bs #1}}}
\newcommand{\ci}[1]{\cih{#1}\texttt{\bs#1}}
%Package
\newcommand{\pai}[1]{%
\index{paquetes!#1@\textsf{#1}}%
\index{#1@\textsf{#1}}%
\textsf{#1}}
% Entrada en el 'indice de entorno
\newcommand{\ei}[1]{%
\index{entornos!\texttt{#1}}%
\index{#1@\texttt{#1}}%
\texttt{#1}}
% Entrada en el 'indice para mensajes
\newcommand{\wni}[1]{%
\index{mensaje!\texttt{#1}}%
\texttt{#1}}
% Entrada en el 'indice de una palabra
\newcommand{\wi}[1]{\index{#1}#1}
%
% Instrucciones de composici'on
%
\newenvironment{command}%
    {\nopagebreak\par\small\addvspace{3.2ex plus 0.8ex minus 0.2ex}%
     \vskip -\parskip
     \noindent%
      \setlength{\arrayrulewidth}{1mm}
          \arrayrulecolor{mybrown!40!black}
     \begin{tabular}{|l|}\rowcolor{mybrown!5}\hline\rule{0pt}{1em}\ignorespaces}%
    {\\\hline\end{tabular}\par\nopagebreak\addvspace{3.2ex plus 0.8ex
        minus 0.2ex}%
     \vskip -\parskip}
%
% Composici'on de fragmentos de c'odigo
%
\newenvironment{code}%
    {\nopagebreak\par\small\addvspace{3.2ex plus 0.8ex minus 0.2ex}%
     \vskip -\parskip
     \noindent%
      \setlength{\arrayrulewidth}{1mm}
          \arrayrulecolor{thered!40!black}
     \begin{tabular}{|l|}\rowcolor{Tan!5}\hline\rule{0pt}{1pt}\ignorespaces}%
    {\\\hline\end{tabular}\par\nopagebreak\addvspace{3.2ex plus 0.8ex
        minus 0.2ex}%
     \vskip -\parskip}
% \cvhrulefill{<color>}{<thickness>}
\newcommand*\cvhrulefill[2]{%
  \leavevmode\color{#1}\leaders\hrule\@height#2\hfill \kern\z@\normalcolor}
% \crule{<color>}{<width>}{<thickness>}
\newcommand\crule[3]{%
  \color{#1}\rule{#2}{#3}\normalcolor}
\newcommand{\codigo}[1]{\lstinline!\\#1!}
% Entorno Intro
\newenvironment{intro}{\sffamily}{\vspace*{2ex minus 1.5ex}}
%%%%%lined
\NewEnviron{lined}[1]%
 {\begin{center}
  \begin{minipage}{#1}\crule{red!40!black}{#1 +0.1\linewidth}{2pt}\vspace{2ex}
   \BODY 
   \crule{red!40!black}{#1 +0.1\linewidth}{2pt}
   \end{minipage}\vspace{2ex}
 \end{center}\vspace{2ex}}
%%%%%%
\protected\def\PdfLaTeX{P\kern -.15em\raisebox{-0.21em}{D}\kern -.05em F\LaTeX}
\protected\def\PdfTeX{P\kern -.15em\raisebox{-0.21em}{D}\kern -.05em F\TeX}
\newenvironment{lcpar}{%
\begingroup
\setlength{\leftskip}{0pt plus 1fil}%
\setlength{\rightskip}{-\leftskip}%
\setlength{\parfillskip}{0pt plus 2fil}
}{%
\par\endgroup
}
\renewcommand*{\LettrineFontHook}
{\bfseries}
\renewcommand*{\LettrineTextFont}
{\bfseries}


 \makeatother
\usepackage{subcaption}
\usepackage{fancyvrb}
\usepackage{cancel}
\usetikzlibrary{fit,shapes}
\usepackage{mivenndiagram}
\usepackage{tikz-cd}% diagramas commutativos
\usetikzlibrary{matrix}
\usepackage{mathtools}
\usepackage[unicode=true,pdfusetitle,
 bookmarks=true,bookmarksnumbered=true,bookmarksopen=true,bookmarksopenlevel=1,
 breaklinks=false,pdfborder={0 0 0},backref=false,colorlinks=false]
 {hyperref}
\hypersetup{
 pdfpagelayout=OneColumn, pdfnewwindow=true, pdfstartview=XYZ, plainpages=false}
 \newsavebox{\LstBox}
 %\graphicspath{ {./img/} }
 \makeatletter
\def\input@path{{./}{./build/}{./styles/}}
\makeatother
 \graphicspath{{./}{./img/}{./build/}}
 \renewcommand{\baselinestretch}{1.5}
\raggedbottom %para que no distibuya los espacios verticales en una hoja
\setcounter{problema}{1}
%\usepackage[spanish]{babel}
 

%}
\newcommand{\triangulo}{\tikz \filldraw[scale=0.5,fill=teal!20] (0,0) -- ++(60:1) -- ++(-60:1) -- cycle ;
                 }
                 \newcommand{\linea}[1]{\overleftrightarrow{#1}}
\newcommand{\degre}{\ensuremath{^\circ}}
\newenvironment{prueba}[2]{\renewcommand{\arraystretch}{1.06}
\vskip 5pt  {
 \colorbox{teal!30}{\color{white}{Prueba:}}
 }\vskip 5pt
$ %
\begin{tabular}
[t]{ll||l}\hline
\multicolumn{2}{c||}{\mbox{Afirmaciones}} & \mbox{Razones}\\\hline\hline
1. & 
#1 
& \multicolumn{1}{|l}{\mbox{Dado}}\\
#2
\end{tabular}
\ \ \ $ 
\vskip 5pt}{\hfill  \raggedright{ \rule{0.5em}{0.5em} }\vskip 5pt }
\newcommand{\problemas}[1]
{
\section{Problemas}
\peque
\begin{changemargin}{23.2cm}{18cm}{0cm}{0cm}{0cm}
\begin{multicols}{2}
 \noindent #1
\end{multicols}
\end{changemargin}
{\setlength{\parindent}{0mm}\color{ptctitle}\rule{\linewidth}{1mm}}
}
\setlength\columnsep{40pt} 
%\renewcommand\columnseprulecolor{\color{ptctitle}}
\setlength{\tabcolsep}{8pt}
%\setlength\columnseprule{2pt} %linea entrecolumnas por defecto 0pt
\def\firstcircle{(0,0) circle (1.5cm)} 
\def\secondcircle{(45:2cm) circle (1.5cm)}
\def\thirdcircle{(0:2cm) circle (1.5cm)} 
\def\rectangulo{(-2,-2) rectangle (4,3.6)}
\def\fourcircle{(1.1,1) circle (1.5cm)}
\def\fivecircle{(-2,1) circle (1.5cm)}
\def\rect{(-4,-2) rectangle (4,3.6)} 
\def\sixcircle{(-2.3,1) circle (0.8cm)} 
\def\sevencircle{(1.1,1) circle (0.8cm)}
\def\myref#1{\thesection .\ref{#1}}

\makeatother

\usepackage[english,spanish]{babel}
\addto\shorthandsspanish{\spanishdeactivate{~<>}}

\begin{document}

\pagecolor{white} \mainmatter \pagestyle{headings} 


\chapter{Teor\'ia de conjuntos y sistemas num\'ericos }

 \chaptertoc  

\begin{competencias}  \begin{lista}  

\item Interpreta correctamente los conceptos de conjunto y elemento. 

\item Argumenta los procedimientos para realizar operaciones entre
conjuntos . 

\item Maneja con criterio las operaciones y sus propiedades en los
diferentes sistemas numéricos.

\end{lista} \end{competencias}

\begin{logros}  \begin{lista}  

\item Identifica conjuntos. 

\item Clasifica conjuntos. 

\item Realiza operaciones entre conjuntos.

\item Deduce las propiedades de los conjuntos a partir de los axiomas,
definiciones y otras propiedades.

\item Aplica bien las propiedades de los conjuntos.

\item Utiliza los diagramas de Venn para representar operaciones
y relaciones entre conjuntos. 

\end{lista} \end{logros}


\section{Introducción}

La \textsf{teoría de conjuntos}, es la rama de las matemáticas a la
que el matemático alemán Georg Cantor dio su primer tratamiento formal
en el siglo XIX. El concepto de conjunto es uno de los más fundamentales
en matemáticas, incluso más que la operación de contar, pues se puede
encontrar, implícita o explícitamente, en todas las ramas de las matemáticas
puras y aplicadas. En su forma explícita, los principios y terminología
de los conjuntos se utilizan para construir proposiciones matemáticas
más claras y precisas y para explicar conceptos abstractos como el
de infinito.


\subsection{Un poco de historia}

En el último cuarto del siglo XIX se vivió un episodio apasionante
de la historia de las matemáticas que las ligaría desde entonces a
la historia de la lógica.

Primero, Georg Boole (1815-1864) en su Mathematical Analysis of Logic
trató de presentar la lógica como parte de las matemáticas. 

Poco después Gottlob Frege (1848-1925) intentó mostrar que la aritmética
era parte de la lógica en su Die Grundlagen der Arithmetik. Pero,
dando un gran paso tanto en la historia de las matemáticas como en
la historia de la lógica, G. Cantor se había adelantado a Frege con
una fundamentación lógica de la aritmética.

Cantor había demostrado que la totalidad de los números naturales
comprendidos en el intervalo de extremos 0 y 1 no es numerable, en
el sentido de que su infinitud no es la de los números naturales.
Como una consecuencia de esa situación, Cantor creó una nueva disciplina
matemática entre 1874 y 1897: la teoría de conjuntos. 

Su obra fue admirada y condenada simultáneamente por sus contemporáneos.

Desde entonces los debates en el seno de la teoría de conjuntos han
sido siempre apasionados, sin duda por hallarse estrechamente conectados
con importantes cuestiones lógicas. 

Según la definición de conjunto de Cantor, éste es “una colección
en un todo de determinados y distintos objetos de nuestra percepción
o nuestro pensamiento, llamados los elementos del conjunto”.

Frege fue uno de los admiradores de la nueva teoría de Cantor, y dio
una definición de conjunto similar. 

En 1903 B. Russell demostraría que la teoría de conjuntos de Cantor
era inconsistente y cuestionaría la definición de conjunto en la teoría
de Cantor.

Pero pronto la teoría axiomática de Zermelo (1908) y refinamientos
de ésta debidos a Fraenkel (1922), Skolem (1923), von Newman (1925)
y otros sentaron las bases para la teoría de conjuntos actual.

Es indiscutible el hecho de que la teoría de conjuntos es una parte
de las matemáticas, es además, la teoría matemática dónde fundamentar
la aritmética y el resto de teorías matemáticas.

Es también indiscutible que es una parte de la lógica y en particular
una parte de la lógica de predicados. 

En esta historia cruzada de las matemáticas, la lógica y los fundamentos
de ambas, la teoría de conjuntos permitiría por un lado una fundación
logicista de las matemáticas; pero por otro lado la teoría de conjuntos
mirada como parte de las matemáticas proporciona el metalenguaje,
el contexto o sustrato de las teorías lógicas. Finalmente, puede ser
completamente expresada en un lenguaje de primer orden y sus axiomas
y teoremas constituyen una teoría de primer orden a la que pueden
aplicarse los resultados generales que se aplican a cualquier teoría
de primer orden.


\subsection{Teoría intuitiva de conjuntos }

La definición inicial de Cantor es totalmente intuitiva: 

\begin{definicionn}{Conjunto según Cantor}un conjunto es cualquier
colección $C$ de objetos determinados y bien distintos de nuestra
percepción o nuestro pensamiento (objetos $x$ que se denominan elementos
de $C$), reunidos en un todo.\end{definicionn}

Igual que en Frege su idea de lo que es un conjunto coincide con la
extensión de un predicado (la colección de objetos que satisface el
predicado). Esta idea sencilla y tan intuitiva resulta ser también
ingenua porque produce enormes contradicciones de inmediato, como
por ejemplo la paradoja de Russell.

Para poder mostrarlo es necesario empezar por formalizar esta teoría
intuitiva que, aparte de los símbolos para los conjuntos y sus elementos
($x$, $C$, etc.), tendrá los símbolos de pertenencia, $\in$ e igualdad
$=$ (de los objetos del lenguaje formal). 

Que $x$ es un elemento del conjunto $C$ se expresa `` $x$ pertenece
a $C$'' o bien $x$ $\in$ $C$. 

Que $x$ no es un elemento de $C$ se expresa `` $x$ no pertenece
a $C$'' o bien $x\notin C$. 

Tendremos en cuenta que no es necesario denotar siempre con mayúsculas
a los conjuntos y con minúsculas a sus elementos, ya que un conjunto
puede ser a su vez un elemento de otro conjunto e incluso podemos
considerar que en nuestra teoría no hay objetos que no sean conjuntos.

El problema ahora el el siguiente:

¿Cómo se determina una colección? 

De acuerdo con la definición intuitiva de Cantor un conjunto queda
definido si es posible describir completamente sus elementos. 

\begin{lista}

\item  El procedimiento mas sencillo de descripción es nombrar cada
uno de sus elementos, esta descripción se llama definición por extensión;
es conocida la notación de encerrar entre llaves los elementos del
conjunto. 

\begin{ejems}{Listar un conjunto }\begin{enumerate} 

\item$A={a,b,c}$. Donde $A$ es el conjunto formado por la colección
de objetos $a,b$ y $c$.

\item $B=\left\{ \oplus,\ominus,\otimes,\oslash,\odot\right\} $
Donde $B$ es el conjunto formado exactamente por esos cinco círculos.
Entonces es cierto que $b\in A$ y que $\oplus\in B$.

\end{enumerate}\end{ejems}

El inconveniente para este método de listado o enumeración de los
elementos del conjunto es que éstos deben poseer un número finito
de elementos y, en la práctica, un número muy pequeño.

¿Entonces qué hacer cuando la colección es infinita, o cuando es finita
pero numerosa? 

\item Cuando el número de elementos del conjunto es infinito (como
el de los número impares) o demasiado numeroso (como el de todas las
palabras que pueden formarse con el alfabeto latino) se utiliza el
método de definición por extensión, que consiste en la descripción
de un conjunto como la extensión de un predicado, esto es, mediante
una o varias propiedades (el predicado) que caracterizan a los elementos
de ese conjunto.

En principio podría tomarse cualquier lengua natural para describir
los objetos (español, inglés, italiano, vasco, catalán, etc), sin
embargo es preferible utilizar un lenguaje formal que ofrezca rigor
y precisión. 

Dicho lenguaje debe ser suficientemente rico; esto es, lo suficientemente
expresivo como para poder describir todas las colecciones matemáticas.
Pero también lo suficientemente restrictivo como para limitarse a
sólo las colecciones de objetos matemáticos. 

Para expresar predicados utilizaremos el lenguaje formal de la la
lógica de predicados de primer orden (el lenguaje de la lógica de
proposiciones con los símbolos lógicos de las conectivas $\sim,\wedge,\vee,\rightarrow,\leftrightarrow$
más los cuantificadores universal $\forall$ y existencial $\exists$)
al que se añade variables, igualdad y el relator binario de pertenencia.

Este lenguaje puede ser ampliado con los símbolos propios de las operaciones,
relaciones o funciones del lenguaje específico de teoría de conjuntos.

En la primera parte, al presentar la Teoría básica de conjuntos, utilizaremos
con frecuencia el lenguaje natural para describir propiedades.

Estas propiedades pueden ser aritméticas ($>,<,/$, etc.) o matemáticas
en general, pero también pueden ser propiedades expresadas en lenguaje
natural (nombres, verbos,...) que describan colecciones no estrictamente
matemáticas.

\begin{ejems}{Descripción de un conjunto por extensión}\begin{enumerate} 

\item $C={x\in I\!\! N/0<x<230000\wedge2/x}$, donde $I\!\! N$ es
el conjunto de los números naturales con la ordenación habitual, $<$
significa “menor que” y $2/x$ significa que “$2$ divide a $x$”. 

\item $D=\{x/x$ es una palabra de $2$ letras del alfabeto griego
(pueden estar repetidas)$\}$

\item $E={x/P_{2}(x)\vee P_{3}(x)\vee\cdots\vee P_{10}(x)}$ . Donde
$P_{i}(x)$ significa “$x$ es una palabra de$i$ letras del alfabeto
griego (pueden estar repetidas).\end{enumerate}\end{ejems} \end{lista}


\subsubsection{Problemas en la teoría intuitiva de conjuntos: la paradoja de Russell
\footnote{La paradoja original era sobre el barbero de un pueblo que afeitaba
a todos los del pueblo que no se afeitaban a sí mismos: ¿Se afeita
entonces el barbero a sí mismo? %
}}

Pero la definición intuitiva de conjunto como el de una colección
de objetos ‘describible’ por un predicado conduce inevitablemente
a ciertas contradicciones que se llaman paradojas, la más célebre
es la conocida como paradoja de Russell: 

Consideremos el conjunto $A={x:x\notin x}$, descrito mediante el
predicado del lenguaje formal $x\notin x$ . 

Obviamente, para cualquier $b$, $b\in A$ si y sólo si $b\notin b$.
Es decir, está en $A$ cuando verifica las condiciones que definen
a $A$. Pero, ¿qué sucede con el propio $A$? 

Evidentemente, $A\in A$ si y sólo si $A\notin A$. Pero este resultado
es contradictorio. En vano se debe intentar descubrir un error en
el razonamiento, más bien parece que el problema proviene de admitir
expresiones como $A\in A$ (o conjuntos como el conjunto de todos
los conjuntos que produce también paradojas). Se ha visto claramente
que el concepto de conjunto no es tan sencillo y que identificarlo
sin mayor investigación con el de colección resulta problemático.
Para evitar la paradoja de Russell, y otras de esta naturaleza, es
necesario tener más cuidado en la definición de conjunto, lo veremos
en lo que sigue. Otras paradojas, de hecho las primeras en descubrirse,
afectaban a colecciones grandes, como por ejemplo la de los ordinales,
o la de todos los conjuntos. entonces estas colecciones no podrían
ser conjuntos. 


\subsubsection{Solución de las paradojas}

Una solución radical al problema de las paradojas es la propuesta
en 1903 por Russell, su Teoría de Tipos. 

Observa que en todas las paradojas conocidas hay una componente de
reflexividad, de circularidad. Técnicamente se evitan las paradojas
al eliminar del lenguaje las formaciones circulares.

Se reconoce que nuestro universo matemático no es plano, sino jerarquizado,
por niveles, y que el lenguaje más adecuado para hablar de un universo
debe tener diversos tipos de variables que correspondan a cada nivel;
en particular, la relación de pertenencia se da entre objetos de distinto
nivel. 

En 1908 Zermelo da como solución la definición axiomática de la Teoría
de Conjuntos, refinada más tarde por Fraenkel, Skolem, von Neumann
y otros. En esta teoría se evita que las colecciones que llevaban
a las paradojas puedan ser conjuntos. De hecho, en la solución de
Zermelo-Fraenkel, una colección de objetos será un conjunto si los
axiomas la respaldan. Dichos axiomas permiten formar conjuntos a partir
de conjuntos previamente construídos y postulan la existencia del
$\textrm{Ø}$ y de al menos un conjunto infinito. Sin embargo, en
la solución de von Neumann se admiten colecciones que no son conjuntos,
las denominadas clases últimas.

Se definen clases mediante propiedades, sin restricción, pero habrá
que mostrar que se trata de conjuntos viendo que pertenecen a alguna
clase. Las clases últimas, como la clase universal o la de los ordinales,
no pertenecen a ninguna otra clase. 


\subsubsection{El Universo matemático}

La idea intuitivamente más fructífera y también la más extendida es
nuestro universo matemático, esto es, el que contiene todas las colecciones
de objetos matemáticos, pero solamente los objetos matemáticos constituyen
una jerarquía de conjuntos, la denominada Jerarquía de Zermelo. 

En la construcción de los conjuntos que formarán la jerarquía se parte
de una colección inicial $M_{0}$ de objetos dados y a continuación
se construye una colección $M_{1}$ de conjuntos de elementos de $M_{0}$
, después una colección $M_{2}$ de conjuntos de objetos de $M_{0}$
y $M_{1}$ , etc. ??, el universo de conjuntos construídos es una
jerarquía.

Para proporcionar mayor precisión debemos responder a las preguntas
siguientes:
\begin{enumerate}
\item ¿Cual será nuestra colección de partida, ?$M_{0}$ 
\item ¿Qué conjuntos de objetos de niveles inferiores se toman para formar
nuevos niveles en la jerarquía? 
\item ¿Hasta dónde se extiende esta jerarquía?. 
\end{enumerate}
Para responder a la primera pregunta debemos considerar si nos interesa
tomar objetos que no sean conjuntos o si nos basta con partir de un
primer nivel que sea sencillamente el conjunto $\emptyset$.

Está claro que se toman sólo objetos matemáticos, pero habrá que ver
que es suficiente y que podremos finalmente contar en la jerarquía
con todos los objetos matemáticos.

Una respuesta a la segunda pregunta que parece razonable es, al ir
tomando nuevos conjuntos, que éstos se puedan describir con nuestro
lenguaje. Al tomar esta opción formamos la Jerarquía de conjuntos
que se pueden construir. Otra posibilidad es tomar como objetos de
un nuevo nivel a todos los posibles. Veremos que esta es la opción
de Zermelo. 

Finalmente, la tercera de las preguntas es hasta donde se extiende
la jerarquía. 

La respuesta es que la jerarquía de conjuntos no tiene fin, siempre
se pueden construir nuevos niveles. Para precisar un poco más esta
imagen intuitiva de nuestro Universo matemático es conveniente contar
con algunas nociones de teoría de conjuntos básica y con el concepto
de ordinal. 


\subsubsection{Teoría axiomática de conjuntos }

Recordemos los componentes de una teoría axiomática: 
\begin{enumerate}
\item El lenguaje o símbolos formales de la teoría. 
\item Los axiomas, que son proposiciones acerca de los objetos de la teoría
y que imponen el funcionamiento de dichos objetos. 
\item Los teoremas, que son todas las proposiciones demostrables con herramientas
lógicas a partir de los axiomas. 
\end{enumerate}
En la teoría de conjuntos axiomática de Zermelo Fraenkel se usará
el lenguaje formal de la lógica de predicados de primer orden. 

Las variables de dicho lenguaje formal se referirán a conjuntos; es
decir, en la interpretación usual todos los objetos son conjuntos.

Es decir, existir será sinónimo de ser un conjunto.

El lenguaje básico sólo tiene el relator binario de pertenencia, pero
se extiende, mediante definiciones pertinentes, para dar cabida a
operaciones. 

Los conceptos primitivos de esta teoría son el de conjunto y el de
pertenencia. 

En realidad la mayoría de los axiomas sirven para garantizar la existencia
de los conjuntos que nos interesa tener. 

Por ello la idea de construcción es esencial en la teoría axiomática
de Zermelo-Fraenkel (que notaremos ZF). En la teoría axiomática de
conjuntos se respeta la idea fundamental de aceptar que una colección
de objetos pueda ser un conjunto, pero se impone la condición de que
todos los objetos de una colección deben haberse formado antes de
definir dicha colección, y de esta manera se evitarán los problemas
que conducen a las paradojas.

Uno de los axiomas de la teoría (se verá más adelante) impondrá esta
restricción: ”Si $X$ es un conjunto ya construido existe un conjunto
$Y$ formado por los elementos de $X$ que satisfacen un predicado
$P$ que los describe (o lo que es lo mismo, una fórmula con al menos
una variable libre)”. Así un predicado describirá un conjunto sólo
si los objetos han sido ya construidos (son de otro conjunto X) y
además satisfacen el predicado. Con esta restricción a la definición
de conjunto de Cantor desaparece la paradoja de Rusell ya que para
que $A={x:x\notin x}$ sea un conjunto se debería tener un conjunto
$X$ a partir del cual construirse; es decir, $A={x\in X:x\notin x}$.
¿Cómo se resuelve la paradoja? 

Al construirse a partir de un conjunto ya construido desaparece el
problema. Ahora, para cualquier $b$ se verifica: $b\in A$ si y sólo
si $b\in X$ y $b\notin b$. En realidad, puesto que la condición
$b\notin b$ la cumplen todos, $A$ será el propio $X$. Además, es
imposible que exista el conjunto de todos los conjuntos. 

Desgraciadamente, Bertrand Russell descubrió que la axiomática de
Frege era contradictoria.

En efecto, uno de los axiomas básicos de Frege afirmaba lo a siguiente: 

Para toda propiedad $\phi(X)$ definida en la teoría, existe un conjunto
$Y$ cuyos elementos son exactamente los conjuntos $X$ que cumplen
$\phi(X)$. En otros términos, Frege postulaba la existencia del conjunto
$Y={X|\phi(X)}$. 

Lo que Russell observó fue que esto podía aplicarse a $\phi(X)\equiv X\notin X$,
que era una propiedad trivialmente definida en la teoría de Frege,
de modo que debía existir un conjunto $R={X:X\notin X}$, que claramente
nos lleva a la contradicción $R\in R\Leftrightarrow R\notin R$. 

A partir de aquí la minuciosa lógica de Frege permitía probar con
el mismo rigor que $2+2=4$ y que $2+2=5$, por lo que su teoría se
volvía inservible. El mismo Russell, junto con A. N. Whitehead, present\'{ó}
un tiempo después otra teoría axiomática que, al menos en apariencia,
estaba exenta de contradicciones, si bien era tan inútil como la de
Frege, esta u vez no por contradictoria sino por complicada. Se trata
de los Principia Mathematica. 

La primera teoría axiomática construida por un matemático a gusto
de los matemáticos fue la de Zermelo. La forma en que Zermelo evitó
la paradoja de Russell fue debilitar el axioma de formación de conjuntos
de Frege, reduciéndolo a: \begin{axioma}{Zermelo}Para toda propiedad
$\phi(X)$ definible en la teoría y todo conjunto $U$ , existe un
conjunto $Y$ cuyos elementos son exactamente los elementos $X\in U$
que cumplen $\phi(X).$ \end{axioma}

Así, lo que Zermelo postulaba era la existencia de 
\[
Y=\left\{ X\in U:\phi\left(X\right)\right\} .
\]


Ahora bien, este axioma sólo permite definir conjuntos a partir de
otros conjuntos, por lo que Zermelo tuvo que añadir otros axiomas
que garantizaran la existencia de aquellos conjuntos necesarios que
no podían obtenerse como subconjuntos de otros conjuntos dados. Enseguida
describiremos con detalle la axiomática de Zermelo, pero antes daremos
a algunas indicaciones sobre la lógica matemática que subyace a toda
teoría de conjuntos moderna.


\subsubsection{El lenguaje de la Teoría de Conjuntos}

Un lenguaje formalizado está constituido por un conjunto de símbolos
básicos y por reglas que nos permiten formar expresiones más complicadas
partir de esos símbolos originales. 

A continuación presentamos el lenguaje formalizado $L_{\in}$ con
le cual escribiremos la Teoría de Conjuntos. 

Este lenguaje puede entenderse de dos maneras distintas: como lenguaje
formal y como abreviaturas de expresiones en español. Esta segunda
interpretación será posiblemente la conveniente en un curso introductorio
como este.

Los símbolos del lenguaje formal de la teoría de conjuntos serán:
\begin{enumerate}
\item Los símbolos de conjuntos serán las letras del alfabeto, mayúsculas
y minúsculas, es decir: Variables: $x,y,z,X,Y,Z,x1,x2,...$, en general,
las últimas letras del alfabeto latino, minúsculas o mayúsculas con
o sin subíndices. Su significado es el habitual en matemáticas y su
rango son los conjuntos.
\item Constantes: $a,b,c,A,B,C,...$ , en general, las primeras letras del
alfabeto latino. Sirven para referirnos a conjuntos específicos.
\item El símbolo de la relación de pertenencia entre conjuntos es $\in$.
\item Los símbolos lógicos de la lógica de predicados: ($\sim$negación),
$\wedge$(conjunción), $\vee$(disyunción), $\rightarrow$(condicional),
$\leftrightarrow$(bicondicional),$\Rightarrow$ (implicación), $\Leftrightarrow$o
$\equiv$ (equivalencia), $\forall$ (cuantificador universal) y $\exists$
(cuantificador existencial). Con estos signos básicos se generan todas
las fórmulas de la teoría de con juntos. \medskip{}
Cualquier cadena finita formada por estos símbolos es una expresión
o del lenguaje, pero no toda expresión es aceptable o significativa.
Sólo aceptaremos aquellas a las que llamaremos Fórmulas de $L_{\in}$ 
\item $($, $)$ (paréntesis). Usados como signos de puntuación
\item Las reglas de formación de fórmulas son las habituales en la lógica
de primer orden. A saber:

\begin{enumerate}
\item $x\in y$ y $x=y$ son fórmulas. Para cualesquiera variables $x$,
$y$. 
\item Si  $\phi$ y $\psi$ son fórmulas, también lo son: $\sim\phi$, $\phi\wedge\psi$,
$\phi\vee\psi$, $\phi\rightarrow\psi$y $\phi\leftrightarrow\psi$. 
\item Si $\phi$ es una fórmula, $\left(\forall x\right)\phi$ y $\left(\exists x\right)\phi$
también lo son.
\end{enumerate}
\end{enumerate}
Solamente aquellas expresiones obtenidas por la aplicación de (un
o número finito de) estas reglas es una fórmula de $L_{\in}$ . 

Entonces el lenguaje $L_{\in}$ se generan mediante las reglas precedentes: 


\section{Relaciones y axiomas en la teoría de conjuntos}

\begin{tndefinido}{Conjunto y elemento}

Los términos conjunto y elemento , los consideramos como primitivos
, esto es no definidos. 

\end{tndefinido}

\begin{definicionn}{ Pertenencia}

$a\in B$ significa que el objeto $a$ es un elemento del conjunto
$B$

\end{definicionn}

\begin{definicionn}{ No pertenece}

$a\notin A$ significa que el objeto $a$ no es un elemento del conjunto
$A.$\end{definicionn}

\begin{ejem} Dado el conjunto $A=\{1,2,3,4\}$ entonces $1\in A$,
porque $1$ es un elemento de $A.$

En cambio $b\notin A$, porque $b$ no es un elemento del conjunto
$A.$

\end{ejem}

\begin{ideas}{Representación}

Se dice que un nombre, diagrama o una proposición representa a un
objeto si explica de forma precisa cuales son sus propiedades. 

\end{ideas}

Como ejemplos de representación están:
\begin{enumerate}
\item La fórmula $y=f(x)$ que representa al una función $f$.
\item Si $A:=\{1,2,3,4,5\}$, en este caso $A$ representa al conjunto $\{1,2,3,4,5\}.$
\item Los grafos dirigidos son diagramas que representan a las relaciones.
\item Una tabla de datos es un diagrama que representa a una relación.
\end{enumerate}
\begin{ideas}{Igualad}

Se dice que dos representaciones son iguales si representan al mismo
objeto,

\end{ideas}

Por ejemplo $x=2$, en este caso los nombres equis y dos representan
al mismo número.

\begin{axioma}{Propiedades de la igualdad}
\begin{enumerate}
\item $\left(\forall x\right)\left(x=x\right),$ propiedad reflexiva.
\item $\left(\forall x\right)\left(\text{\ensuremath{\forall}\ensuremath{y}}\right)\left(x=y\Longleftrightarrow y=x\right),$
propiedad simétrica.
\item $\left(\forall x\right)\left(\text{\ensuremath{\forall}\ensuremath{y}}\right)\left(\forall z\right)\left(x=y\wedge y=z\Rightarrow x=z\right),$
propiedad transitiva.
\end{enumerate}
\end{axioma}

El siguiente axioma nos muestra el uso de la idea de igualad. 

\begin{axioma}{ (Axioma de extensión)}\label{A1}

Dos conjuntos son iguales si y solo si tienen los mismos elementos.
Esto es: el conjunto $A$ es igual al conjunto $B$ si todo elemento
de $A$ es elemento de $B$ y si todo elemento de $B$ es elemento
de $A$. 

En el lenguaje $L_{\in}$ se expresa.

$\forall x(x\in A\Leftrightarrow x\lyxmathsym{∈}B)\Rightarrow\left(A=B\right)$.
\end{axioma}

Este axioma nos permite demostrar la unicidad de muchos conjuntos
definidos de alguna forma concreta. Haremos un ejemplo, dejando el
resto de casos de unicidad como ejercicio, explícito o implícito: 

\begin{ejem}{ Igualdad de conjuntos }

Por ejemplo A = \{ Los ríos de de América\} y B = \{ Los Ríos de Colombia\}.

Es claro que en Colombia no están todos los ríos de América, por tanto
$A\neq B$, Es decir que los conjuntos $A$ y $B$ no tienen los mismos
elementos.\end{ejem}

Veamos otro ejemplo :

El primer axioma de Zermelo, afirma que si dos conjuntos tienen los
mismos elementos entonces son iguales (el recíproco es un caso particular
de un principio lógico: si $X=Y$ entonces todo lo que vale para $X$
vale para $Y$ ).

Según hemos comentado en la introducción , el problema que presenta
la fundamentación de la teoría de conjuntos es que no podemos permitirnos
el lujo de postular que toda propiedad define un conjunto. 

En su lugar, la teoría de Zermelo postula la existencia de conjuntos
definidos por ciertas propiedades inofensivas (no como $X\notin X$).
Tal vez el conjunto más inofensivo de todos sea el que nos da el axioma
del conjunto vacío 

\begin{axioma}{ Conjunto vacío} Existe un conjunto que no contiene
ningún elemento.
\[
\left(\exists X\forall x\right)x\notin X
\]


\end{axioma}

Este axioma afirma la existencia de un conjunto que no tiene elementos.
Dicho conjunto es único, pues dos conjuntos sin elementos tendrían
los mismos elementos, como lo probaremos mas adelante. Esto nos permite
definir el término
\[
\textrm{\ensuremath{\emptyset}}\equiv\left\{ X:\left(\forall U\right)U\notin X\right\} .
\]


\nota\ Observe que $\emptyset$ no es un signo del lenguaje de la
teoría de conjuntos, si no una abreviatura de un término que puede
eliminarse de cualquier expresión sin más que sustituirla por el miembro
derecho de la definición. 

\begin{lema}{ Unicidad del vacío} Existe un único conjunto que no
contiene ningún elemento.\end{lema} 

\begin{dems}

Según el axioma de extensión, dos conjuntos A y B son iguales si y solo si \[ \forall Z (Z\in A\Leftrightarrow Z\in B). \] Por otro lado, el axioma del conjunto vacío dice \begin{equation}\label{vacio} \exists Y\forall Z{\sim}(Z\in Y) \end{equation} Si suponemos que existen conjuntos $Y$ y $Y'$ que satisfacen la condición~\eqref{vacio} y $Y\neq Y'$; entonces es claro que, para todo $Z$, se satisface \[ Z\in Y\Leftrightarrow Z\in Y' \] pues ambas expresiones son falsas. Así, por el axioma de extensión resulta que $Y=Y'$.

\end{dems}

\begin{definicionn}{ Subconjunto}\label{subset}Decimos que $X$
es subconjunto de $Y$ , en símbolos , $X\subseteq Y$ , si y sólo
si todo elemento de $X$ es un elemento de $Y$. O sea,\label{subc}
\[
X\subseteq Y\Leftrightarrow\forall x(x\in X\Rightarrow x\in Y).
\]


\end{definicionn} Usando la definición \myref{subset} se puede escribirse
más abreviado el axioma de extensión;
\[
\left(\forall X\text{\ensuremath{\forall}}Y\right)(X\subseteq Y\wedge Y\subseteq X\Rightarrow X=Y).
\]


\nota Si tenemos dos conjuntos $A$ y $B$ tal que $A\subseteq B$,
de la definición \myref{subset} se sigue que es suficiente que exista
al menos un elemento del conjunto $A$ que no sea elemento de $B$
para que $A$ no sea subconjunto de $B,$ en este caso se dice que
$A\nsubseteq B.$

\begin{ejem}{ Subconjuntos }

Dados dos conjuntos $A=\left\{ 1,3,5\right\} $ y $B=\left\{ 1,2,3,4,5,6,7\right\} $,
que relación hay entre ellos

\end{ejem}

\solucion  La relación que existe entre ellos es que $A$ es subconjunto
de $B$, es decir, $A\subseteq B.$ En efecto se observa por simple
inspección que todo elemento de $A$ es también elemento de $B.$

\begin{definicionn}{Subconjunto propio}\label{subcp}

Decimos que un conjunto $X$ es un subconjunto propio de un conjunto
$Y$, o parte de $Y$, si se verifica $X\subseteq Y$ y además existe
algún $x\in Y$ tal que $x\notin X.$

\end{definicionn}

\notacion Si un conjunto $X$ es subconjunto propio de un conjunto
$Y$ se denota $X\subset Y.$

La definición de subconjunto propio también se puede expresar de la
siguiente forma;
\[
X\subset Y\Leftrightarrow X\subseteq Y\,\wedge X\neq Y.
\]


\begin{ejem}{Subconjunto Propio}

Dados los conjuntos $A=\left\{ 2,4,6\right\} $ y $B=\left\{ 1,2,3,4,5,6\right\} $.
Indicar si $A$ es subconjunto propio de $B,$ 

\end{ejem}

\solucion Es evidente que $A\subseteq B$, además los elementos de
$B$ $1,3$ y $5$ no pertenecen al conjunto $A,$ por tanto existe
un elemento de $B$ que no está en $A.$ Por lo que podemos asegurar
que $A\subset B$ de acuerdo con la definición \myref{subcp}.

\begin{definicionn}{Comparables}

Se dice que dos conjuntos $X$ e $Y$ son comparables si $A\subseteq B\vee B\subseteq A$.

y se dice que no son comparables si $A\nsubseteq B\wedge B\nsubseteq A$.

\end{definicionn}

\begin{ejem}{Comparables}\label{cc}

Dados los conjuntos $A=\left\{ a,e,i\right\} $ y $B=\left\{ a,e,o,i,u\right\} .$
Determine si los conjuntos $A$ y $B$ son comparables.

\end{ejem}

\solucion

Es evidente que $A\subseteq B,$ por tanto los conjuntos $A$ y $B$
son comparables.

\begin{ejem}{Comparables}

Dados los conjuntos $M=\left\{ 1,5,7,8\right\} $ y $N=\left\{ 2,5,6,8,9\right\} .$
Determine si los conjuntos $M$ y $N$ son comparables.

\end{ejem}

\solucion Los conjuntos $M$ y $N$ no son comparables ya que solo
el elemento $5\in M$ está en $N$. Por tanto $M\nsubseteq N$ y solo
el $5\in N$ está en $M$ de lo que se concluye que $N\nsubseteq M,$
es decir $M\nsubseteq N\wedge N\nsubseteq M$ y de acuerdo con la
definición \myref{cc} los conjuntos $M$ y $N$ no son comparables.

\begin{axioma}{ Especificación o separación} \label{A3}A todo conjunto
$A$ y a toda condición o fórmula $\psi=S(x)$, corresponde un conjunto
$B$ cuyos elementos son precisamente aquellos elementos de $A$ que
cumplen $S(x)$. \end{axioma}

El axioma de especificación en el lenguaje $L_{\in}$ se escribe 
\[
\left(\forall A\exists B\right)\left(\forall z\right)\left(z\in B\Leftrightarrow\left(z\in A\wedge S\left(x\right)\right)\right)\,\mbox{ó \ensuremath{B=\left\{ x\in A\,:\,\psi=S(x)\right\} .}}
\]


Este axioma nos dice que para cualquier propiedad (expresada por $S(x)$)
y cualquier conjunto $A$ existe el subconjunto de $A$ formado por
los elementos que verifican esa propiedad. Obviamente este conjunto
es único. 

Este Axioma ayuda a construir subconjuntos de un conjunto dado.

\begin{definicionn}{}Si $S(x)$ es una fórmula de $L_{\in}$ y $A$
un conjunto, el conjunto cuya existencia está garantizada por el axioma
\myref{A3} se denotará con el símbolo ${x\in A\,:\, S(x)}$ y se
lee “el conjunto de los elementos de $A$ tales que cumplen $S(x)$”.
\end{definicionn}

Por último, cabe destacar que este no es propiamente un axioma si
no más bien un esquema. En efecto, para cada fórmula $S(x)$ de $L_{\in}$
tenemos un axioma distinto, o sea, hay una cantidad ilimitada de instancias
para este axioma.

Recordemos que la paradoja de Russell se produce al tratar de construir
el conjunto de todos los conjuntos que verifican una propiedad cualquiera
$S(x)$. Este axioma limita nuestra capacidad de formar conjuntos
de objetos que verifican una cierta propiedad, sólo podemos referirnos
a aquellos elementos que perteneciendo a un cierto conjunto dado,
verifican la propiedad en cuestión. Veamos que esta restricción evita
que se produzca la paradoja.

Para ello tratemos de formar la clase de Russell. Dado un conjunto
$A$ , el axioma de extensión nos permite formar el conjunto 
\[
R=\left\{ x\in A:x\notin x\right\} 
\]
En este caso tenemos que si $R\in A$ y $R\notin R,$

lo cual es contradicción, luego $R\notin R$, lo que, que a diferencia
de antes, no es contradictorio, sólo implica que $R\notin A$.

\begin{teorema}{}

No hay ningún conjunto $A$ tal que $\forall x\left(x\in A\leftrightarrow x\notin x\right),$
es decir el conjunto $R=\left\{ x\,:\, x\notin x\right\} $ no existe.

\end{teorema}

\begin{dems}

Vamos a efectuar una prueba indirecta.

Es decir suponemos que existe un conjunto $A$ tal que 
\[
\forall x\left(x\in A\leftrightarrow x\notin x\right).
\]
Este vale en particular para el caso $x=A,$ de modo que 
\[
A\in A\leftrightarrow A\notin A
\]
lo cual es contradictorio. 

\end{dems}

Si bien hay un conjunto mínimo, el conjunto vacío, veremos ahora que
no hay un conjunto máximo. El proceso de construcción de conjuntos
no puede darse nunca por acabado.

\begin{teorema}{Conjunto de todos los conjuntos}

No existe el conjunto de todos los conjuntos, es decir $\left(\sim\exists A\forall x\right)\left(x\in A\right)$
\end{teorema}

\begin{dems}

Supongamos que si existe el conjunto de todos los conjuntos y llamaremos
$A$, con la propiedad de que $\left(\forall x\right)\left(x\in A\right)$.
Entonces podemos aplicar el axioma \ref{A3} (axioma de separación)
al conjunto $A$ con la proposición $x\notin x$, obteniendo de ese
modo el conjunto $B=\left\{ x\,:\, x\in A\wedge x\notin x\right\} .$
Sin embargo la propiedad $x\in A\wedge x\notin x$ se puede sustituir
por $x\notin x$, porque la propiedad $x\in A$ la verifican todos
los objetos$x.$ al efectuar la sustitución obtenemos el conjunto
$B=\left\{ x\,:\, x\notin x\right\} $, pero ese conjunto no existe
, lo cual es una contradición.

\end{dems}

\begin{axioma}{ Axioma de Pares}

Dados dos conjuntos $X$ e $Y$, existe un conjunto cuyos únicos elementos
son $X$ e $Y.$

Su expresión en el lenguaje $L_{\in}$ es 
\[
\left(\forall X\forall Y\exists Z\right)\forall x\left(x\in Z\Leftrightarrow\left(x=X\vee x=Y\right)\right)\,\acute{o}\: Z=\left\{ X,Y\right\} .
\]


\end{axioma}

Resulta claro por el axioma \ref{A1} que este conjunto es único.
Lo denotaremos 
\[
\{X,Y\}
\]
 y lo llamaremos par no ordenado $X,Y$.

El axioma \ref{A1}también garantiza de un conjunto cuyo único elemento
es el conjunto $X$

\[
\left\{ X,X\right\} =X,
\]


\begin{ejem}{ conjunto $\left\{ \emptyset\right\} $ }

Tomando $X=Y=\emptyset,$ llegamos a la conclusión de que $\left\{ \emptyset\right\} $
es un conjunto no vacío, ya que contiene un elemento. Ahora si tomamos
$X=\emptyset,\, Y=\left\{ \emptyset\right\} ,$ llegamos a la conclusión
que $\left\{ \emptyset,\left\{ \emptyset\right\} \right\} $ es un
conjunto con dos elementos. 

\end{ejem}


\section{Operaciones entre conjuntos}

En realidad, los axiomas que tenemos hasta ahora sólo garantiza la
existencia del conjunto vacío, y la existencia de conjuntos más pequeños
a partir de conjuntos ya conocidos. los siguientes axiomas nos permitirán
construir conjuntos más grandes.

El siguiente axioma nos permite construir uniones arbitrarias de conjuntos
(siempre que los conjuntos que queramos anexar formen un conjunto:
recordemos que no existe el conjunto de todos los conjuntos).

\begin{axioma}{Union de conjuntos}\label{union}

Para todo conjunto $S,$ existe un conjunto, que denotaremos $\bigcup S,$
tal que $x\in\bigcup S$ si y sólo si $x\in X$ para algún $X\text{\ensuremath{\in}\ensuremath{S}.}$

\end{axioma}

\nota  El conjunto $\bigcup S$ es la unión de los subconjuntos $X$
de $S.$

\begin{ejem}{Union de conjuntos} Si $S=\left\{ X,Y\right\} ,$ entonces
obtenemos la unión de $X$ e $Y,$ que denotamos $X\cup Y.$ Además,
tomando $S=\left\{ X\cup Y,Z\right\} ,$ podemos obtener $\left(X\cup Y\right)\cup Z$
y en general, la unión de un número finito de conjuntos.

\end{ejem}

\nota Si $S=\left\{ X,Y\right\} $ el conjunto $\bigcup S$ se puede
definir como: 
\[
\bigcup S=X\cup Y=\{x\,:\: x\in X\vee x\in Y,\: X,Y\in S\}.
\]


\begin{ejem}{Union}

Dado el conjunto $S=\left\{ \left\{ 1,2,3\right\} ,\left\{ 0,2,4,6\right\} ,\left\{ 5,7,9\right\} ,\left\{ a,b,c,d\right\} \right\} $.
Determine $\bigcup S.$

\end{ejem}

\solucion Sean $X_{1}=\left\{ 1,2,3\right\} ,\, X_{2}=\left\{ 0,2,4,6\right\} ,\, X_{3}=\left\{ 5,7,9\right\} $
y $X_{4}=\left\{ a,b,c,d\right\} ,$ tenemos que los elementos $1,2,3$
están $\bigcup S$ porque son elementos $X_{1}$ de la misma forma
podemos decir que los elementos $0,2,4,6;$ $5,7,9;$ y $a,b,c,d$
están en $\bigcup S$ porque pertenecen a los conjuntos $X_{2},X_{3}$
y $X_{4}$ respectivamente, entonces de acuerdo con el axioma \myref{union}
$\bigcup S=X_{1}\cup X_{2}\cup X_{3}\cup X_{4}=\left\{ 0,1,2,3,4,5,6,7,9,a,b,c,d\right\} $.

Aplicando el axioma de separación podemos demostrar la existencia
del la intersección de conjuntos

\begin{lema}{Intersección de conjuntos}\label{interseccion}

Dados dos conjuntos, $X,Y,$ existe un (único) conjunto $Z$ tal que
$x\in Z$ si y sólo si $x\in X$ y $x\in Y.$

\end{lema}

\begin{dems} 

Definamos la propiedad $P\left(x\right)$ que sea $x\in Y.$ Entonces,
por el axioma de separación , existe el conjunto $Z=\left\{ x\in X\,:\, P\left(x\right)\right\} $
, que es el conjunto buscado. La demostración de la unicidad se deja
como ejercicio, esto se demuestra a partir del axioma de Extensión.

\end{dems}

\begin{definicionn}{Intersecci\'on de conjuntos}\label{inter}

El conjunto $Z$ cuya existencia acabamos de demostrar se llama intersección
de $X$ e $Y,$y se denota $Z=X\cap Y.$

De acuerdo con el lema \myref{interseccion} podemos definir la intersección
entre dos conjuntos $X$ e $Y$ como: 
\[
\mbox{\ensuremath{X}\ensuremath{\cap}\ensuremath{Y=\ensuremath{\left\{ x\,:\, x\in X\wedge x\in Y\right\} }.}}
\]


\end{definicionn}

\begin{ejem}{Intersección}

Dados conjuntos $A=\left\{ -1,0,1,2,3,4,5\right\} $ y $B=\left\{ 1,2,3,4,5,6,7,8,9\right\} .$
Determine $A\cap B.$ 

\end{ejem}

\solucion Sean las proposiciones abiertas $A_{x}=x\in A$, $B_{x}=x\in B$
y $\left(A\cap B\right)_{x}=x\in\left(A\cap B\right)\equiv x\in X\wedge x\in B.$
Usemos las tablas de verdad para resolver la proposición $x\in X\wedge x\in B.$

\begin{table}[H]
\centering

\caption{Tabla de verdad de la Intersección de conjuntos.}


\setlength\arrayrulewidth{1pt}\arrayrulecolor{ptctitle} 

\begin{tabular}{cc|c|c}
\arrayrulecolor{ptctitle}\hline\cellcolor{ptctitle!50}$x$ & \cellcolor{ptctitle!50}$A_{x}$ & \cellcolor{ptctitle!50}$B_{x}$ & \cellcolor{ptctitle!50}$\left(A\cap B\right)_{x}$\tabularnewline
\hline\cellcolor{ptcbackground}$-1$ & \cellcolor{ptcbackground} $v$ & \cellcolor{ptcbackground} $f$ & \cellcolor{ptcbackground}$f$\tabularnewline
\hline\cellcolor{gray!50}$0$ & \cellcolor{gray!50} $v$ & \cellcolor{gray!50} $f$ & \cellcolor{gray!50}$f$\tabularnewline
\hline\cellcolor{ptcbackground}$1$ & \cellcolor{ptcbackground} $v$ & \cellcolor{ptcbackground} $v$ & \cellcolor{ptcbackground}$v$\tabularnewline
\hline\cellcolor{gray!50}$2$ & \cellcolor{gray!50} $v$ & \cellcolor{gray!50} $v$ & \cellcolor{gray!50}$v$\tabularnewline
\hline\cellcolor{ptcbackground}$3$ & \cellcolor{ptcbackground} $v$ & \cellcolor{ptcbackground} $v$ & \cellcolor{ptcbackground} $v$\tabularnewline
\hline\cellcolor{gray!50}$4$ & \cellcolor{gray!50} $v$ & \cellcolor{gray!50} $v$ & \cellcolor{gray!50} $v$\tabularnewline
\hline\cellcolor{ptcbackground}$5$ & \cellcolor{ptcbackground} $v$ & \cellcolor{ptcbackground} $v$ & \cellcolor{ptcbackground} $v$\tabularnewline
\hline\cellcolor{gray!50}$6$ & \cellcolor{gray!50} $f$ & \cellcolor{gray!50} $v$ & \cellcolor{gray!50} $f$\tabularnewline
\hline\cellcolor{ptcbackground}$7$ & \cellcolor{ptcbackground} $f$ & \cellcolor{ptcbackground} $v$ & \cellcolor{ptcbackground} $f$\tabularnewline
\hline\cellcolor{gray!50}$8$ & \cellcolor{gray!50} $f$ & \cellcolor{gray!50} $v$ & \cellcolor{gray!50} $f$\tabularnewline
\hline\cellcolor{ptcbackground}$9$ & \cellcolor{ptcbackground} $f$ & \cellcolor{ptcbackground} $v$ & \cellcolor{ptcbackground} $f$\tabularnewline
\end{tabular}

\label{tinter}
\end{table}


En la ultima columna de la tabla \ref{tinter} tenemos que los elementos
que están en $A\cap B$ son $1,2,3,4$ y $5$ ya que $\left(A\cap B\right)_{x}$
es verdadera . Entonces
\[
A\cap B=\left\{ 1,2,3,4,5\right\} .
\]
 \begin{definicionn}{Conjuntos Disjuntos}

Sean $X$ e $Y$ se dicen que son disjuntos $\left(\nexists x\right)\left(x\in X\vee x\in Y\right)$ 

\end{definicionn}

\begin{teorema}{Diferencia de conjuntos}

Dados dos conjuntos $X,Y,$ existe un (único) conjunto $Z$ tal que
$x\in Z$ si y sólo si $x\text{\ensuremath{\in}\ensuremath{X}}$y
$x\text{\ensuremath{\notin}\ensuremath{Y}.}$ Dicho conjunto se llama
diferencia de los conjuntos $X$ e $Y.$ 

\end{teorema}

La demostración queda de tarea.

\begin{definicionn}{Diferencia simétrica}

Dados dos conjuntos $X,Y,$ se llama diferencia simétrica de $X$
e $Y$ al conjunto $X\bigtriangleup Y:=\left(X-Y\right)\cup\left(Y-X\right)$

\end{definicionn}

\begin{axioma}{Conjunto Potencia} Dado cualquier conjunto $X,$ existe
un conjunto, que denotaremos $\mathcal{P}\left(X\right)$ tal que
$x\in\mathcal{P}\left(X\right)$ si y sólo si $x$ es un subconjunto
de $X.$ 

\end{axioma}


\paragraph{Algunas propiedades del Conjunto Potencia}

  \def\po#1{\mathcal{P}(#1)}
Recuerda que si $A$ es un conjunto, la colección que tiene por elementos a los subconjuntos de $A$ es un conjunto al que denotamos $\po{A}$ y llamamos el conjunto potencia de $A$. Por ejemplo, 
\begin{enumerate} 
\item $\po{\emptyset}=\{\emptyset\}$ 
\item $\po{\{a\}}=\{\emptyset, \{a\}\}$ 
\item $\po{\{a,b\}}=\{\emptyset, \{a\},\{b\},\{a,b\}\}$ 
\end{enumerate}
Nota que el conjunto potencia de un conjunto $A$ siempre tiene como elementos al conjunto vacío $\emptyset$ y a $A$ mismo. En particular,  $\po{A}$ siempre es diferente del vacío. 

\begin{definicionn}{Conjunto universal}\label{uni}

Sea $X$ un conjunto y sea $P(x)$ una condición definida $\psi=P\left(x\right):=P_{1}\left(x\right)\wedge P_{2}\left(x\right)\wedge P_{3}\left(x\right)\wedge\cdots\wedge P_{n}\left(x\right)$,
entonces por el axioma de separación existe un conjunto definido $U:=\{x\in X\,:\,\psi=P\left(x\right).$
A este conjunto se le llama conjunto universal o de referencia. 

\end{definicionn}

Teniendo el conjunto universal como referencia podemos aplicar el
axioma de separación de la siguiente manera:

Sea $U$ el conjunto Universal y sea la condición $\psi=P^{j}\left(x\right):=\{P_{1}\left(x\right)\wedge P_{2}\left(x\right)\wedge P_{3}\left(x\right)\wedge\cdots\wedge P_{j}\left(x\right)$
ó cualquier conjunción de propiedades $P_{i}(x)$\}, con $i\leq j<n$.
Por el axioma de separación existe un conjunto $A_{j}=\left\{ x\in U\,:\,\psi=P\left(x\right)\right\} ,$
pero en este caso $A_{j}$ no es único , si no que es una colección
de subconjuntos de $U.$

Por esta razón es que el conjunto Universal lo podemos definir como
el conjunto que contiene todos los elementos que poseen las propiedades
de referencia que nos interesa.

\nota Observe que el conjunto $X$ utilizado para definir el conjunto
universal es su vez un conjunto universal. Pero esto de ningún modo
establece una contradicción, porque el conjunto universal es solo
un conjunto de referencia que tomamos para construir una colección
de subconjuntos, es decir cualquier conjunto con dos o más elementos
nos puede servir de referencia.

\begin{definicionn}{Complemento de un conjunto}

Sean $X,Y$ conjuntos tales que, $Y\subset X,$ definimos utilizando
el axioma de separación, un conjunto con la propiedad: $x\in X\wedge x\notin Y$
dicho conjunto llameremos complemento del conjunto $Y$ con respecto
al conjunto $X$ y lo notaremos $Y_{X}^{c}$.

\end{definicionn}

\nota En el caso que definamos un conjunto universal $U$ tal que
$X\subset U$ denotamos el complemento de $X$ con respecto a $U$
como $X^{c}$.


\section{Propiedades de los conjuntos}

Ahora podemos hacer una relación entre la teoría axiomática de conjuntos
y la teoría intuitiva vista en secundaria.

Sea $U$ el conjunto referencia definido en la definición \ref{uni}
podemos establecer el producto $U\times U$ y luego las operaciones
de la forma %
\begin{tabular}{cccc}
$\oplus$ : & $U\times U$ & $\rightarrow$ & %
$U$%
\tabularnewline
 & $\left(A,B\right)$ & $\mapsto$ & $C=A\oplus B$\tabularnewline
\end{tabular}.


\paragraph{Unión de conjuntos}

Sean $A,B\subseteq U$ se define el conjunto $A\text{\ensuremath{\cup}\ensuremath{B=\ensuremath{\left\{ x\,:\, x\in A\vee x\in B\right\} }}}$.


\paragraph{Intersección de conjuntos}

Sean $A,B\subseteq U$ se define el conjunto $A\text{\ensuremath{\cap}\ensuremath{B=\ensuremath{\left\{ x\,:\, x\in A\wedge x\in B\right\} }}}$.


\paragraph{Algunas propiedades de la unión y la intersección son las siguientes.}

\begin{teorema}{Propiedades de la unión y la intersección} Si $A$, $B$ y $C$ son conjuntos, entonces: \begin{enumerate} 
\item $A\cap A=A$ y $A\cup A=A$ \hfill {\sc (Idempotencia)} 
\item $A\cap B=B\cap A$ y $A\cup B=B\cup A$ \hfill {\sc (Conmutatividad)} 
\item $(A\cap B)\cap C=A\cap(B\cap C)$ y $(A\cup B)\cup C=A\cup(B\cup C)$ \hfill {\sc (Asociatividad)}
\item $C\cup (A\cap B)=(C\cup A)\cap (C\cup B)$ y $C\cap (A\cup B)=(C\cap A)\cup (C\cap B)$ \hfill {\sc (Distributividad)}
\item  $A\cap \emptyset= \emptyset$ y $A\cup \emptyset =A$ 
\end{enumerate}
\end{teorema}


\paragraph{Diferencia de conjuntos}

Sean $A,B\subseteq U$ se define el conjunto $A\backslash B=A\text{-\ensuremath{B=\left\{ x\,:\, x\in A\wedge x\notin B\right\} }}$.

\newcommand{\dif}{\backslash}
\begin{teorema}{Propiedades de la diferencia} Si $A$ es un conjunto, entonces  
\begin{enumerate} 
\item $A-\emptyset = A$ 
\item $\emptyset- A= \emptyset$    
\item $A- A=\emptyset$ 
\end{enumerate} 
\end{teorema}


\section{Representación de conjuntos}

Cuando tenemos definido un conjunto de referencia podemos usar diagrama
para representar los conjuntos sus operaciones y sus relaciones. estos
diagramas se llaman de Venn y consisten el dibujar un rectángulo que
representa al conjunto de referencia y dentro de el dibujamos con
ovalos o circunferencias sus subconjuntos.


\subsection{Diagrama de Venn}

Sea $U$ el conjunto de referencia y $A\subseteq U$ el siguiente
diagrama los representa.

\begin{figure}[H]
\centering
\begin{tikzpicture}[scale=0.7]  \fill[gray!20]\rectangulo;         \fill[blue!20] \fourcircle ;   \draw[thick] \fourcircle node [above](c2) {$B$};     \draw[thick] \rectangulo node [below left of=c2, yshift=2.0cm, xshift=2.5cm]  {$U$};     \end{tikzpicture} 
\caption{Diagrama de Venn- Euler} 
\label{venn}
\end{figure}  

En los diagramas de Venn-Euler se representa el conjunto de referencia
con un rectángulo, y los subconjuntos se representan con círculos
o elipses, como se muestra en la figura \ref{venn}. 


\subsubsection{Subconjuntos}

\begin{figure}[H]
\centering
 \begin{subfigure}[b]{0.30\textwidth}
\centering
\begin{tikzpicture}[scale=0.5,every node/.style={scale=0.5}] 
\fill[gray!20]\rect;        
\fill[blue!20] \fourcircle ;    
     \fill[red!20] \fivecircle ;  
      \draw[thick] \fivecircle node [above](c3) {$A$}; 
 \draw[thick] \fourcircle node [above](c2) {$B$};  
  \draw[thick] \rect node [below left of=c2, yshift=2.3cm, xshift=2.7cm]  {$U$};
    \end{tikzpicture}   
 \caption{$A\nsubseteq B$}   
 \label{subfig1}    
\end{subfigure}\hspace{25pt}     
   \begin{subfigure}[b]{0.30\textwidth} 
\centering
   \begin{tikzpicture}[scale=0.5,every node/.style={scale=0.5}] 
   \fill[gray!30,thick] \rectangulo;  
        \fill[red!30,thick] \secondcircle;   
       \fill[blue!30,thick] \sevencircle;  
\draw[thick] \sevencircle node [above](c1) {$A$};
    \draw[thick] \secondcircle node [above right of=c1](c2) {$B$};  
      \draw[thick] \rectangulo node [below right of=c2, yshift=1.8cm, xshift=1.1cm]  {$U$};
                            \end{tikzpicture}                
   \caption{$A \subseteq B$}    
\label{subfig2}  
                 \end{subfigure} 
          \begin{subfigure}[b]{0.30\textwidth}
\centering
               \begin{tikzpicture}[scale=0.5,every node/.style={scale=0.5}]                    \fill[gray!30,thick] \rectangulo;
          \fill[red!30,thick] \secondcircle; 
         \fill[blue!30,thick] \firstcircle;
    \draw[thick]\firstcircle node[below](c1) {$A$}; 
   \draw[thick] \secondcircle node [above](c2) {$B$};
       \draw[thick] \rectangulo node [below left of=c2, yshift=2.3cm, xshift=2.7cm]  {$U$};
         \end{tikzpicture}   
       \subcaption{$A\nsubseteq B$}
    \label{subfig3}    
      \end{subfigure} 
        \caption{Contenencia De Conjuntos} 
         \label{subcfig}
        \end{figure} 

En la figura \ref{subcfig} observamos en la figuras \ref{subfig1}
y \ref{subfig3} que el circulo que representa al conjunto $A$ no
se encuentra totalmente en el interior del circulo que representa
al conjunto $B$, por tanto decimos que $A\nsubseteq B,$ mientras
que el figura \ref{subfig2} si se da que el circulo que representa
al conjunto $A$ si se encuentra en el interior del circulo que representa
al conjunto $B$, por tanto podemos decir que $A\subset B.$

\begin{ejem}{Subconjunto}\label{sub2}

Consideremos los siguientes conjuntos: $A=\left\{ 1,3,5,7\right\} ,\, B=\left\{ 1,3,5,7,9,11\right\} ,\, M=\left\{ a,b,c,d,e\right\} $
y $N=\left\{ b,c,d,m,n\right\} .$ Que podemos afirmar sobre la relación
entre $A$ y $B$, e $M$ y $N.$ 

\end{ejem}

\solucion 
\begin{lyxlist}{00.00.0000}
\item [{i)}] $A\subseteq B,$ porque todos los elementos de $A$ están
en $B.$
\item [{ii)}] $M\nsubseteq N,$ porque algunos elementos de $M$, $m\text{\,\ \ensuremath{\mbox{y }}}n$
no están en $N.$
\end{lyxlist}
Este ejemplo se puede representar usando los diagramas de Venn-Euler
como se muestra en la figura \ref{sub}

\begin{figure}[H] 
\centering 
\begin{subfigure}[b]{0.40\textwidth}  
 \centering 
   \begin{tikzpicture}[scale=0.8,every node/.style={scale=0.8}] 
            \fill[red!30,thick] \secondcircle; 
         \fill[blue!30,thick] \sevencircle; 
         \draw[thick] \sevencircle node [above](c1) {$A$};
         \draw[thick] \secondcircle node [above right of=c1](c2) {$B$};
         \node at (1,2.3) {11};           \node at (2.5,1.5) {9};
          \node at (0.6,0.8) {1};           \node at (1,0.6) {4};
          \node at (1.4,0.9) {3};           \node at (0.8,1.3) {5};  
                     \end{tikzpicture} 
     \caption{$A \subseteq B$} 
     \label{subfig4}   
   \end{subfigure} 
  \begin{subfigure}[b]{0.40\textwidth}
   \centering 
    \begin{tikzpicture}[scale=0.8,every node/.style={scale=0.8}] 
                            \fill[red!30,thick] \secondcircle; 
         \fill[blue!30,thick] \firstcircle;
          \draw[thick]\firstcircle node[below](c1) {$M$}; 
         \draw[thick] \secondcircle node [above](c2) {$N$};
          \node at (0.6,0.8) {c};           \node at (0.6,-0.5) {e}; 
         \node at (-0.5,0.1) {a};           \node at (1,2.5) {m};
          \node at (2,1) {n};           \node at (1.1,0.2){d}; 
         \node at (0.2,1.2) {b};  
           \end{tikzpicture}  
     \subcaption{$A\nsubseteq B$}  
    \label{subfig5}   
     \end{subfigure}       
  \caption{Ejemplo \ \myref{sub2}} 
\label{sub}
  \end{figure}   


\paragraph{Unión de conjuntos}

Sean $A,B\subseteq U$ el conjunto $A\text{\ensuremath{\cup}\ensuremath{B}}$
se representa 

\begin{figure}[H]
\centering
\begin{venndiagram2sets}[shadeA=red,shadeB=red,tikzoptions={scale=0.8,thick,opacity=0.8}]
\fillNotAorB 
\fillA\fillB
\end{venndiagram2sets}
\end{figure}

Sean $A,B\subseteq U$ el conjunto $A\text{\ensuremath{\cap}\ensuremath{B}}$
se representa 

\begin{figure}[H]
\centering
\begin{venndiagram2sets}[shadeA=red,shadeB=red,tikzoptions={scale=0.8,thick,opacity=0.8}]
\fillA
\fillB
\fillNotAorNotB 
\end{venndiagram2sets}
\end{figure}

Sean $A,B\subseteq U$ el conjunto $A\text{-\ensuremath{B}}$ se representa 

\begin{figure}[H]
\centering
\begin{venndiagram2sets}[shadeA=red,tikzoptions={scale=0.8,thick,opacity=0.8}]
\fillNotB
\fillA \fillB

\end{venndiagram2sets}
\end{figure}

\label{sec:problema} \problemas{ 

\begin{enumerate} 
\item Demuestra que si $A$ es un conjunto, la colección de todos los objetos que no son elementos de $A$ no es un conjunto. 
\item $A\cap B=A\dif (A\dif B)$. 
\item Si $A\cup B\subset C$, entonces $A\dif B=A\cap (C\dif B)$. 
\item $A\subset B$ si y sólo si para todo conjunto $E$, $E\dif B\subset E\dif A$. 
\item $A\dif (B\dif C)=(A\dif B)\cup (A\cap C)$.
\item Muestra por medio de ejemplos que las siguientes proposiciones son falsas:  \begin{enumerate}   
\item $A\dif B=B\dif A$.   
\item Si $A\subset B\cup C$ entonces $A\subset B$ o $A\subset C$. 
\end{enumerate} 
\newcommand{\tri}{\triangle} 
\item Definimos la \emph{diferencia simétrica} de $A$ y $B$ como el conjunto $(A\dif B)\cup (B\dif A)$ y lo denotamos por $A\tri B$. Demuestra que:  
\begin{enumerate}    
\item $A\tri \emptyset=A$.    
\item $A\tri B=\emptyset$ si y sólo si $A=B$.  
 \item Si $A\tri B= A\tri C$, entonces $B=C$.  
\end{enumerate} 
\item $A\times B=\emptyset$ si y sólo si $A=\emptyset$ o $B=\emptyset$. 
\item $A\times B=B\times A$ si y sólo si $A=B$. 
\item $A\times (B\cup C)=(A\times B)\cup (B\times C)$. 
\item $A\times (B\cap C)=(A\times B)\cap (B\times C)$. 
\item Usando las leyes de De Morgan y el ejercicio 4 enuncia y demuestra las igualdades duales de los siguientes enunciados:
 \begin{enumerate}   
\item $A\cup(B\cup C)=(A\cup B)\cup C$.    
\item $A\cap\big(\underset{i\in I}{\bigcup}B_i\big)=            \underset{i\in I}{\bigcup}(A\cap B_i)$  
\end{enumerate} 
\item $\po{A} \cap \po{B}=\po{A\cap B}$ 
\item $\po{A} \cup \po{B}\subset\po{A\cup B}$  
\item Muestra con un ejemplo que la igualdad en el ejercicio 14 no siempre se cumple.  \end{enumerate}
%%%%%%%%%%%%%%
\item  Determinar por extensión los siguientes conjuntos
\begin{enumerate}
\item $A=\left\{ x\in I\! N\,:\, x\leq3\vee5<x<7\right\} $
\item $B=\left\{ y\in\mathcal{\mathfrak{\mathcal{\mathbb{Z}}}}\,:\, y=x^{2}-1\wedge-1<x<3,\, x\in\mathbb{Z}\right\} $
\item $C=\left\{ y\in\mathrm{\mathbb{Z\,\mathrm{:}\,\mathrm{y}=\mathrm{3-5x\wedge-2<x<5\wedge3\leq x\leq8,}\,\mathrm{x}\in Z}}\right\} $
\end{enumerate}
\item  Determinar por compresión el conjunto $T=\left\{ -1,1,2\right\} $

\item  Si $A=\left\{ 2,3,5,7\right\} $, indica cuales de las siguientes
afirmaciones son verdaderas y cuales son falsas
\begin{enumerate}
\item $5\in A$
\item $3\subset A$
\item $\left\{ 7\right\} \subset A$
\item $\left\{ 3,5\right\} \in A.$
\end{enumerate}
\item  Si $A=\left\{ x\in I\! N\,:\, x\leq2\vee x=7\right\} ,$ hallar
todos los subconjuntos propios de $A.$

\item  Dados los siguientes conjuntos

$A=\left\{ y\in\mathbb{Z}\::\, y=7x+2\wedge x\in\mathbb{Z}\right\} ,\,$

$B=\left\{ y\in\mathbb{Z}\,:\: y=7x-26\wedge x\in\mathbb{Z}\right\} ,\,$ 

$C=\left\{ y\in\mathbb{Z}\text{\,:\,\ \ensuremath{y}\ensuremath{=4x+1\wedge x\in\mathbb{Z}}}\right\} $
y 

$D=\left\{ y\in\mathbb{Z}\,:\: y=2x+1\wedge x\in\mathbb{Z}\right\} ,$
analizar y justificar debidamente su conclusión en cada caso:
\begin{enumerate}
\item $A=B$
\item $C=D$
\item $C\subset D$
\item $A\cup B$=$A$
\item $C\cap D=\left\{ \right\} $
\end{enumerate}
\item  Sean $U=\left\{ 1,2,3,4,5,6,7,8,9,0\right\} ,\, A=\left\{ 2,4,6,8\right\} $
y $C=\left\{ 3,4,5\right\} .$ Al hallar un subconjunto $X$ de $U$
tal que $X\subset C,\, x\nsubseteq A,\, X\nsubseteq B,$ cuántas soluciones
existen.

\item Cuántos de los siguientes conjuntos son vacíos:
\begin{enumerate}
\item $A=\left\{ x\in U\,:\: x\notin U\right\} $
\item $B=\left\{ x\in\mathbb{Z}\,:\: y=x^{3}=3,\, y\in\mathbb{Z}\right\} $
\item $C=\left\{ y\in I\! R\,:\, y=\frac{1}{x},x\in\mathbb{Z}\right\} $
\item $D=\left\{ x\in\mathbb{Q}\,:\, x^{2}-1=2\right\} $
\end{enumerate}
\item Si $A,\, B$ y $C$ son conjuntos tal que $A\subseteq B\subseteq C$
¿Cuál es la relación entre $C-B$ y $C-A$?

\item  Si $A,\, B,\, C$ y $D$ son conjuntos. Demostrar las siguientes
propiedades
\begin{enumerate}
\item Si $A\subseteq U,\: A\cup U=U$
\item $B\subseteq A\cup B$
\item Si $A\subseteq C\wedge B\subseteq C\Rightarrow A\cup B\subseteq C$
\item $A\cap B\subseteq B$
\item Si $A\subseteq C$ y $B\subseteq D\Rightarrow\mbox{\ensuremath{A\cap B\subseteq C\cap D}}$
\item Si $A\subseteq U$, $A\cap U=A$
\item Si $A\subseteq B\Rightarrow A\cap C\subseteq B\subseteq C,\forall C$
\item $A-A=\emptyset$
\item $\emptyset-A=\emptyset$
\item $A\cap\left(B-C\right)=\left(A\cap B\right)-\left(A\cap C\right)$
\item $A-\emptyset=\emptyset$
\item $A-B\neq B-A$
\item $\left(A-B\right)\subseteq A$
\item Si $A\subseteq B\Rightarrow A-C\subseteq B-C,\,\forall C$
\item Si $A\subseteq B\Rightarrow A-B=\emptyset$
\item $B\cap\left(A-B\right)=\emptyset$
\item Si $A$ y $B$ son disjuntos, entonces $A\cap B=\emptyset$
\end{enumerate}
\item  Si $A,\, B$ y $C$ son conjuntos y $U$ el conjunto definido
en \myref{uni} demostrar las siguientes propiedades
\begin{enumerate}
\item $\left(A^{c}\right)^{c}=A$
\item $A\cap A^{c}=\emptyset$
\item Si $A\subseteq U,$ entonces $A\cup A_{U}^{c}=U$
\item Si $A\subseteq B,$ entonces $B^{c}\subseteq A^{c}$
\item $\left(A\cup B\right)^{c}=A^{c}\cap B^{c}$
\item $\left(A\cap B\right)^{c}=A^{c}\cup B^{c}$
\item $A\triangle A=\emptyset$
\item $A\triangle\emptyset=A$
\item $A\triangle B=B\triangle A$
\item $\left(A\triangle B\right)\triangle C=A\triangle\left(B\triangle C\right)$
\item $\left(A\triangle B\right)\cap C=\left(A\cap C\right)\triangle\left(B\cap C\right)$
\item $\left(A\triangle B\right)\cup\left(B\triangle C\right)=\left(A\cup B\cup C\right)-\left(A\cap B\cap C\right)$
\item $\emptyset_{r}^{c}=U$
\item $U^{c}=\emptyset_{r}$
\end{enumerate}
\item  Dados los conjuntos $A=\left\{ a,c,d\right\} ,\, B=\left\{ e,f,g\right\} $
y $C=\left\{ l,e,j,k\right\} ,$ hallar $A\cup\left(B\cap C\right).$

\item Si $U=\left\{ a,b,c,d,e\right\} ,$ $A\cup B=\{a,c\}$ y $A-B=\left\{ b\right\} .$
Hallar $A$ y $B.$

\item  Considere los conjuntos $A=\left\{ x\in I\! N\,:\, x\right\} \mbox{es divisor de \ensuremath{12}}$,
$B=\left\{ x\in I\! N\,:\, x\mbox{ es divisor de \ensuremath{18}}\right\} $
y $C=\left\{ x\in I\! N\,:\, x\mbox{ es divisor de \ensuremath{16}}\right\} .$
Hallar 
\begin{enumerate}
\item $\left(A-B\right)\cap\left(B-C\right)$
\item $\left(A-B\right)\cup\left(B-C\right).$
\end{enumerate}
\item Dados los conjuntos $A=\left\{ 1,2,5,7,8\right\} ,$ $B=\left\{ 2,3,4,7,9\right\} ,$
$C=\left\{ 1,3,5,6,8\right\} $ y $U=\left\{ x\in I\! N\,:\, x<10\right\} .$
Hallar
\begin{enumerate}
\item $\left[\left(A\cup B\right)-\left(A\cap C\right)\right]^{c}$
\item $\left[\left(A\cap B\right)-\left(A\cup B\right)\right]^{c}$
\item $\left[\left(A-B\right)\cup\left(A-C\right)\right]^{c}$
\item $\left[\left(A^{c}-B\right)\cap\left(A-C\right)\right]^{c}$
\item $\left[\left(C-B^{c}\right)-\left(A^{c}\cup c\right)\right]^{c}$
\item $\left(A^{c}-B^{c}\right)\triangle\left(B^{c}\cup C\right)^{c}$.
\end{enumerate}
\item Utilice el diagrama de Venn-Euler para demostrar las siguientes
propiedades
\begin{enumerate}
\item $A\cup B=B\cup A$
\item $A\cap B=B\cap A$
\item $\left(A\cap B\right)\cap C=A\cap\left(B\cap C\right)$
\item $\left(A\cup B\right)\cup C=A\cup\left(B\cup C\right)$
\item $A^{c}\cap B^{c}=\left(A\cup B\right)^{c}$
\item $A^{c}\cup B^{c}=\left(A\cup B\right)^{c}$
\item $A-B=A\cap B^{c}$
\item $A\subseteq A\cup B.$
\end{enumerate}
\item Utilice el diagrama de Venn-Euler para demostrar las siguientes
propiedades
\begin{enumerate}
\item $A\cup B=B\cup A$
\item $A\cap B=B\cap A$
\item $\left(A\cap B\right)\cap C=A\cap\left(B\cap C\right)$
\item $\left(A\cup B\right)\cup C=A\cup\left(B\cup C\right)$
\item $A^{c}\cap B^{c}=\left(A\cup B\right)^{c}$
\item $A^{c}\cup B^{c}=\left(A\cup B\right)^{c}$
\item $A-B=A\cap B^{c}$
\item $A\subseteq A\cup B.$
\end{enumerate}
\item Utilice el diagrama de Venn-Euler para demostrar las siguientes
propiedades
\begin{enumerate}
\item $A\cup B=B\cup A$
\item $A\cap B=B\cap A$
\item $\left(A\cap B\right)\cap C=A\cap\left(B\cap C\right)$
\item $\left(A\cup B\right)\cup C=A\cup\left(B\cup C\right)$
\item $A^{c}\cap B^{c}=\left(A\cup B\right)^{c}$
\item $A^{c}\cup B^{c}=\left(A\cup B\right)^{c}$
\item $A-B=A\cap B^{c}$
\item $A\subseteq A\cup B.$
\end{enumerate}
\item Utilice las tablas de verdad para demostrar las siguientes
propiedades
\begin{enumerate}
\item $A\cup B=B\cup A$
\item $A\cap B=B\cap A$
\item $\left(A\cap B\right)\cap C=A\cap\left(B\cap C\right)$
\item $\left(A\cup B\right)\cup C=A\cup\left(B\cup C\right)$
\item $A^{c}\cap B^{c}=\left(A\cup B\right)^{c}$
\item $A^{c}\cup B^{c}=\left(A\cup B\right)^{c}$
\item $A-B=A\cap B^{c}$
\item $A\subseteq A\cup B.$
\end{enumerate}
\item Utilice las leyes de inferencia para simplificar las siguientes
proposiciones:
\begin{enumerate}
\item $A\cup\emptyset$
\item $A\cap U$
\item $A\cap\emptyset$
\item $A\cap A^{c}$
\item $\left(A^{c}\right)^{c}$
\item $\left(A-B\right)^{c}$
\end{enumerate}
\item Pruebe: $A\cap B=A\Rightarrow A\subseteq B$

\item  Sean $A$ y $B$ dos conjuntos no comparables. Hacer el diagrama
lineal de $A,$ $B,$ $A\cap B,$ $A\cup B$ y el vacío. 


				 

}
%\layout
%\pagevalues

\end{document}

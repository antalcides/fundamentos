\documentclass[10pt]{article} %tipo documento y tipo letra

\definecolor{azulc}{cmyk}{0.72,0.58,0.42,0.20} % color titulo
\definecolor{naranja}{cmyk}{0.21,0.5,1,0.03} % color leccion

\def\eje{\centerline{\textbf{ EJERCICIOS.}}} % definiciones propias

\begin{document}

\begin{quote}

% inicio encabezado
{\color{gray}
\begin{tabular}{@{\extracolsep{\fill}}lcr}
\docLink{geometria21.tex}{\includegraphics{../../images/navegacion/anterior.gif}}
\begin{tabular}{l}
{\color {darkgray} {\small \'{A}ngulos Semejantes }} \\ \\ \\
\end{tabular} &
\docLink[_top]{../../index.html}{\includegraphics{../../images/navegacion/inicio.gif}}
\docLink{../../docs_curso/contenido.html}{\includegraphics{../../images/navegacion/contenido.gif}}
\docLink{../../docs_curso/descripcion.html}{\includegraphics{../../images/navegacion/descripcion.gif}}
\docLink{../../docs_curso/profesor.html}{\includegraphics{../../images/navegacion/profesor.gif}}
& \begin{tabular}{r}
{\color {darkgray} {\small Volumen de un Prisma}} \\ \\ \\
\end{tabular}
\docLink{geometria23.tex}{\includegraphics{../../images/navegacion/siguiente.gif}}
\\ \hline
\end{tabular}
}
%fin encabezado

%nombre capitulo
\begin{center}
\colorbox{azulc}{{\color{white} \large CAP�TULO 3}}  {\large
{\color{ azulc} MODULO DE GEOMETR�A}}
\end{center}

\newline

%nombre leccion

\colorbox{naranja}{{\color{white} \normalsize  Lecci\'on 3.18. }}
{\normalsize {\color{naranja}  Algunos Elementos de Geometr�a del
Espacio}}

\newline

El estudio de las figuras tridimensionales es llamado
geo\-me\-tr�a de los s�lidos. Cuando hablamos de los pol�gonos,
insistimos en diferenciar entre el pol�gono y la regi�n poligonal
(Un pol�gono es la frontera de una regi�n poligonal y la regi�n es
la uni�n de la frontera con su interior).

\bigskip

Una distinci�n similar se hace con las figuras tridimensionales,
una superficie es la frontera de una regi�n tridimensional. Un
s�lido es la uni�n de la frontera y la regi�n del espacio
encerrada por la superficie. Por ejemplo una caja de cart�n es una
superficie y un bloque de ladrillo es un s�lido.

\bigskip

\textbf{Definici�n 3.18.1. } Ll�mase poliedro un cuerpo o s�lido
geom�trico limitado por planos. Las intersecciones de estos planos
forman pol�gonos llamados caras del poliedro; los lados de las
caras se llaman aristas y las intersecciones de las aristas se
llaman v�rtices.

\bigskip

\bigskip

\[
\begin{array}{cc}
\includegraphics{imagenes/poliedro.gif} & \includegraphics{imagenes/poliedro1.gif}
\end{array}
\]

\bigskip


Diagonal de un poliedro es toda recta que une dos v�rtices no
situados en una misma cara.

\bigskip

\textbf{Definici�n 3.18.2. } Un poliedro regular es aquel cuyas
caras son pol�gonos regulares iguales.
\bigskip

\begin{center}
\includegraphics{imagenes/poliedro2.gif} \\
\includegraphics{imagenes/poliedro3.gif}
\end{center}

\bigskip

De acuerdo al n�mero de caras los poliedros se clasifican en
tetraedros, pentaedros, hexaedros, etc., seg�n tengan cuatro,
cinco, seis caras.

\bigskip

\textbf{Definici�n 3.18.3. } Ll�mase prisma al poliedro que tiene
dos caras iguales y paralelas y las dem�s caras son
paralelogramos.

\bigskip

\[
\includegraphics{imagenes/poliedro_reg5.gif}
\]

\bigskip

Las caras iguales y paralelas se denominan bases y las dem�s caras
laterales. Si las bases son tri�ngulos, cuadril�teros, pent�gonos
\ldots se habla de prismas rectangulares, cuadrangulares o
pentagonales y si las caras laterales son perpendiculares se habla
de prisma recto.

\bigskip

\textbf{Definici�n 3.18.4. } Un paralelep�pedo es un prisma cuyas
bases son paralelogramos, es decir sus seis caras son
paralelogramos. (Si sus bases son rect�ngulos se habla de
paralelep�pedo rect�ngulo, si sus caras laterales son
perpendiculares a las bases se habla de paralelep�pedo recto)

\bigskip

\[
\includegraphics{imagenes/paralelepipido.gif}
\]

\bigskip

" El cubo es el paralelep�pedo rect�ngulo cuyas seis caras son
cuadrados"

%Pie de p�gina
\newline

{\color{gray}
\begin{tabular}{@{\extracolsep{\fill}}lcr}
\hline \\
\docLink{geometria21.tex}{\includegraphics{../../images/navegacion/anterior.gif}}
\begin{tabular}{l}
{\color {darkgray} {\small \'{A}ngulos Semejantes }} \\ \\ \\
\end{tabular} &
\docLink[_top]{../../index.html}{\includegraphics{../../images/navegacion/inicio.gif}}
\docLink{../../docs_curso/contenido.html}{\includegraphics{../../images/navegacion/contenido.gif}}
\docLink{../../docs_curso/descripcion.html}{\includegraphics{../../images/navegacion/descripcion.gif}}
\docLink{../../docs_curso/profesor.html}{\includegraphics{../../images/navegacion/profesor.gif}}
& \begin{tabular}{r}
{\color {darkgray} {\small Volumen de un Prisma}} \\ \\ \\
\end{tabular}
\docLink{geometria23.tex}{\includegraphics{../../images/navegacion/siguiente.gif}}
\end{tabular}
}

\end{quote}

\newline

\begin{flushright}
\includegraphics{../../images/interfaz/copyright.gif}
\end{flushright}
\end{document}

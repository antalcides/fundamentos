\documentclass[10pt]{article} %tipo documento y tipo letra

\definecolor{azulc}{cmyk}{0.72,0.58,0.42,0.20} % color titulo
\definecolor{naranja}{cmyk}{0.50,0.42,0.42,0.06} % color leccion

\def\Reales{\mathbb{R}}
\def\Naturales{\mathbb{N}}
\def\Enteros{\mathbb{Z}}
\def\Racionales{\mathbb{Q}}
\def\Irr{\mathbb{I}}
\def\contradiccion{($\rightarrow \leftarrow$)}


\begin{document}

\begin{quote}

% inicio encabezado
{\color{gray}
\begin{tabular}{@{\extracolsep{\fill}}lcr}
\docLink{taller1.tex}{\includegraphics{../../../images/navegacion/anterior.gif}}
\begin{tabular}{l}
{\color{darkgray}\small Taller 1 } \\ \\ \\
\end{tabular}&
\docLink[_top]{../../../index.html}{\includegraphics{../../../images/navegacion/inicio.gif}}
\docLink{../../../docs_curso/contenido.html}{\includegraphics{../../../images/navegacion/contenido.gif}}
\docLink{../../../docs_curso/descripcion.html}{\includegraphics{../../../images/navegacion/descripcion.gif}}
\docLink{../../../docs_curso/profesor.html}{\includegraphics{../../../images/navegacion/profesor.gif}}
&
\\ \hline
\end{tabular}
}
%fin encabezado

%nombre capitulo
\begin{center}
\colorbox{azulc}{{\color{white} \large FUNCIONES}}  {\large
{\color{ azulc} }}
\end{center}

\newline

%nombre leccion

\colorbox{naranja}{{\color{white} \normalsize  TALLER 2 (POLINOMIOS)}}
{\normalsize {\color{naranja} }}

\newline

\begin{enumerate}

\item Halle todos los valores enteros $b$ y $c$ tales que cada uno
de los siguientes polinomios pueda escribirse como producto de
polinomios de grado uno y coeficientes enteros.
\begin{enumerate}
\item $x^{2}+3x+c$ \item $2x^{2}+bx-3$
\end{enumerate}

\item En cada caso determine si existen enteros $x$ y $y$ que
satisfagan la ecuaci�n
\begin{enumerate}
\item $x^{2}-y^{2}=17$ \item $x^{2}-y^{2}=12$
\end{enumerate}

\item En cada caso divida $f\left( x\right) $ por $p\left(
x\right) $
\begin{enumerate}
\item $f\left( x\right) =27x^{4}-36x^{2}+x+9$, \quad $p\left(
x\right) =3x^{2}+x+2$ \item $f\left( x\right)=7x+6$, \quad
$p\left( x\right) =2x^{2}-3$ \item $f\left( x\right) =x^{2}+x+1$,
\quad $p\left( x\right)=x^{2}+1$
\end{enumerate}

\item Use divisi�n sint�tica para determinar el cociente y el
residuo de dividir $f\left( x\right) $ por $p\left( x\right) $ en
cada uno de los siguientes casos.
\begin{enumerate}
\item $f\left( x\right) =x^{5}-3x^{2}+4x+5$, \quad $p\left(
x\right) =x-\frac{1}{4}$ \item $f\left( x\right)
=-4x^{6}-21x^{5}+26x^{3}-27x$, \quad $p\left( x\right) =x+5$ \item
$f\left( x\right) =12x^{3}+5x^{2}-11x-6$, \quad $p\left( x\right)
=3x+2$ \quad (Sugerencia: $3x+2=3\left(x+\frac{2}{3}\right) $).
\end{enumerate}

\item Usando el teorema del residuo. Determine el residuo de
dividir $f\left( x\right) $ por $p\left( x\right) $ en cada uno de
los siguientes casos.
\begin{enumerate}
\item $f\left( x\right) =2x^{7}-3x^{5}+4x-2$, \quad $p\left(
x\right) =x+2$ \item $f\left( x\right)
=\dfrac{x^{4}}{27}-\dfrac{x^{3}}{9}+x+1$, \quad $p\left( x\right)
=x-3$ \item $f\left( x\right)
=6x^{100}+10x^{85}-x^{38}+4x^{17}-12$, \quad $p\left( x\right)
=x-1$
\end{enumerate}

\item En cada caso demuestre que $p\left( x\right)$ es factor de
$f\left(x\right) $
\begin{enumerate}
\item $f\left( x\right)=6x^{100}+10x^{85}-x^{38}+4x^{17}-12$,
\quad $p\left(x\right) =x-1$ \item $f\left( x\right)
=4x^{3}+4x^{2}-x-1$, \quad $p\left( x\right) =x+1$ \item $f\left(
x\right) =5x^{12}-20480$, \quad $p\left(x\right) =x-2$ \item
$f\left( x\right) =x^{49}+3^{49}$, \quad $p\left( x\right) =x+3$
\item $f\left( x\right) =6x^{3}-7x^{2}+1$, \quad $p\left( x\right)
=x-\frac{1}{2}$
\end{enumerate}

\item Para qu� valores de $k$, $f\left( x\right) $ es divisible
por el polinomio lineal?
\begin{enumerate}
\item $f\left( x\right) =x^{3}-k^{2}x^{2}-8kx-16$, $x-4$ \item
$f\left( x\right) =kx^{3}-17x^{2}-4kx+5$, $x-5$
\end{enumerate}

\item Halle todos los valores de $k$ para los cuales el residuo de
dividir \linebreak$f\left( x\right)=3x^{2}+4kx^{2}+6,$ por $x+2$
sea $-2.$

\item Si se divide $f\left( x\right) =x^{2}-5x+5$, por $x-c$ y se
obtiene como residuo $r=-1$ a qu� es igual $c$?

\item Halle el polinomio $f\left( x\right)$ de coeficientes reales
y coeficiente principal $1$, que tiene el grado y las ra�ces que
se indican
\begin{enumerate}
\item grado $4$, ra�ces $-\frac{1}{2},1,3,-3$ \item grado $4$,
ra�ces $-\frac{1}{4},-3i,0,-3$ \item grado $5$, ra�ces $-3i,1+i,2$
\end{enumerate}

\item Determine el polinomio $f\left( x\right) $ de grado 3 y
coeficientes reales que tenga los ceros indicados y satisfaga la
condici�n dada
\begin{enumerate}
\item $-1,2,3,$ $f\left( -2\right) =80$ \item $-3,-2,0,$ $f\left(
-4\right) =16$ \item $-2i,2i,3,$ $f\left(1\right) =20$
\end{enumerate}

\item Halle el polinomio $f\left( x\right) $ de grado 7 para el
cual $-2$ y $2$ son ra�ces de multiplicidad 2, 0 es ra�z de
multiplicidad 3 y $f\left( -1\right) =27$

\item Para cada uno de los siguientes polinomios halle todas sus
ra�ces y escr�balo como producto de factores lineales
\begin{enumerate}
\item $x^{3}-x^{2}+x-1$ \item $2x^{4}+x^{3}-3x^{2}-x+1$ \item
$8x^{6}+7x^{3}-1$ \item $x^{6}-64$
\end{enumerate}

\item Halle todas las soluciones de cada una de las siguientes
ecuaciones
\begin{enumerate}
\item $x^{\frac{2}{5}}-x^{\frac{1}{5}}-2=0$ \item
$\sqrt{x}-6\sqrt[4]{x}+9=0$ \item $\sqrt{2x+3}-\sqrt{x-2}-2=0$
\item $\sqrt{4x-3x}+\sqrt{3x-9}=\sqrt{3x-14}$
\end{enumerate}

\end{enumerate}
%Pie de p�gina
\newline

{\color{gray}
\begin{tabular}{@{\extracolsep{\fill}}lcr}
\hline \\
\docLink{taller1.tex}{\includegraphics{../../../images/navegacion/anterior.gif}}
\begin{tabular}{l}
{\color{darkgray}\small Taller 1 } \\ \\ \\
\end{tabular}&
\docLink[_top]{../../../index.html}{\includegraphics{../../../images/navegacion/inicio.gif}}
\docLink{../../../docs_curso/contenido.html}{\includegraphics{../../../images/navegacion/contenido.gif}}
\docLink{../../../docs_curso/descripcion.html}{\includegraphics{../../../images/navegacion/descripcion.gif}}
\docLink{../../../docs_curso/profesor.html}{\includegraphics{../../../images/navegacion/profesor.gif}}
&
}

\end{quote}

\newline

\begin{flushright}
\includegraphics{../../../images/interfaz/copyright.gif}
\end{flushright}
\end{document}

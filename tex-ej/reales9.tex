\documentclass[10pt]{article} %tipo documento y tipo letra

\definecolor{azulc}{cmyk}{0.72,0.58,0.42,0.20} % color titulo
\definecolor{naranja}{cmyk}{0.21,0.5,1,0.03} % color leccion

\def\eje{\centerline{\textbf{ EJERCICIOS.}}} % definiciones propias

\begin{document}

\begin{quote}

% inicio encabezado
{\color{gray}
\begin{tabular}{@{\extracolsep{\fill}}lcr}
\docLink{reales8.tex}{\includegraphics{../../images/navegacion/anterior.gif}}
\begin{tabular}{l}
{\color{darkgray}\small Expresiones Decimales} \\ \\ \\
\end{tabular} &
\docLink[_top]{../../index.html}{\includegraphics{../../images/navegacion/inicio.gif}}
\docLink{../../docs_curso/contenido.html}{\includegraphics{../../images/navegacion/contenido.gif}}
\docLink{../../docs_curso/descripcion.html}{\includegraphics{../../images/navegacion/descripcion.gif}}
\docLink{../../docs_curso/profesor.html}{\includegraphics{../../images/navegacion/profesor.gif}}
& \begin{tabular}{r}
{\color{darkgray}\small Axioma de Completez} \\ \\ \\
\end{tabular}
\docLink{reales10.tex}{\includegraphics{../../images/navegacion/siguiente.gif}}
\\ \hline
\end{tabular}
}
%fin encabezado

%nombre capitulo
\begin{center}
\colorbox{azulc}{{\color{white} \large CAP�TULO 1}}  {\large
{\color{ azulc} LOS NUMEROS REALES}}
\end{center}

\newline

%nombre leccion

\colorbox{naranja}{{\color{white} \normalsize  Lecci\'on 1.9. }}
{\normalsize {\color{naranja} Densidad de los N�meros Racionales y de los N�meros Irracionales en \ $\mathbb{R}$}}

\newline

Dados dos n�meros reales diferentes $x$ y $y$, su promedio
$\dfrac{x+y}{2}$ esta comprendido entre $x$ y $y$. Por lo tanto,
entre dos n�meros reales sin importar lo cercano que se
encuentren, hay una infinidad de n�meros reales. Esto implica que
dado un n�mero real cualquiera $x$ no tienen sentido expresiones
tales como " el n�mero real siguiente a $x$" o " el n�mero real
anterior a $x$".

\bigskip

Usando nuestra caracterizaci�n de los n�meros reales como
expresiones decimales, podemos refinar el resultado anterior y
establecer los siguientes resultados:

\bigskip

\textbf{Resultado 1.} Entre dos n�meros reales diferentes hay un
n�mero racional, y por lo tanto hay infinitos n�meros racionales
entre ellos.

\bigskip

\textbf{Resultado 2.} Entre dos n�meros reales diferentes hay un
n�mero irracional, y por lo tanto hay infinitos n�meros
irracionales entre ellos.

\bigskip

Los resultados 1 y 2 se describen en lenguaje matem�tico diciendo,
respectivamente, que el conjunto de los n�meros racionales es
\textit{denso} en el conjunto de los n�meros reales y que el
conjunto de los n�meros irracionales es denso en el conjunto de
los n�meros reales.

\bigskip

\textbf{Ejemplo 1.15. } Construyamos dos n�meros racionales y dos
n�meros irracionales entre $x=1,24$ y $y=1,2401$.

\bigskip

Usando expresiones decimales peri�dicas tenemos que

\bigskip

\[a=1,24005 \qquad\ \text{y} \qquad b=1,240\overline{03}\]

\bigskip

son dos n�meros racionales entre $x$ y $y$.

\bigskip

Usando expresiones decimales no peri�dicas tenemos que

\bigskip

\[t=1,24002000200002\cdots \quad \text{y} \qquad s=1,2400201001000100001\cdots\]

\bigskip

son dos n�meros irracionales entre $x$ y $y$.

%Pie de p�gina
\newline

{\color{gray}
\begin{tabular}{@{\extracolsep{\fill}}lcr}
\hline \\
\docLink{reales8.tex}{\includegraphics{../../images/navegacion/anterior.gif}}
\begin{tabular}{l}
{\color{darkgray}\small Expresiones Decimales} \\ \\ \\
\end{tabular} &
\docLink[_top]{../../index.html}{\includegraphics{../../images/navegacion/inicio.gif}}
\docLink{../../docs_curso/contenido.html}{\includegraphics{../../images/navegacion/contenido.gif}}
\docLink{../../docs_curso/descripcion.html}{\includegraphics{../../images/navegacion/descripcion.gif}}
\docLink{../../docs_curso/profesor.html}{\includegraphics{../../images/navegacion/profesor.gif}}
& \begin{tabular}{r}
{\color{darkgray}\small Axioma de Completez} \\ \\ \\
\end{tabular}
\docLink{reales10.tex}{\includegraphics{../../images/navegacion/siguiente.gif}}
\end{tabular}
}

\end{quote}

\newline

\begin{flushright}
\includegraphics{../../images/interfaz/copyright.gif}
\end{flushright}
\end{document}

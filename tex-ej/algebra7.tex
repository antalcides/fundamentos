\documentclass[10pt]{article} %tipo documento y tipo letra

\definecolor{azulc}{cmyk}{0.72,0.58,0.42,0.20} % color titulo
\definecolor{naranja}{cmyk}{0.21,0.5,1,0.03} % color leccion

\def\eje{\centerline{\textbf{ EJERCICIOS.}}} % definiciones propias

\begin{document}

\begin{quote}

% inicio encabezado
{\color{gray}
\begin{tabular}{@{\extracolsep{\fill}}lcr}
\docLink{algebra6.tex}{\includegraphics{../../images/navegacion/anterior.gif}}
\begin{tabular}{l}
{\color{darkgray}\small Polinomios} \\ \\ \\
\end{tabular} &
\docLink[_top]{../../index.html}{\includegraphics{../../images/navegacion/inicio.gif}}
\docLink{../../docs_curso/contenido.html}{\includegraphics{../../images/navegacion/contenido.gif}}
\docLink{../../docs_curso/descripcion.html}{\includegraphics{../../images/navegacion/descripcion.gif}}
\docLink{../../docs_curso/profesor.html}{\includegraphics{../../images/navegacion/profesor.gif}}
& \begin{tabular}{r}
{\color{darkgray}\small Multiplicaci�n} \\ \\ \\
\end{tabular}
\docLink{algebra8.tex}{\includegraphics{../../images/navegacion/siguiente.gif}}
\\ \hline
\end{tabular}
}
%fin encabezado

%nombre capitulo
\begin{center}
\colorbox{azulc}{{\color{white} \large CAP�TULO 2}}  {\large
{\color{ azulc} FUNDAMENTOS DE ALGEBRA}}
\end{center}

\newline

%nombre leccion

\colorbox{naranja}{{\color{white} \normalsize  Lecci\'on 2.7. }}
{\normalsize {\color{naranja} Suma y Resta de Polinomios}}

\newline

El procedimiento para sumar y restar polinomios esta basado
directamente en las propiedades asociativas y conmutativas de la
adici�n y multiplicaci�n de n�meros reales, y en la propiedad
distributiva de la multiplicaci�n con respecto a la adici�n. Es
decir, esta basado en las propiedades P.2, P.3 y P.4 de los
n�meros reales, estudiadas en el cap�tulo 1. B�sicamente lo que se
hace es agrupar y reducir los t�rminos semejantes. Se llaman
\textit{t�rminos semejantes} los que difieren �nicamente en su
coeficiente num�rico, por ejemplo $5x^{2}y$ y $-7x^{2}y$ son
t�rminos semejantes.

\bigskip

\textbf{Ejemplo 2.23. } Efectuemos la siguiente suma

\bigskip

\[(4x^{2}-2xy+2y^{2}+7x)+(6xy-5y^{2}+3x^{2})\]

\bigskip

Primero utilizamos las propiedades conmutativa y asociativa para
agrupar los t�rminos semejantes, obteniendo

\bigskip

\[(4x^{2}+3x^{2})+(-2xy+6xy)+(2y^{2}-5y^{2})+7x,\]

\bigskip

luego, aplicamos la propiedad distributiva para reducir los
t�rminos semejantes, obteniendo como resultado final

\bigskip

\[7x^{2}+4xy-3y^{2}+7x.\]

\bigskip

En la pr�ctica se escriben en fila los polinomios a sumar, de tal
forma que las columnas contengan solo t�rminos semejantes. El
ejemplo anterior nos queda de la siguiente forma

\bigskip

\begin{center}
\renewcommand{\tabcolsep}{0cm}
\begin{tabular}{cccc}
$4x^2$ & $-2xy$ & $+2y^2$ & $+7x$\\
$3x^2$ & $+6xy$ & $-5y^2$ & \\ \hline $7x^2$ & $+4xy$ & $-3y^2$ &
$+7x$
\end{tabular}
\end{center}

\bigskip

donde la tercera l�nea presenta el resultado despu�s de reducir
los t�rminos semejantes.

\bigskip

Si tenemos en cuenta la definici�n de resta, el problema de restar
dos polinomios se transforma en el problema de sumar dos
polinomios.

\bigskip

\textbf{Ejemplo 2.24. } Efectuemos la siguiente resta de
polinomios

\bigskip

\[(5x^{2}+3xy-y^{3})-(2x^{2}-4xy+1)\]

\bigskip

La resta anterior se convierte en la suma

\bigskip

\[(5x^{2}+3xy-y^{3})+(-2x^{2}+4xy-1)\]

\bigskip

que es igual a

\bigskip

\[3x^{2}+7xy-y^{3}-1.\]

%Pie de p�gina
\newline

{\color{gray}
\begin{tabular}{@{\extracolsep{\fill}}lcr}
\hline \\
\docLink{algebra6.tex}{\includegraphics{../../images/navegacion/anterior.gif}}
\begin{tabular}{l}
{\color{darkgray}\small Polinomios} \\ \\ \\
\end{tabular} &
\docLink[_top]{../../index.html}{\includegraphics{../../images/navegacion/inicio.gif}}
\docLink{../../docs_curso/contenido.html}{\includegraphics{../../images/navegacion/contenido.gif}}
\docLink{../../docs_curso/descripcion.html}{\includegraphics{../../images/navegacion/descripcion.gif}}
\docLink{../../docs_curso/profesor.html}{\includegraphics{../../images/navegacion/profesor.gif}}
& \begin{tabular}{r}
{\color{darkgray}\small Multiplicaci�n} \\ \\ \\
\end{tabular}
\docLink{algebra8.tex}{\includegraphics{../../images/navegacion/siguiente.gif}}
\end{tabular}
}

\end{quote}

\newline

\begin{flushright}
\includegraphics{../../images/interfaz/copyright.gif}
\end{flushright}
\end{document}

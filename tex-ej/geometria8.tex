\documentclass[10pt]{article} %tipo documento y tipo letra

\definecolor{azulc}{cmyk}{0.72,0.58,0.42,0.20} % color titulo
\definecolor{naranja}{cmyk}{0.21,0.5,1,0.03} % color leccion

\def\eje{\centerline{\textbf{ EJERCICIOS.}}} % definiciones propias

\begin{document}

\begin{quote}

% inicio encabezado
{\color{gray}
\begin{tabular}{@{\extracolsep{\fill}}lcr}
\docLink{geometria7.tex}{\includegraphics{../../images/navegacion/anterior.gif}}
\begin{tabular}{l}
{\color {darkgray} {\small \'{A}ngulos y Medidas}} \\ \\ \\
\end{tabular} &
\docLink[_top]{../../index.html}{\includegraphics{../../images/navegacion/inicio.gif}}
\docLink{../../docs_curso/contenido.html}{\includegraphics{../../images/navegacion/contenido.gif}}
\docLink{../../docs_curso/descripcion.html}{\includegraphics{../../images/navegacion/descripcion.gif}}
\docLink{../../docs_curso/profesor.html}{\includegraphics{../../images/navegacion/profesor.gif}}
& \begin{tabular}{r}
{\color {darkgray} {\small Propiedades de los \'{Angulos}}} \\ \\ \\
\end{tabular}
\docLink{geometria9.tex}{\includegraphics{../../images/navegacion/siguiente.gif}}
\\ \hline
\end{tabular}
}
%fin encabezado

%nombre capitulo
\begin{center}
\colorbox{azulc}{{\color{white} \large CAP�TULO 3}}  {\large
{\color{ azulc} MODULO DE GEOMETR�A}}
\end{center}

\newline

%nombre leccion

\colorbox{naranja}{{\color{white} \normalsize  Lecci\'on 3.8. }}
{\normalsize {\color{naranja} Medida de \'{A}ngulos}}

\newline

La medida de un �ngulo indica la cantidad de abertura del interior
del �ngulo. Para medirlo inicialmente la unidad de medida que se
toma es el {\bfseries grado}, para  indicar que se hace referencia
a la medida se acostumbra a escribir $m$ $\angle ABC$. Un �ngulo
de un grado (1�) est� determinado por la divisi�n de un c�rculo en
360 partes iguales, y se toma el �ngulo determinado por el centro
del c�rculo (v�rtice) y dos puntos de subdivisi�n consecutivos
sobre la circunferencia. La circunferencia completa tiene 360�.

\bigskip

\[
\begin{array}{ccc}
\includegraphics{/imagenes/medida_angulos1.gif}
\end{array}
\]

\bigskip

La medida del exterior de un �ngulo es 360� menos la medida del
interior.

\bigskip

\[
\begin{array}{ccc}
\includegraphics{/imagenes/medida_angulos2.gif}
\end{array}
\]

\bigskip

Si $\overrightarrow{VC}$ (salvo el punto  $V$) est� en el interior
de  $\angle AVB$  entonces
\[m \angle AVC + m \angle CVB = m  \angle AVB,\]

\bigskip

\[
\begin{array}{ccc}
\includegraphics{/imagenes/medida_angulos3.gif}
\end{array}
\]

\bigskip

\textbf{Ejemplo 3.1. } \quad\begin{enumerate} \item  En la figura
determinar: $m\angle CPD$

\bigskip

\[
\begin{array}{ccc}
\includegraphics{/imagenes/medida_angulos4.gif}
\end{array}
\]

\bigskip

\begin{itemize}
\item $\overrightarrow{PB}$ (excepto por el punto $P$) es interior
al �ngulo $\angle APC$, entonces $m\angle APC = m \angle APB + m
\angle BPC = 125^{o}$.

\bigskip

\item Pensando a $P$ como centro de un c�rculo, podemos afirmar
que: $m \angle CPD = 360^{o} - (90^{o} + 45^{o} + 80^{o}) =
145^{o}$.
\end{itemize}

\bigskip

\textbf{nota} $\overrightarrow{VR}$ bisecta al �ngulo $\angle PVQ$
s� y s�lo s� $\overrightarrow{VR}$ (excepto por el punto V) est�
en el interior del �ngulo $\angle PVQ$ y $m$$\angle PVR$ =
$m$$\angle RVQ$.

\bigskip

\item Si en la figura $\overrightarrow{VR}$ bisecta el �ngulo
$\angle PVQ$ y si $m\angle PVR = 5x - 11$ y $m \angle RVQ = 2x +
25$, encontrar $m\angle PVQ$.

\bigskip

\[
\begin{array}{ccc}
\includegraphics{/imagenes/medida_angulos5.gif}
\end{array}
\]

\bigskip

\begin{align*}
m\angle PVR &= m\angle RVQ\\
5x - 11 &= 2x + 25\\
3x &= 36\\
x &= 12
\end{align*}

\bigskip

Sustituyendo $x$, se tiene que:
\begin{align*}
m\angle PVR &= 5(12) - 11 = 49\\
m\angle RVQ &= 2(12) + 25 = 49
\end{align*}
De donde $m\angle PVQ$ = 98

\end{enumerate}

\bigskip

\textbf{nota} Existen l�mites para la informaci�n que usted puede
suponer cuando se presenta un dibujo.

\bigskip

Dada la figura:

\bigskip

\[
\begin{array}{ccc}
\includegraphics{/imagenes/nota_import.gif}
\end{array}
\]

\bigskip

Usted puede asumir:

\bigskip

\begin{itemize}
\item \textbf{Colinealidad} de los puntos que aparecen marcados
sobre la recta.

\bigskip

$E$, $D$, $B$ y $C$ est�n todos sobre la recta $m$ y $D$ est�
entre $B$ y $E$.

\bigskip

\item \textbf{Intersecci�n} de las rectas en un punto dado.

\bigskip

$m$ y $l$ se interceptan en el punto $B$.

\bigskip

\item \textbf{Puntos} en el interior de un �ngulo, sobre un lado,
o en el exterior del �ngulo. Por ejemplo $F$ est� en el interior
del $\angle ABC$, $G$ est� en el exterior y $A$ est� sobre el
lado.
\end{itemize}

\bigskip

No se puede asumir:

\bigskip

\begin{itemize}
\item \textbf{Colinealidad} de tres o mas puntos que no est�n
dibujados sobre la recta.

\bigskip

Por ejemplo no se puede asumir que $F$ est� entre $A$ y $C$.

\bigskip

\item \textbf{Paralelismo} entre rectas.

\bigskip

No se puede asumir por ejemplo que $m\parallel n$, si no se dice
expl�citamente en el enunciado del problema.

\bigskip

\item \textbf{Medida} de �ngulos y longitud de segmentos.

\bigskip

Por ejemplo no se puede asumir que  $\overrightarrow{BF}$ bisecte
a $\angle ABC$ o que $DE=DB$.

\bigskip

\end{itemize}

\bigskip

Para asumir informaci�n como esta acerca de una figura, debe ser
especificada en el enunciado del problema.


%Pie de p�gina
\newline

{\color{gray}
\begin{tabular}{@{\extracolsep{\fill}}lcr}
\hline \\
\docLink{geometria7.tex}{\includegraphics{../../images/navegacion/anterior.gif}}
\begin{tabular}{l}
{\color {darkgray} {\small \'{A}ngulos y Medidas}} \\ \\ \\
\end{tabular} &
\docLink[_top]{../../index.html}{\includegraphics{../../images/navegacion/inicio.gif}}
\docLink{../../docs_curso/contenido.html}{\includegraphics{../../images/navegacion/contenido.gif}}
\docLink{../../docs_curso/descripcion.html}{\includegraphics{../../images/navegacion/descripcion.gif}}
\docLink{../../docs_curso/profesor.html}{\includegraphics{../../images/navegacion/profesor.gif}}
& \begin{tabular}{r}
{\color {darkgray} {\small Propiedades de los \'{Angulos}}} \\ \\ \\
\end{tabular}
\docLink{geometria9.tex}{\includegraphics{../../images/navegacion/siguiente.gif}}
\end{tabular}
}

\end{quote}

\newline

\begin{flushright}
\includegraphics{../../images/interfaz/copyright.gif}
\end{flushright}
\end{document}

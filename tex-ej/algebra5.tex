\documentclass[10pt]{article} %tipo documento y tipo letra

\definecolor{azulc}{cmyk}{0.72,0.58,0.42,0.20} % color titulo
\definecolor{naranja}{cmyk}{0.21,0.5,1,0.03} % color leccion

\def\eje{\centerline{\textbf{ EJERCICIOS.}}} % definiciones propias

\begin{document}

\begin{quote}

% inicio encabezado
{\color{gray}
\begin{tabular}{@{\extracolsep{\fill}}lcr}
\docLink{algebra4.tex}{\includegraphics{../../images/navegacion/anterior.gif}}
\begin{tabular}{l}
{\color{darkgray}\small Exponentes Racionales} \\ \\ \\
\end{tabular} &
\docLink[_top]{../../index.html}{\includegraphics{../../images/navegacion/inicio.gif}}
\docLink{../../docs_curso/contenido.html}{\includegraphics{../../images/navegacion/contenido.gif}}
\docLink{../../docs_curso/descripcion.html}{\includegraphics{../../images/navegacion/descripcion.gif}}
\docLink{../../docs_curso/profesor.html}{\includegraphics{../../images/navegacion/profesor.gif}}
& \begin{tabular}{r}
{\color{darkgray}\small Polinomios} \\ \\ \\
\end{tabular}
\docLink{algebra6.tex}{\includegraphics{../../images/navegacion/siguiente.gif}}
\\ \hline
\end{tabular}
}
%fin encabezado

%nombre capitulo
\begin{center}
\colorbox{azulc}{{\color{white} \large CAP�TULO 2}}  {\large
{\color{ azulc} FUNDAMENTOS DE ALGEBRA}}
\end{center}

\newline

%nombre leccion

\colorbox{naranja}{{\color{white} \normalsize  Lecci\'on 2.5. }}
{\normalsize {\color{naranja} Radicales}}

\newline

En algunas ocasiones es mas ventajoso expresar las cantidades en
t�rminos de radicales que en t�rminos de exponentes racionales.
Las leyes de los radicales se siguen inmediatamente de las leyes
de los exponentes. Si $m$ y $n$ son enteros positivos y $a$ y $b$
son n�meros reales positivos, entonces

\bigskip

\begin{enumerate}
\item $\sqrt[n]{ab}=\sqrt[n]{a}\sqrt[n]{b}$ \item
$\sqrt[n]{\dfrac{a}{b}}=\dfrac{\sqrt[n]{a}}{\sqrt[n]{b}}$ \item
$\sqrt[m]{\sqrt[n]{a}}=\sqrt[mn]{a}$
\end{enumerate}

\bigskip


\textbf{Ejemplo 2.13. } \quad\begin{itemize} \item
$\sqrt{180}=\sqrt{36\cdot 5}=\sqrt{36}\sqrt{5}=6\sqrt{5}$ \item
$\sqrt[3]{\dfrac{250}{27}}=\dfrac{\sqrt[3]{250}}{\sqrt[3]{27}}=
\dfrac{\sqrt[3]{125\cdot 2}}{3}= \dfrac{\sqrt[3]{125}\cdot
\sqrt[3]{2}}{3}=\dfrac{5 \sqrt[3]{2}}{3}$ \item
$\sqrt[6]{81}=\sqrt[3]{\sqrt{81}}=\sqrt[3]{9}$ \item Si $x$ es un
n�mero real, entonces $\sqrt{x^{2}}=\left| x\right|$ (Por qu�?).
\end{itemize}

\bigskip

Al simplificar un radical, lo hacemos de tal forma que no existan
potencias $n$-�simas de radicales cuyo �ndice es $n$, no se tengan
fracciones bajo el signo de radical y los radicales tengan el
�ndice m�s peque�o posible.

\bigskip

\textbf{Ejemplo 2.14. } \quad\begin{itemize} \item
$\sqrt[4]{48x^{6}y^{9}}=\sqrt[4]{(16x^{4}y^{8})(3x^{2}y)}=
\sqrt[4]{16x^{4}y^{8}}\sqrt[4]{3x^{2}y}=2xy^{2}\sqrt[4]{3x^{2}y}$
\item
$\sqrt[3]{\dfrac{x}{y^{2}}}=\dfrac{\sqrt[3]{x}}{\sqrt[3]{y^{2}}}=
\dfrac{\sqrt[3]{x}\sqrt[3]{y}}{\sqrt[3]{y^{2}}\sqrt[3]{y}}=
\dfrac{\sqrt[3]{xy}}{y}$
\end{itemize}

\bigskip

En la simplificaci�n de radicales generalmente es mas f�cil
trabajarlos como exponentes racionales

\bigskip

\textbf{Ejemplo 2.15. }
\[\sqrt[4]{\dfrac{256a^{2}b^{4}}{c^{2}}}=\left(\dfrac{2^{8}a^{2}b^{4}}{c^{2}}\right)^{\frac{1}{4}}=\dfrac{2^{2}a^{\frac{1}{2}}b}{c^{\frac{1}{2}}}=\dfrac{4ba^{\frac{1}{2}}}{c^{\frac{1}{2}}}\cdot
\dfrac{c^{\frac{1}{2}}}{c^{\frac{1}{2}}}=\dfrac{4ba^{\frac{1}{2}}c^{\frac{1}{2}}}{c}=\dfrac{4b}{c}\sqrt{ac}\]

\bigskip

Para terminar, queremos se�alar un error muy frecuente en el uso
de los radicales. Este error consiste en creer que se tiene la
siguiente f�rmula \textbf{incorrecta}

\bigskip

\[\sqrt[n]{a\pm b}=\sqrt[n]{a}\pm \sqrt[n]{b}\]

\bigskip

\textbf{Ejemplo 2.16. } \quad\begin{itemize} \item
$\sqrt{9+16}\neq\sqrt{9}+\sqrt{16}$, pues
$\sqrt{9+16}=\sqrt{25}=5$ y $\sqrt{9}+\sqrt{16}=3+4=7$. \item
$\sqrt[3]{4^{3}-2^{3}}\neq \sqrt[3]{4^{3}}-\sqrt[3]{2^{3}}$, pues
$\sqrt[3]{4^{3}-2^{3}}=\sqrt[3]{56}$ y
$\sqrt[3]{4^{3}}-\sqrt[3]{2^{3}}=4-2=2$.
\end{itemize}

%Pie de p�gina
\newline

{\color{gray}
\begin{tabular}{@{\extracolsep{\fill}}lcr}
\hline \\
\docLink{algebra4.tex}{\includegraphics{../../images/navegacion/anterior.gif}}
\begin{tabular}{l}
{\color{darkgray}\small Exponentes Racionales} \\ \\ \\
\end{tabular} &
\docLink[_top]{../../index.html}{\includegraphics{../../images/navegacion/inicio.gif}}
\docLink{../../docs_curso/contenido.html}{\includegraphics{../../images/navegacion/contenido.gif}}
\docLink{../../docs_curso/descripcion.html}{\includegraphics{../../images/navegacion/descripcion.gif}}
\docLink{../../docs_curso/profesor.html}{\includegraphics{../../images/navegacion/profesor.gif}}
& \begin{tabular}{r}
{\color{darkgray}\small Polinomios} \\ \\ \\
\end{tabular}
\docLink{algebra6.tex}{\includegraphics{../../images/navegacion/siguiente.gif}}
\end{tabular}
}

\end{quote}

\newline

\begin{flushright}
\includegraphics{../../images/interfaz/copyright.gif}
\end{flushright}
\end{document}

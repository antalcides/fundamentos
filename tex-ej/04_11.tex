\documentclass[10pt]{article} %tipo documento y tipo letra

\definecolor{azulc}{cmyk}{0.72,0.58,0.42,0.20} % color titulo
\definecolor{naranja}{cmyk}{0.21,0.5,1,0.03} % color leccion

\def\eje{\centerline{\textbf{ EJERCICIOS.}}} % definiciones propias

\begin{document}

\begin{quote}

% inicio encabezado
{\color{gray}
\begin{tabular}{@{\extracolsep{\fill}}lcr}
\docLink{04_10.tex}{\includegraphics{../../images/navegacion/anterior.gif}}
\begin{tabular}{l}
{\color{darkgray}\small $e^x$} \\ \\ \\
\end{tabular} &
\docLink[_top]{../../index.html}{\includegraphics{../../images/navegacion/inicio.gif}}
\docLink{../../docs_curso/contenido.html}{\includegraphics{../../images/navegacion/contenido.gif}}
\docLink{../../docs_curso/descripcion.html}{\includegraphics{../../images/navegacion/descripcion.gif}}
\docLink{../../docs_curso/profesor.html}{\includegraphics{../../images/navegacion/profesor.gif}}
& \begin{tabular}{r}
{\color{darkgray}\small Composici�n} \\ \\ \\
\end{tabular}
\docLink{04_12.tex}{\includegraphics{../../images/navegacion/siguiente.gif}}
\\ \hline
\end{tabular}
}
%fin encabezado

%nombre capitulo
\begin{center}
\colorbox{azulc}{{\color{white} \large CAP�TULO 4}}  {\large
{\color{ azulc} FUNCIONES}}
\end{center}

\newline

%nombre leccion

\colorbox{naranja}{{\color{white} \normalsize  Lecci\'on 4.11. }}
{\normalsize {\color{naranja} Operaciones de Funciones}}

\newline

As� como los n�meros reales se suman, restan, multiplican y
dividen, tambi�n las funciones se operan.

\bigskip

\textbf{Definici�n 4.11.1. } Una {\bf operaci�n entre funciones}
es una regla que asocia a dos de ellas en una tercera funci�n,
precisando cual es la imagen de un n�mero real $x$ por esta
�ltima.

\bigskip

\colorbox{naranja}{{\color{white} \normalsize 4.11.1. }}
{\normalsize {\color{naranja} �lgebra de funciones}}

\bigskip

Es posible obtener nuevas funciones operando directamente las
im�genes, que representan n�meros reales.

\bigskip

\textbf{Definici�n 4.11.2. } Dadas las funciones $f$ y $g$ se
definen

\bigskip

\begin{center}\begin{tabular}{lcl}
{\bf la suma de} $f$ {\bf y} $g$: & & $(f+g)(x)=f(x)+g(x)$\\
{\bf la resta o la diferencia de} $f$ {\bf y} $g$: & & $(f-g)(x)=f(x)-g(x)$ \\
{\bf el producto de} $f$ {\bf y} $g$: & & $(f\cdot g)(x)=f(x)\cdot g(x)$ \\
{\bf el cociente de} $f$ {\bf y} $g$: & &
$(\frac{f}{g})(x)=\frac{f(x)}{g(x)}$ si $g(x)\neq $ $0$
\end{tabular}
\end{center}

\bigskip

Para determinar sus dominio notemos que al definir la imagen de
$x$, deben estar definidas tanto $f(x)$ como $g(x)$ y que en el
caso del cociente, $g(x)$ debe ser distinto de $0$. As�, si
$D_{h}$ denota el dominio de la funci�n $h$ entonces

\bigskip

\begin{align*}
D_{f+g}&=D_{f-g}=D_{f\cdot g}=D_{f}\cap D_{g}\\
D_{\frac{f}{g}}&=D_{f}\cap D_{g}-\{x/g(x)=0\}
\end{align*}

\bigskip

\textbf{Ejemplo 4.13. } \quad \begin{enumerate} \item Sean
$f(x)=4-2x$ y $g(x)=16-x^{2}$

\bigskip

\begin{align*}
(f+g)(x)&=\left( 4-2x\right) +\left( 16-x^{2}\right)
=20-2x-x^{2}\\
(f-g)(x)&=4-2x-\left( 16-x^{2}\right) =-12-2x+x^{2}\\
(f\cdot
g)(x)&=\left(4-2x\right)\left(16-x^{2}\right)=64-32x-4x^{2}+2x^{3}\\
\left(\frac{f}{g}\right)(x)&=\frac{4-2x}{16-x^{2}}
\end{align*}

\bigskip

Ahora bien,

\bigskip

\[D_{f}=\mathbf{R}, \quad D_{g}=\mathbf{R}\]

\bigskip

En consecuencia

\bigskip

\begin{align*}
D_{f+g}&=D_{f-g}=D_{f\cdot g}=\mathbf{R}\quad\text{y}\\
D\frac{f}{g}&=\mathbf{R-}\left\{
x\in\mathbf{R/}16-x^{2}=0\right\}=\mathbf{R-}\left\{ -4,4\right\}
\end{align*}

\bigskip

Estas son sus gr�ficas

\bigskip

\begin{center}
\includegraphics{/imagenes/4_33.gif} \\
\includegraphics{/imagenes/4_34.gif} \\
\includegraphics{/imagenes/4_35.gif} \\
\includegraphics{/imagenes/4_36.gif}
\end{center}

\bigskip

\item Como caso particular del producto tenemos
$f^{2}(x)=(f(x))^{2}$ y m�s generalmente $f^{n}(x)=(f(x))^{n}$.

\bigskip

As�, para $f(x)=4-2x$ tenemos que

\bigskip

\begin{align*}
f^{2}(x)&=\left( 4-2x\right)^{2}=16-16x+4x^{2}\qquad\text{y}\\
f^{4}(x)&=\left( 4-2x\right)^{4}=\left( 16-16x+4x^{2}\right)
\left( 16-16x+4x^{2}\right) \\
&=256-512x+384x^{2}-128x^{3}+16x^{4}
\end{align*}

\bigskip

Estas son sus gr�ficas:

\bigskip

${\color{navy}f},{\color{blue}f^{2}},{\color{magenta}f^{4}}$ \\

\bigskip

\begin{center}
\includegraphics{/imagenes/4_37.gif} \\
\end{center}

\bigskip

\item Sean $f(x)=e^{x}$ y $g(x)=e^{-x}$
\begin{align*}
(f+g)(x)&=e^{x}+e^{-x}\\
(f-g)(x)&=e^{x}-e^{-x}\\
(f\cdot g)(x)&=e^{x}e^{-x}=e^{x-x}=e^{0}=1\\
\left(\frac{f}{g}\right)(x)&=\frac{e^{x}}{e^{-x}}=e^{x}e^{x}=\left(e^{x}\right)^{2}=e^{2x}
\end{align*}

\bigskip

Ahora bien, $D_{f}=\mathbf{R}$, $D_{g}=\mathbf{R}$ En consecuencia
$D_{f+g}=D_{f-g}=D_{f\cdot g}=\mathbf{R}$ y
$D_{\frac{f}{g}}=\mathbf{R-}\left\{ x\in
\mathbf{R/}e^{x}=0\right\}$. Como $e^{x}>0$ para todo $x$ entonces
$D_{\frac{f}{g}}=\mathbf{R}$

\bigskip

Estas son sus gr�ficas

\bigskip

\begin{center}
\includegraphics{imagenes/4_38.gif} \\
\includegraphics{imagenes/4_39.gif} \\
\end{center}

\bigskip

\begin{center}
\includegraphics{imagenes/4_40.gif} \\
\includegraphics{imagenes/4_41.gif} \\
\end{center}
\end{description}

\bigskip

\colorbox{naranja}{{\color{white} \normalsize 4.11.1. }}
{\normalsize {\color{naranja} Propiedades}}

\bigskip

Las propiedades que tienen las operaciones definidas en
$\mathbf{R}$ son heredadas por la suma y el producto de funciones.
As� tenemos :

\bigskip

\begin{enumerate}
\item La asociatividad de la suma pues, dadas $f$, $g$ y $h$
funciones,
\[\left( \left( f+g\right) +h\right) \left( x\right)=\left(f+g\right) \left( x\right) +h\left( x\right) =\left( f\left(x\right) +g\left( x\right) \right) +h\left(
x\right)\] Puesto que $f\left( x\right)$, $g\left( x\right) $ y
$h\left(x\right) $ representan n�meros reales, esta �ltima suma es
igual a
\[f\left( x\right) +\left( g\left( x\right) +h\left( x\right)\right) =f\left( x\right) +\left( g+h\right) \left(x\right)\]
As�:

\bigskip

\[\left( f+g\right) +h=f+\left( g+h\right)\]

\bigskip

De manera similar se prueba la asociatividad del producto.

\bigskip

\item La conmutatividad del producto pues

\bigskip

\[\left( f\cdot g\right)\left(x\right) =f\left( x\right) g\left( x\right),\]
$f\left( x\right) $ y $g\left( x\right) $ representan n�meros
entonces este producto es igual a:

\bigskip

\[g\left( x\right) f\left( x\right) =\left( g\cdot f\right) \left(x\right).\]

\bigskip

As� $f\cdot g=g\cdot f$. De manera similar se prueba la
conmutatividad de la suma.

\bigskip

\item Sea $\overline{0}$ la funci�n que a todo n�mero real $x$
asocia el n�mero $0$, es decir $\overline{0}\left( x\right) =0$ y
sea $\overline{1}$ la funci�n que a cada real $x$ asocia el n�mero
$1$. Entonces, por una parte,
\[\left( f+\overline{0}\right) \left( x\right) =f\left( x\right) +\overline{0}\left( x\right) =f\left( x\right) +0=f\left( x\right)\]
es decir,
\[f+\overline{0}=f\]
Por otra parte,
\[\left( f\cdot \overline{1}\right) \left( x\right) =f\left(x\right) \overline{1}\left( x\right) =f\left( x\right) \cdot 1=f\left(x\right)\]
es decir
\[f\cdot \overline{1}=f\]
As�, entre las funciones existen m�dulos para la suma y para el
producto.

\item Si la funci�n $-f$ se define por
\[\left( -f\right) \left( x\right) =-f\left( x\right)\]
y, para aquellos $x$ para los cuales $f\left( x\right) $
diferentes de $0$ se define $\frac{1}{f}$ por
\[\left( \frac{1}{f}\right) \left( x\right) =\frac{1}{f\left(x\right) }\]
entonces
\[\left( f+\left( -f\right) \right) \left( x\right) =f\left(x\right) +\left(-f\right) \left( x\right) =f\left( x\right) -f\left( x\right) =0=\overline{0}x\]
es decir
\[f+\left( -f\right) =\overline{0}\]
y
\[\left( f\cdot \frac{1}{f}\right) \left( x\right) =f\left(x\right) \cdot \frac{1}{f\left( x\right) }=1=\overline{1}\left(x\right)\]
es decir,
\[f\cdot \frac{1}{f}=\overline{1.}\]
\end{enumerate}

\bigskip

As�, entre las funciones existen inversos para la suma e inversos
parciales para el producto.


%Pie de p�gina
\newline

{\color{gray}
\begin{tabular}{@{\extracolsep{\fill}}lcr}
\hline \\
\docLink{04_10.tex}{\includegraphics{../../images/navegacion/anterior.gif}}
\begin{tabular}{l}
{\color{darkgray}\small $e^x$} \\ \\ \\
\end{tabular} &
\docLink[_top]{../../index.html}{\includegraphics{../../images/navegacion/inicio.gif}}
\docLink{../../docs_curso/contenido.html}{\includegraphics{../../images/navegacion/contenido.gif}}
\docLink{../../docs_curso/descripcion.html}{\includegraphics{../../images/navegacion/descripcion.gif}}
\docLink{../../docs_curso/profesor.html}{\includegraphics{../../images/navegacion/profesor.gif}}
& \begin{tabular}{r}
{\color{darkgray}\small Composici�n} \\ \\ \\
\end{tabular}
\docLink{04_12.tex}{\includegraphics{../../images/navegacion/siguiente.gif}}
\end{tabular}
}

\end{quote}

\newline

\begin{flushright}
\includegraphics{../../images/interfaz/copyright.gif}
\end{flushright}
\end{document}

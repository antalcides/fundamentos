\documentclass[10pt]{article} %tipo documento y tipo letra

\definecolor{azulc}{cmyk}{0.72,0.58,0.42,0.20} % color titulo
\definecolor{naranja}{cmyk}{0.21,0.5,1,0.03} % color leccion

\def\eje{\centerline{\textbf{ EJERCICIOS.}}} % definiciones propias

\begin{document}

\begin{quote}

% inicio encabezado
{\color{gray}
\begin{tabular}{@{\extracolsep{\fill}}lcr}
\docLink{geometria5.tex}{\includegraphics{../../images/navegacion/anterior.gif}}
\begin{tabular}{l}
{\color {darkgray} {\small Regiones Poligonales}} \\ \\ \\
\end{tabular} &
\docLink[_top]{../../index.html}{\includegraphics{../../images/navegacion/inicio.gif}}
\docLink{../../docs_curso/contenido.html}{\includegraphics{../../images/navegacion/contenido.gif}}
\docLink{../../docs_curso/descripcion.html}{\includegraphics{../../images/navegacion/descripcion.gif}}
\docLink{../../docs_curso/profesor.html}{\includegraphics{../../images/navegacion/profesor.gif}}
& \begin{tabular}{r}
{\color {darkgray} {\small \'{A}ngulos y Medidas}} \\ \\ \\
\end{tabular}
\docLink{geometria7.tex}{\includegraphics{../../images/navegacion/siguiente.gif}}
\\ \hline
\end{tabular}
}
%fin encabezado

%nombre capitulo
\begin{center}
\colorbox{azulc}{{\color{white} \large CAP�TULO 3}}  {\large
{\color{ azulc} MODULO DE GEOMETR�A}}
\end{center}

\newline

%nombre leccion

\colorbox{naranja}{{\color{white} \normalsize  Lecci\'on 3.6. }}
{\normalsize {\color{naranja} Tri�ngulos - Clasificaci�n seg�n
medida de lados }}

\newline

Con la caracterizaci�n anterior ya podemos definir de manera
precisa tri�ngulo " un pol�gono de tres lados" y mencionar
tri�ngulos con caracter�sticas especiales que reciben a su vez
nombres espec�ficos. Si se considera, por ejemplo, la medida de
los lados del tri�ngulo, se presentan tres posibilidades: que
todas las medidas de los lados sean iguales, que al menos dos sean
iguales o que ninguno de los lados tenga la misma medida que otro.
Un tri�ngulo equil�tero tiene sus tres lados de la misma medida.
Un tri�ngulo is�sceles tiene al menos dos de sus lados de la misma
medida (Todo tri�ngulo equil�tero es en consecuencia is�sceles).
Un tri�ngulo que no tiene lados de la misma medida es llamado
escaleno.

\bigskip

\[
\begin{array}{ccc}
\includegraphics{/imagenes/isoceles.gif} & \includegraphics{/imagenes/escaleno.gif} &
\includegraphics{/imagenes/equilatero.gif} \\
Is�sceles & Escaleno & Equil�tero
\end{array}
\]

%Pie de p�gina
\newline

{\color{gray}
\begin{tabular}{@{\extracolsep{\fill}}lcr}
\hline \\
\docLink{geometria5.tex}{\includegraphics{../../images/navegacion/anterior.gif}}
\begin{tabular}{l}
{\color {darkgray} {\small Regiones Poligonales}} \\ \\ \\
\end{tabular} &
\docLink[_top]{../../index.html}{\includegraphics{../../images/navegacion/inicio.gif}}
\docLink{../../docs_curso/contenido.html}{\includegraphics{../../images/navegacion/contenido.gif}}
\docLink{../../docs_curso/descripcion.html}{\includegraphics{../../images/navegacion/descripcion.gif}}
\docLink{../../docs_curso/profesor.html}{\includegraphics{../../images/navegacion/profesor.gif}}
& \begin{tabular}{r}
{\color {darkgray} {\small \'{A}ngulos y Medidas}} \\ \\ \\
\end{tabular}
\docLink{geometria7.tex}{\includegraphics{../../images/navegacion/siguiente.gif}}
\end{tabular}
}

\end{quote}

\newline

\begin{flushright}
\includegraphics{../../images/interfaz/copyright.gif}
\end{flushright}
\end{document}

\documentclass[10pt]{article} %tipo documento y tipo letra

\definecolor{azulc}{cmyk}{0.72,0.58,0.42,0.20} % color titulo
\definecolor{naranja}{cmyk}{0.50,0.42,0.42,0.06} % color leccion

\def\Reales{\mathbb{R}}
\def\Naturales{\mathbb{N}}
\def\Enteros{\mathbb{Z}}
\def\Racionales{\mathbb{Q}}
\def\Irr{\mathbb{I}}
\def\contradiccion{($\rightarrow \leftarrow$)}


\begin{document}

\begin{quote}

% inicio encabezado
{\color{gray}
\begin{tabular}{@{\extracolsep{\fill}}lcr}
\docLink{taller2.tex}{\includegraphics{../../../images/navegacion/anterior.gif}}
\begin{tabular}{l}
{\color{darkgray}\small Taller 2 } \\ \\ \\
\end{tabular}&
\docLink[_top]{../../../index.html}{\includegraphics{../../../images/navegacion/inicio.gif}}
\docLink{../../../docs_curso/contenido.html}{\includegraphics{../../../images/navegacion/contenido.gif}}
\docLink{../../../docs_curso/descripcion.html}{\includegraphics{../../../images/navegacion/descripcion.gif}}
\docLink{../../../docs_curso/profesor.html}{\includegraphics{../../../images/navegacion/profesor.gif}}
& \begin{tabular}{r}
{\color{darkgray}\small Taller 4} \\ \\ \\
\end{tabular}
\docLink{taller4.tex}{\includegraphics{../../../images/navegacion/siguiente.gif}}
\\ \hline
\end{tabular}
}
%fin encabezado

%nombre capitulo
\begin{center}
\colorbox{azulc}{{\color{white} \large TRIGONOMETR\'{I}A}}  {\large
{\color{ azulc} }}
\end{center}

\newline

%nombre leccion

\colorbox{naranja}{{\color{white} \normalsize  TALLER 3}}
{\normalsize {\color{naranja} }}

\newline

\begin{enumerate}

\item Encuentre la soluci�n de cada ecuaci�n en el intervalo
$\left[ 0,\pi \right]$:
\begin{enumerate}
\item  $sen t = \frac{1}{2}$ \item  $\sec \beta =2$ \item
$\sec^{2}\alpha -4=0$ \item  $2sen^{2}u=1-sen u$ \item
$2\cos^{2}t+3\cos t+1 = 0$
\end{enumerate}

\item Encuentre la soluci�n de cada ecuaci�n en el intervalo
$\left[ 0,2\pi \right]$:
\begin{enumerate}
\item  $sen x - \cos x = 0$ \item  $sen^{2}\theta -sen\theta -6=0$
\item  $2\tan t - \sec^{2}t = 0$
\end{enumerate}

\item En un d�a despejado con $D$ horas de iluminaci�n, la
intensidad de la luz solar $I$ (en calor�as/cm$^{2}$) se puede
calcular mediante: $I = I_{m}sen^{3}\left( \dfrac{\pi
t}{D}\right)$, $0\leq t\leq D$, donde $I_{m}$ es la intensidad
m�xima. Un dermat�logo recomienda protegerse del sol cuando la
intensidad $I$ revase el 75\% de la m�xima. Si $D=12$ horas,
calcule el n�mero de horas para las que se requiere protecci�n en
un d�a despejado.

\item Un jard�n triangular tiene lados que miden 42, 50 y 63 m.
Encuentre la medida del �ngulo menor.

\item Dos autom�viles parten de la intersecci�n de dos carreteras
rectas, y viajan a lo largo de ellas a una velocidad de 55 millas
por hora y 65 millas por hora, respectivamente. Si el �ngulo de
intersecci�n de las carreteras mide $72^{\circ }$, �qu� tan
separados est�n los autom�viles despu�s de 36 minutos?

\item Las boyas $A$, $B$ y $C$, marcan los v�rtices de una pista
triangular de carreras en un lago. Las boyas $A$ y $B$, distan
4200 pies, las boyas $A$ y $C$ distan 3800 pies y el �ngulo
$\angle CAB$ mide $100^{\circ }$. Si la lancha ganadora de la
carrera recorri� la pista en 6.4 segundos, �cu�l fue su promedio
en millas por hora?

\item Un camino recto hace un �ngulo de $15^{\circ }$ con la
horizontal. Cuando el �ngulo de elevaci�n del sol es de
$57^{\circ}$, un poste vertical que est� a un lado del camino,
proyecta una sombra de 75 pies de largo, directamente cuesta
abajo. Calcule la longitud del poste.

\item Un trotador corre a una velocidad constante de una milla
cada 8 minutos en direcci�n S$40^{\circ }$E durante 20 minutos y
luego en direcci�n N$20^{\circ }$E durante los siguientes 16
minutos. Calcule la distancia desde el punto final al punto de
partida.

\end{enumerate}

%Pie de p�gina
\newline

{\color{gray}
\begin{tabular}{@{\extracolsep{\fill}}lcr}
\hline \\
\docLink{taller2.tex}{\includegraphics{../../../images/navegacion/anterior.gif}}
\begin{tabular}{l}
{\color{darkgray}\small Taller 2 } \\ \\ \\
\end{tabular}&
\docLink[_top]{../../../index.html}{\includegraphics{../../../images/navegacion/inicio.gif}}
\docLink{../../../docs_curso/contenido.html}{\includegraphics{../../../images/navegacion/contenido.gif}}
\docLink{../../../docs_curso/descripcion.html}{\includegraphics{../../../images/navegacion/descripcion.gif}}
\docLink{../../../docs_curso/profesor.html}{\includegraphics{../../../images/navegacion/profesor.gif}}
& \begin{tabular}{r}
{\color{darkgray}\small Taller 4} \\ \\ \\
\end{tabular}
\docLink{taller4.tex}{\includegraphics{../../../images/navegacion/siguiente.gif}}
\end{tabular}
}

\end{quote}

\newline

\begin{flushright}
\includegraphics{../../../images/interfaz/copyright.gif}
\end{flushright}
\end{document}

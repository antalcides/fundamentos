%tipo documento y tipo letra
% color titulo
% color leccion
% definiciones propias


\documentclass[10pt]{article}
%%%%%%%%%%%%%%%%%%%%%%%%%%%%%%%%%%%%%%%%%%%%%%%%%%%%%%%%%%%%%%%%%%%%%%%%%%%%%%%%%%%%%%%%%%%%%%%%%%%%%%%%%%%%%%%%%%%%%%%%%%%%%%%%%%%%%%%%%%%%%%%%%%%%%%%%%%%%%%%%%%%%%%%%%%%%%%%%%%%%%%%%%%%%%%%%%%%%%%%%%%%%%%%%%%%%%%%%%%%%%%%%%%%%%%%%%%%%%%%%%%%%%%%%%%%%
%TCIDATA{OutputFilter=Latex.dll}
%TCIDATA{Version=5.00.0.2570}
%TCIDATA{<META NAME="SaveForMode" CONTENT="1">}
%TCIDATA{LastRevised=Thursday, November 04, 2004 16:53:18}
%TCIDATA{<META NAME="GraphicsSave" CONTENT="32">}

\definecolor{azulc}{cmyk}{0.72,0.58,0.42,0.20}
\definecolor{naranja}{cmyk}{0.21,0.5,1,0.03}
\def\eje{\centerline{\textbf{ EJERCICIOS.}}}

\begin{document}

\begin{quote}
% inicio encabezado
{\color{gray}
\begin{tabular}{@{\extracolsep{\fill}}lcr}
\docLink{04_07.tex}{\includegraphics{../../images/navegacion/anterior.gif}}
\begin{tabular}{l}
{\color{darkgray}{\small Funciones Cuadr\'{a}ticas}} \\
\\
\\
\end{tabular}
& \docLink[_top]{../../index.html}{%
\includegraphics{../../images/navegacion/inicio.gif}} %
\docLink{../../docs_curso/contenido.html}{%
\includegraphics{../../images/navegacion/contenido.gif}} %
\docLink{../../docs_curso/descripcion.html}{%
\includegraphics{../../images/navegacion/descripcion.gif}} %
\docLink{../../docs_curso/profesor.html}{%
\includegraphics{../../images/navegacion/profesor.gif}} &
\begin{tabular}{r}
{\color{darkgray}{\small Logaritmos}} \\
\\
\\
\end{tabular}
\docLink{04_09.tex}{\includegraphics{../../images/navegacion/siguiente.gif}}
\\ \hline
\end{tabular}
} %fin encabezado

%nombre capitulo
\end{quote}

\begin{center}
\colorbox{azulc}{{\color{white} \large CAP�TULO 4}}  {\large
{\color{ azulc} FUNCIONES}}
\end{center}

\begin{quote}
\newline

%nombre leccion

\colorbox{naranja}{{\color{white} \normalsize  Lecci\'on 4.8. }}
{\normalsize {\color{naranja} Funciones Exponenciales y Logar\'{\i}tmicas}}

\newline

Consideremos el siguiente problema:

\bigskip

Una poblaci�n $P_{\circ }$ de bacterias presente en un instante
$t=0$ se duplica cada hora. transcurrida la primera, la segunda y
la tercera horas, las poblaciones son respectivamente de
$2P_{\circ }$, $2^{2}P_{\circ }$, $2^{3}P_{\circ }$,
$2^{4}P_{\circ }$ m�s generalmente, transcurridas $k$ horas la
poblaci�n es $2^{k}P_{\circ }$.

\bigskip

Ahora bien si $P$ es la poblaci�n presente en un instante $t$ y la
poblaci�n presente media hora despu�s se expresa en la forma $rP$,
entonces en una hora la poblaci�n es, de una parte $r\cdot
rp=r^{2}P$ y, de otra parte es $2P$. As� $r^{2}P=2P$, de donde
$r=2^{\frac{1}{2}}$.

\bigskip

M�s generalmente, si $p$ y $q$ son\ enteros positivos entonces
$2^{\frac{P}{q}}$ es el factor por el que queda multiplicada la
poblaci�n, transcurridos $p$ periodos de $\frac{1}{q}$ horas.

\bigskip

Como el tiempo es continuo y no se restringe solamente a valores
racionales, para un n�mero real $t>0$, $2^{t}$ es el factor por el
que se multiplica la poblaci�n transcurrido un tiempo $t$.

\bigskip

El significado que tiene $2^{t}$ cuando $t$ es un n�mero
irracional es el valor al cual se acerca $2^{\frac{P}{q}}$ para
sucesivas aproximaciones decimales racionales de $t$. As�, $1.4$,
$1.41$, $1.414$, $1.4142$, $1.41421$, $1.414213$,\ldots, son
aproximaciones decimales racionales de $\sqrt{2}$ entonces
$2^{1.4}$, $2^{1.41}$, $2^{1.414}$, $2^{1.4142}$, $2^{1.41421}$,
$2^{1.414213}$,\ldots, se acerca al valor $2^{\sqrt{2}}$.

\bigskip

El ejemplo permite introducir el tema de las {\bf funciones
exponenciales} de base $a$ para $a$ un n�mero real positivo, es
decir, funciones de la forma $f\left( x\right)=a^{x}$ con $a>0$ y
$x$ una variable que recorre el conjunto $\mathbf{R}$.

\bigskip

Anotamos en primer lugar que las {\bf leyes de los exponentes} son
v�lidas en este caso, es decir, cualesquiera sean $a$ y $b$
n�meros reales positivos y $x$ y $y$ n�meros reales, racionales o
irracionales, tenemos:

\bigskip

\begin{itemize}
\item $a^{x}a^{y}=a^{x+y}$ \item $\left( a^{x}\right)^{y}=a^{xy}$
\item $\left( ab\right)^{x}=a^{x}b^{x}$ \item $a^{-x}=\left(
a^{-1}\right)^{x}=\left( a^{x}\right)^{-1}=\frac{1}{a^{x}}$ \item
$a^{0}=1$
\end{itemize}

\bigskip

Cualquiera sea $a>0$, $f\left( x\right)=a^{x}>0$ para todo $x$.
Con el objeto de analizar otras propiedades distinguimos entre
$a>1$ y $a<1$.

\bigskip

Supongamos $a>1$ y sea $z=\frac{P}{q}$ un racional positivo.
Tenemos entonces $a^{z}=a^{\frac{P}{q}}>1$.

\bigskip

Supongamos que $x$ e $y$ son n�meros racionales tales que $x<y$.
Como $0<y-x$ entonces $1<a^{y-x}$ y, en consecuencia,
$a^{x}=a^{x}\cdot 1<a^{x}a^{y-x}=a^{y}$. Esto es, a mayor valor
del exponente corresponde un mayor valor de la funci�n en el caso
de los exponentes racionales y como para los irracionales la
funci�n se define a partir de aproximaciones decimales racionales,
en general, a mayor valor del exponente corresponde un mayor valor
de la funci�n. Decimos que la {\bf funci�n es creciente}.

\bigskip

Ahora si $0<a<1$, sea $b=\frac{1}{a}$. Como$1<b$ y entonces la
funci�n $g\left( x\right) =b^{x}$ se comporta como en el caso
anterior, es decir, si $x$ e $y$ son n�meros reales tales que
$x<y$ entonces $b^{x}<b^{y}$ esto es $\left( \frac{1}{a}\right)
^{x}<\left( \frac{1}{a}\right) ^{y}$ o, de otra manera,
$\frac{1}{a^{x}}<\frac{1}{a^{y}}$ lo cual implica, $a^{y}<a^{x}$.
As�, en este caso, a mayor valor del exponente corresponde un
menor valor de la funci�n. Decimos que la {\bf funci�n es
decreciente}.

\bigskip

Estas son gr�ficas de funciones exponenciales de base $a$ para
diferentes valores de $a$
\end{quote}

\bigskip

\begin{center}
\includegraphics{imagenes/4_28.gif}
\end{center}

\bigskip

\begin{quote}
Si $a>1$ y $x$ e $y$ son n�meros reales tales que $x\neq y$
entonces $x<y $ o $y<x$. Si $x<y$ entonces $a^{x}<a^{y}$, en
particular $a^{x}\neq a^{y}$. Si $y<x$, $a^{y}<a^{x}$ y tambi�n
$a^{x}\neq a^{y}$. El caso $0<a<1$ es similar.

%Pie de p�gina
\newline

{\color{gray}
\begin{tabular}{@{\extracolsep{\fill}}lcr}
\hline
&  &  \\
\docLink{04_07.tex}{\includegraphics{../../images/navegacion/anterior.gif}}
\begin{tabular}{l}
{\color{darkgray}{\small Funciones Cuadr\'{a}ticas}} \\
\\
\\
\end{tabular}
& \docLink[_top]{../../index.html}{%
\includegraphics{../../images/navegacion/inicio.gif}} %
\docLink{../../docs_curso/contenido.html}{%
\includegraphics{../../images/navegacion/contenido.gif}} %
\docLink{../../docs_curso/descripcion.html}{%
\includegraphics{../../images/navegacion/descripcion.gif}} %
\docLink{../../docs_curso/profesor.html}{%
\includegraphics{../../images/navegacion/profesor.gif}} &
\begin{tabular}{r}
{\color{darkgray}{\small Logaritmos}} \\
\\
\\
\end{tabular}
\docLink{04_09.tex}{\includegraphics{../../images/navegacion/siguiente.gif}}%
\end{tabular}
}
\end{quote}

\newline

\begin{flushright}
\includegraphics{../../images/interfaz/copyright.gif}
\end{flushright}

\end{document}

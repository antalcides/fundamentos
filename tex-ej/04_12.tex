\documentclass[10pt]{article} %tipo documento y tipo letra

\definecolor{azulc}{cmyk}{0.72,0.58,0.42,0.20} % color titulo
\definecolor{naranja}{cmyk}{0.21,0.5,1,0.03} % color leccion

\def\eje{\centerline{\textbf{ EJERCICIOS.}}} % definiciones propias

\begin{document}

\begin{quote}

% inicio encabezado
{\color{gray}
\begin{tabular}{@{\extracolsep{\fill}}lcr}
\docLink{04_11.tex}{\includegraphics{../../images/navegacion/anterior.gif}}
\begin{tabular}{l}
{\color{darkgray}\small Operaciones} \\ \\ \\
\end{tabular} &
\docLink[_top]{../../index.html}{\includegraphics{../../images/navegacion/inicio.gif}}
\docLink{../../docs_curso/contenido.html}{\includegraphics{../../images/navegacion/contenido.gif}}
\docLink{../../docs_curso/descripcion.html}{\includegraphics{../../images/navegacion/descripcion.gif}}
\docLink{../../docs_curso/profesor.html}{\includegraphics{../../images/navegacion/profesor.gif}}
& \begin{tabular}{r}
{\color{darkgray}\small Cap. 5. Trigonometr�a} \\ \\ \\
\end{tabular}
\docLink{../cap5/trigo1.tex}{\includegraphics{../../images/navegacion/siguiente.gif}}
\\ \hline
\end{tabular}
}
%fin encabezado

%nombre capitulo
\begin{center}
\colorbox{azulc}{{\color{white} \large CAP�TULO 4}}  {\large
{\color{ azulc} FUNCIONES}}
\end{center}

\newline

%nombre leccion

\colorbox{naranja}{{\color{white} \normalsize  Lecci\'on 4.12. }}
{\normalsize {\color{naranja} Composici�n de Funciones}}

\newline

Existe otra manera de obtener una funci�n a partir de dos
funciones dadas y es haciendo actuar una funci�n despu�s de la
otra: se define la {\bf funci�n compuesta} de las funciones $f$ y
$g$, denotada con $f\circ g$, as�:

\bigskip

\[\left( f\circ g\right) \left(x\right)=f\left(g\left(x\right)\right)\]

\bigskip

Esta funci�n puede aplicarse a los n�meros reales $x$ para los
cuales est�n definidas tanto $g\left( x\right)$ como
$f\left(g\left( x\right) \right) $, as�:

\bigskip

\[D_{f\circ g}=\left\{ x\in D_{g}/g\left( x\right) \in
D_{f}\right\}\]

\bigskip

\textbf{Ejemplo 4.15. } \quad\begin{enumerate} \item Sean v
$f(x)=4-2x$ y $g(x)=16-x^{2}$

\bigskip

\begin{align*}
\left( f\circ g\right) \left( x\right) &=f\left( g\left( x\right)
\right) =f\left( 16-x^{2}\right) =4-2\left( 16-x^{2}\right)
=-28+2x^{2}\\
D_{f\circ g}&=\left\{ x\in D_{g}/g\left( x\right) \in
D_{f}\right\}\\
&=\left\{ x\in \mathbf{R}/g\left( x\right) \in \mathbf{R}\right\}
=\mathbf{R}\\
\left( g\circ f\right) \left( x\right) &=g\left( f\left( x\right)
\right) =g\left( 4-2x\right) =16-\left( 4-2x\right)^{2}=16x-4x^{2}\\
D_{g\circ f}&=\left\{ x\in D_{f}/g\left( x\right) \in
D_{g}\right\}\\
&=\left\{ x\in \mathbf{R}/g\left( x\right) \in \mathbf{R}\right\}
=\mathbf{R}
\end{align*}

\bigskip

Estas son sus gr�ficas

\bigskip

\begin{center}
\includegraphics{imagenes/4_42.gif} \\
\includegraphics{imagenes/4_43.gif} \\
\end{center}

\bigskip

Note que las funciones $f\circ g$ y $g\circ f$ son distintas.

\bigskip

\item Si $f\left( x\right)=\frac{3}{x+1}$, entonces
\begin{enumerate}
\item $f\left( x^{2}\right)=\frac{3}{x^{2}+1}$, en particular,
$f\left(2^{2}\right) =\frac{3}{5}$ \item $f(x)^{2}=\left(
\frac{3}{x+1}\right)^{2}=\frac{9}{\left( x+1\right)^{2}}$, en
particular, $f\left( 2\right)^{2}=1$ \item $\left( f\circ f\right)
\left( x\right) =f\left( f\left( x\right)\right)=f\left(
\frac{3}{x+1}\right)=\frac{3}{\frac{3}{x+1}+1}=\frac{3\left(x+1\right)
}{x+4}$.

\bigskip

Tambi�n

\bigskip

$f\left(f\left(x\right)\right)=\frac{3}{f\left(x\right)+1}$. En
particular, $\left( f\circ f\right) \left( 2\right) =\frac{3}{2}$
\end{enumerate}

\bigskip

En general $f\left( x^{2}\right)\neq f(x)^{2}$,
$f\left(x^{2}\right) \neq f\left( f\left( x\right) \right)$  y
$f(x)^{2}\neq f\left( f\left( x\right) \right)$.

\bigskip

\item La funci�n $h\left( x\right) =\sqrt{x^{2}-1}$ puede
obtenerse en la forma $\left( f\circ g\right)\left(x\right)$ para
que $g\left(x\right) =x^{2}-1$ y $f\left( x\right)=\sqrt{x}$.

\item La funci�n $k\left( x\right)=\frac{2}{x^{2}+2x+3}$ puede
expresarse como una funci�n compuesta, de m�s de una manera. As�
por ejemplo,

\bigskip

Si $f_{1}\left( x\right)=\frac{1}{x}$ y
$g_{1}\left(x\right)=\frac{x^{2}+2x+3}{2}$ entonces

\bigskip

\begin{align*}
(f_{1}\circ g_{1})\left(x\right)&=f_{1}\left(g_{1}\left(x\right)
\right)=\frac{1}{g_{1}\left(x\right)}=\frac{1}{\frac{x^{2}+2x+3}{2}}\\
&=\frac{2}{x^{2}+2x+3}=k\left(x\right)
\end{align*}

\bigskip

Si $f_{2}\left(x\right)=\frac{2}{x^{2}+2}$ y $g_{2}\left(
x\right)=x+1$ entonces

\bigskip

\begin{align*}
(f_{2}\circ
g_{2})\left(x\right)&=f_{2}\left(g_{2}\left(x\right)\right)=f_{2}\left(x+1\right)=\frac{2}{\left(x+1\right)^{2}+2}\\
&=\frac{2}{x^{2}+2x+3}=k\left(x\right)
\end{align*}

\bigskip

Adem�s, si $f_{3}\left(x\right)=\frac{2}{x}$,
$g_{3}\left(x\right)=x+3$ y $h\left( x\right)=x^{2}+2x$ entonces

\bigskip

\begin{align*}
\left( \left( f_{3}\circ g_{3}\right) \circ h\right)
\left(x\right)&=\left( f_{3}\circ g_{3}\right)
h\left(x\right)=f_{3}\left( g_{3}\left( h\left( x\right) \right)
\right)\\
&=f_{3}\left( g_{3}\left( x^{2}+2x\right) \right)
=f_{3}\left(x^{2}+2x+3\right)\\
&=\frac{2}{x^{2}+2x+3}
\end{align*}

\bigskip

Tambi�n

\bigskip

\[\left( f_{3}\circ \left( g_{3}\circ h\right) \right) \left(x\right) =f_{3}\left( \left( g_{3}\circ h\right) \left( x\right)\right) =f_{3}\left( g_{3}\left( h\left( x\right) \right) \right)=k\left(x\right)\]

\bigskip

\item Sean $a>0$ y $f(x)=a^{x}$ y $g(x)=\log _{a}x$
\begin{align*}
\left( f\circ
g\right)\left(x\right)&=f\left(g\left(x\right)\right)
=f\left(\log_{a}x\right)=a^{\log_{a}x}=x\\
D_{f\circ g}&=\left\{ x\in D_{g}/g\left( x\right) \in
D_{f}\right\}\\
&=\left\{ x\in \mathbf{R},\text{ }x>0/g\left( x\right) \in
\mathbf{R}\right\} =\left\{ x\in \mathbf{R},\text{ }x>0\right\}
=\mathbf{R}^{+}\\
\left( g\circ f\right) \left( x\right)&=g\left( f\left( x\right)
\right) =g\left( a^{x}\right) =\log _{a}a^{x}=x\\
D_{g\circ f}&=\left\{ x\in D_{f}/f\left( x\right) \in
D_{g}\right\}\\
&=\left\{ x\in \mathbf{R}/f\left( x\right) >0\right\} =\mathbf{R}
\end{align*}
\end{enumerate}

\bigskip

\colorbox{naranja}{{\color{white} \normalsize  Lecci\'on 4.12.1.
}} {\normalsize {\color{naranja} Propiedades}}

\bigskip

Examinemos ahora las propiedades de la composici\'{o}n de
funciones:

\bigskip

\begin{enumerate}
\item La asociatividad se verifica en general pues
\begin{align*}
\left( \left( f\circ g\right) \circ h\right) \left(
x\right)&=\left( f\circ g\right) \left( h\left( x\right) \right)
=f\left( g\left( h\left( x\right) \right) \right)\\
\left( f\circ \left( g\circ h\right) \left( x\right)
\right)&=f\left( g\left( h\left( x\right) \right) \right)
\end{align*}
as�

\bigskip

\[\left( f\circ g\right) \circ h=f\circ \left(g\circ h\right)\]

\bigskip

\item La conmutatividad no se verifica en general como lo vimos
anteriormente.

\item Sea $I$ la funci�n identidad que a cada $x$ asocia $x$, es
decir, $I\left( x\right)=x$. Para toda funci�n $f$ tenemos
\begin{align*}
\left( f\circ I\right) \left( x\right) &=\ f\left( I\left(x\right)\right) =f\left( x\right)\\
\left( I\circ f\right) \left( x\right) &=I\left(
f\left(x\right)\right) =f\left( x\right)
\end{align*}
esto es,

\bigskip

\[f\circ I=f \quad\text{e}\quad I\circ f=f\]

\end{enumerate}

%Pie de p�gina
\newline

{\color{gray}
\begin{tabular}{@{\extracolsep{\fill}}lcr}
\hline \\
\docLink{04_11.tex}{\includegraphics{../../images/navegacion/anterior.gif}}
\begin{tabular}{l}
{\color{darkgray}\small Operaciones} \\ \\ \\
\end{tabular} &
\docLink[_top]{../../index.html}{\includegraphics{../../images/navegacion/inicio.gif}}
\docLink{../../docs_curso/contenido.html}{\includegraphics{../../images/navegacion/contenido.gif}}
\docLink{../../docs_curso/descripcion.html}{\includegraphics{../../images/navegacion/descripcion.gif}}
\docLink{../../docs_curso/profesor.html}{\includegraphics{../../images/navegacion/profesor.gif}}
& \begin{tabular}{r}
{\color{darkgray}\small Cap. 5. Trigonometr�a} \\ \\ \\
\end{tabular}
\docLink{../cap5/trigo1.tex}{\includegraphics{../../images/navegacion/siguiente.gif}}
\end{tabular}
}

\end{quote}

\newline

\begin{flushright}
\includegraphics{../../images/interfaz/copyright.gif}
\end{flushright}
\end{document}

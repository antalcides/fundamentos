\documentclass[10pt]{article} %tipo documento y tipo letra

\definecolor{azulc}{cmyk}{0.72,0.58,0.42,0.20} % color titulo
\definecolor{naranja}{cmyk}{0.21,0.5,1,0.03} % color leccion

\def\eje{\centerline{\textbf{ EJERCICIOS.}}} % definiciones propias

\begin{document}

\begin{quote}

% inicio encabezado
{\color{gray}
\begin{tabular}{@{\extracolsep{\fill}}lcr}
\docLink{reales3.tex}{\includegraphics{../../images/navegacion/anterior.gif}}
\begin{tabular}{l}
{\color{darkgray}\small Diferencia y Cociente} \\ \\ \\
\end{tabular} &
\docLink[_top]{../../index.html}{\includegraphics{../../images/navegacion/inicio.gif}}
\docLink{../../docs_curso/contenido.html}{\includegraphics{../../images/navegacion/contenido.gif}}
\docLink{../../docs_curso/descripcion.html}{\includegraphics{../../images/navegacion/descripcion.gif}}
\docLink{../../docs_curso/profesor.html}{\includegraphics{../../images/navegacion/profesor.gif}}
& \begin{tabular}{r}
{\color{darkgray}\small Representaci�n Geom�trica} \\ \\ \\
\end{tabular}
\docLink{reales5.tex}{\includegraphics{../../images/navegacion/siguiente.gif}}
\\ \hline
\end{tabular}
}
%fin encabezado

%nombre capitulo
\begin{center}
\colorbox{azulc}{{\color{white} \large CAP�TULO 1}}  {\large
{\color{ azulc} LOS NUMEROS REALES}}
\end{center}

\newline

%nombre leccion

\colorbox{naranja}{{\color{white} \normalsize  Lecci\'on 1.4. }}
{\normalsize {\color{naranja} Orden en los N�meros Reales}}

\newline

El conjunto $\mathbb{R}$ de los n�meros reales contiene un
subconjunto especial llamado el conjunto de los \textit{n�meros
positivos}, que representamos por $\mathbb{P}$, cuyas propiedades
b�sicas son:

\bigskip

\begin{description}
\item[P.7] Si $x$ y $y$ pertenecen a $\mathbb{P}$, entonces $x+y$
pertenece a $\mathbb{P}$.

\item[P.8] Si $x$ y $y$ pertenecen a $\mathbb{P}$, entonces $xy$
pertenece a $\mathbb{P}$.

\item[P.9] Si $x$ es un n�mero real, se cumple exactamente una de
las siguientes relaciones.

\bigskip

\[x\in \mathbb{P}, \quad x=0, \quad -x\in \mathbb{P}\]
\end{description}

\bigskip

Estas propiedades las completamos con la siguiente definici�n:

\bigskip

\textbf{Definici�n 1.4.1.} Si $x$ y $y$ son n�meros reales,
entonces

\bigskip

\begin{itemize}
\item $x<y$ significa que $y-x\in \mathbb{P}$.

\item $x>y$ significa que $y<x$.

\item $x\leq y$ significa que $x<y$ o $x=y$.

\item $ x\geq y$ significa que $x>y$ o $x=y$.
\end{itemize}

\bigskip

De acuerdo a la definici�n anterior tenemos que

\bigskip

\begin{center}
$x>0$ si y s�lo si $x$ es positivo, es decir, $x>0$ si y s�lo si
$x\in \mathbb{P}$.
\end{center}

\bigskip

Utilizando la notaci�n anterior, podemos expresar las propiedades
b�sicas de los n�meros positivos de la siguiente forma:

\bigskip

\begin{description}
\item[P.7] Si $x>0$ y $y>0$ entonces $x+y>0$.

\item[P.8] Si $x>0$ y $y>0$ entonces $xy>0$.

\item[P.9] Si $x$ es un n�mero real, se cumple exactamente una de
las siguientes relaciones

\[x>0, \quad x=0, \quad -x>0\]
\end{description}

\bigskip

Las siguientes terminolog�as se usan con frecuencia:

\bigskip

\begin{itemize}
\item Si $x<0$ decimos que $x$ es \textit{negativo}.

\item Si $x\geq 0$ decimos que $x$ es \textit{no negativo}.

\item $x<y<z$ significa que $x<y$ y $y<z$.

\item $x\leq y<z$ significa que $x\leq y$ y $y<z$.

\item $x<y\leq z$ significa que $x<y$ y $y\leq z$.

\item $x\leq y\leq z$ significa que $x\leq y$ y $y\leq z$.
\end{itemize}

\bigskip

De estas propiedades se deducen las reglas usuales que rigen las
operaciones con desigualdades. Como ejemplo podemos citar algunas
de ellas de uso muy frecuente:

\bigskip

\begin{itemize}

\bigskip

\item Si $x$ y $y$ son n�meros reales tales que $x<y$, entonces
$x+z<y+z$ para todo n�mero real $z$.

\item Si $x,y$ y $z$ son n�meros reales tales que $x<y$ y $z>0$,
entonces $xz<yz$.

\item Si $x,y$ y $z$ son n�meros reales tales que $x<y$ y $z<0$,
entonces $xz>yz$.

\item Si $x>0$ entonces $-x<0$ y si $x<0$ entonces $-x>0$.

\item Si $x>0$ entonces $\dfrac{1}{x}>0$ y si $x<0$ entonces
$\dfrac{1}{x}<0$.

\item Si $x$ y $y$ son n�meros reales tales que $xy>0$, entonces
$x>0$ y $y>0$, o, $x<0$ y $y<0$.

\item Si $x$ y $y$ son n�meros reales tales que $xy<0$, entonces
$x>0 $ y $y<0$, o, $x<0$ y $y>0$.
\end{itemize}

\bigskip

\textbf{Ejemplo 1.3.}

\bigskip

\begin{itemize}
\item $4<7$ pues $7-4=3>0$

\item $-10<-5$ pues $(-5)-(-10)=-5+10=5>0$

\item $-6<0$ pues $0-(-6)=6>0$

\item $3<\dfrac{10}{3}<4$ pues $3<\dfrac{10}{3}$ y
$\dfrac{10}{3}<4$

\item Si $-2x<-8$ entonces $x>4$

\item Si $-3\leq 2x-5\leq 7$ entonces $2\leq 2x\leq 12$ y por lo
tanto $1\leq x\leq 6$

\item Para todo n�mero real $a$ se tiene que $a^{2}\geq 0$ \item
Si $a$ y $b$ son n�meros reales tales que $a<b$ entonces
$a<\dfrac{a+b}{2}<b$
\end{itemize}

%Pie de p�gina
\newline

{\color{gray}
\begin{tabular}{@{\extracolsep{\fill}}lcr}
\hline \\
\docLink{reales3.tex}{\includegraphics{../../images/navegacion/anterior.gif}}
\begin{tabular}{l}
{\color{darkgray}\small Diferencia y Cociente} \\ \\ \\
\end{tabular} &
\docLink[_top]{../../index.html}{\includegraphics{../../images/navegacion/inicio.gif}}
\docLink{../../docs_curso/contenido.html}{\includegraphics{../../images/navegacion/contenido.gif}}
\docLink{../../docs_curso/descripcion.html}{\includegraphics{../../images/navegacion/descripcion.gif}}
\docLink{../../docs_curso/profesor.html}{\includegraphics{../../images/navegacion/profesor.gif}}
& \begin{tabular}{r}
{\color{darkgray}\small Representaci�n Geom�trica} \\ \\ \\
\end{tabular}
\docLink{reales5.tex}{\includegraphics{../../images/navegacion/siguiente.gif}}
\end{tabular}
}

\end{quote}

\newline

\begin{flushright}
\includegraphics{../../images/interfaz/copyright.gif}
\end{flushright}
\end{document}

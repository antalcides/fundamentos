\documentclass[10pt]{article} %tipo documento y tipo letra

\definecolor{azulc}{cmyk}{0.72,0.58,0.42,0.20} % color titulo
\definecolor{naranja}{cmyk}{0.21,0.5,1,0.03} % color leccion

\def\eje{\centerline{\textbf{ EJERCICIOS.}}} % definiciones propias

\begin{document}

\begin{quote}

% inicio encabezado
{\color{gray}
\begin{tabular}{@{\extracolsep{\fill}}lcr}
\docLink{algebra10.tex}{\includegraphics{../../images/navegacion/anterior.gif}}
\begin{tabular}{l}
{\color{darkgray}\small Productos Notables} \\ \\ \\
\end{tabular} &
\docLink[_top]{../../index.html}{\includegraphics{../../images/navegacion/inicio.gif}}
\docLink{../../docs_curso/contenido.html}{\includegraphics{../../images/navegacion/contenido.gif}}
\docLink{../../docs_curso/descripcion.html}{\includegraphics{../../images/navegacion/descripcion.gif}}
\docLink{../../docs_curso/profesor.html}{\includegraphics{../../images/navegacion/profesor.gif}}
& \begin{tabular}{r}
{\color{darkgray}\small Desigualdades} \\ \\ \\
\end{tabular}
\docLink{algebra12.tex}{\includegraphics{../../images/navegacion/siguiente.gif}}
\\ \hline
\end{tabular}
}
%fin encabezado

%nombre capitulo
\begin{center}
\colorbox{azulc}{{\color{white} \large CAP�TULO 2}}  {\large
{\color{ azulc} FUNDAMENTOS DE ALGEBRA}}
\end{center}

\newline

%nombre leccion

\colorbox{naranja}{{\color{white} \normalsize  Lecci\'on 2.12. }}
{\normalsize {\color{naranja} Fracciones Algebraicas}}

\newline

Una \textit{fracci�n algebraica} es el cociente de dos expresiones
algebraicas. Si la fracci�n algebraica es el cociente de dos
polinomios, la llamamos una \textit{fracci�n racional}. Algunos
ejemplos son

\bigskip

\[\dfrac{3x^{2}-5x+1}{2x+7}, \quad \dfrac{7x-\sqrt{x^{2}-5}}{x^{\frac{2}{3}}+1}, \quad \dfrac{5x^{2}y-x^{3}+6y^{2}}{2xy-y^{4}}, \quad
\dfrac{\sqrt{x}-1}{x-\sqrt{x}}\]

\bigskip

La primera y tercera fracciones son fracciones racionales.

\bigskip

La mayor�a de las fracciones que consideramos, son fracciones
racionales en una sola variable. Como la divisi�n por cero no es
posible, siempre que tratemos con fracciones, supondremos
impl�citamente que los denominadores son diferentes de cero.

\bigskip

\colorbox{naranja}{{\color{white} \normalsize  Lecci\'on 2.12.1.
}} {\normalsize {\color{naranja} Simplificaci�n de fracciones}}

\bigskip

En el trabajo con fracciones, se acostumbra a simplificarlas hasta
donde sea posible, de tal manera que obtengamos fracciones donde
el numerador y el denominador no tengan factores comunes. El
principio b�sico para simplificar fracciones es la relaci�n
siguiente, que mencionamos en el cap�tulo 1,

\bigskip

\[\dfrac{xz}{yz}=\dfrac{x}{y}\quad \text{si} \quad z\neq 0\]

\bigskip

Este principio, nos indica que podemos cancelar los factores
comunes distintos de cero que aparecen en el numerador y el
denominador de una fracci�n.

\bigskip

\textbf{Ejemplo 2.34. } Simplifiquemos algunas fracciones

\begin{itemize}
\item [a)]
$\dfrac{2x^{2}+5x-3}{x^{2}-2x-15}=\dfrac{(2x-1)(x+3)}{(x-5)(x+3)}=\dfrac{2x-1}{x-5}$.
\item [b)]
$\dfrac{x^{2}-y^{2}}{x^{3}-y^{3}}=\dfrac{(x-y)(x+y)}{(x-y)(x^{2}+xy+y^{2})}=\dfrac{x+y}{x^{2}+xy+y^{2}}$.
\item [c)]
$\dfrac{x(x+y)(x^{3}-y^{3})}{(x-y)(x^{2}-y^{2})}=\dfrac{x(x+y)(x-y)(x^{2}+xy+y^{2})}{(x-y)(x-y)(x+y)}=\dfrac{x(x^{2}+xy+y^{2})}{x-y}$.
\end{itemize}

\bigskip

Algunos errores muy frecuentes en la simplificaci�n de fracciones
se presentan por aplicaci�n de las siguientes f�rmulas incorrectas

\bigskip

\begin{align*}
&\dfrac{x+y}{x+z}=\dfrac{y}{z}\\
&\dfrac{x+y}{x}=y \qquad\text{(F�rmulas incorrectas)}\\
&\dfrac{x+y}{x}=1+y
\end{align*}

\bigskip

\textbf{Ejemplo 2.35. } \quad \begin{itemize} \item [a)]
$\dfrac{5x+3}{5x+8}$ no es igual a $\dfrac{3}{8}$ \item [b)]
$\dfrac{3x+4y}{3x}$ no es igual a $4y$ ni $1+4y$
\end{itemize}

\bigskip

\colorbox{naranja}{{\color{white} \normalsize  Lecci\'on 2.12.2.
}} {\normalsize {\color{naranja} Operaciones con fracciones}}

\bigskip

Las operaciones de suma resta multiplicaci�n y divisi�n de
fracciones se basan en las propiedades que mencionamos en el
cap�tulo 1 y que para comodidad repetimos ahora. Estas propiedades
son:

\bigskip

\begin{itemize}
\item $\dfrac{x}{z}\pm \dfrac{y}{w}=\dfrac{xw\pm yz}{zw}$ \item
$\dfrac{x}{z}\cdot \dfrac{y}{w}=\dfrac{xy}{zw}$ \item
$\dfrac{x}{z}\div
\dfrac{y}{w}=\frac{\dfrac{x}{z}}{\dfrac{y}{w}}=\dfrac{x}{z}\cdot
\dfrac{w}{y}=\dfrac{xw}{yz}$
\end{itemize}

\bigskip

Cuando realizamos operaciones con fracciones debemos simplificar
el resultado hasta donde sea posible. En los casos de
multiplicaci�n y divisi�n de fracciones, cuando sea factible, se
simplifican numeradores y denominadores, antes de realizar las
operaciones mas complejas.

\bigskip

\textbf{Ejemplo 2.36. } \quad\begin{itemize} \item [a)]
\begin{align*}
\dfrac{x}{x+y}+\dfrac{3}{x^{2}-y^{2}}&=\dfrac{x(x^{2}-y^{2})+3(x+y)}{(x+y)(x^{2}-y^{2})}=\dfrac{x(x-y)(x+y)+3(x+y)}{(x+y)(x^{2}-y^{2})}\\
&=\dfrac{(x(x-y)+3)(x+y)}{(x^{2}-y^{2})(x+y)}=\dfrac{x(x-y)+3}{(x^{2}-y^{2})}=\dfrac{x^{2}-xy+3}{(x^{2}-y^{2})}.
\end{align*}
\item [b)]
\begin{align*}
\dfrac{(4x^{2}+4xy-3y^{2})}{(x^{2}-2xy-3y^{2})}\cdot
\dfrac{(2x^{2}-xy-3y^{2})}{(4x^{2}-9y^{2})}&=\dfrac{(4x^{2}+4xy-3y^{2})(2x^{2}-xy-3y^{2})}{(x^{2}-2xy-3y^{2})(4x^{2}-9y^{2})}\\
&=\dfrac{(2x+3y)(2x-y)(2x-3y)(x+y)}{(x-3y)(x+y)(2x-3y)(2x+3y)}\\
&=\dfrac{2x-y}{x-3y}.
\end{align*}
\item [c)]
\begin{align*}
\dfrac{\dfrac{a^{2}-2ab+b^{2}}{a^{2}-2ab}}{\dfrac{a^{2}-b^{2}}{a^{2}-ab-2b^{2}}}&=\dfrac{(a^{2}-2ab+b^{2})(a^{2}-ab-2b^{2})}{(a^{2}-2ab)(a^{2}-b^{2})}\\
&=\dfrac{(a-b)^{2}(a-2b)(a+b)}{a(a-2b)(a-b)(a+b)}=\dfrac{a-b}{a}
\end{align*}
\end{itemize}

\bigskip

Cuando se suman o restan dos o mas fracciones algebraicas, es
aconsejable escribir todas las fracciones con el mismo denominador
pues en este caso las operaciones resultan inmediatas si aplicamos
repetidamente las f�rmulas

\bigskip

\[\dfrac{x}{y}+\dfrac{z}{y}=\dfrac{x+z}{y} \quad \text{ y } \quad
\dfrac{x}{y}-\dfrac{z}{y}=\dfrac{x-z}{y}\]

\bigskip

Cualquier denominador com�n funciona, pero el mas utilizado es el
\textit{m�nimo com�n denominador} (m.c.d.), que podemos encontrar
de la siguiente forma: Primero factorizamos todos los
denominadores y luego formamos un producto que contenga a todos
los factores que aparezcan en cualquiera de los denominadores,
elevados a la mayor potencia conque se presenten en ellos. Este
producto es el m�nimo com�n denominador.

\bigskip

\textbf{Ejemplo 2.37. } Hallemos el m.c.d. de las siguientes
expresiones

\bigskip

\[x^{2}-y^{2}, \quad x^{2}+2xy+y^{2}, \quad x^{2}-3xy+2y^{2} \quad \text{ y } \quad
x^{3}+y^{3}.\]

\bigskip

Tenemos

\bigskip

\begin{align*}
x^{2}-y^{2}&=(x-y)(x+y)\\
x^{2}+2xy+y^{2}&=(x+y)^{2}\\
x^{2}-3xy+2y^{2}&=(x-2y)(x-y)\\
x^{3}+y^{3}&=(x+y)(x^{2}-xy+y^{2})
\end{align*}

\bigskip

Luego el m�nimo com�n denominador es

\bigskip

\[(x-y)(x+y)^{2}(x-2y)(x^{2}-xy+y^{2}).\]

\bigskip

\textbf{Ejemplo 2.38. } Hallemos

\bigskip

\[\dfrac{2x+1}{x^{2}-4}+\dfrac{x}{x+2}-\dfrac{3x-1}{x^{2}-4x+4}\]

\bigskip

El m�nimo com�n denominador es $(x-2)^{2}(x+2)$. Por lo tanto

\bigskip

\[\dfrac{2x+1}{x^{2}-4}=\dfrac{2x+1}{(x-2)(x+2)}=\dfrac{(2x+1)(x-2)}{(x-2)^{2}(x+2)}, \qquad
\dfrac{x}{x+2}=\dfrac{x(x-2)^{2}}{(x-2)^{2}(x+2)}\]

\bigskip

y

\bigskip

\[\dfrac{3x-1}{x^{2}-4x+4}=\dfrac{3x-1}{(x-2)^{2}}=\dfrac{(3x-1)(x+2)}{(x-2)^{2}(x+2)},\]

\bigskip

luego {\small
\begin{align*}
\dfrac{2x+1}{x^{2}-4}+\dfrac{x}{x+2}-\dfrac{3x-1}{x^{2}-4x+4}&=\dfrac{(2x+1)(x-2)+x(x-2)^{2}-(3x-1)(x+2)}{(x-2)^{2}(x+2)}\\
&=\dfrac{x^{3}-5x^{2}-4x}{(x-2)^{2}(x+2)}
\end{align*}}

\bigskip

\colorbox{naranja}{{\color{white} \normalsize  Lecci\'on 2.12.2.
}} {\normalsize {\color{naranja} Racionalizaci�n de fracciones}}

\bigskip

Algunas veces se hace necesario expresar una fracci�n de tal
manera que su numerador o su denominador no contenga radicales. El
proceso a seguir se conoce con el nombre de racionalizaci�n del
numerador o el denominador, seg�n sea el caso y lo ilustramos en
los siguientes ejemplos.

\bigskip

\textbf{Ejemplo 2.39. } \quad\begin{itemize} \item [a)]
Racionalicemos el denominador en la siguiente expresi�n

\bigskip

\[\dfrac{h}{\sqrt{x+h}-\sqrt{x}}\]

\bigskip

Para eliminar los radicales en el denominador nos basamos en el
producto notable

\bigskip

\[(a-b)(a+b)=a^{2}-b^{2} \quad \text{ con }\quad a=\sqrt{x+h} \quad \text{ y }\quad
b=\sqrt{x}\]

\bigskip

Tenemos

\bigskip

\begin{align*}
\dfrac{h}{\sqrt{x+h}-\sqrt{x}}&=\dfrac{h(\sqrt{x+h}+\sqrt{x})}{(\sqrt{x+h}-\sqrt{x})(\sqrt{x+h}+\sqrt{x})}=\dfrac{h(\sqrt{x+h}+\sqrt{x})}{(\sqrt{x+h})^{2}-(\sqrt{x})^{2}}\\
&=\dfrac{h(\sqrt{x+h}+\sqrt{x})}{(x+h)-x}=\dfrac{h(\sqrt{x+h}+\sqrt{x})}{h}=(\sqrt{x+h}+\sqrt{x})
\end{align*}
\item [b)] Racionalicemos el numerador en la expresi�n

\[\dfrac{\sqrt[3]{x}-\sqrt[3]{y}}{x-y}\]

\bigskip

En este caso nos basaremos en el producto notable

\bigskip

\[(a-b)(a^{2}+ab+b^{2})=a^{3}-b^{3}\]

\bigskip

con $a=\sqrt[3]{x}$ y $b=\sqrt[3]{y}$.

\bigskip

Tenemos

\bigskip

\begin{align*}
\dfrac{\sqrt[3]{x}-\sqrt[3]{y}}{x-y}&=\dfrac{(\sqrt[3]{x}-\sqrt
[3]{y})((\sqrt[3]{x})^{2}+\sqrt[3]{x}\sqrt[3]{y}+(\sqrt[3]{y})^{2})}{(x-y)((\sqrt[3]{x})^{2}+\sqrt[3]{x}\sqrt[3]{y}+(\sqrt[3]{y})^{2})}\\
&=\dfrac{(\sqrt[3]{x})^{3}-(\sqrt[3]{y})^{3}}{(x-y)(\sqrt[3]{x^{2}}+\sqrt[3]{xy}+\sqrt[3]{y^{2}})}\\
&=\dfrac{(x-y)}{(x-y)(\sqrt[3]{x^{2}}+\sqrt[3]{xy}+\sqrt[3]{y^{2}})}\\
&=\dfrac{1}{(\sqrt[3]{x^{2}}+\sqrt[3]{xy}+\sqrt[3]{y^{2}})}
\end{align*}
\end{itemize}

%Pie de p�gina
\newline

{\color{gray}
\begin{tabular}{@{\extracolsep{\fill}}lcr}
\hline \\
\docLink{algebra10.tex}{\includegraphics{../../images/navegacion/anterior.gif}}
\begin{tabular}{l}
{\color{darkgray}\small Productos Notables} \\ \\ \\
\end{tabular} &
\docLink[_top]{../../index.html}{\includegraphics{../../images/navegacion/inicio.gif}}
\docLink{../../docs_curso/contenido.html}{\includegraphics{../../images/navegacion/contenido.gif}}
\docLink{../../docs_curso/descripcion.html}{\includegraphics{../../images/navegacion/descripcion.gif}}
\docLink{../../docs_curso/profesor.html}{\includegraphics{../../images/navegacion/profesor.gif}}
& \begin{tabular}{r}
{\color{darkgray}\small Desigualdades} \\ \\ \\
\end{tabular}
\docLink{algebra12.tex}{\includegraphics{../../images/navegacion/siguiente.gif}}
\end{tabular}
}

\end{quote}

\newline

\begin{flushright}
\includegraphics{../../images/interfaz/copyright.gif}
\end{flushright}
\end{document}

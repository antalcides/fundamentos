\documentclass[10pt]{article} %tipo documento y tipo letra

\definecolor{azulc}{cmyk}{0.72,0.58,0.42,0.20} % color titulo
\definecolor{naranja}{cmyk}{0.21,0.5,1,0.03} % color leccion

\def\eje{\centerline{\textbf{ EJERCICIOS.}}} % definiciones propias

\begin{document}

\begin{quote}

% inicio encabezado
{\color{gray}
\begin{tabular}{@{\extracolsep{\fill}}lcr}
\docLink{trigo13.tex}{\includegraphics{../../images/navegacion/anterior.gif}}
\begin{tabular}{l}
{\color{darkgray}\small F�rmulas Suma y Resta} \\ \\ \\
\end{tabular} &
\docLink[_top]{../../index.html}{\includegraphics{../../images/navegacion/inicio.gif}}
\docLink{../../docs_curso/contenido.html}{\includegraphics{../../images/navegacion/contenido.gif}}
\docLink{../../docs_curso/descripcion.html}{\includegraphics{../../images/navegacion/descripcion.gif}}
\docLink{../../docs_curso/profesor.html}{\includegraphics{../../images/navegacion/profesor.gif}}
& \begin{tabular}{r}
{\color{darkgray}\small Aplicaciones} \\ \\ \\
\end{tabular}
\docLink{trigo15.tex}{\includegraphics{../../images/navegacion/siguiente.gif}}
\\ \hline
\end{tabular}
}
%fin encabezado

%nombre capitulo
\begin{center}
\colorbox{azulc}{{\color{white} \large CAP�TULO 5}}  {\large
{\color{ azulc} NOTAS DE TRIGONOMETRIA}}
\end{center}

\newline

%nombre leccion

\colorbox{naranja}{{\color{white} \normalsize  Lecci\'on 5.13. }}
{\normalsize {\color{naranja} \'{A}ngulos M�ltiples}}

\newline

A partir de las f�rmulas del seno y coseno para la suma de dos
�ngulos se pueden encontrar f�rmulas para calcular los valores de
seno y coseno del doble de un �ngulo:

\bigskip

\begin{align*}
sen 2t &= 2sen t \cos t\\
\cos 2t &= \cos^{2}t-sen^{2}t\\
\tan 2t &= \dfrac{2 \tan t}{1-\tan^{2}t}
\end{align*}

\bigskip

Para demostrar estas afirmaciones, se toma $2t = t + t$, y se
aplican las f�rmulas respectivas:

\bigskip

\begin{align*}
sen 2t &= sen\left(t+t\right) =sen t \cos t+\cos t sen t
=2sen t \cos t\\
\cos 2t &= \cos\left(t+t\right) =\cos t \cos t-sen t sen t
=\cos^{2}t-sen^{2}t\\
\tan 2t &= \tan\left(t+t\right) =\dfrac{\tan t+\tan t}{1-\tan t
\tan t} =\dfrac{2\tan t}{1+\tan^{2}t}
\end{align*}

\bigskip

En cada uno de los siguientes ejemplos hallaremos: $sen 2\theta$,
$\cos 2 \theta$, $\tan 2\theta$, haciendo uso de la informaci�n
dada.

\bigskip

\textbf{Ejemplo 5.35. } $sen\theta =-\dfrac{4}{5}$; $270^{\circ
}\leq \theta \leq 360^{\circ }$

\bigskip

\emph{Soluci�n}:

\bigskip

Como $\theta $ pertenece al cuarto cuadrante $\cos\theta $ es
positivo.

\bigskip

Usando la identidad fundamental:

\bigskip

\begin{align*}
\cos^{2}\theta &=1-\left( -\dfrac{4}{5}\right)
^{2}=\dfrac{9}{25}\\
\cos\theta &=\pm \sqrt{\dfrac{9}{25}}\\
\cos\theta &=\dfrac{3}{5}
\end{align*}

\bigskip

\begin{align*}
sen 2\theta &=2sen\theta \cos\theta =2\left(-\dfrac{4}{5}\right)
\left( \dfrac{3}{5}\right) =-\dfrac{24}{25}\\
\cos 2\theta &=\cos^{2}\theta -sen^{2}\theta
=\dfrac{9}{25}-\dfrac{16}{25}=-\dfrac{7}{25}
\end{align*}

\bigskip

$\tan2\theta $ puede hallarse directamente calculando $\dfrac{sen
2\theta}{\cos 2\theta }$

\bigskip

\[\tan 2\theta =\dfrac{24}{7}\]

\bigskip

\textbf{Ejemplo 5.36. }

\bigskip

$\sec \theta =-3$; $180^{\circ }< \theta < 270^{\circ }$

\bigskip

\emph{Soluci�n}:

\bigskip

$\cos\theta =\dfrac{1}{\sec\theta }=-\dfrac{1}{3}$

\bigskip

Por las identidades Pitag�ricas podemos afirmar:

\bigskip

\[sen \theta =\pm \sqrt{1-\frac{1}{9}}\]

\bigskip

Como $\theta $ est� en el tercer cuadrante:

\bigskip

\begin{align*}
sen\theta &=-\sqrt{\dfrac{8}{9}}=-\dfrac{2\sqrt{2}}{3}\\
sen 2\theta &=2sen\theta \cos\theta
=2\left(-\dfrac{2\sqrt{2}}{3}\right) \left( -\dfrac{1}{3}\right)\\
sen 2\theta &=\dfrac{4}{9}\sqrt{2}
\end{align*}

\bigskip

\begin{align*}
\cos 2\theta &=\cos^{2}\theta-sen^{2}\theta=\dfrac{1}{9}-\dfrac{8}{9}\\
\cos  2\theta &=-\dfrac{7}{9}
\end{align*}

\bigskip

\[\tan \ 2\theta =-\dfrac{4}{7}\sqrt{2}\]

%Pie de p�gina
\newline

{\color{gray}
\begin{tabular}{@{\extracolsep{\fill}}lcr}
\hline \\
\docLink{trigo13.tex}{\includegraphics{../../images/navegacion/anterior.gif}}
\begin{tabular}{l}
{\color{darkgray}\small F�rmulas Suma y Resta} \\ \\ \\
\end{tabular} &
\docLink[_top]{../../index.html}{\includegraphics{../../images/navegacion/inicio.gif}}
\docLink{../../docs_curso/contenido.html}{\includegraphics{../../images/navegacion/contenido.gif}}
\docLink{../../docs_curso/descripcion.html}{\includegraphics{../../images/navegacion/descripcion.gif}}
\docLink{../../docs_curso/profesor.html}{\includegraphics{../../images/navegacion/profesor.gif}}
& \begin{tabular}{r}
{\color{darkgray}\small Aplicaciones} \\ \\ \\
\end{tabular}
\docLink{trigo15.tex}{\includegraphics{../../images/navegacion/siguiente.gif}}
\end{tabular}
}

\end{quote}

\newline

\begin{flushright}
\includegraphics{../../images/interfaz/copyright.gif}
\end{flushright}
\end{document}

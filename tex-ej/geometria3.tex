\documentclass[10pt]{article} %tipo documento y tipo letra

\definecolor{azulc}{cmyk}{0.72,0.58,0.42,0.20} % color titulo
\definecolor{naranja}{cmyk}{0.21,0.5,1,0.03} % color leccion

\def\eje{\centerline{\textbf{ EJERCICIOS.}}} % definiciones propias

\begin{document}

\begin{quote}

% inicio encabezado
{\color{gray}
\begin{tabular}{@{\extracolsep{\fill}}lcr}
\docLink{geometria2.tex}{\includegraphics{../../images/navegacion/anterior.gif}}
\begin{tabular}{l}
{\color {darkgray} {\small Algo de Historia}} \\ \\ \\
\end{tabular} &
\docLink[_top]{../../index.html}{\includegraphics{../../images/navegacion/inicio.gif}}
\docLink{../../docs_curso/contenido.html}{\includegraphics{../../images/navegacion/contenido.gif}}
\docLink{../../docs_curso/descripcion.html}{\includegraphics{../../images/navegacion/descripcion.gif}}
\docLink{../../docs_curso/profesor.html}{\includegraphics{../../images/navegacion/profesor.gif}}
& \begin{tabular}{r}
{\color {darkgray} {\small Pol�gonos}} \\ \\ \\
\end{tabular}
\docLink{geometria4.tex}{\includegraphics{../../images/navegacion/siguiente.gif}}
\\ \hline
\end{tabular}
}
%fin encabezado

%nombre capitulo
\begin{center}
\colorbox{azulc}{{\color{white} \large CAP�TULO 3}}  {\large
{\color{ azulc} MODULO DE GEOMETR�A}}
\end{center}

\newline

%nombre leccion

\colorbox{naranja}{{\color{white} \normalsize  Lecci\'on 3.1. }}
{\normalsize {\color{naranja} Introducci�n}}

\newline

Siguiendo la tradici�n iniciada por un gran matem�tico alem�n
David Hibert (1862-1943), consideraremos en este escrito el punto,
la recta y el plano como t�rminos indefinidos. Las nociones
intuitivas y sus posibles descripciones nos permitir�n usarlos en
la reconstrucci�n de algunos elementos de la geo\-me\-tr�a
euclidiana, que muy seguramente ya han sido trabajados en la
b�sica.

\bigskip

Tomando t�rminos como puntos de partida, podemos recordar unas
primeras nociones importantes:

\bigskip

\begin{itemize}
\item Una figura es un conjunto de puntos. \item El espacio es el
conjunto de todos los puntos. \item Tres o mas puntos son
colineales s� y solamente s� ellos est�n sobre la misma recta.
\item Cuatro o mas puntos son coplanares s� y solamente s� ellos
est�n en un mismo plano.
\end{itemize}

\bigskip

Las figuras que est�n en un plano tales como cuadrados, c�rculos y
tri�ngulos son bidimensionales (tienen dos dimensiones),

\bigskip

\[
\begin{array}{ccc}
\includegraphics{/imagenes/cuadrado.gif} & \includegraphics{/imagenes/circulo.gif} & \includegraphics{/imagenes/triangulo.gif}
\end{array}
\]

\bigskip

las esferas, cajas, cubos y los objetos reales son figuras
tridimensionales.

\bigskip

\[
\begin{array}{ccc}
\includegraphics{/imagenes/cuborectangular.gif} & \includegraphics{/imagenes/cuborectangular2.gif}
\end{array}
\]

\bigskip

Diferentes maneras de describir los puntos y las rectas han dado
origen a distintas geo\-me\-tr�as (en la geo\-me\-tr�a euclidiana,
por ejemplo, un punto se describe como una localizaci�n y en la
geo\-me\-tr�a anal�tica un punto es un par ordenado de n�meros).
Para hacer mas clara la descripci�n de los puntos y las rectas se
plantean los postulados, que adem�s de servir para explicar los
t�rminos indefinidos sirven de punto de partida para deducir y
probar otros enunciados. Los postulados en la geo\-me\-tr�a
euclidiana, que como lo comentamos en el aparte anterior plante�
Euclides en los {\bf Elementos} pueden ser resumidos de la
siguiente manera:

\bigskip

\begin{itemize}
\item Dos puntos determinan una recta. (A trav�s de cualesquiera
dos puntos, pasa exactamente una recta) \item Una recta contiene
infinitos puntos \item Dada una recta en un plano, existe por lo
menos un punto en el plano que no est� en la recta. \item Dado un
plano en el espacio, existe al menos un punto en el espacio que no
est� en el plano. \item Dos rectas diferentes se intersectan a lo
mas en un punto.
\end{itemize}

\bigskip

De los postulados anteriores se derivan definiciones, teoremas y
caracterizaciones que permitir�n posteriormente solucionar
problemas.

\bigskip

\textbf{Definici�n 3.3.1.} Dos rectas que est�n en un mismo plano
son paralelas s� y solo s� ellas no tienen puntos en com�n o son
id�nticas.

\bigskip

\textbf{Definici�n 3.3.2.} El segmento (o segmento de recta) con
puntos extremos $A$ y $B$, notado $\overline{AB}$, es el conjunto
formado por los puntos $A$ y $B$ y por todos los puntos ubicados
entre ellos dos. La longitud del segmento $\overline{AB}$ (notada
$AB$) se define como la distancia entre $A$ y $B$.

\bigskip

\[
\begin{array}{ccc}
\includegraphics{/imagenes/segmento_recta.gif}
\end{array}
\]

\bigskip

\textbf{Definici�n 3.3.3.} El rayo con punto extremo $A$ y que
contiene un punto $B$, que notaremos $\overline{AB}$, consiste en
todos los puntos sobre el segmento $\overline{AB}$ y todos los
puntos que cumplen que $B$ est� entre ellos y $A$.

\bigskip

\[
\begin{array}{ccc}
\includegraphics{/imagenes/segmento_recta_inclinado.gif}
\end{array}
\]

\bigskip

{\bf PROPIEDAD ADITIVA}

\bigskip

Si $B$ est� sobre el segmento $\overline{AC}$, entonces
$AB+BC=AC$.

\bigskip

\[
\begin{array}{ccc}
\includegraphics{/imagenes/segmento_recta_1.gif}
\end{array}
\]

%Pie de p�gina
\newline

{\color{gray}
\begin{tabular}{@{\extracolsep{\fill}}lcr}
\hline \\
\docLink{geometria2.tex}{\includegraphics{../../images/navegacion/anterior.gif}}
\begin{tabular}{l}
{\color {darkgray} {\small Algo de Historia}} \\ \\ \\
\end{tabular} &
\docLink[_top]{../../index.html}{\includegraphics{../../images/navegacion/inicio.gif}}
\docLink{../../docs_curso/contenido.html}{\includegraphics{../../images/navegacion/contenido.gif}}
\docLink{../../docs_curso/descripcion.html}{\includegraphics{../../images/navegacion/descripcion.gif}}
\docLink{../../docs_curso/profesor.html}{\includegraphics{../../images/navegacion/profesor.gif}}
& \begin{tabular}{r}
{\color {darkgray} {\small Pol�gonos}} \\ \\ \\
\end{tabular}
\docLink{geometria4.tex}{\includegraphics{../../images/navegacion/siguiente.gif}}
\end{tabular}
}

\end{quote}

\newline

\begin{flushright}
\includegraphics{../../images/interfaz/copyright.gif}
\end{flushright}
\end{document}

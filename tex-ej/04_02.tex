\documentclass[10pt]{article} %tipo documento y tipo letra

\definecolor{azulc}{cmyk}{0.72,0.58,0.42,0.20} % color titulo
\definecolor{naranja}{cmyk}{0.21,0.5,1,0.03} % color leccion

\def\eje{\centerline{\textbf{ EJERCICIOS.}}} % definiciones propias

\begin{document}

\begin{quote}

% inicio encabezado
{\color{gray}
\begin{tabular}{@{\extracolsep{\fill}}lcr}
\docLink{04_01.tex}{\includegraphics{../../images/navegacion/anterior.gif}}
\begin{tabular}{l}
{\color{darkgray}\small Plano Cartesiano} \\ \\ \\
\end{tabular} &
\docLink[_top]{../../index.html}{\includegraphics{../../images/navegacion/inicio.gif}}
\docLink{../../docs_curso/contenido.html}{\includegraphics{../../images/navegacion/contenido.gif}}
\docLink{../../docs_curso/descripcion.html}{\includegraphics{../../images/navegacion/descripcion.gif}}
\docLink{../../docs_curso/profesor.html}{\includegraphics{../../images/navegacion/profesor.gif}}
& \begin{tabular}{r}
{\color{darkgray}\small Funciones y sus Gr�ficas} \\ \\ \\
\end{tabular}
\docLink{04_03.tex}{\includegraphics{../../images/navegacion/siguiente.gif}}
\\ \hline
\end{tabular}
}
%fin encabezado

%nombre capitulo
\begin{center}
\colorbox{azulc}{{\color{white} \large CAP�TULO 4}}  {\large
{\color{ azulc} FUNCIONES}}
\end{center}

\newline

%nombre leccion

\colorbox{naranja}{{\color{white} \normalsize  Lecci\'on 4.2. }}
{\normalsize {\color{naranja} F�rmula de la Distancia}}

\newline

Disponiendo de un sistema de coordenadas se puede, mediante el uso
del teorema de Pit�goras, hallar la distancia entre dos puntos del
plano.

\bigskip

Consideremos los puntos $P(x_{1},y_{1})$ y $Q(x_{2},y_{2})$. La
distancia entre ellos es la hipotenusa del tri�ngulo rect�ngulo
cuyos v�rtices son $P(x_{1},y_{1})$, $Q(x_{2},y_{2})$ y
$R(x_{2},y_{1})$. En el v�rtice $R$ est� el �ngulo recto.

\bigskip

\begin{center}
\includegraphics{imagenes/4_1.gif}
\end{center}

\bigskip

Entonces, la distancia entre $P$ y $Q$ es

\bigskip

\[d(P,Q)=\sqrt{(x_{1}-x_{2})^{2}+(y_{2}-y_{1})^{2}}\]

\bigskip

Note que, por tratarse de cuadrados de n�meros reales,

\bigskip

\[(x_{2}-x_{1})^{2}=(x_{1}-x_{2})^{2} \quad\text{y}\quad (y_{2}-y_{1})^{2}=(y_{1}-y_{2})^{2}\]

\bigskip

as� que $d(P,Q)=d(Q,P)$.

\bigskip

Otras {\bf propiedades de la distancia} son las siguientes:

\bigskip

\begin{enumerate}
\item $d(P,Q)\geq 0$ y $d(P,Q)=0$ si y solo si $P=Q$ \item Si $R$
es cualquier otro punto del plano $d(P,Q)\leq d(P,R)+d(R,Q)$. Esta
es la desigualdad triangular. La igualdad se presenta cuando los
puntos est�n alineados
\end{enumerate}

\bigskip

\begin{center}
\includegraphics{imagenes/4_2.gif}
\end{center}

\bigskip

\textbf{Ejemplo 4.1. } \quad\begin{enumerate}

\bigskip

\item \quad
\begin{enumerate}
\item Si $P(-1,3)$ y $Q(2,-1)$ entonces
\[d(P,Q)=\sqrt{(2-(-1))^{2}+(-1-3)^{2}}=\sqrt{9+16}=\sqrt{25}=5.\]
\item Si $P\left(\sqrt{2},\sqrt{3}\right)$ y
$Q\left(-\sqrt{3},\sqrt{2}\right)$, entonces
\begin{align*}
d(P,Q)&=\sqrt{\left(-\sqrt{3}-\sqrt{2}\right)^{2}+\left(\sqrt{2}-\sqrt{3}\right)^{2}}\\
&=\sqrt{3+2\sqrt{6}+2+2-2\sqrt{6}+3}\\
&=\sqrt{10}
\end{align*}

\end{enumerate}

\bigskip

\item Los puntos $P(6,-3)$, $Q(1,-5)$ y $R(-3,5)$ son los v�rtices
de un tri�ngulo rect�ngulo. En efecto,

\bigskip

\begin{align*}
d(P,Q)^{2}&=(1-6)^{2}+(-5+3)^{2}=25+4=29\\
d(Q,R)^{2}&=(-3-1)^{2}+(5+5)^{2}=16+100=116\\
d(P,R)^{2}&=(-3-6)^{2}+(5+3)^{2}=81+64=145
\end{align*}

\bigskip

as� $d(P,R)^{2}=d(P,Q)^{2}+d(Q,R)^{2}$.

\bigskip

$P$, $Q$ y $R$ son los v�rtices de un tri�ngulo rect�ngulo cuyos
catetos son el segmento que une $P$ y $Q$ y el segmento que une
$Q$ y $R$ y cuya hipotenusa es el segmento que une los puntos $P$
y $R$.

\bigskip

\item Sean $P(x_{1},y_{1})$ y $Q(x_{2},y_{2})$. Buscamos el punto
medio del segmento que une estos puntos. Supongamos que $x_{1}<
x_{2}$.

\bigskip

Sobre el eje $X$, el punto medio entre $(x_{1},0)$ y $(x_{2},0)$
tiene segunda coordenada 0 y primera coordenada dada as�

\bigskip

\[x_{1}+\frac{x_{2}-x_{1}}{2}=\frac{x_{1}+x_{2}}{2}\]

\bigskip

De manera de similar, sobre el eje $Y$, el punto medio entre
$(0,y_{1})$ y $(0,y_{2})$ es el punto
$\left(0,\frac{y_{1}+y_{2}}{2}\right)$.

\bigskip

En consecuencia, el punto medio entre $P$ y $Q$ es el punto de
coordenadas
$\left(\frac{x_{1}+x_{2}}{2},\frac{y_{1}+y_{2}}{2}\right)$.

\end{enumerate}

%Pie de p�gina
\newline

{\color{gray}
\begin{tabular}{@{\extracolsep{\fill}}lcr}
\hline \\
\docLink{04_01.tex}{\includegraphics{../../images/navegacion/anterior.gif}}
\begin{tabular}{l}
{\color{darkgray}\small Plano Cartesiano} \\ \\ \\
\end{tabular} &
\docLink[_top]{../../index.html}{\includegraphics{../../images/navegacion/inicio.gif}}
\docLink{../../docs_curso/contenido.html}{\includegraphics{../../images/navegacion/contenido.gif}}
\docLink{../../docs_curso/descripcion.html}{\includegraphics{../../images/navegacion/descripcion.gif}}
\docLink{../../docs_curso/profesor.html}{\includegraphics{../../images/navegacion/profesor.gif}}
& \begin{tabular}{r}
{\color{darkgray}\small Funciones y sus Gr�ficas} \\ \\ \\
\end{tabular}
\docLink{04_03.tex}{\includegraphics{../../images/navegacion/siguiente.gif}}
\end{tabular}
}

\end{quote}

\newline

\begin{flushright}
\includegraphics{../../images/interfaz/copyright.gif}
\end{flushright}
\end{document}

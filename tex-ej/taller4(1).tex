\documentclass[10pt]{article} %tipo documento y tipo letra

\definecolor{azulc}{cmyk}{0.72,0.58,0.42,0.20} % color titulo
\definecolor{naranja}{cmyk}{0.50,0.42,0.42,0.06} % color leccion

\def\Reales{\mathbb{R}}
\def\Naturales{\mathbb{N}}
\def\Enteros{\mathbb{Z}}
\def\Racionales{\mathbb{Q}}
\def\Irr{\mathbb{I}}
\def\contradiccion{($\rightarrow \leftarrow$)}


\begin{document}

\begin{quote}

% inicio encabezado
{\color{gray}
\begin{tabular}{@{\extracolsep{\fill}}lcr}
\docLink{taller3.tex}{\includegraphics{../../../images/navegacion/anterior.gif}}
\begin{tabular}{l}
{\color{darkgray}\small Taller 3 } \\ \\ \\
\end{tabular}&
\docLink[_top]{../../../index.html}{\includegraphics{../../../images/navegacion/inicio.gif}}
\docLink{../../../docs_curso/contenido.html}{\includegraphics{../../../images/navegacion/contenido.gif}}
\docLink{../../../docs_curso/descripcion.html}{\includegraphics{../../../images/navegacion/descripcion.gif}}
\docLink{../../../docs_curso/profesor.html}{\includegraphics{../../../images/navegacion/profesor.gif}}
&
\\ \hline
\end{tabular}
}
%fin encabezado

%nombre capitulo
\begin{center}
\colorbox{azulc}{{\color{white} \large TRIGONOMETR\'{I}A}}  {\large
{\color{ azulc} }}
\end{center}

\newline

%nombre leccion

\colorbox{naranja}{{\color{white} \normalsize  TALLER 4}}
{\normalsize {\color{naranja} }}

\newline

\begin{description}

\item[I.]  De las siguientes afirmaciones:
\begin{enumerate}
\item  $sen 90=1$ \item  $\tan(x+\dfrac{\pi }{2})= \tan x$ \item
$\cos\left(11\dfrac{\pi}{5}\right)=\cos\left(\dfrac{\pi}{5}\right)$
\item $\tan\left( x+5\pi \right) =\tan x$
\end{enumerate}
Son verdaderas:
\begin{description}
\item[a.]  1 y 4 \item[b.]  2 y 3 \item[c.]  1 y 3 \item[d.]  2 y
4
\end{description}

\item[II.] Si $0\leq x\leq 3\pi$, entonces:
\begin{enumerate}
\item $sen3x=0$ no tiene soluci�n. \item $\cos3\pi =0$ tiene una
�nica soluci�n \item $\tan 3\pi =0$, tiene infinitas soluciones.
\item $sen 3x =\cos 3x$, no tiene soluci�n
\end{enumerate}

\item[III.] Una las siguientes afirmaciones es verdadera:

\begin{enumerate}
\item  El periodo de $\cos\left( \dfrac{\pi }{2}x+3\right)$ es
$\pi$. \item  La amplitud de $sen\left( \pi x+1\right)+2$ es 3.
\item  $sen\left( x-\dfrac{\pi }{2}\right)$, est� desfasado, con
respecto a $sen x$, $\dfrac{\pi }{2}$ unidades a la izquierda.
\item  La amplitud de $-2\cos\left( 2x+1\right)$ es -2.
\end{enumerate}

\item[IV.]  Vientos dominantes han ocasionado la inclinaci�n de
$11^{\circ }$ de un viejo �rbol hacia el Este desde la vertical.
\begin{enumerate}
\item El sol en el Oeste est� a 32$^{\circ }$ arriba de la
horizontal. \item El �rbol mide 114 pies de la corona al suelo.
\end{enumerate}
Si se quiere hallar la longitud de la sombra:
\begin{description}
\item[a.]  La informaci�n 1 es suficiente pero 2 no lo es.
\item[b.]  La informaci�n 2 es suficiente pero 1 no lo es.
\item[c.]  Se necesitan las dos informaciones. \item[d.]
Cualquiera de las dos informaciones es suficiente.
\end{description}

\item[V.]  De las siguientes afirmaciones:
\begin{enumerate}
\item $\cos\left( t+\pi \right) =-\cos t$ \item $sen\left(
t+\dfrac{3\pi }{2}\right) =-sen t$ \item $sen\left( t-\dfrac{\pi
}{2}\right) =-\cos t$ \item $\cos\left( t+\dfrac{\pi }{2}\right)
=sen t$
\end{enumerate}
Son verdaderas:
\begin{description}
\item[a.]  1 y 3 \item[b.]  2 y 4 \item[c.]  3 y 4 \item[d.]  1 y
2
\end{description}

\end{description}
%Pie de p�gina
\newline

{\color{gray}
\begin{tabular}{@{\extracolsep{\fill}}lcr}
\hline \\
\docLink{taller3.tex}{\includegraphics{../../../images/navegacion/anterior.gif}}
\begin{tabular}{l}
{\color{darkgray}\small Taller 3 } \\ \\ \\
\end{tabular}&
\docLink[_top]{../../../index.html}{\includegraphics{../../../images/navegacion/inicio.gif}}
\docLink{../../../docs_curso/contenido.html}{\includegraphics{../../../images/navegacion/contenido.gif}}
\docLink{../../../docs_curso/descripcion.html}{\includegraphics{../../../images/navegacion/descripcion.gif}}
\docLink{../../../docs_curso/profesor.html}{\includegraphics{../../../images/navegacion/profesor.gif}}
&
\end{tabular}
}

\end{quote}

\newline

\begin{flushright}
\includegraphics{../../../images/interfaz/copyright.gif}
\end{flushright}
\end{document}

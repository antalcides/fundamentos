\documentclass[10pt]{article} %tipo documento y tipo letra

\definecolor{azulc}{cmyk}{0.72,0.58,0.42,0.20} % color titulo
\definecolor{naranja}{cmyk}{0.21,0.5,1,0.03} % color leccion

\def\eje{\centerline{\textbf{ EJERCICIOS.}}} % definiciones propias

\begin{document}

\begin{quote}

% inicio encabezado
{\color{gray}
\begin{tabular}{@{\extracolsep{\fill}}lcr}
\docLink{trigo10.tex}{\includegraphics{../../images/navegacion/anterior.gif}}
\begin{tabular}{l}
{\color{darkgray}\small Gr�ficas Sinusoidales} \\ \\ \\
\end{tabular} &
\docLink[_top]{../../index.html}{\includegraphics{../../images/navegacion/inicio.gif}}
\docLink{../../docs_curso/contenido.html}{\includegraphics{../../images/navegacion/contenido.gif}}
\docLink{../../docs_curso/descripcion.html}{\includegraphics{../../images/navegacion/descripcion.gif}}
\docLink{../../docs_curso/profesor.html}{\includegraphics{../../images/navegacion/profesor.gif}}
& \begin{tabular}{r}
{\color{darkgray}\small Ecuaciones Trigonom�tricas} \\ \\ \\
\end{tabular}
\docLink{trigo12.tex}{\includegraphics{../../images/navegacion/siguiente.gif}}
\\ \hline
\end{tabular}
}
%fin encabezado

%nombre capitulo
\begin{center}
\colorbox{azulc}{{\color{white} \large CAP�TULO 5}}  {\large
{\color{ azulc} NOTAS DE TRIGONOMETRIA}}
\end{center}

\newline

%nombre leccion

\colorbox{naranja}{{\color{white} \normalsize  Lecci\'on 5.10. }}
{\normalsize {\color{naranja} Identidades Trigonom�tricas}}

\newline

Cuando una expresi�n contiene t�rminos con funciones
trigonom�tricas, se dice que es una expresi�n trigonom�trica.

\bigskip

Muchas veces dichas expresiones presentan formas complicadas que
pueden reemplazarse por expresiones equivalentes.

\bigskip

Una Identidad Trigonom�trica es una igualdad entre expresiones
trigonom�tricas, que es verdadera para todos los valores para los
que dicha expresi�n tenga sentido.

\bigskip

Esto es: Si $f\left(t\right)$ y $g\left(t\right)$ son expresiones
trigonom�tricas, f$\left(t\right)=g\left(t\right)$ es una
identidad trigonom�trica, si la igualdad se cumple para todo $t$
que est� en el dominio de $f$ y en el de $g$.

\bigskip

\colorbox{naranja}{{\color{white} 5.10.1. }}{\color{naranja}
Identidades fundamentales}

\bigskip

\begin{enumerate}
\item Rec�procas:

\bigskip

A partir de las definiciones de las funciones trigonom�tricas de
�ngulos en posici�n can�nica, deducimos

\bigskip

\[\csc \theta =\dfrac{1}{sen\theta}, \quad\sec \theta =\dfrac{1}{\cos\theta }, \quad\cot \theta =\dfrac{1}{\tan\theta}\]

\bigskip

\item Igualmente, teniendo en cuenta las definiciones dadas:

\bigskip

\[\tan\theta =\dfrac{sen\theta}{\cos\theta },\quad \cot\theta =\dfrac{\cos\theta }{sen\theta}\]

\bigskip

\item Identidades Pitag�ricas:

\bigskip

\begin{enumerate}
\item  $sen^{2}t+\cos^{2}t=1$

\bigskip

Recordamos que si $\left( x,y\right) $ es un punto que est� en el
lado final de un �ngulo en posici�n can�nica y
$r=\sqrt{x^{2}+y^{2}}$

\bigskip

\[sen t =\dfrac{y}{r}, \qquad \cos t = \dfrac{x}{r}\]

\bigskip

Entonces:

\bigskip

\begin{align*}
sen^{2}t+\cos^{2}t&=\dfrac{y^{2}}{r^{2}}+\dfrac{x^{2}}{r^{2}}\\
&= \dfrac{y^{2}+x^{2}}{r^{2}}\\
&=\dfrac{r^{2}}{r^{2}}=1
\end{align*}

\bigskip

\item $1+\tan^{2}t=\sec^{2}t$

\bigskip

Se obtiene dividiendo $sen^{2}t+\cos^{2}t=1$ , por $\cos^{2}t$

\bigskip

\item $1+\cot^{2}\theta =\csc^{2}\theta $

\bigskip

Si se divide la igualdad: $sen^{2}\theta +\cos^{2}\theta =1$, por
$sen^{2}\theta $

\bigskip

\begin{align*}
1+\dfrac{\cos^{2}\theta}{sen^{2}\theta}&=\dfrac{1}{sen^{2}\theta}\\
1+\cot^{2}\theta &=\csc^{2}\theta
\end{align*}
\end{enumerate}
\end{enumerate}

\bigskip

A partir de estas identidades es posible obtener otras m�s
complejas. No hay realmente un m�todo especial para demostrar que
una igualdad es una identidad, pero en general se aconseja iniciar
con el lado que parezca m�s complejo y hacer las transformaciones
que se considere adecuadas, para obtener la expresi�n del otro
extremo de la igualdad. No es bueno transformar los dos extremos
simult�neamente por que se estar�a suponiendo que la igualdad es
verdadera.

\bigskip

\textbf{Ejemplo 5.20. } Haciendo uso de las identidades
fundamentales encuentre los valores de las funciones
trigonom�tricas del �ngulo $\theta$ si: $\tan \theta
=-\dfrac{3}{4}$ y $sen \theta> 0$

\bigskip

\emph{Soluci�n}:

\bigskip

\begin{align*}
\cot \theta &=\dfrac{1}{\tan\theta }\\
\cot \theta &=\dfrac{4}{-3}
\end{align*}

\bigskip

\begin{align*}
\sec^{2}\theta &=1+\tan^{2}\theta\\
\sec^{2}\theta &=1+\dfrac{9}{16}\\
\sec^{2}\theta &=\dfrac{25}{16}\\
\sec \theta &=\pm \sqrt{\frac{25}{16}}
\end{align*}

\bigskip

Como la tangente es negativa y el seno positivo, $\theta $ est� en
el segundo cuadrante, por lo tanto la secante es negativa.

\bigskip

\[\sec \theta =-\dfrac{5}{4}\]

\bigskip

Esto nos permite concluir que:

\bigskip

\[\cos \theta =-\dfrac{4}{5}\]

\bigskip

Haciendo uso de la identidad : $1+\cot^{2}\theta =\csc^{2}\theta $

\bigskip

\begin{align*}
1+\frac{16}{9}&=\csc^{2}\theta\\
\csc^{2}\theta &=\dfrac{25}{9}\\
\csc\theta &=\pm \sqrt{\frac{25}{9}}
\end{align*}

\bigskip

Como $sen \theta > 0$

\bigskip

\[\csc\theta =\dfrac{5}{3}\]

\bigskip

Entonces: $sen \theta =\dfrac{3}{5}$.

\bigskip

En los siguientes ejemplos vamos a demostrar algunas identidades
trigonom�tricas:

\bigskip

\textbf{Ejemplo 5.21. } $\csc \theta -sen\theta =\cot\theta
\cos\theta$.

\bigskip

\emph{Soluci�n}:

\bigskip

\begin{align*}
\csc\theta -sen\theta &=\dfrac{1}{sen\theta }-sen\theta\\
&=\dfrac{1-sen^{2}\theta }{sen\theta }\\
&=\dfrac{\cos^{2}\theta }{sen \theta }\\
&=\dfrac{\cos\theta }{sen \theta }\cos\theta \\
&=\cot\theta \cos\theta
\end{align*}

\bigskip

\textbf{Ejemplo 5.22. } $\tan t +2\cos t \csc t = \sec t \csc t +
\cot t$

\bigskip

\emph{Soluci�n}:

\bigskip

\begin{align*}
\tan t + 2\cos t \csc t &=\dfrac{sen t}{\cos t}+2\dfrac{\cos t}{sen t}\\
&=\dfrac{sen^{2}t+2\cos^{2}t}{\cos t sen t}\\
&=\dfrac{sen^{2}t+\cos^{2}t+\cos^{2}t}{\left( \cos t\right)
\left( sen t\right)}\\
&=\dfrac{1}{\left( \cos t\right) \left( sen t\right)
}+\dfrac{\cos^{2}t}{\left( \cos t\right) \left( sen t\right)}\\
&=\left( \sec t\right) \left( \csc t\right) +\dfrac{\cos
t}{sen t}\\
&=\left( \sec t\right) \left( \csc t\right) + \cot t
\end{align*}

\bigskip

\textbf{Ejemplo 5.23. } $\left( \sec u-\tan u\right) \left( \csc
u+1\right) =\cot u$

\bigskip

\emph{Soluci�n}:

\bigskip

\begin{align*}
\left(\sec u-\tan u\right) \left( \csc u+1\right) &=\left(
\dfrac{1}{\cos u}-\dfrac{sen u}{\cos u} \right) \left(
\dfrac{1}{sen u}+1\right)\\
&=\left( \dfrac{1-sen u}{\cos u}\right) \left( \dfrac{1+sen
u}{sen u}\right)\\
&=\dfrac{1-sen^{2}u}{\left( \cos u\right) \left( sen u\right)}\\
&=\dfrac{\cos^{2}u}{\left( \cos u\right) \left( sen u\right) }\\
&=\dfrac{\cos u}{sen u}=\cot u
\end{align*}

\bigskip

\textbf{Ejemplo 5.24. } $sen^{4}r-\cos^{4}r=sen^{2}r-\cos^{2}r$

\bigskip

\emph{Soluci�n}:

\bigskip

\begin{align*}
sen^{4}r-\cos^{4}r&=\left( sen^{2}r-\cos^{2}r\right)
\left(sen^{2}r+\cos^{2}r\right)\\
&= sen^{2}r-\cos^{2}r
\end{align*}

\bigskip

\textbf{Ejemplo 5.25. } $\dfrac{1}{1-\cos
v}+\dfrac{1}{1+\cos\text{ v}}=2\csc^{2}v$

\bigskip

\emph{Soluci�n}:

\bigskip

\begin{align*}
\dfrac{1}{1-\cos v}+\dfrac{1}{1+\cos\text{ v}}&=\dfrac{\left(
1+\cos\text{ v}\right) +\left( 1-\cos\text{ v}\right) }{\left(
1-\cos\text{ v}\right) \left( 1+\cos\text{ v}\right) }\\
&=\dfrac{2}{1-\cos^{2}v}\\
&=\dfrac{2}{sen^{2}v}\\
&=2\csc^{2}v
\end{align*}

\bigskip

Observe que en cada una de las demostraciones anteriores:

\bigskip

\begin{itemize}
\item  Se inici� en el lado m�s complejo. \item  Se efectuaron las
operaciones b�sicas. \item  Se hizo uso de la factorizaci�n \item
Se emplearon identidades fundamentales.
\end{itemize}

%Pie de p�gina
\newline

{\color{gray}
\begin{tabular}{@{\extracolsep{\fill}}lcr}
\hline \\
\docLink{trigo10.tex}{\includegraphics{../../images/navegacion/anterior.gif}}
\begin{tabular}{l}
{\color{darkgray}\small Gr�ficas Sinusoidales} \\ \\ \\
\end{tabular} &
\docLink[_top]{../../index.html}{\includegraphics{../../images/navegacion/inicio.gif}}
\docLink{../../docs_curso/contenido.html}{\includegraphics{../../images/navegacion/contenido.gif}}
\docLink{../../docs_curso/descripcion.html}{\includegraphics{../../images/navegacion/descripcion.gif}}
\docLink{../../docs_curso/profesor.html}{\includegraphics{../../images/navegacion/profesor.gif}}
& \begin{tabular}{r}
{\color{darkgray}\small Ecuaciones Trigonom�tricas} \\ \\ \\
\end{tabular}
\docLink{trigo12.tex}{\includegraphics{../../images/navegacion/siguiente.gif}}
\end{tabular}
}

\end{quote}

\newline

\begin{flushright}
\includegraphics{../../images/interfaz/copyright.gif}
\end{flushright}
\end{document}

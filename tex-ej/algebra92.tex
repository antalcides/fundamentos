\documentclass[10pt]{article} %tipo documento y tipo letra

\definecolor{azulc}{cmyk}{0.72,0.58,0.42,0.20} % color titulo
\definecolor{naranja}{cmyk}{0.21,0.5,1,0.03} % color leccion

\def\eje{\centerline{\textbf{ EJERCICIOS.}}} % definiciones propias

\begin{document}

\begin{quote}

% inicio encabezado
{\color{gray}
\begin{tabular}{@{\extracolsep{\fill}}lcr}
\docLink{algebra9.tex}{\includegraphics{../../images/navegacion/anterior.gif}}
\begin{tabular}{l}
{\color{darkgray}\small Divisi�n de Polinomios} \\ \\ \\
\end{tabular} &
\docLink[_top]{../../index.html}{\includegraphics{../../images/navegacion/inicio.gif}}
\docLink{../../docs_curso/contenido.html}{\includegraphics{../../images/navegacion/contenido.gif}}
\docLink{../../docs_curso/descripcion.html}{\includegraphics{../../images/navegacion/descripcion.gif}}
\docLink{../../docs_curso/profesor.html}{\includegraphics{../../images/navegacion/profesor.gif}}
& \begin{tabular}{r}
{\color{darkgray}\small Productos Notables} \\ \\ \\
\end{tabular}
\docLink{algebra10.tex}{\includegraphics{../../images/navegacion/siguiente.gif}}
\\ \hline
\end{tabular}
}
%fin encabezado

%nombre capitulo
\begin{center}
\colorbox{azulc}{{\color{white} \large CAP�TULO 2}}  {\large
{\color{ azulc} FUNDAMENTOS DE ALGEBRA}}
\end{center}

\newline

%nombre leccion

\colorbox{naranja}{{\color{white} \normalsize  Lecci\'on 2.10. }}
{\normalsize {\color{naranja} Ra�ces de Polinomios}}

\newline

Una ra�z o un cero de un polinomio $p(x)\in
\mathbf{R}\left[x\right] $ es un n�mero complejo $z$ (real o no)
que verifica $p(z)=0$ (o, de otra, manera, es una soluci�n de la
ecuaci�n $p(x)=0$).

\bigskip

\textbf{Ejemplo 2.28. } \quad \begin{enumerate} \item Si
$z=-\dfrac{1}{2}+\dfrac{\sqrt{3}}{2}i$, entonces
$z^{2}=-\dfrac{1}{2}-\dfrac{\sqrt{3}}{2}i$ y $1+z+z^{2}=0$, es
decir, $z$ es ra�z del polinomio.

\bigskip

Adem�s, $\overline{z}$, el conjugado de $z$ es
$\overline{z}=-\dfrac{1}{2}-\dfrac{\sqrt{3}}{2}i$, as�
$\overline{z}=z^{2}$ y $\overline{z}^{2}=z$. Entonces
$1+\overline{z}+\overline{z}^{2}=1+z^{2}+z=0$, es decir, tambi�n
$\overline{z}$ es ra�z del polinomio.

\bigskip

Ahora bien $z^{3}-1=\left( z-1\right) \left( 1+z+z^{2}\right)$ y
$\overline{z}^{3}-1=\left( \overline{z}-1\right) \left(
1+\overline{z}+ \overline{z}^{2}\right)$ entonces $z$ y
$\overline{z}$, adem�s de 1 son ra�ces de $x^{3}-1$.

\bigskip

\item Para hallar las ra�ces del polinomio de grado 2,
$ax^{2}+bx+c$ con $a\neq 0,$ o dicho de otra manera, para hallar
las soluciones de la ecuaci�n

\bigskip

\[ax^{2}+bx+c=0\]

\bigskip

basta hallar las soluciones de la ecuaci�n

\bigskip

\[x^{2}+\frac{b}{a}x+\frac{c}{a}=0\]

\bigskip

puesto que
$ax^{2}+bx+c=a\left(x^{2}+\dfrac{b}{a}x+\dfrac{c}{a}\right)$ y
$a\neq 0$.

\bigskip

Para hacerlo, completamos cuadrados en la ecuaci�n

\bigskip

\[x^{2}+\frac{b}{a}x=-\frac{c}{a}\]

\bigskip

\[x^{2}+\frac{b}{a}x+\left( \frac{b}{a}x\right)^{2}=\left( \frac{b}{a}x\right)
^{2}-\frac{c}{a}\]

\bigskip

esto es

\bigskip

\[\left(x+\frac{b}{2a}x\right)^{2}=\frac{b^{2}}{4a^{2}}-\frac{c}{a}=\frac{b^{2}-4ac}{4a^{2}}\]

\bigskip

as�, $x+\dfrac{b}{2a}x=\pm \sqrt{\dfrac{b^{2}-4ac}{4a^{2}}}$

\bigskip

de donde $x=-\dfrac{b}{2a}\pm
\dfrac{\sqrt{b^{2}-4ac}}{2a}=\dfrac{-b\pm \sqrt{b^{2}-4ac}}{2a}$
\end{enumerate}

%Pie de p�gina
\newline

{\color{gray}
\begin{tabular}{@{\extracolsep{\fill}}lcr}
\hline \\
\docLink{algebra9.tex}{\includegraphics{../../images/navegacion/anterior.gif}}
\begin{tabular}{l}
{\color{darkgray}\small Divisi�n de Polinomios} \\ \\ \\
\end{tabular} &
\docLink[_top]{../../index.html}{\includegraphics{../../images/navegacion/inicio.gif}}
\docLink{../../docs_curso/contenido.html}{\includegraphics{../../images/navegacion/contenido.gif}}
\docLink{../../docs_curso/descripcion.html}{\includegraphics{../../images/navegacion/descripcion.gif}}
\docLink{../../docs_curso/profesor.html}{\includegraphics{../../images/navegacion/profesor.gif}}
& \begin{tabular}{r}
{\color{darkgray}\small Productos Notables} \\ \\ \\
\end{tabular}
\docLink{algebra10.tex}{\includegraphics{../../images/navegacion/siguiente.gif}}
\end{tabular}
}

\end{quote}

\newline

\begin{flushright}
\includegraphics{../../images/interfaz/copyright.gif}
\end{flushright}
\end{document}

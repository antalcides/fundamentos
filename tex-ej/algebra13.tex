\documentclass[10pt]{article} %tipo documento y tipo letra

\definecolor{azulc}{cmyk}{0.72,0.58,0.42,0.20} % color titulo
\definecolor{naranja}{cmyk}{0.21,0.5,1,0.03} % color leccion

\def\eje{\centerline{\textbf{ EJERCICIOS.}}} % definiciones propias

\begin{document}

\begin{quote}

% inicio encabezado
{\color{gray}
\begin{tabular}{@{\extracolsep{\fill}}lcr}
\docLink{algebra12.tex}{\includegraphics{../../images/navegacion/anterior.gif}}
\begin{tabular}{l}
{\color{darkgray}\small Desigualdades} \\ \\ \\
\end{tabular} &
\docLink[_top]{../../index.html}{\includegraphics{../../images/navegacion/inicio.gif}}
\docLink{../../docs_curso/contenido.html}{\includegraphics{../../images/navegacion/contenido.gif}}
\docLink{../../docs_curso/descripcion.html}{\includegraphics{../../images/navegacion/descripcion.gif}}
\docLink{../../docs_curso/profesor.html}{\includegraphics{../../images/navegacion/profesor.gif}}
& \begin{tabular}{r}
{\color{darkgray}\small Valor Absoluto} \\ \\ \\
\end{tabular}
\docLink{algebra14.tex}{\includegraphics{../../images/navegacion/siguiente.gif}}
\\ \hline
\end{tabular}
}
%fin encabezado

%nombre capitulo
\begin{center}
\colorbox{azulc}{{\color{white} \large CAP�TULO 2}}  {\large
{\color{ azulc} FUNDAMENTOS DE ALGEBRA}}
\end{center}

\newline

%nombre leccion

\colorbox{naranja}{{\color{white} \normalsize  Lecci\'on 2.14. }}
{\normalsize {\color{naranja} Desigualdades Cuadr�ticas}}

\newline

Una desigualdad se llama cuadr�tica si tiene alguna de las formas
siguientes:

\bigskip

\begin{gather*}
ax^{2}+bx+c>0\\
ax^{2}+bx+c\geq 0\\
ax^{2}+bx+c\leq 0\\
ax^{2}+bx+c<0
\end{gather*}

\bigskip

con $a\neq 0$.

\bigskip

Antes de indicar como se resuelven estas desigualdades, recordamos
que las soluciones de la ecuaci�n cuadr�tica $ax^{2}+bx+c=0$ donde
$a\neq 0$ son

\bigskip

\[r_{1}=\dfrac{-b+\sqrt{b^{2}-4ac}}{2a} \qquad \text{y}
\qquad r_{2}=\dfrac{-b-\sqrt{b^{2}-4ac}}{2a}\]

\bigskip

Adem�s, f�cilmente se verifica que $r_{1}$ y $r_{2}$ satisfacen
las siguientes relaciones

\bigskip

\[r_{1}+r_{2}=-\dfrac{b}{a}, \qquad r_{1}r_{2}=\dfrac{c}{a} \qquad \text{y} \qquad
ax^{2}+bx+c=a(x-r_{1})(x-r_{2})\]

\bigskip

La �ltima f�rmula nos proporciona un m�todo para factorizar
cualquier trinomio de la forma $ax^{2}+bx+c$ en todos los casos
posibles.

\bigskip

Veamos ahora como se resuelven las desigualdades cuadr�ticas. Una
primera simplificaci�n que podemos hacer es suponer que $a>0$,
pues en caso contrario, multiplicando la desigualdad por $-1$,
esta se transforma en otra desigualdad cuadr�tica con $a>0$.

\bigskip

Se presentan dos casos

\bigskip

\underline{\textit{Caso 1 }} \qquad Si $b^{2}-4ac\geq 0$.

\bigskip

En este caso la ecuaci�n cuadr�tica $ax^{2}+bx+c=0$ tiene ra�ces
reales $r_{1}$ y $r_{2}$, podemos factorizar el trinomio
$ax^{2}+bx+c$ en la forma $a(x-r_{1})(x-r_{2})$, y la desigualdad
se resuelve como en el ejemplo \docLink{algebra11.tex}{2.39}.

\bigskip

\underline{\textit{Caso 2}} \qquad Si $b^{2}-4ac<0$.

\bigskip

En este caso las ra�ces de la ecuaci�n $ax^{2}+bx+c=0$ no son
reales, sino complejas, y la factorizaci�n $a(x-r_{1})(x-r_{2})$
no sirve para resolver la desigualdad.

\bigskip

Para resolver la desigualdad en este caso procedemos de la
siguiente forma:

\bigskip

Completando el cuadrado tenemos

\bigskip

\begin{align*}
ax^{2}+bx+c&=a(x^{2}+\dfrac{b}{a}x)+c\\
&=a\left[x^{2}+\dfrac{b}{a}x+\left(\dfrac{b}{2a}\right)^{2}-\left(\dfrac{b}{2a}\right)^{2}\right]+c\\
&=a\left[x^{2}+\dfrac{b}{a}x+\left(\dfrac{b}{2a}\right)^{2}\right]+\left(c-\dfrac{b^{2}}{4a}\right)\\
&=a\left(x+\dfrac{b}{2a}\right)^{2}+\dfrac{4ac-b^{2}}{4a}
\end{align*}

\bigskip

Por lo tanto las desigualdades cuadr�ticas se transforman en su
orden en

\bigskip

\begin{gather*}
a(x+\dfrac{b}{2a})^{2}>\dfrac{b^{2}-4ac}{4a}\\
a(x+\dfrac{b}{2a})^{2}\geq \dfrac{b^{2}-4ac}{4a}\\
a(x+\dfrac{b}{2a})^{2}\leq \dfrac{b^{2}-4ac}{4a}\\
a(x+\dfrac{b}{2a})^{2}<\dfrac{b^{2}-4ac}{4a}
\end{gather*}

\bigskip

Como estamos suponiendo que $a>0$ y sabemos que $b^{2}-4ac<0$, las
dos primeras desigualdades son v�lidas para todo n�mero real y las
dos �ltimas para ninguno.

\bigskip

\textbf{Ejemplo 2.46. } Resolvamos la desigualdad

\bigskip

$3x^{2}-10x+2\leq 0$.

\bigskip

En este caso $b^{2}-4ac=(-10)^{2}-4\cdot 3\cdot 2=76\geq 0$. Por
lo tanto la ecuaci�n $3x^{2}-10x+2=0$ tiene ra�ces reales que son

\bigskip

\begin{align*}
r_{1}&=\dfrac{10+\sqrt{76}}{6}=\dfrac{10+2\sqrt{19}}{6}=\dfrac{5+\sqrt{19}}{3}
\qquad \text{y} \\
r_{2}&=\dfrac{10-\sqrt{76}}{6}=\dfrac{10-2\sqrt{19}}{6}=\dfrac{5-\sqrt{19}}{3}
\end{align*}

\bigskip

Luego la factorizaci�n de $3x^{2}-10x+2$ es

\bigskip

\[3x^{2}-10x+2=3\left[x-\left(\dfrac{5+\sqrt{19}}{3}\right)\right]\left[x-\left(\dfrac{5-\sqrt{19}}{3}\right)\right],\]

\bigskip

y la desigualdad original es equivalente a

\bigskip

\[3\left[x-\left(\dfrac{5+\sqrt{19}}{3}\right)\right]\left[x-\left(\dfrac{5-\sqrt{19}}{3}\right)\right]\leq 0\]

\bigskip

Elaborando el diagrama de signos tenemos

\bigskip

\[
\includegraphics{img/img5.gif}
\]

\bigskip

Vemos que la soluci�n de la desigualdad es el intervalo
$\left[\dfrac{5-\sqrt{19}}{3},\dfrac{5+\sqrt{19}}{3}\right]$

\bigskip

\textbf{Ejemplo 2.47. } Resolvamos la desigualdad $2x^{2}+4x+5\geq
0$.

\bigskip

En este caso tenemos que $b^{2}-4ac=4^{2}-4\cdot 2\cdot 5=-24<0$.
Por lo tanto la ecuaci�n $2x^{2}+4x+5=0$ no tiene ra�ces reales y
de acuerdo a la teor�a desarrollada, el conjunto soluci�n de la
desigualdad $2x^{2}+4x+5\geq 0$ es todo $\mathbb{R}$.

\bigskip

\textbf{Ejemplo 2.48. }

\bigskip

Resolvamos la desigualdad $-5x^{2}+7x-6>0$.

\bigskip

La desigualdad es equivalente a $5x^{2}-7x+6<0$.

\bigskip

Para esta �ltima desigualdad tenemos que
$b^{2}-4ac=(-7)^{2}-4\cdot 5\cdot 6=-71<\dot{0}$. Por lo tanto la
ecuaci�n $5x^{2}-7x+6=0$ no tiene ra�ces reales y de acuerdo a la
teor�a desarrollada, el conjunto soluci�n de la desigualdad
$5x^{2}-7x+6<0$ es $\emptyset$. Es decir, la desigualdad original
$-5x^{2}+7x-6>0$ no tiene soluciones reales.

\bigskip

Para terminar esta secci�n, recalcamos que cuando $a>0$ y
$b^{2}-4ac<0$, las desigualdades cuadr�ticas, o tienen como
conjunto soluci�n todo $\mathbb{R}$, o no tienen soluciones
reales.

%Pie de p�gina
\newline

{\color{gray}
\begin{tabular}{@{\extracolsep{\fill}}lcr}
\hline \\
\docLink{algebra12.tex}{\includegraphics{../../images/navegacion/anterior.gif}}
\begin{tabular}{l}
{\color{darkgray}\small Desigualdades} \\ \\ \\
\end{tabular} &
\docLink[_top]{../../index.html}{\includegraphics{../../images/navegacion/inicio.gif}}
\docLink{../../docs_curso/contenido.html}{\includegraphics{../../images/navegacion/contenido.gif}}
\docLink{../../docs_curso/descripcion.html}{\includegraphics{../../images/navegacion/descripcion.gif}}
\docLink{../../docs_curso/profesor.html}{\includegraphics{../../images/navegacion/profesor.gif}}
& \begin{tabular}{r}
{\color{darkgray}\small Valor Absoluto} \\ \\ \\
\end{tabular}
\docLink{algebra14.tex}{\includegraphics{../../images/navegacion/siguiente.gif}}
\end{tabular}
}

\end{quote}

\newline

\begin{flushright}
\includegraphics{../../images/interfaz/copyright.gif}
\end{flushright}
\end{document}

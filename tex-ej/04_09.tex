\documentclass[10pt]{article} %tipo documento y tipo letra

\definecolor{azulc}{cmyk}{0.72,0.58,0.42,0.20} % color titulo
\definecolor{naranja}{cmyk}{0.21,0.5,1,0.03} % color leccion

\def\eje{\centerline{\textbf{ EJERCICIOS.}}} % definiciones propias

\begin{document}

\begin{quote}

% inicio encabezado
{\color{gray}
\begin{tabular}{@{\extracolsep{\fill}}lcr}
\docLink{04_08.tex}{\includegraphics{../../images/navegacion/anterior.gif}}
\begin{tabular}{l}
{\color{darkgray}\small Exponenciales} \\ \\ \\
\end{tabular} &
\docLink[_top]{../../index.html}{\includegraphics{../../images/navegacion/inicio.gif}}
\docLink{../../docs_curso/contenido.html}{\includegraphics{../../images/navegacion/contenido.gif}}
\docLink{../../docs_curso/descripcion.html}{\includegraphics{../../images/navegacion/descripcion.gif}}
\docLink{../../docs_curso/profesor.html}{\includegraphics{../../images/navegacion/profesor.gif}}
& \begin{tabular}{r}
{\color{darkgray}\small Logaritmo Natural} \\ \\ \\
\end{tabular}
\docLink{04_10.tex}{\includegraphics{../../images/navegacion/siguiente.gif}}
\\ \hline
\end{tabular}
}
%fin encabezado

%nombre capitulo
\begin{center}
\colorbox{azulc}{{\color{white} \large CAP�TULO 4}}  {\large
{\color{ azulc} FUNCIONES}}
\end{center}

\newline

%nombre leccion

\colorbox{naranja}{{\color{white} \normalsize  Lecci\'on 4.9. }}
{\normalsize {\color{naranja} Logaritmos}}

\newline

Si $y$ es un n�mero positivo y $y=a^{x}$ para alg�n n�mero real
$x$ entonces se dice que $x$ es el {\bf logaritmo en base} $a$
{\bf de} $y$. Se denota por $\log _{a}y$.

\bigskip

As�, tenemos:

\bigskip

\begin{itemize}
\item $y=a^{x}\Longleftrightarrow \log _{a}y=x$ \item $\log
_{a}\left(a^{x}\right)=x$, $x\in \mathbf{R}$ \item $a^{\log
_{a}y}=y$, $y\in \mathbf{R}^{+}$
\end{itemize}

\bigskip

La relaci�n existente entre los logaritmos y las exponenciales de
base $a$ permite demostrar las siguientes

\bigskip

\medskip{\bf Propiedades de los logaritmos}

\bigskip

Si $a$, $b$, $x$ y $y$ son n�meros reales positivos entonces:

\bigskip

\begin{enumerate}
\item $\log_{a}x+\log_{a}y=\log_{a}xy$ \item
$\log_{a}x-\log_{a}y=\log_{a}\left(\frac{x}{y}\right)$ \item
$\log_{a}x^{y}=y\log_{a}x$ \item $\log _{b}x=\frac{\log
_{a}x}{\log _{a}b}$
\end{enumerate}

\bigskip

\textit{Demostraci�n. } Sean

\bigskip

\begin{align*}
r&=\log _{a}x\\
s&=\log_{a}y\\
t&=\log _{a}xy\\
u&=\log_{a}\left(\frac{x}{y}\right)\\
b&=\log _{a}x^{y}
\end{align*}

\bigskip

\begin{enumerate}
\item $a^{r+s}=a^{r}a^{s}=xy$ y $a^{t}=xy$

\bigskip

As�, $a^{r+s}=a^{t}$ entonces $r+s=t$, es decir $\log
_{a}x+\log_{a}y=\log _{a}xy$.

\item $a^{r-s}=a^{r}a^{-s}=\frac{a}{a^{s}}=\frac{x}{y}$ y
$a^{u}=\frac{x}{y}$

\bigskip

As�, $a^{r-s}=a^{u}$ de donde

\bigskip

$\log_{a}x-\log_{a}y=\log_{a}\left(\frac{x}{y}\right)$

\item $a^{v}=x^{y}$ y $a^{yr}=\left(a^{r}\right)^{y}=x^{y}$

\bigskip

As�, $a^{v}=a^{yr}$ entonces $\log_{a}x^{y}=y\log_{a}x \Box$
\end{enumerate}

\bigskip

{\bf Notaci�n}: cuando $a=10$, $\log_{a}x$ se denota simplemente
por $\log x$.

\bigskip

\textbf{Ejemplo 4.12. } \quad\begin{enumerate} \item
$\log_{2}32=\log_{2}2^{5}=5$

\item
$\log_{\frac{1}{2}}\left(4\right)=\log_{\frac{1}{2}}\left(2^{2}\right)=\log_{\frac{1}{2}}\left\{\left[\left(\frac{1}{2}\right)^{-1}\right]^{2}\right\}
=\log_{\frac{1}{2}}\left[\left(\frac{1}{2}\right)^{-2}\right] =-2$

\item Supongamos que $4^{2x}=64$ entonces
$2x=\log_{4}64=\log_{4}4^{3}=3$. En consecuencia, $x=\frac{3}{2}$

\item Si $\log_{6}\left( 2x+2\right)=3$ entonces
$6^{\log_{6}\left(2x+2\right)}=6^{3}$ es decir, $2x+2=216$ de
donde $x=107$

\item Si $\log \left( x-2\right) =1-\log \left( x+1\right)$
entonces
\[10^{\log\left( x-2\right) }=10^{1-\log \left(x+1\right)}=\frac{10}{10^{\log \left( x+1\right)}}\]
esto es $x-2=\frac{10}{x+1}$ de donde $x^{2}-x-12=0$.

\bigskip

Esta ecuaci�n tiene dos soluciones: $x=4$ y $x=-3$. Pero para
$x=-3$, $\log \left( x-2\right)$ y $\log \left( x+1\right)$ no
est�n definidas. As�, la �nica soluci�n es $x=4$.

\item Usando un solo logaritmo, vamos a expresar la suma

\bigskip

\[7\log_{a}\left( x+1\right) +\frac{1}{2}\log_{a}\left(3x+6\right) -\log_{a}\left( x^{2}+3x+2\right) \]

\bigskip

La expresi�n dada es igual a

\bigskip

\[\log_{a}\left( x+1\right)^{7}+\log_{a}\sqrt{3x+6}-\log_{a}\left( x^{2}+3x+2\right) \]

\bigskip

y a su vez esta suma es igual a

\bigskip

\[\log _{a}\frac{\left( x+1\right) ^{7}\sqrt{3x+6}}{x^{2}+3x+2}=\log _{a}\frac{\left( x+1\right) ^{6}\sqrt{3x+6}}{x+2}\]
\end{enumerate}

\bigskip

As� como deducimos las propiedades de los logaritmos a partir de
las propiedades de las exponenciales, podemos decir, que si $a>1$,
a un mayor valor de $x$ corresponde un mayor valor de $\log_{a}x$
y si $0<a<1$, a un mayor valor de $x$ corresponde un menor valor
de $\log_{a}x$.

\bigskip

La funci�n logaritmo en base $a$, $\log_{a}$, se define asociando
a cada n�mero real positivo $x$ el n�mero real $\log_{a}x$.

\bigskip

La observaci�n anterior permite decir que si $a>1$, la funci�n
$\log_{a}$ es creciente y si $0<a<1$, la funci�n es decreciente.

\bigskip

De otra parte, note que si una pareja $\left( x,y\right)$
pertenece a la gr�fica de la funci�n exponencial de base $a$ es
porque $y=a^{x}$ , es decir, $\log_{a}y=x$, entonces la pareja
$\left( y,x\right)$ pertenece a la gr�fica de la funci�n
$\log_{a}$. De igual manera, si una pareja $\left( u,v\right)$
pertenece a la gr�fica de la funci�n $\log _{a}$, entonces la
pareja $\left( v,u\right) $ pertenece a la gr�fica de la funci�n
exponencial de base $a$. As�, la gr�fica de la funci�n $\log_{a}$
se puede obtener reflejando la gr�fica de la funci�n exponencial
de base $a$ en la recta de ecuaci�n $y=x$.

\bigskip

Estas son las gr�ficas de las funciones $\log_{a}$ para los
valores de $a$ tomados como base de las funciones exponenciales
trazadas anteriormente.

\bigskip

\begin{center}
\includegraphics{/imagenes/4_29.gif}
\end{center}

%Pie de p�gina
\newline

{\color{gray}
\begin{tabular}{@{\extracolsep{\fill}}lcr}
\hline \\
\docLink{04_08.tex}{\includegraphics{../../images/navegacion/anterior.gif}}
\begin{tabular}{l}
{\color{darkgray}\small Exponenciales} \\ \\ \\
\end{tabular} &
\docLink[_top]{../../index.html}{\includegraphics{../../images/navegacion/inicio.gif}}
\docLink{../../docs_curso/contenido.html}{\includegraphics{../../images/navegacion/contenido.gif}}
\docLink{../../docs_curso/descripcion.html}{\includegraphics{../../images/navegacion/descripcion.gif}}
\docLink{../../docs_curso/profesor.html}{\includegraphics{../../images/navegacion/profesor.gif}}
& \begin{tabular}{r}
{\color{darkgray}\small Logaritmo Natural} \\ \\ \\
\end{tabular}
\docLink{04_10.tex}{\includegraphics{../../images/navegacion/siguiente.gif}}
\end{tabular}
}

\end{quote}

\newline

\begin{flushright}
\includegraphics{../../images/interfaz/copyright.gif}
\end{flushright}
\end{document}

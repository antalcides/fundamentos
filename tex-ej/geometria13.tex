\documentclass[10pt]{article} %tipo documento y tipo letra

\definecolor{azulc}{cmyk}{0.72,0.58,0.42,0.20} % color titulo
\definecolor{naranja}{cmyk}{0.21,0.5,1,0.03} % color leccion

\def\eje{\centerline{\textbf{ EJERCICIOS.}}} % definiciones propias

\begin{document}

\begin{quote}

% inicio encabezado
{\color{gray}
\begin{tabular}{@{\extracolsep{\fill}}lcr}
\docLink{geometria12.tex}{\includegraphics{../../images/navegacion/anterior.gif}}
\begin{tabular}{l}
{\color {darkgray} {\small \'{A}ngulos Tri�ngulo}} \\ \\ \\
\end{tabular} &
\docLink[_top]{../../index.html}{\includegraphics{../../images/navegacion/inicio.gif}}
\docLink{../../docs_curso/contenido.html}{\includegraphics{../../images/navegacion/contenido.gif}}
\docLink{../../docs_curso/descripcion.html}{\includegraphics{../../images/navegacion/descripcion.gif}}
\docLink{../../docs_curso/profesor.html}{\includegraphics{../../images/navegacion/profesor.gif}}
& \begin{tabular}{r}
{\color {darkgray} {\small Tipos Cuadrilateros}} \\ \\ \\
\end{tabular}
\docLink{geometria15.tex}{\includegraphics{../../images/navegacion/siguiente.gif}}
\\ \hline
\end{tabular}
}
%fin encabezado

%nombre capitulo
\begin{center}
\colorbox{azulc}{{\color{white} \large CAP�TULO 3}}  {\large
{\color{ azulc} MODULO DE GEOMETR�A}}
\end{center}

\newline

%nombre leccion

%nombre leccion

\colorbox{naranja}{{\color{white} \normalsize  Lecci\'on 3.12. }}
{\normalsize {\color{naranja} Tri�ngulos - Algunas Relaciones}}

\newline

\colorbox{naranja}{{\color{white} \normalsize  Lecci\'on 3.12. }}
{\normalsize {\color{naranja} Congruencia de Tri�ngulos}}

\bigskip

\[
\includegraphics{imagenes/triangulos_relaciones.gif}
\]

\bigskip

\[
\begin{array}{cc}
\includegraphics{imagenes/triangulos_relaciones1.gif} & \includegraphics{imagenes/triangulos_relaciones2.gif}
\end{array}
\]

\bigskip

Observando cada uno de los pares de figuras anteriores, podemos
afirmar que en cada caso el par de figuras tiene las mismas
medidas de sus lados, las mismas medidas de sus �ngulos, una
figura se puede superponer sobre la otra si la trasladamos, la
rotamos o la reflejamos o si efectuamos consecutivamente mas de
una de estas transformaciones. Se dice que estos pares de figuras
son congruentes.

\bigskip

En s�ntesis se dice que: Dos figuras $F$ y $G$ son congruentes si
una cualquiera de ellas resulta de trasladar, rotar o reflejar la
otra.

\bigskip

Se escribe  $F\cong G$.

\bigskip

\[
\includegraphics{imagenes/triangulos_relaciones3.gif}
\]

\bigskip

En la figura se dicen correspondientes los lados $\overline{AB}$ y
${A}^{{\prime }}\overline{B}^{{\prime }}$, $\overline{AC}$ y
${A}^{{\prime }}\overline{C}^{{\prime }}$, $\overline{ BC}$ y
${B}^{{\prime }}\overline{C}^{{\prime }}$; de manera similar los
�ngulos $A$, ${A}^{{\prime }}$, $B$, ${B}^{{\prime }}$, y $C$
${C}^{{\prime }}$ son correspondientes

\bigskip

N�tese que dos segmentos son congruentes s� y solamente si tienen
la misma longitud y dos �ngulos son congruentes s� y solo s�
tienen la misma medida. Si dos figuras son congruentes, cualquier
par de lados o de �ngulos correspondientes son congruentes. Usando
las propiedades anteriores es posible presentar unos criterios que
nos permiten decidir cuando dos tri�ngulos son congruentes. Si
analizamos las construcciones del taller 1 podremos interpretar
los criterios que se presentan a continuaci�n:

\bigskip

\begin{itemize}
\item Si en los tri�ngulos $ABC$ y $DEF$, se tiene que
$\overline{AB}\cong \overline{DE}$, $\overline{AC}\cong
\overline{DF}$ y $\overline{BC}\cong \overline{DF}$ entonces
$\Delta ABC\cong \Delta DEF$.

\bigskip

\[
\includegraphics{imagenes/triangulos_relaciones4.gif}
\]

\bigskip

\item Si en los tri�ngulos $ABC$ y $DEF$, se tiene que
$\overline{AB}\cong \overline{DE}$, $\overline{AC}\cong
\overline{DF}$  y  $\angle A\cong \angle FDE$ entonces $\Delta
ABC\cong \Delta DEF$.

\bigskip

\[
\includegraphics{imagenes/triangulos_relaciones5.gif}
\]

\bigskip

\item Si en los tri�ngulos $ABC$ y $DEF$, se tiene que
$\overline{AB}\cong \overline{DE}$,  $\angle A\cong \angle FDE$ y
$\angle B = \angle FED$ entonces $\Delta ABC\cong \Delta DEF$.

\bigskip

\[
\includegraphics{imagenes/triangulos_relaciones6.gif}
\]

\end{itemize}

\bigskip

\textbf{Ejemplo 3.5. } Usando solamente la informaci�n se�alada,
seleccionar pares de tri�ngulos que sean congruentes. Justificar
cada escogencia con los anteriores criterios

\bigskip

\[
\begin{array}{cc}
\includegraphics{imagenes/triangulos_relaciones7.gif} & \includegraphics{imagenes/triangulos_relaciones8.gif}
\end{array}
\]

\bigskip

\[
\includegraphics{imagenes/triangulos_relaciones9.gif}
\]

\bigskip

\begin{description}
\item[(a)] $\Delta ABC\cong \Delta FED$  porque tienen un par de
lados correspondientes congruente y dos pares de �ngulos
correspondientes congruentes. \item[(b)] No se puede concluir que
sean congruentes porque el �ngulo congruente no est� comprendido
entre el par de lados congruentes. \item[(c)] $\Delta CAD\cong
\Delta ISO$  porque dos pares de lados y el �ngulo que ellos
determinan son congruentes.
\end{description}

%Pie de p�gina
\newline

{\color{gray}
\begin{tabular}{@{\extracolsep{\fill}}lcr}
\hline \\
\docLink{geometria12.tex}{\includegraphics{../../images/navegacion/anterior.gif}}
\begin{tabular}{l}
{\color {darkgray} {\small \'{A}ngulos Tri�ngulo}} \\ \\ \\
\end{tabular} &
\docLink[_top]{../../index.html}{\includegraphics{../../images/navegacion/inicio.gif}}
\docLink{../../docs_curso/contenido.html}{\includegraphics{../../images/navegacion/contenido.gif}}
\docLink{../../docs_curso/descripcion.html}{\includegraphics{../../images/navegacion/descripcion.gif}}
\docLink{../../docs_curso/profesor.html}{\includegraphics{../../images/navegacion/profesor.gif}}
& \begin{tabular}{r}
{\color {darkgray} {\small Tipos Cuadrilateros}} \\ \\ \\
\end{tabular}
\docLink{geometria15.tex}{\includegraphics{../../images/navegacion/siguiente.gif}}
\end{tabular}
}

\end{quote}

\newline

\begin{flushright}
\includegraphics{../../images/interfaz/copyright.gif}
\end{flushright}
\end{document}

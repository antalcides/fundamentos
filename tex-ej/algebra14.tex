\documentclass[10pt]{article} %tipo documento y tipo letra

\definecolor{azulc}{cmyk}{0.72,0.58,0.42,0.20} % color titulo
\definecolor{naranja}{cmyk}{0.21,0.5,1,0.03} % color leccion

\def\eje{\centerline{\textbf{ EJERCICIOS.}}} % definiciones propias

\begin{document}

\begin{quote}

% inicio encabezado
{\color{gray}
\begin{tabular}{@{\extracolsep{\fill}}lcr}
\docLink{algebra13.tex}{\includegraphics{../../images/navegacion/anterior.gif}}
\begin{tabular}{l}
{\color{darkgray}\small Desigualdades Cuadr�ticas} \\ \\ \\
\end{tabular} &
\docLink[_top]{../../index.html}{\includegraphics{../../images/navegacion/inicio.gif}}
\docLink{../../docs_curso/contenido.html}{\includegraphics{../../images/navegacion/contenido.gif}}
\docLink{../../docs_curso/descripcion.html}{\includegraphics{../../images/navegacion/descripcion.gif}}
\docLink{../../docs_curso/profesor.html}{\includegraphics{../../images/navegacion/profesor.gif}}
& \begin{tabular}{r}
{\color{darkgray}\small Cap. 3. Geometr�a} \\ \\ \\
\end{tabular}
\docLink{../cap3/geometria1.tex}{\includegraphics{../../images/navegacion/siguiente.gif}}
\\ \hline
\end{tabular}
}
%fin encabezado

%nombre capitulo
\begin{center}
\colorbox{azulc}{{\color{white} \large CAP�TULO 2}}  {\large
{\color{ azulc} FUNDAMENTOS DE ALGEBRA}}
\end{center}

\newline

%nombre leccion

\colorbox{naranja}{{\color{white} \normalsize  Lecci\'on 2.15. }}
{\normalsize {\color{naranja} Desigualdades con Valor Absoluto}}

\newline

En el cap�tulo 1 definimos el valor absoluto de un n�mero real
$x$, que representamos por $\left| x\right|$, mediante

\bigskip

\[\left| x\right| =
\begin{cases}
x &\text{ \ si \ }x\geq 0 \\
-x &\text{ \ si \ }x<0
\end{cases}
\]

\bigskip

Tambi�n observamos en dicho cap�tulo que $\left| x\right|$
representa la distancia del origen al punto $x$, y de forma mas
general que $\left| x_{1}-x_{2}\right|$ representa la distancia
entre $x_{1}$ y $x_{2}$.

\bigskip

Las propiedades siguientes del valor absoluto nos indican que este
se comporta muy bien con respecto a la multiplicaci�n y la
divisi�n, pero no as� con respecto a la adici�n y la sustracci�n.

\bigskip

\textbf{Propiedades del valor absoluto.} Si $x$ y $y$ son n�meros
reales arbitrarios entonces

\bigskip

\begin{enumerate}
\item $\left| -x\right| =\left| x\right| $ \item $\left| xy\right|
=\left| x\right| \left| y\right| $ \item $\left|
\dfrac{x}{y}\right| =\dfrac{\left| x\right| }{\left| y\right|}$,
$y\neq 0)$ \item $\left| x+y\right| \leq \left| x\right| +\left|
y\right| $ (Desigualdad triangular) \item $\left| x\right| -\left|
y\right| \leq \left| x-y\right| $  y  $\left| y\right| -\left|
x\right| \leq \left|x-y\right| $

\bigskip

La interpretaci�n geom�trica de $\left| x\right| $ nos proporciona
una justificaci�n de las siguientes dos propiedades

\bigskip

Sea $a\geq 0$. Entonces

\bigskip

\item $\left| x\right| \leq a$ es equivalente a $-a\leq x\leq a$
\item $\left| x\right| \geq a$ es equivalente a $x\geq a$ o $x\leq
-a$

\bigskip

Gr�ficamente tenemos

\bigskip

\[
\includegraphics{img/img6.gif}
\]

\bigskip

Otra propiedad del valor absoluto, muy utilizada en la soluci�n de
desigualdades, es la siguiente

\bigskip

\item $\left| x\right| \leq \left| y\right| $ es equivalente a
$x^{2}\leq y^{2}$

\end{enumerate}

\bigskip

En las propiedades (6) a (8) el s�mbolo $\leq $ puede remplazarse
por $<$.

\bigskip

\textbf{Ejemplo 2.49. } Resolvamos la desigualdad $\left|
3-4x\right| \leq 7$.

\bigskip

Utilizando la propiedad (6), tenemos la siguiente cadena de
desigualdades equivalentes:

\bigskip

\begin{gather*}
\left| 3-4x\right| \leq 7\\
-7\leq 3-4x\leq 7\\
-10\leq -4x\leq 4\\
-\dfrac{10}{4}\leq -x\leq 1\\
\dfrac{10}{4}\geq x\geq -1
\end{gather*}

\bigskip

Por lo tanto, la soluci�n de la desigualdad es el intervalo
$\left[-1,\dfrac{10}{4}\right]$.

\bigskip

\textbf{Ejemplo 2.50. } Resolvamos la desigualdad $\left|
5x+14\right|
>10$.

\bigskip

La propiedad (7) nos dice que la desigualdad es equivalente a

\bigskip

\[5x+14>10 \qquad \text{o} \qquad 5x+14<-10\]

\bigskip

Resolviendo

\bigskip

\[5x>-4 \qquad \text{o} \qquad 5x<-24\]

\bigskip

o sea

\bigskip

\[x>-\dfrac{4}{5} \qquad \text{o} \qquad x<-\dfrac{24}{5}\]

\bigskip

Por lo tanto, la soluci�n de la desigualdad dada es
$\left(-\infty,-\dfrac{24}{5}\right)\cup
\left(-\dfrac{4}{5},\infty \right)$

\bigskip

\textbf{Ejemplo 2.51. } Resolvamos la desigualdad $\left|
\dfrac{2x-1}{x+3}\right| \geq 3$.

\bigskip

Utilizando la propiedad (8) del valor absoluto, tenemos la
siguiente cadena de desigualdades equivalentes:

\bigskip

\begin{align*}
\left| \dfrac{2x-1}{x+3}\right| &\geq 3\\
\dfrac{\left| 2x-1\right| }{\left| x+3\right| } &\geq 3\\
\left| 2x-1\right| &\geq 3\left| x+3\right|\\
(2x-1)^{2} &\geq 9(x+3)^{2}\\
(2x-1)^{2}-9(x+3)^{2} &\geq 0\\
[(2x-1)-3(x+3)][(2x-1)+3(x+3)] &\geq 0\\
(-x-10)(5x+8) &\geq 0
\end{align*}

\bigskip

Elaborando un diagrama de signos tenemos

\bigskip

\begin{center}
\begin{tabular}{lccc}
Signo de $(-x-10)$ & + & - & -\\
Signo de $(5x+8)$ & - & - & +\\ \hline Signo de $(-x-10)(5x+8)$ &
- & + & -
\end{tabular}
\end{center}

\bigskip

Vemos que la soluci�n de la desigualdad es
$\left[-10,-\dfrac{8}{5}\right]$.

%Pie de p�gina
\newline

{\color{gray}
\begin{tabular}{@{\extracolsep{\fill}}lcr}
\hline \\
\docLink{algebra13.tex}{\includegraphics{../../images/navegacion/anterior.gif}}
\begin{tabular}{l}
{\color{darkgray}\small Desigualdades Cuadr�ticas} \\ \\ \\
\end{tabular} &
\docLink[_top]{../../index.html}{\includegraphics{../../images/navegacion/inicio.gif}}
\docLink{../../docs_curso/contenido.html}{\includegraphics{../../images/navegacion/contenido.gif}}
\docLink{../../docs_curso/descripcion.html}{\includegraphics{../../images/navegacion/descripcion.gif}}
\docLink{../../docs_curso/profesor.html}{\includegraphics{../../images/navegacion/profesor.gif}}
& \begin{tabular}{r}
{\color{darkgray}\small Cap. 3. Geometr�a} \\ \\ \\
\end{tabular}
\docLink{../cap3/geometria1.tex}{\includegraphics{../../images/navegacion/siguiente.gif}}
\end{tabular}
}

\end{quote}

\newline

\begin{flushright}
\includegraphics{../../images/interfaz/copyright.gif}
\end{flushright}
\end{document}

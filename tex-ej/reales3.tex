\documentclass[10pt]{article} %tipo documento y tipo letra

\definecolor{azulc}{cmyk}{0.72,0.58,0.42,0.20} % color titulo
\definecolor{naranja}{cmyk}{0.21,0.5,1,0.03} % color leccion

\def\eje{\centerline{\textbf{ EJERCICIOS.}}} % definiciones propias

\begin{document}

\begin{quote}

% inicio encabezado
{\color{gray}
\begin{tabular}{@{\extracolsep{\fill}}lcr}
\docLink{reales2.tex}{\includegraphics{../../images/navegacion/anterior.gif}}
\begin{tabular}{l}
{\color{darkgray}\small Adici�n y Multiplicaci�n} \\ \\ \\
\end{tabular} &
\docLink[_top]{../../index.html}{\includegraphics{../../images/navegacion/inicio.gif}}
\docLink{../../docs_curso/contenido.html}{\includegraphics{../../images/navegacion/contenido.gif}}
\docLink{../../docs_curso/descripcion.html}{\includegraphics{../../images/navegacion/descripcion.gif}}
\docLink{../../docs_curso/profesor.html}{\includegraphics{../../images/navegacion/profesor.gif}}
& \begin{tabular}{r}
{\color{darkgray}\small Orden} \\ \\ \\
\end{tabular}
\docLink{reales4.tex}{\includegraphics{../../images/navegacion/siguiente.gif}}
\\ \hline
\end{tabular}
}
%fin encabezado

%nombre capitulo
\begin{center}
\colorbox{azulc}{{\color{white} \large CAP�TULO 1}}  {\large
{\color{ azulc} LOS NUMEROS REALES}}
\end{center}

\newline

%nombre leccion

\colorbox{naranja}{{\color{white} \normalsize  Lecci\'on 1.3. }}
{\normalsize {\color{naranja} Diferencia y Cociente en los N�meros Reales}}

\newline

La diferencia y el cociente de dos n�meros reales se pueden
expresar en t�rminos de la adici�n y la multiplicaci�n de acuerdo
con las siguientes definiciones:

\bigskip

\textbf{Definici�n 1.3.1.} Si $x$ y $y$ son n�meros reales, la
diferencia de $x$ y $y$ es
\[x-y=x+(-y)\]

\bigskip

\textbf{Definici�n 1.3.2.} Si $x$ y $y$ son n�meros reales con
$y\neq 0$, el cociente de $x$ por $y$ es
\[\dfrac{x}{y}=x\dfrac{1}{y}\]

\bigskip

Hacemos notar que como el n�mero cero carece de inverso
multiplicativo, la divisi�n por cero no est� definida.

\bigskip

La importancia de las propiedades b�sicas P.1 a P.6 es que a
partir de ellas se pueden deducir todas las dem�s propiedades
relativas a la adici�n y multiplicaci�n de n�meros reales. A
manera de ejemplo podemos citar algunas de ellas de uso muy
frecuente:

\bigskip

\begin{itemize}

\item\textit{Propiedades cancelativas}:

\bigskip

\begin{itemize}
\item Si $x,y$ y $z$ son n�meros reales tales que $x+y=x+z$,
entonces $y=z$. \item Si $x,y$ y $z$ son n�meros reales tales que
$xy=xz$ y $x\neq 0$ entonces $y=z$.
\end{itemize}

\bigskip

\item \textit{Unicidad de los inversos}:

\bigskip

\begin{itemize}
\item Si $x$ y $y$ son n�meros reales tales que $x+y=0$, entonces
$y=-x$. \item Si $x$ y $y$ son n�meros reales con $x\neq 0$ tales
que $xy=1$, entonces $y=\dfrac{1}{x}$.
\end{itemize}

\bigskip

\item \textit{Reglas de los signos}:

\bigskip

Si $x$ y $y$ son n�meros reales arbitrarios, entonces
\begin{itemize}
\item $(-x)y=-(xy)$, \item $x(-y)=-(xy)$ \item $(-x)(-y)=xy$
\end{itemize}

\bigskip

\item \textit{Regla de los signos para las fracciones}:

\bigskip

Si $x$ y $y$ son n�meros reales con $y\neq 0$, entonces
\[-\dfrac{x}{y}=\dfrac{-x}{y}=\dfrac{x}{-y}\]

\bigskip

\item \textit{Simplificaci�n de fracciones}:

\bigskip

Si $x,y$ y $z$ son n�meros reales con $y\neq 0$ y $z\neq 0$ ,
entonces

\bigskip

\[\dfrac{xz}{yz}=\dfrac{x}{y}\]

\bigskip

\item \textit{Operaciones con fracciones}:

\bigskip

Si $x,y,z$ y $w$ son n�meros reales con $z\neq 0$ y $w\neq 0,$
entonces

\bigskip

\begin{itemize}
\item $\dfrac{x}{z}+\dfrac{y}{w}=\dfrac{xw+yz}{zw}$ \item
$\dfrac{x}{z}\cdot \dfrac{y}{w}=\dfrac{xy}{zw}$
\end{itemize}

\bigskip

Si adem�s, $y\neq 0$,  entonces

\bigskip

\begin{itemize}
\item $\dfrac{\dfrac{x}{z}}{\dfrac{y}{w}}=\dfrac{x}{z}\cdot
\dfrac{w}{y}=\dfrac{xw}{yz}$
\end{itemize}
\end{itemize}

\bigskip

\textbf{Ejemplo 1.2. } En la siguiente lista utilizamos algunas de
las propiedades mencionadas. Siempre suponemos que los
denominadores de las fracciones son diferentes de cero.

\bigskip

\begin{itemize}
\item Si $4x=12$ entonces $x=3$\quad Propiedad cancelativa.

\item $(-8x)(-7y)=56xy$\quad Regla de los signos.

\item $-\dfrac{1-x}{x+2}=\dfrac{x-1}{x+2}$\quad Regla de los
signos para fracciones.

\item $\dfrac{5}{x}+\dfrac{4}{y}=\dfrac{5y+4x}{xy}$\quad Suma de
fracciones.

\item $\dfrac{-2xy}{-5z}=\dfrac{2xy}{5z}$\quad Regla de los
signos.

\item $\dfrac{(x+y)(x-5y)}{(x+y)(3x+y)}=\dfrac{x-5y}{3x+y}$\quad
Simplificaci�n de fracciones.

\item $\dfrac{a-1}{a+1}\cdot
\dfrac{a+2}{a-3}=\dfrac{(a-1)(a+2)}{(a+1)(a-3)}$\quad

\bigskip

Multiplicaci�n de fracciones.

\bigskip

\item $\dfrac{\frac{2}{3}}{\frac{-4}{5}}=\dfrac{2\cdot 5}{3\cdot (-4)}=\dfrac{10}{%
-12}=-\dfrac{10}{12}$\quad Divisi�n de fracciones y regla de los
signos.
\end{itemize}


\bigskip


En el pr�ximo cap�tulo usaremos intensivamente las propiedades
b�sicas de la adici�n y multiplicaci�n de n�meros reales, y las
propiedades que de ellas se deducen.

%Pie de p�gina
\newline

{\color{gray}
\begin{tabular}{@{\extracolsep{\fill}}lcr}
\hline \\
\docLink{reales2.tex}{\includegraphics{../../images/navegacion/anterior.gif}}
\begin{tabular}{l}
{\color{darkgray}\small Adici�n y Multiplicaci�n} \\ \\ \\
\end{tabular} &
\docLink[_top]{../../index.html}{\includegraphics{../../images/navegacion/inicio.gif}}
\docLink{../../docs_curso/contenido.html}{\includegraphics{../../images/navegacion/contenido.gif}}
\docLink{../../docs_curso/descripcion.html}{\includegraphics{../../images/navegacion/descripcion.gif}}
\docLink{../../docs_curso/profesor.html}{\includegraphics{../../images/navegacion/profesor.gif}}
& \begin{tabular}{r}
{\color{darkgray}\small Orden} \\ \\ \\
\end{tabular}
\docLink{reales4.tex}{\includegraphics{../../images/navegacion/siguiente.gif}}
\end{tabular}
}

\end{quote}

\newline

\begin{flushright}
\includegraphics{../../images/interfaz/copyright.gif}
\end{flushright}
\end{document}

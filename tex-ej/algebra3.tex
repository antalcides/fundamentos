\documentclass[10pt]{article} %tipo documento y tipo letra

\definecolor{azulc}{cmyk}{0.72,0.58,0.42,0.20} % color titulo
\definecolor{naranja}{cmyk}{0.21,0.5,1,0.03} % color leccion

\def\eje{\centerline{\textbf{ EJERCICIOS.}}} % definiciones propias

\begin{document}

\begin{quote}

% inicio encabezado
{\color{gray}
\begin{tabular}{@{\extracolsep{\fill}}lcr}
\docLink{algebra2.tex}{\includegraphics{../../images/navegacion/anterior.gif}}
\begin{tabular}{l}
{\color{darkgray}\small Exponentes Positivos} \\ \\ \\
\end{tabular} &
\docLink[_top]{../../index.html}{\includegraphics{../../images/navegacion/inicio.gif}}
\docLink{../../docs_curso/contenido.html}{\includegraphics{../../images/navegacion/contenido.gif}}
\docLink{../../docs_curso/descripcion.html}{\includegraphics{../../images/navegacion/descripcion.gif}}
\docLink{../../docs_curso/profesor.html}{\includegraphics{../../images/navegacion/profesor.gif}}
& \begin{tabular}{r}
{\color{darkgray}\small Exponentes Racionales} \\ \\ \\
\end{tabular}
\docLink{algebra4.tex}{\includegraphics{../../images/navegacion/siguiente.gif}}
\\ \hline
\end{tabular}
}
%fin encabezado

%nombre capitulo
\begin{center}
\colorbox{azulc}{{\color{white} \large CAP�TULO 2}}  {\large
{\color{ azulc} FUNDAMENTOS DE ALGEBRA}}
\end{center}

\newline

%nombre leccion

\colorbox{naranja}{{\color{white} \normalsize  Lecci\'on 2.3. }}
{\normalsize {\color{naranja} Exponentes Enteros}}

\newline

La definici�n de $a^{n}$ se puede extender al caso de exponente
cero y exponentes negativos de acuerdo con la siguiente definici�n

\bigskip

\textbf{Definici�n 2.3.1. } Si $a$ es un n�mero real con $a\neq
0$, definimos

\bigskip

\begin{align*}
a^{0}&=1\\
a^{-n}&=\dfrac{1}{a^{n}}
\end{align*}

\bigskip

donde $n$ es un entero positivo.

\bigskip

\clearpage \textbf{Ejemplo 2.6. } \quad\begin{itemize} \item
$7^{-3}=\dfrac{1}{7^{3}}=\frac{1}{343}$ \item
$(-4)^{-5}=\dfrac{1}{(-4)^{5}}=-\dfrac{1}{1024}$ \item $5^{0}=1$
\item $(2+\sqrt{3})^{0}=1$
\end{itemize}

\bigskip

Estamos en capacidad de demostrar las principales propiedades de
los exponentes, cuando estos son enteros arbitrarios. Estas
propiedades se conocen con el nombre de leyes de los exponentes y
son las siguientes.

\bigskip

\textbf{Leyes de los Exponentes} Si $a,b$ son n�meros reales
diferentes de cero, y $m,n$ son enteros arbitrarios entonces,

\bigskip

\begin{enumerate}
\item $a^{m}a^{n}=a^{m+n}$ \item $(a^{m})^{n}=a^{mn}$ \item
$(ab)^{n}=a^{n}b^{n}$ \item
$\left(\dfrac{a}{b}\right)^{n}=\dfrac{a^{n}}{b^{n}}$ \item
$\dfrac{a^{m}}{a^{n}}=a^{m-n}$ \item
$\dfrac{a^{m}}{a^{n}}=\dfrac{1}{a^{n-m}}$
\end{enumerate}

\bigskip

Generalmente usamos exponentes no negativos, en consecuencia
empleamos la propiedad (5) si $m\geq n$ y la propiedad (6) si
$m<n$. Las propiedades anteriores se extienden de manera natural
al caso en que intervienen varios enteros o varios factores. Por
ejemplo tenemos $\left( abc\right)^{n}=a^{n}b^{n}c^{n}$,
$a^{m}a^{n}a^{p}=a^{m+n+p}$ etc.

\bigskip

\textbf{Ejemplo 2.7. } Si las letras representan n�meros
diferentes de cero tenemos:

\bigskip

\begin{itemize}
\item $x^{-2}x^{5}=x^{-2+5}=x^{3}$ \item
$(y^{-3})^{4}=y^{(-3).4}=y^{-12}=\dfrac{1}{y^{12}}$ \item
$b^{2}b^{3}b^{5}=b^{2+3+5}=b^{10}$ \item
$(x^{2}y^{-3}z^{4})^{2}=(x^{2})^{2}(y^{-3})^{2}(z^{4})^{2}=x^{4}y^{-6}z^{8}$
\item $\dfrac{3^{7}}{3^{5}}=3^{7-5}=3^{2}=9$
\item $\left(\dfrac{5a}{6}\right)^{3}=\dfrac{(5a)^{3}}{6^{3}}=\dfrac{5^{3}a^{3}}{6^{3}}=%
\dfrac{125a^{3}}{216}$
\end{itemize}


%Pie de p�gina
\newline

{\color{gray}
\begin{tabular}{@{\extracolsep{\fill}}lcr}
\hline \\
\docLink{algebra2.tex}{\includegraphics{../../images/navegacion/anterior.gif}}
\begin{tabular}{l}
{\color{darkgray}\small Exponentes Positivos} \\ \\ \\
\end{tabular} &
\docLink[_top]{../../index.html}{\includegraphics{../../images/navegacion/inicio.gif}}
\docLink{../../docs_curso/contenido.html}{\includegraphics{../../images/navegacion/contenido.gif}}
\docLink{../../docs_curso/descripcion.html}{\includegraphics{../../images/navegacion/descripcion.gif}}
\docLink{../../docs_curso/profesor.html}{\includegraphics{../../images/navegacion/profesor.gif}}
& \begin{tabular}{r}
{\color{darkgray}\small Exponentes Racionales} \\ \\ \\
\end{tabular}
\docLink{algebra4.tex}{\includegraphics{../../images/navegacion/siguiente.gif}}
\end{tabular}
}

\end{quote}

\newline

\begin{flushright}
\includegraphics{../../images/interfaz/copyright.gif}
\end{flushright}
\end{document}

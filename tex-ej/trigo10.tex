\documentclass[10pt]{article} %tipo documento y tipo letra

\definecolor{azulc}{cmyk}{0.72,0.58,0.42,0.20} % color titulo
\definecolor{naranja}{cmyk}{0.21,0.5,1,0.03} % color leccion

\def\eje{\centerline{\textbf{ EJERCICIOS.}}} % definiciones propias

\begin{document}

\begin{quote}

% inicio encabezado
{\color{gray}
\begin{tabular}{@{\extracolsep{\fill}}lcr}
\docLink{trigo9.tex}{\includegraphics{../../images/navegacion/anterior.gif}}
\begin{tabular}{l}
{\color{darkgray}\small Expresiones con Seno y Coseno} \\ \\ \\
\end{tabular} &
\docLink[_top]{../../index.html}{\includegraphics{../../images/navegacion/inicio.gif}}
\docLink{../../docs_curso/contenido.html}{\includegraphics{../../images/navegacion/contenido.gif}}
\docLink{../../docs_curso/descripcion.html}{\includegraphics{../../images/navegacion/descripcion.gif}}
\docLink{../../docs_curso/profesor.html}{\includegraphics{../../images/navegacion/profesor.gif}}
& \begin{tabular}{r}
{\color{darkgray}\small Identidades} \\ \\ \\
\end{tabular}
\docLink{trigo11.tex}{\includegraphics{../../images/navegacion/siguiente.gif}}
\\ \hline
\end{tabular}
}
%fin encabezado

%nombre capitulo
\begin{center}
\colorbox{azulc}{{\color{white} \large CAP�TULO 5}}  {\large
{\color{ azulc} NOTAS DE TRIGONOMETRIA}}
\end{center}

\newline

%nombre leccion

\colorbox{naranja}{{\color{white} \normalsize  Lecci\'on 5.9. }}
{\normalsize {\color{naranja} Gr�ficas Sinusoidales}}

\newline

En general: funciones que pueden representarse por:

\bigskip

\[y = A sen\left( Bx+C\right), \quad y = A \cos\left( Bx+C\right) \]

\bigskip

o una combinaci�n de estas puede ser usada para obtener
conocimientos en los mundos f�sico,econ�mico, pol�tico, art�sticos
y seguramente en muchos otros. Veamos matem�ticamente como se
interpreta una de estas funciones.

\bigskip

{\bf Caracter�sticas de la Funci�n:} $y = Asen\left(Bx+C\right)$,
$B> 0$.

\bigskip

\begin{enumerate}
\item La amplitud es el mayor valor que toma la funci�n.

\begin{itemize}
\item Como $-1\leq sen\left( Bx+C\right) \leq 1$. \item Si $A\geq
0$ , $-A\leq Asen\left(Bx+C\right) \leq A$. \item Si $A\leq 0$,
$-A\geq Asen\left( Bx+C\right)\geq A$.
\end{itemize}

\bigskip

Por lo tanto Amplitud de la funci�n es $\left| A\right| $.

\bigskip

\item La funci�n es peri�dica y el periodo es $\dfrac{2\pi }{B}$.

\item La gr�fica, con respecto a la de la funci�n seno est�
desplazada $-\dfrac{C}{B}$ unidades a la derecha o a la izquierda,
seg�n si $C$ es negativo o positivo respectivamente.
\end{enumerate}

\bigskip

\textbf{Ejemplo 5.17. } $y=2sen 3x$

\bigskip

Amplitud:2, \quad Periodo: $\dfrac{2\pi }{3}$

\bigskip

\begin{center}
\includegraphics{img/grafica_trigonometrica1.gif}
\end{center}

\bigskip

\end{description}

\bigskip

\textbf{Ejemplo 5.18. } $y=-3sen\left( 2x-\frac{\pi }{3}\right)$

\bigskip

Amplitud: $\left| -3\right|$, \quad Periodo: $\dfrac{2\pi }{2}$,
\quad Desfase: $\dfrac{\pi }{6}$

\bigskip

\begin{center}
\includegraphics{img/grafica_trigonometrica2.gif}
\end{center}

\bigskip

\end{description}

\bigskip

\textbf{Ejemplo 5.19. } Un Tsunami es una ola de marea ocasionada
por un terremoto bajo el mar. Estas olas pueden medir m�s de 100
pies de altura y pueden viajar a grandes velocidades. A veces los
ingenieros representan estas olas por expresiones trigonom�tricas
de la forma: $y= a\cos\left(bt\right)$ y utilizan estas
representaciones para calcular la efectividad de los muros
rompeolas. Supongamos que una ola en el instante $t=0$ tiene una
altura de $y=25$ pies, viaja a raz�n de 180 pies por segundo con
un periodo de 30 min.

\bigskip

La expresi�n del movimiento de las olas es:

\bigskip

\[y=a\cos\left( bt\right)\]

\bigskip

Como para $t=0$  $y = 25$ pies, entonces:

\bigskip

$25=a\cos \left( b0\right)$, as� $a = 25$ pies.

\bigskip

El periodo es 30 min. por lo tanto: $\dfrac{2\pi
}{b}=30m\acute{\imath}n$.

\bigskip

De donde: $b=\dfrac{\pi }{15}$.

\bigskip

La ecuaci�n es: $y=25\cos \frac{\pi }{15}t$.

\bigskip

Podemos calcular la distancia entre dos crestas consecutivas:

\bigskip

Como recorre 180 pies en un segundo, recorrer� 10800 pies en un
minuto.

\bigskip

La longitud de onda es la distancia entre dos crestas
consecutivas, como el periodo es 30 min., en 30 minutos recorrer�:
$\left( 10800\right) \left( 30\right) =324.000$ pies

%Pie de p�gina
\newline

{\color{gray}
\begin{tabular}{@{\extracolsep{\fill}}lcr}
\hline \\
\docLink{trigo9.tex}{\includegraphics{../../images/navegacion/anterior.gif}}
\begin{tabular}{l}
{\color{darkgray}\small Expresiones con Seno y Coseno} \\ \\ \\
\end{tabular} &
\docLink[_top]{../../index.html}{\includegraphics{../../images/navegacion/inicio.gif}}
\docLink{../../docs_curso/contenido.html}{\includegraphics{../../images/navegacion/contenido.gif}}
\docLink{../../docs_curso/descripcion.html}{\includegraphics{../../images/navegacion/descripcion.gif}}
\docLink{../../docs_curso/profesor.html}{\includegraphics{../../images/navegacion/profesor.gif}}
& \begin{tabular}{r}
{\color{darkgray}\small Identidades} \\ \\ \\
\end{tabular}
\docLink{trigo11.tex}{\includegraphics{../../images/navegacion/siguiente.gif}}
\end{tabular}
}

\end{quote}

\newline

\begin{flushright}
\includegraphics{../../images/interfaz/copyright.gif}
\end{flushright}
\end{document}

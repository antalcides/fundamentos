\documentclass[10pt]{article} %tipo documento y tipo letra

\definecolor{azulc}{cmyk}{0.72,0.58,0.42,0.20} % color titulo
\definecolor{naranja}{cmyk}{0.21,0.5,1,0.03} % color leccion

\def\eje{\centerline{\textbf{ EJERCICIOS.}}} % definiciones propias

\begin{document}

\begin{quote}

% inicio encabezado
{\color{gray}
\begin{tabular}{@{\extracolsep{\fill}}lcr}
\docLink{geometria6.tex}{\includegraphics{../../images/navegacion/anterior.gif}}
\begin{tabular}{l}
{\color {darkgray} {\small Tipos de Tri�ngulos}} \\ \\ \\
\end{tabular} &
\docLink[_top]{../../index.html}{\includegraphics{../../images/navegacion/inicio.gif}}
\docLink{../../docs_curso/contenido.html}{\includegraphics{../../images/navegacion/contenido.gif}}
\docLink{../../docs_curso/descripcion.html}{\includegraphics{../../images/navegacion/descripcion.gif}}
\docLink{../../docs_curso/profesor.html}{\includegraphics{../../images/navegacion/profesor.gif}}
& \begin{tabular}{r}
{\color {darkgray} {\small Medidad de \'{A}ngulos}} \\ \\ \\
\end{tabular}
\docLink{geometria8.tex}{\includegraphics{../../images/navegacion/siguiente.gif}}
\\ \hline
\end{tabular}
}
%fin encabezado

%nombre capitulo
\begin{center}
\colorbox{azulc}{{\color{white} \large CAP�TULO 3}}  {\large
{\color{ azulc} MODULO DE GEOMETR�A}}
\end{center}

\newline

%nombre leccion

\colorbox{naranja}{{\color{white} \normalsize  Lecci\'on 3.7. }}
{\normalsize {\color{naranja} �ngulos y Sus Medidas}}

\newline

\textbf{Definici�n 3.7.1. } Un �ngulo es la uni�n de dos rayos que
tienen el mismo punto inicial. (O la abertura comprendida entre
dos rectas trazadas desde un mismo punto).

\bigskip

\[
\begin{array}{ccc}
\includegraphics{/imagenes/angulo1.gif}
\end{array}
\]

\bigskip

En el pol�gono $ABCD$ el $\angle B$ es la uni�n de $\overline{BA}$
y $\overline{BC}$.

\bigskip

Los lados del �ngulo son los dos rayos que lo forman, el v�rtice
del �ngulo es el punto inicial com�n de los dos rayos. Los �ngulos
pueden ser determinados por segmentos como en el pol�gono, pero se
considera que los lados del �ngulo son rayos (o rectas).

\bigskip

\[
\begin{array}{ccc}
\includegraphics{/imagenes/angulo2.gif}
\end{array}
\]

\bigskip

La magnitud de un �ngulo no depende de la longitud de sus lados,
sino de la abertura o separaci�n que hay entre ellos.

\bigskip

Un �ngulo puede considerarse generado por dos rayos de los cuales
uno permanece fijo y el otro gira alrededor de un punto fijo del
primero.

\bigskip

Si en la figura siguiente se supone que los dos rayos
$\overrightarrow{OA}$ y $\overrightarrow{OB}$ coinciden, el �ngulo
es nulo.

\bigskip

Al moverse el rayo $\overrightarrow{OB}$ alrededor del punto en
sentido contrario al movimiento de las manecillas del reloj,
permaneciendo fijo el rayo $\overrightarrow{OA}$, el �ngulo es
sucesivamente: menor que un recto, igual a un recto (un cuarto de
vuelta), menor que dos rectos, igual a dos rectos (media vuelta),
menor que tres rectos, igual a tres rectos (tres cuartos de
vuelta), menor que cuatro rectos, e igual a cuatro rectos (una
vuelta) cuando el rayo m�vil coincide con el rayo fijo.

\bigskip

\[
\begin{array}{ccc}
\includegraphics{/imagenes/angulo3.gif}
\end{array}
\]

\bigskip

Todo �ngulo, excepto el �ngulo cero, separa el plano en dos
conjuntos y si el �ngulo no es llano, es posible referirse a uno
de estos conjuntos como el interior del �ngulo y al otro como el
exterior del �ngulo.

\bigskip

\[
\begin{array}{ccc}
\includegraphics{/imagenes/angulo4.gif}
\end{array}
\]

%Pie de p�gina
\newline

{\color{gray}
\begin{tabular}{@{\extracolsep{\fill}}lcr}
\hline \\
\docLink{geometria6.tex}{\includegraphics{../../images/navegacion/anterior.gif}}
\begin{tabular}{l}
{\color {darkgray} {\small Tipos de Tri�ngulos}} \\ \\ \\
\end{tabular} &
\docLink[_top]{../../index.html}{\includegraphics{../../images/navegacion/inicio.gif}}
\docLink{../../docs_curso/contenido.html}{\includegraphics{../../images/navegacion/contenido.gif}}
\docLink{../../docs_curso/descripcion.html}{\includegraphics{../../images/navegacion/descripcion.gif}}
\docLink{../../docs_curso/profesor.html}{\includegraphics{../../images/navegacion/profesor.gif}}
& \begin{tabular}{r}
{\color {darkgray} {\small Medidad de \'{A}ngulos}} \\ \\ \\
\end{tabular}
\docLink{geometria8.tex}{\includegraphics{../../images/navegacion/siguiente.gif}}
\end{tabular}
}

\end{quote}

\newline

\begin{flushright}
\includegraphics{../../images/interfaz/copyright.gif}
\end{flushright}
\end{document}

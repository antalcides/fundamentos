\documentclass[10pt]{article} %tipo documento y tipo letra

\definecolor{azulc}{cmyk}{0.72,0.58,0.42,0.20} % color titulo
\definecolor{naranja}{cmyk}{0.50,0.42,0.42,0.06} % color leccion

\def\Reales{\mathbb{R}}
\def\Naturales{\mathbb{N}}
\def\Enteros{\mathbb{Z}}
\def\Racionales{\mathbb{Q}}
\def\Irr{\mathbb{I}}
\def\contradiccion{($\rightarrow \leftarrow$)}


\begin{document}

\begin{quote}

% inicio encabezado
{\color{gray}
\begin{tabular}{@{\extracolsep{\fill}}lcr}
&
\docLink[_top]{../../../index.html}{\includegraphics{../../../images/navegacion/inicio.gif}}
\docLink{../../../docs_curso/contenido.html}{\includegraphics{../../../images/navegacion/contenido.gif}}
\docLink{../../../docs_curso/descripcion.html}{\includegraphics{../../../images/navegacion/descripcion.gif}}
\docLink{../../../docs_curso/profesor.html}{\includegraphics{../../../images/navegacion/profesor.gif}}
& \begin{tabular}{r}
{\color{darkgray}\small Taller 2} \\ \\ \\
\end{tabular}
\docLink{taller2.tex}{\includegraphics{../../../images/navegacion/siguiente.gif}}
\\ \hline
\end{tabular}
}
%fin encabezado

%nombre capitulo
\begin{center}
\colorbox{azulc}{{\color{white} \large TRIGONOMETR\'{I}A}}  {\large
{\color{ azulc} }}
\end{center}

\newline

%nombre leccion

\colorbox{naranja}{{\color{white} \normalsize  TALLER 1 }}
{\normalsize {\color{naranja} }}

\newline

\begin{enumerate}
\item  La rueda delantera del triciclo de Pedro tiene un di�metro
de 10 pulgadas. �Qu� tan lejos llegar� pedaleando 60 revoluciones?

\item  Un reloj tiene el minutero y el horario del mismo tama�o,
miden seis pulgadas y llegan hasta la orilla de la car�tula del
reloj. Encuentre el �rea de la regi�n angular entre las dos
manecillas a las 5:40 a.m.

\item  Indique en grados y en radianes el �ngulo central $\theta$,
en una circunferencia de radio 4 cm., subtendido por el arco $S$
de longitud 7 cm.

\item Halle el �rea del sector determinado por un �ngulo central
de medida $50^{\circ }$ en un c�rculo de di�metro 16 m.

\item  Encuentre el �rea de la regi�n sombreada.

\begin{center}
\includegraphics{imagenes/taller1.gif}
\end{center}

\item Se utiliza una gran polea de 3 pies de di�metro para
levantar cargas. Halle la distancia que la carga es levantada si
la polea gira un �ngulo de $7\dfrac{\pi }{4}$ radianes.

\item  Un guarda bosques que est� a 200 pies de la base de un
�rbol, observa que el �ngulo entre el suelo y la parte superior
del �rbol es de $60^{\circ }$. Calcule la altura del �rbol.

\item Se desea construir una rampa de 24 pies de largo que se
levante a una altura de 5 pies sobre el nivel del suelo. Calcule
el �ngulo de la rampa con la horizontal.

\item  Una escalera que mide 20 pies se apoya en un edificio y el
�ngulo entre ambos es de $22^{\circ }$.
\begin{enumerate}
\item Calcule la distancia del pie de la escalera al piso. \item
Si la distancia del pie de la escalera al edificio aumenta 3 pies,
�aproximadamente cu�nto bajar� del edificio la parte alta de la
escalera?
\end{enumerate}

\item En cada caso halle los valores exactos de la seis funciones
trigonom�tricas:
\begin{enumerate}
\item  El punto de coordenadas $\left(
-\frac{3}{5},-\frac{4}{5}\right)$ pertenece al lado final del
�ngulo. \item  $\left( 30,-40\right)$ es un punto del lado final
del �ngulo. \item  El lado final est� en el tercer cuadrante y
sobre la recta $3y- 4x = 0$.
\end{enumerate}

\item Encuentre los valores de las seis funciones trigonom�tricas
del �ngulo $\theta$, si $sen\theta =\dfrac{3}{5}$, $0<\theta
<\dfrac{\pi }{2}$.

\item  Si $\tan\alpha =2$, $\pi < \alpha < 3\dfrac{\pi }{2}$,
encuentre los valores de $sen\alpha$, $\cos\alpha $.
\end{enumerate}

%Pie de p�gina
\newline

{\color{gray}
\begin{tabular}{@{\extracolsep{\fill}}lcr}
\hline \\
&
\docLink[_top]{../../../index.html}{\includegraphics{../../../images/navegacion/inicio.gif}}
\docLink{../../../docs_curso/contenido.html}{\includegraphics{../../../images/navegacion/contenido.gif}}
\docLink{../../../docs_curso/descripcion.html}{\includegraphics{../../../images/navegacion/descripcion.gif}}
\docLink{../../../docs_curso/profesor.html}{\includegraphics{../../../images/navegacion/profesor.gif}}
& \begin{tabular}{r}
{\color{darkgray}\small Taller 2} \\ \\ \\
\end{tabular}
\docLink{taller2.tex}{\includegraphics{../../../images/navegacion/siguiente.gif}}
\end{tabular}
}

\end{quote}

\newline

\begin{flushright}
\includegraphics{../../../images/interfaz/copyright.gif}
\end{flushright}
\end{document}

\documentclass[10pt]{article} %tipo documento y tipo letra

\definecolor{azulc}{cmyk}{0.72,0.58,0.42,0.20} % color titulo
\definecolor{naranja}{cmyk}{0.21,0.5,1,0.03} % color leccion

\def\eje{\centerline{\textbf{ EJERCICIOS.}}} % definiciones propias

\begin{document}

\begin{quote}

% inicio encabezado
{\color{gray}
\begin{tabular}{@{\extracolsep{\fill}}lcr}
\docLink{algebra92.tex}{\includegraphics{../../images/navegacion/anterior.gif}}
\begin{tabular}{l}
{\color{darkgray}\small Ra�ces} \\ \\ \\
\end{tabular} &
\docLink[_top]{../../index.html}{\includegraphics{../../images/navegacion/inicio.gif}}
\docLink{../../docs_curso/contenido.html}{\includegraphics{../../images/navegacion/contenido.gif}}
\docLink{../../docs_curso/descripcion.html}{\includegraphics{../../images/navegacion/descripcion.gif}}
\docLink{../../docs_curso/profesor.html}{\includegraphics{../../images/navegacion/profesor.gif}}
& \begin{tabular}{r}
{\color{darkgray}\small Fracciones} \\ \\ \\
\end{tabular}
\docLink{algebra11.tex}{\includegraphics{../../images/navegacion/siguiente.gif}}
\\ \hline
\end{tabular}
}
%fin encabezado

%nombre capitulo
\begin{center}
\colorbox{azulc}{{\color{white} \large CAP�TULO 2}}  {\large
{\color{ azulc} FUNDAMENTOS DE ALGEBRA}}
\end{center}

\newline

%nombre leccion

\colorbox{naranja}{{\color{white} \normalsize  Lecci\'on 2.11. }}
{\normalsize {\color{naranja} Productos Notables y Factorizaci�n de Polinomios}}

\newline

Los siguientes productos se utilizan con tanta frecuencia en
�lgebra que no solo merecen destacarse sino que es aconsejable
memorizarlos. Se conocen con el nombre de \textit{productos
notables} y son:

\bigskip

\begin{enumerate}
\item $a(x+y)=ax+ay$ \item $(x+y)(x-y)=x^{2}-y^{2}$ \item
$(x+y)^{2}=x^{2}+2xy+y^{2}$ \item $(x-y)^{2}=x^{2}-2xy+y^{2}$
\item $(x+y)^{3}=x^{3}+3x^{2}y+3xy^{2}+y^{3}$ \item
$(x-y)^{3}=x^{3}-3x^{2}y+3xy^{2}-y^{3}$ \item
$(x+a)(x+b)=x^{2}+(a+b)x+ab$ \item
$(ax+b)(cx+d)=acx^{2}+(ad+bc)x+bd$ \item
$(x-y)(x^{2}+xy+y^{2})=x^{3}-y^{3}$ \item
$(x+y)(x^{2}-xy+y^{2})=x^{3}+y^{3}$
\end{enumerate}

\bigskip

La validez de los productos anteriores se comprueba f�cilmente
realizando las multiplicaciones correspondientes. Las letras que
intervienen en las f�rmulas, pueden reemplazarse por expresiones
algebraicas arbitrarias.

\bigskip

\textbf{Ejemplo 2.29. } \quad \begin{itemize} \item [a)]
$(3\sqrt{x+y}-2\sqrt{x})(3\sqrt{x+y}+2\sqrt{x})=(3\sqrt{x+y)})^{2}-(2\sqrt{x})^{2}=9(x+y)-4x=5x-9y$
\item [b)]
$(3t^{2}+4s)^{2}=(3t^{2})^{2}+2(3t^{2})(4s)+(4s)^{2}=9t^{4}+24t^{2}s+16s^{2}$
\item [c)]
\begin{align*}
(a+b-c)^{3}&=[(a+b)-c]^{3}\\
&=(a+b)^{3}-3(a+b)^{2}c+3(a+b)c^{2}-c^{3}\\
&=(a^{3}+3a^{2}b+3ab^{2}+b^{3})-3(a^{2}+2ab+b^{2})c+3ac^{2}\\
&\quad+3bc^{2}-c^{3}
\end{align*}

\bigskip

Observamos que consideramos $(a+b)$ como un solo t�rmino.

\bigskip

\item [d)]
$(2x+5y)(5x-3y)=10x^{2}+(-6+25)xy-15y^{2}=10x^{2}+19xy-15y^{2}$
\item [e)]
\begin{align*}
(2a-5b)(4a^{2}+10ab+25b^{2})&=(2a-5b)((2a)^{2}+(2a)(5b)+(5b)^{2})\\
&=(2a)^{3}-(5b)^{3}=8a^{3}-25b^{3}
\end{align*}
\end{itemize}

\bigskip

Factorizar una expresi�n algebraica, es expresarla como producto
de expresiones mas simples llamadas \textit{factores} de la
expresi�n original. En general la factorizaci�n de expresiones
algebraicas puede ser muy complicada y nos limitaremos por ahora a
considerar algunos casos sencillos, que se derivan de las f�rmulas
de los productos notables cuando se leen de derecha a izquierda.

\bigskip

\textbf{Ejemplo 2.30. (Factor com�n)} Para factorizar las
siguientes expresiones utilizamos el producto notable (1), donde
la letra $a$ se conoce con el nombre de un factor com�n.

\bigskip

\begin{itemize}
\item [a)] $3x^{3}-6x+9=3(x^{3}-2x+3)$. El factor com�n es $3$.
\item [b)] $(5x-2y)x^{2}-(5x-2y)6xy=(5x-2y)(x^{2}-6xy)$. El factor
com�n es $5x-2y$. \item [c)]
$y^{6}-y^{4}=y^{4}(y^{2}-1)=y^{4}(y-1)(y+1)$.
\end{itemize}

\bigskip

\textbf{Ejemplo 2.31. (Diferencia de cuadrados)} Para factorizar
las siguientes expresiones utilizamos el producto notable (2).

\bigskip

\begin{itemize}
\item [a)]
$16t^{2}-81r^{2}s^{4}=(4t)^{2}-(9rs^{2})^{2}=(4t+9rs^{2})(4t-9rs^{2})$.
\item [b)]
\begin{align*}
\dfrac{9}{z^{4}}-25x^{4}&=\left(\dfrac{3}{z^{2}}\right)^{2}-(5x^{2})^{2}=\left(\dfrac{3}{z^{2}}+5x^{2}\right)\left(\dfrac{3}{z^{2}}-5x^{2}\right)\\
&=\left(\dfrac{3}{z^{2}}+5x^{2}\right)\left(\left(\dfrac{\sqrt{3}}{z}\right)^{2}-(\sqrt{5}x)^{2}\right)\\
&=\left(\dfrac{3}{z^{2}}+5x^{2}\right)\left(\dfrac{\sqrt{3}}{z}+\sqrt{5}x\right)\left(\dfrac{\sqrt{3}}{z}-\sqrt{5}x\right)
\end{align*}
\end{itemize}

\bigskip

\textbf{Ejemplo 2.32. (Suma y diferencia de cubos)} Utilizando los
productos notables (9) y (10) tenemos las siguientes
factorizaciones

\begin{itemize}
\item [a)]
\begin{align*}
27a^{3}+8b^{3}&=(3a)^{3}+(2b)^{3}=(3a+2b)((3a)^{2}-(3a)(2b)+(2b)^{2})\\
&=(3a+2b)(9a^{2}-6ab+4b^{2})
\end{align*}
\item [b)]
$\left(\dfrac{t-s}{4r}\right)^{3}-125y^{3}=\left(\dfrac{t-s}{4r}-5y\right)\left(\left(\dfrac{t-s}{4r}\right)^{2}+\left(\dfrac{t-s}{4r}\right)(5y)+(5y)^{2}\right)$
\end{itemize}

\bigskip

\textbf{Ejemplo 2.33. }[Factorizaci�n de trinomios] Los productos
notables (3), (4), (7) y (8) nos permiten factorizar varias clases
de trinomios, como en los casos siguientes:

\bigskip

\begin{itemize}
\item [a)]
$9x^{2}+12xy+4y^{2}=(3x)^{2}+2(3x)(2y)+(2y)^{2}=(3x+2y)^{2}$ \item
[b)]
$81z^{6}-90z^{3}w^{2}+25w^{4}=(9z^{3})^{2}-2(9z^{3})(5w^{2})+(5w^{2})^{2}=(9z^{3}-5w^{2})^{2}$
\item [c)] Factoricemos el trinomio $x^{2}+3x-4$.

\bigskip

Debemos tener $x^{2}+3x-4=(x+a)(x+b)=x^{2}+(a+b)x+ab$, luego
tenemos que buscar dos n�meros $a$ y $b$ tales que $a+b=3$ y
$ab=-4$. F�cilmente encontramos que $a=4$ y $b=-1$. En
consecuencia

\bigskip

\[x^{2}+3x-4=(x+4)(x-1)\]

\bigskip

\item [d)] Factoricemos el trinomio $6x^{2}+11x-10$

\bigskip

Debemos tener $6x^{2}+11x-10=(ax+b)(cx+d)=acx^{2}+(ad+bc)x+bd$.
Luego tenemos que buscar n�meros $a,b,c$ y $d$ tales que
$ac=6,bd=-10$ y $ad+bc=11$. Observamos que $a$ y $c$ son positivos
y que $b$ y $d$ son de signos opuestos. Por ensayo y error
llegamos a la combinaci�n correcta

\bigskip

\[6x^{2}+11x-10=(2x+5)(3x-2)\]

\bigskip

La factorizaci�n de este trinomio, tambi�n puede efectuarse
reduci�ndolo a un caso similar al de la parte c) que es mas
sencillo. Veamos como se procede.

\bigskip

\begin{align*}
6x^{2}+11x-10&=\dfrac{1}{6}[(6x)^{2}+11(6x)-60]\\
&=\dfrac{1}{6}(u^{2}+11u-60), \qquad \text{donde} \quad u=6x\\
&=\dfrac{1}{6}(u+15)(u-4)\\
&=\dfrac{1}{6}(6x+15)(6x-4)\\
&=\left(\dfrac{6x+15}{3}\right)\left(\dfrac{6x-4}{2}\right)=(2x+5)(3x-2)
\end{align*}

\item [e)] Factoricemos el trinomio $4x^{2}-4xy-3y^{2}$

\bigskip

Procediendo como en el ejemplo anterior tenemos:

\bigskip

\begin{align*}
4x^{2}-4xy-3y^{2}&=\dfrac{1}{4}[(4x)^{2}-4y(4x)-12y^{2}]\\
&=\dfrac{1}{4}(u^{2}-4yu-12y^{2}), \quad \text{donde} \quad u=4x\\
&=\dfrac{1}{4}(u-6y)(u+2y)\\
&=\dfrac{1}{4}(4x-6y)(4x+2y)\\
&=\left(\dfrac{4x-6y}{2}\right)\left(\dfrac{4x+2y}{2}\right)=(2x-3y)(2x+y)
\end{align*}
\end{itemize}

\bigskip

Algunas veces hay necesidad de agrupar o manipular
convenientemente los t�rminos para poder factorizar las
expresiones, como en el ejemplo siguiente.

\bigskip

\textbf{Ejemplo 2.33. } \quad \begin{itemize} \item [a)]
\begin{align*}
4x^{3}+4x^{2}-9x-9&=(4x^{3}+4x^{2})-(9x+9)=4x^{2}(x+1)-9(x+1)\\
&=(4x^{2}-9)(x+1)=(2x-3)(2x+3)(x+1)
\end{align*}
\item [b)]
\begin{align*}
a^{3}-a^{2}b-ab^{2}+b^{3}&=(a^{3}-a^{2}b)-(ab^{2}-b^{3})=a^{2}(a-b)-b^{2}(a-b)\\
&=(a^{2}-b^{2})(a-b)=(a-b)(a+b)(a-b)\\
&=(a+b)(a-b)^{2}
\end{align*}
\item [c)]
\begin{align*}
y^{4}+4&=y^{4}+4y^{2}-4y^{2}+4=(y^{4}+4y^{2}+4)-4y^{2}\\
&=(y^{2}+2)^{2}-4y^{2}=(y^{2}+2-2y)(y^{2}+2+2y)
\end{align*}
\end{itemize}

%Pie de p�gina
\newline

{\color{gray}
\begin{tabular}{@{\extracolsep{\fill}}lcr}
\hline \\
\docLink{algebra92.tex}{\includegraphics{../../images/navegacion/anterior.gif}}
\begin{tabular}{l}
{\color{darkgray}\small Ra�ces} \\ \\ \\
\end{tabular} &
\docLink[_top]{../../index.html}{\includegraphics{../../images/navegacion/inicio.gif}}
\docLink{../../docs_curso/contenido.html}{\includegraphics{../../images/navegacion/contenido.gif}}
\docLink{../../docs_curso/descripcion.html}{\includegraphics{../../images/navegacion/descripcion.gif}}
\docLink{../../docs_curso/profesor.html}{\includegraphics{../../images/navegacion/profesor.gif}}
& \begin{tabular}{r}
{\color{darkgray}\small Fracciones} \\ \\ \\
\end{tabular}
\docLink{algebra11.tex}{\includegraphics{../../images/navegacion/siguiente.gif}}
\end{tabular}
}

\end{quote}

\newline

\begin{flushright}
\includegraphics{../../images/interfaz/copyright.gif}
\end{flushright}
\end{document}

\documentclass[10pt]{article} %tipo documento y tipo letra

\definecolor{azulc}{cmyk}{0.72,0.58,0.42,0.20} % color titulo
\definecolor{naranja}{cmyk}{0.21,0.5,1,0.03} % color leccion

\def\eje{\centerline{\textbf{ EJERCICIOS.}}} % definiciones propias

\begin{document}

\begin{quote}

% inicio encabezado
{\color{gray}
\begin{tabular}{@{\extracolsep{\fill}}lcr}
\docLink{geometria11.tex}{\includegraphics{../../images/navegacion/anterior.gif}}
\begin{tabular}{l}
{\color {darkgray} {\small Rectas Perpendiculares}} \\ \\ \\
\end{tabular} &
\docLink[_top]{../../index.html}{\includegraphics{../../images/navegacion/inicio.gif}}
\docLink{../../docs_curso/contenido.html}{\includegraphics{../../images/navegacion/contenido.gif}}
\docLink{../../docs_curso/descripcion.html}{\includegraphics{../../images/navegacion/descripcion.gif}}
\docLink{../../docs_curso/profesor.html}{\includegraphics{../../images/navegacion/profesor.gif}}
& \begin{tabular}{r}
{\color {darkgray} {\small Tri�ngulos - Relaciones}} \\ \\ \\
\end{tabular}
\docLink{geometria13.tex}{\includegraphics{../../images/navegacion/siguiente.gif}}
\\ \hline
\end{tabular}
}
%fin encabezado

%nombre capitulo
\begin{center}
\colorbox{azulc}{{\color{white} \large CAP�TULO 3}}  {\large
{\color{ azulc} MODULO DE GEOMETR�A}}
\end{center}

\newline

%nombre leccion

\colorbox{naranja}{{\color{white} \normalsize  Lecci\'on 3.11. }}
{\normalsize {\color{naranja} Los \'{A}ngulos en un Pol�gono}}

\newline

\colorbox{naranja}{{\color{white} \normalsize  Lecci\'on 3.11.1.
}} {\normalsize {\color{naranja} Los �ngulos de un tri�ngulo}}

\newline

\textbf{Teorema 3.11.1. } La suma de las medidas de los �ngulos de
un tri�ngulo es $180^o$.

\bigskip

\textit{Demostraci�n.} Dado el tri�ngulo $ABC$,

\bigskip

\[
\begin{array}{ccc}
\includegraphics{imagenes/trianguloabc.gif}
\end{array}
\]

\bigskip

debemos probar que $m\angle A + m\angle B + m\angle C =
{180}^{{o}}$. Dibujamos $\overline{BD}$, con $\overline{BD}
\parallel \overline{AC}$. Existe solamente una recta que cumple
esta condici�n. Se tiene por  adici�n de �ngulos que:  $m\angle 1
+ m\angle 2 + m\angle 3 = 180$. Dado que las rectas son paralelas,
por propiedad de �ngulos alternos internos entonces $m\angle 1=
m\angle A$ y $m\angle 3 = m\angle C$, pero adem�s $\angle 2=
\angle B$, sustituyendo se obtiene que $m\angle A + m\angle B +
m\angle C = 180$.

\bigskip

\textbf{Ejemplo 3.4. } Si en el tri�ngulo $ABC$ las medidas de los
�ngulos est�n en raz�n 1:2:3. Determinar las medidas de los
�ngulos.

\bigskip

\[
\begin{array}{ccc}
\includegraphics{imagenes/trianguloabc1.gif}
\end{array}
\]

\bigskip

\emph{Soluci�n}. Que las medidas de los �ngulos est�n en raz�n
1:2:3, significa que para alg�n n�mero real $x$, si $x$ es la
medida de uno de los �ngulos en grados, las otras medidas son $2x$
y $3x$. Si aplicamos el teorema tenemos:

\bigskip

\[m\angle A + m\angle B + m\angle C = {180}^{{o}}\]

\bigskip

entonces,

\bigskip

\[x + 2x + 3x = 180\]

\bigskip

De donde

\bigskip

\[6x = 180,\qquad  x = 30\]

\bigskip

Los �ngulos del tri�ngulo tienen en consecuencia medidas 30, 60 y
90 grados.

\bigskip

\colorbox{naranja}{{\color{white} \normalsize Lecci\'on 3.11.2. }}
{\normalsize {\color{naranja} Suma de las medidas de los �ngulos
de un cuadril�tero}}

\newline

Conociendo la suma de las medidas de los �ngulos internos de un
tri�ngulo, podemos determinar la suma de las medidas de los
�ngulos de cualquier pol�gono convexo. Observemos el cuadril�tero
de la figura

\bigskip

\[
\begin{array}{ccc}
\includegraphics{imagenes/cuadrilatero1.gif}
\end{array}
\]

\bigskip

La suma de las medidas de los �ngulos del cuadril�tero QUAD es:

\bigskip

\[S = m \angle U + m \angle A +  m \angle D + m \angle Q.\]

\bigskip

Si dibujamos el segmento $\overline{AQ}$  se determinan dos
tri�ngulos y entonces

\bigskip

\[S = m \angle U + (m \angle 1 + m \angle 2) + m \angle D + (m \angle 3 + m \angle 4) = 180 + 180 =
360\]

\bigskip

En conclusi�n: La suma de las medidas de los �ngulos internos de
un cuadril�tero convexo es 360 grados.
%Pie de p�gina
\newline

{\color{gray}
\begin{tabular}{@{\extracolsep{\fill}}lcr}
\hline \\
\docLink{geometria11.tex}{\includegraphics{../../images/navegacion/anterior.gif}}
\begin{tabular}{l}
{\color {darkgray} {\small Rectas Perpendiculares}} \\ \\ \\
\end{tabular} &
\docLink[_top]{../../index.html}{\includegraphics{../../images/navegacion/inicio.gif}}
\docLink{../../docs_curso/contenido.html}{\includegraphics{../../images/navegacion/contenido.gif}}
\docLink{../../docs_curso/descripcion.html}{\includegraphics{../../images/navegacion/descripcion.gif}}
\docLink{../../docs_curso/profesor.html}{\includegraphics{../../images/navegacion/profesor.gif}}
& \begin{tabular}{r}
{\color {darkgray} {\small Tri�ngulos - Relaciones}} \\ \\ \\
\end{tabular}
\docLink{geometria13.tex}{\includegraphics{../../images/navegacion/siguiente.gif}}
\end{tabular}
}

\end{quote}

\newline

\begin{flushright}
\includegraphics{../../images/interfaz/copyright.gif}
\end{flushright}
\end{document}

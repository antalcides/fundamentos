\documentclass[10pt]{article} %tipo documento y tipo letra

\definecolor{azulc}{cmyk}{0.72,0.58,0.42,0.20} % color titulo
\definecolor{naranja}{cmyk}{0.21,0.5,1,0.03} % color leccion

\def\eje{\centerline{\textbf{ EJERCICIOS.}}} % definiciones propias

\begin{document}

\begin{quote}

% inicio encabezado
{\color{gray}
\begin{tabular}{@{\extracolsep{\fill}}lcr}
\docLink{reales11.tex}{\includegraphics{../../images/navegacion/anterior.gif}}
\begin{tabular}{l}
{\color{darkgray}\small N�meros Complejos} \\ \\ \\
\end{tabular} &
\docLink[_top]{../../index.html}{\includegraphics{../../images/navegacion/inicio.gif}}
\docLink{../../docs_curso/contenido.html}{\includegraphics{../../images/navegacion/contenido.gif}}
\docLink{../../docs_curso/descripcion.html}{\includegraphics{../../images/navegacion/descripcion.gif}}
\docLink{../../docs_curso/profesor.html}{\includegraphics{../../images/navegacion/profesor.gif}}
& \begin{tabular}{r}
{\color{darkgray}\small Cap. 2. Algebra} \\ \\ \\
\end{tabular}
\docLink{../cap2/algebra1.tex}{\includegraphics{../../images/navegacion/siguiente.gif}}
\\ \hline
\end{tabular}
}
%fin encabezado

%nombre capitulo
\begin{center}
\colorbox{azulc}{{\color{white} \large CAP�TULO 1}}  {\large
{\color{ azulc} LOS NUMEROS REALES}}
\end{center}

\newline

%nombre leccion

\colorbox{naranja}{{\color{white} \normalsize  Lecci\'on 1.12. }}
{\normalsize {\color{naranja} Ra�ces Cuadradas de N�meros Reales
Negativos}}

\newline

El n�mero complejo $i$ es $\sqrt{-1}$. �Qu� podemos decir acerca
de las ra�ces cuadradas de los n�meros reales negativos en
general?

\bigskip

Si $a$ es un real negativo entonces $-a$ es un real positivo y
$\sqrt{-a}$ (la ra�z cuadrada positiva de $-a$), es un n�mero
real. Tenemos

\bigskip

\[(\sqrt{-a}i)^2=(\sqrt{-a})^2 i^2=a\text{\quad y\quad} (-\sqrt{-a}i)^2=a\]

\bigskip

As� que $\sqrt{-a}i$ y $-\sqrt{-a}i$ son las ra�ces cuadradas de
$a$ para $a<0$.

\bigskip

La \textbf{ra�z cuadrada principal} de $a$ es $\sqrt{-a}i$. Es
denotada por $\sqrt{a}$.

\bigskip

\textbf{Ejemplo 1.29. }
\begin{enumerate}
\item $\sqrt{-4}=\sqrt{4}i=2i$\\
$\sqrt{-7}=\sqrt{7}i$

\item Si $b$ y $c$ son n�meros reales positivos
$\sqrt{b}\sqrt{c}=\sqrt{bc}$. Sin embargo, en el caso de los
reales negativos la igualdad \textbf{no es v�lida}. En efecto,

\bigskip

$\sqrt{-4}\sqrt{-4}=(2i)(2i)=-4$ mientras que
$\sqrt{(-4)(-4)}=\sqrt{16}=4$\\
$\sqrt{-4}\sqrt{-7}=(2i)\left(\sqrt{7}i\right)=-2\sqrt{7}$
mientras que $\sqrt{(-4)(-7)}=\sqrt{28}=2\sqrt{7}$

\item
$\left(\sqrt{18}+\sqrt{-9}\right)\left(\sqrt{2}-\sqrt{-6}\right)=\left(\sqrt{18}-\sqrt{9}i\right)\left(\sqrt{2}-\sqrt{6}i\right)$\\
$=\left(\sqrt{18}\sqrt{2}-\sqrt{9}\sqrt{6}\right)+\left(-\sqrt{18}\sqrt{6}-\sqrt{9}\sqrt{2}\right)i$\\
$=\left(6-3\sqrt{6}\right)-\left(6\sqrt{3}+3\sqrt{2}\right)i$
\end{enumerate}

%Pie de p�gina
\newline

{\color{gray}
\begin{tabular}{@{\extracolsep{\fill}}lcr}
\hline \\
\docLink{reales11.tex}{\includegraphics{../../images/navegacion/anterior.gif}}
\begin{tabular}{l}
{\color{darkgray}\small N�meros Complejos} \\ \\ \\
\end{tabular} &
\docLink[_top]{../../index.html}{\includegraphics{../../images/navegacion/inicio.gif}}
\docLink{../../docs_curso/contenido.html}{\includegraphics{../../images/navegacion/contenido.gif}}
\docLink{../../docs_curso/descripcion.html}{\includegraphics{../../images/navegacion/descripcion.gif}}
\docLink{../../docs_curso/profesor.html}{\includegraphics{../../images/navegacion/profesor.gif}}
& \begin{tabular}{r}
{\color{darkgray}\small Cap. 2. Algebra} \\ \\ \\
\end{tabular}
\docLink{../cap2/algebra1.tex}{\includegraphics{../../images/navegacion/siguiente.gif}}
\end{tabular}
}

\end{quote}

\newline

\begin{flushright}
\includegraphics{../../images/interfaz/copyright.gif}
\end{flushright}
\end{document}

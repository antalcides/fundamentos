\documentclass[10pt]{article} %tipo documento y tipo letra

\definecolor{azulc}{cmyk}{0.72,0.58,0.42,0.20} % color titulo
\definecolor{naranja}{cmyk}{0.21,0.5,1,0.03} % color leccion

\def\eje{\centerline{\textbf{ EJERCICIOS.}}} % definiciones propias

\begin{document}

\begin{quote}

% inicio encabezado
{\color{gray}
\begin{tabular}{@{\extracolsep{\fill}}lcr}
\docLink{04_06.tex}{\includegraphics{../../images/navegacion/anterior.gif}}
\begin{tabular}{l}
{\color{darkgray}\small Rectas Paralelas} \\ \\ \\
\end{tabular} &
\docLink[_top]{../../index.html}{\includegraphics{../../images/navegacion/inicio.gif}}
\docLink{../../docs_curso/contenido.html}{\includegraphics{../../images/navegacion/contenido.gif}}
\docLink{../../docs_curso/descripcion.html}{\includegraphics{../../images/navegacion/descripcion.gif}}
\docLink{../../docs_curso/profesor.html}{\includegraphics{../../images/navegacion/profesor.gif}}
& \begin{tabular}{r}
{\color{darkgray}\small Exponenciales} \\ \\ \\
\end{tabular}
\docLink{04_08.tex}{\includegraphics{../../images/navegacion/siguiente.gif}}
\\ \hline
\end{tabular}
}
%fin encabezado

%nombre capitulo
\begin{center}
\colorbox{azulc}{{\color{white} \large CAP�TULO 4}}  {\large
{\color{ azulc} FUNCIONES}}
\end{center}

\newline

%nombre leccion

\colorbox{naranja}{{\color{white} \normalsize  Lecci\'on 4.7. }}
{\normalsize {\color{naranja} Funciones Cuadr�ticas}}

\newline

Recordemos que el �rea de un cuadrado de lado $l$ est� dada por

\bigskip

\[A_{1}\left( l\right) =l^{2}\]

\bigskip

y el �rea de un c�rculo de radio $r$ est� dada por

\bigskip

\[A_{2}\left( r\right) =\pi r^{2}.\]

\bigskip

De otra parte, cuando se lanza una pelota hacia arriba con una
velocidad inicial $v_{0}$, la altura $h$ que alcanza $t$ segundos
despu�s de lanzada es $h\left( t\right)=-\frac{gt^{2}}{2}+v_{0}t$,
donde $g$ es la aceleraci�n debida a la fuerza de gravedad.

\bigskip

\textbf{Definici�n 4.7.1. } Una funci�n $f$ es una {\bf funci�n
cuadr�tica} si la imagen de la {\bf variable independiente} $x$ se
expresa en la forma

\bigskip

\[f(x)=ax^{2}+bx+c,\]

\bigskip

para algunas constantes $a$, $b$ y $c$ con $a\neq 0$.

\bigskip

\textbf{Ejemplo 4.10. } \quad\begin{enumerate} \item Si
$D=\mathbf{R}$ y $g(x)=x^{2}$ para todo $x\in D$, la gr�fica de
$g$ es una par�bola que abre hacia arriba.

\bigskip

\begin{center}
\begin{tabular}{ccc}
\includegraphics{imagenes/4_25.gif} &
\includegraphics{imagenes/4_26.gif}  \\
$y=x^{2}$ & $y=x^{2}+3x+1$
\end{tabular}
\end{center}

\bigskip

\item La gr�fica de $f(x)=x^{2}+3x+1$ es una par�bola.

\bigskip

La gr�fica de toda funci�n cuadr�tica $f(x)=ax^{2}+bx+c$, $a\neq
0$ es una par�bola que se abre hacia arriba si $a > 0$ o se abre
hacia abajo si $a < 0$. El v�rtice tiene abscisa $\frac{-b}{2a}$.
En particular, si $f(x)=x^{2}$ la gr�fica de $f$, como ya vimos,
es una par�bola que pasa por el punto $(0,0)$ y se abre hacia
arriba.

\bigskip

De otra parte, la gr�fica de la ecuaci\'{o}n $y=-x^{2}+1$ es una
par�bola que abre hacia abajo. El v�rtice tiene abscisa
$\frac{-b}{2a}=0$ y ordenada $f(0)=1$. El punto $\left(0,1\right)$
es el punto de corte de la gr�fica y el eje $Y$. Los puntos de
corte con el eje $X$ tienen ordenada $0$ y su abscisa verifica
entonces la ecuaci�n $0=-x^{2}+1$. En consecuencia, $x=\pm 1$ y
los puntos son $\left(-1,0\right)$ y $\left( 1,0\right)$.

\bigskip

\begin{center}
\includegraphics{imagenes/4_27.gif}
\end{center}

\end{description}

\bigskip

\textbf{Ejemplo 4.11. } La aceleraci�n debida a la gravedad es de
$g=32.2pies/s$, una pelota lanzada a una velocidad $v_{0}$ alcanza
la m�xima altura en $t=\frac{v_{0}}{g}$.

%Pie de p�gina
\newline

{\color{gray}
\begin{tabular}{@{\extracolsep{\fill}}lcr}
\hline \\
\docLink{04_06.tex}{\includegraphics{../../images/navegacion/anterior.gif}}
\begin{tabular}{l}
{\color{darkgray}\small Rectas Paralelas} \\ \\ \\
\end{tabular} &
\docLink[_top]{../../index.html}{\includegraphics{../../images/navegacion/inicio.gif}}
\docLink{../../docs_curso/contenido.html}{\includegraphics{../../images/navegacion/contenido.gif}}
\docLink{../../docs_curso/descripcion.html}{\includegraphics{../../images/navegacion/descripcion.gif}}
\docLink{../../docs_curso/profesor.html}{\includegraphics{../../images/navegacion/profesor.gif}}
& \begin{tabular}{r}
{\color{darkgray}\small Exponenciales} \\ \\ \\
\end{tabular}
\docLink{04_08.tex}{\includegraphics{../../images/navegacion/siguiente.gif}}
\end{tabular}
}

\end{quote}

\newline

\begin{flushright}
\includegraphics{../../images/interfaz/copyright.gif}
\end{flushright}
\end{document}

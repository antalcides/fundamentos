\documentclass[10pt]{article} %tipo documento y tipo letra

\definecolor{azulc}{cmyk}{0.72,0.58,0.42,0.20} % color titulo
\definecolor{naranja}{cmyk}{0.21,0.5,1,0.03} % color leccion

\def\eje{\centerline{\textbf{ EJERCICIOS.}}} % definiciones propias

\begin{document}

\begin{quote}

% inicio encabezado
{\color{gray}
\begin{tabular}{@{\extracolsep{\fill}}lcr}
\docLink{algebra1.tex}{\includegraphics{../../images/navegacion/anterior.gif}}
\begin{tabular}{l}
{\color{darkgray}\small Expresiones Algebraicas} \\ \\ \\
\end{tabular} &
\docLink[_top]{../../index.html}{\includegraphics{../../images/navegacion/inicio.gif}}
\docLink{../../docs_curso/contenido.html}{\includegraphics{../../images/navegacion/contenido.gif}}
\docLink{../../docs_curso/descripcion.html}{\includegraphics{../../images/navegacion/descripcion.gif}}
\docLink{../../docs_curso/profesor.html}{\includegraphics{../../images/navegacion/profesor.gif}}
& \begin{tabular}{r}
{\color{darkgray}\small Exponentes Enteros} \\ \\ \\
\end{tabular}
\docLink{algebra3.tex}{\includegraphics{../../images/navegacion/siguiente.gif}}
\\ \hline
\end{tabular}
}
%fin encabezado

%nombre capitulo
\begin{center}
\colorbox{azulc}{{\color{white} \large CAP�TULO 2}}  {\large
{\color{ azulc} FUNDAMENTOS DE ALGEBRA}}
\end{center}

\newline

%nombre leccion

\colorbox{naranja}{{\color{white} \normalsize  Lecci\'on 2.2. }}
{\normalsize {\color{naranja} Exponentes Enteros Positivos}}

\newline

En nuestros ejemplos de expresiones algebraicas, hemos utilizado
s�mbolos tales como $x^{3}$ y $a^{3}$. Un s�mbolo como $x^{2}$
representa el producto $x\cdot x$, similarmente $a^{3}$ representa
el producto $a\cdot a\cdot a$. En general establecemos la
siguiente definici�n.

\bigskip

\textbf{Definici�n 2.2.1. } Si $a$ es un n�mero real arbitrario y
$n$ es un entero positivo, definimos $a^{n}$ como el producto de
$n$ factores iguales a $a$. Es decir

\bigskip

\[a^{n}=\underbrace{a\cdot a\cdot a\cdots a}_{\text{$n$ factores de $a$}}\]

\bigskip

En el s�mbolo $a^{n}$, a se llama la base y $n$ el exponente o la
potencia. Tambi�n decimos que $a^{n}$ es $a$ elevado a la potencia
$n$.

\bigskip

\textbf{Ejemplo 2.4. } \quad\begin{itemize} \item $2^{4}=2\cdot
2\cdot 2\cdot 2=16$ \item $(-3)^{5}=(-3)\cdot (-3)\cdot (-3)\cdot
(-3)\cdot (-3)=-243$ \item
$\left(\dfrac{1}{4}\right)^{3}=\left(\dfrac{1}{4}\right)\cdot
\left(\dfrac{1}{4}\right)\cdot \left(\dfrac{1}{4}\right)=
\dfrac{1}{64}$ \item $(\sqrt{2})^{6}=(\sqrt{2})\cdot
(\sqrt{2})\cdot (\sqrt{2})\cdot (\sqrt{2})\cdot (\sqrt{2})\cdot
(\sqrt{2})=8$
\end{itemize}

\bigskip

A partir de la definici�n 2.2.1 podemos comprobar las siguientes
propiedades b�sicas que satisfacen los exponentes enteros
positivos. Si $a$ y $b$ son n�meros reales arbitrarios y $m$ y $n$
son enteros positivos, entonces tenemos:

\bigskip

\begin{align*}
a^{m}a^{n}&=a^{m+n}\\
(a^{m})^{n}&=a^{mn}\\
(ab)^{n}&=a^{n}b^{n}
\end{align*}

\bigskip

El siguiente razonamiento es una justificaci�n de la primera de
las propiedades mencionadas.

\bigskip

\[a^{m}a^{n}= \underbrace{a\cdot a\cdot a\cdots a}_{\text{$m$ factores de
$a$}}\cdot \underbrace{a\cdot a\cdot a\cdots a}_{\text{$n$
factores de $a$}}= \underbrace{a\cdot a\cdot a\cdots
a}_{\text{$m+n$ factores de $a$}}\]

\bigskip

Se pueden hacer justificaciones similares para comprobar las otras
dos propiedades.

\bigskip

\textbf{Ejemplo 2.5. } \quad \begin{itemize} \item $5^{2}\cdot
5^{3}=5^{2+3}=5^{5}=625$ \item $(2^{2})^{4}=2^{2\cdot
4}=2^{8}=256$ \item $\left( 3\cdot 5\right) ^{2}=3^{2}\cdot
5^{2}=9\cdot 25=225$ \item $(ab)^{3}=a^{3}b^{3}$
\end{itemize}

%Pie de p�gina
\newline

{\color{gray}
\begin{tabular}{@{\extracolsep{\fill}}lcr}
\hline \\
\docLink{algebra1.tex}{\includegraphics{../../images/navegacion/anterior.gif}}
\begin{tabular}{l}
{\color{darkgray}\small Expresiones Algebraicas} \\ \\ \\
\end{tabular} &
\docLink[_top]{../../index.html}{\includegraphics{../../images/navegacion/inicio.gif}}
\docLink{../../docs_curso/contenido.html}{\includegraphics{../../images/navegacion/contenido.gif}}
\docLink{../../docs_curso/descripcion.html}{\includegraphics{../../images/navegacion/descripcion.gif}}
\docLink{../../docs_curso/profesor.html}{\includegraphics{../../images/navegacion/profesor.gif}}
& \begin{tabular}{r}
{\color{darkgray}\small Exponentes Enteros} \\ \\ \\
\end{tabular}
\docLink{algebra3.tex}{\includegraphics{../../images/navegacion/siguiente.gif}}
\end{tabular}
}

\end{quote}

\newline

\begin{flushright}
\includegraphics{../../images/interfaz/copyright.gif}
\end{flushright}
\end{document}

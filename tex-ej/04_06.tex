\documentclass[10pt]{article} %tipo documento y tipo letra

\definecolor{azulc}{cmyk}{0.72,0.58,0.42,0.20} % color titulo
\definecolor{naranja}{cmyk}{0.21,0.5,1,0.03} % color leccion

\def\eje{\centerline{\textbf{ EJERCICIOS.}}} % definiciones propias

\begin{document}

\begin{quote}

% inicio encabezado
{\color{gray}
\begin{tabular}{@{\extracolsep{\fill}}lcr}
\docLink{04_05.tex}{\includegraphics{../../images/navegacion/anterior.gif}}
\begin{tabular}{l}
{\color{darkgray}\small Funciones Lineales} \\ \\ \\
\end{tabular} &
\docLink[_top]{../../index.html}{\includegraphics{../../images/navegacion/inicio.gif}}
\docLink{../../docs_curso/contenido.html}{\includegraphics{../../images/navegacion/contenido.gif}}
\docLink{../../docs_curso/descripcion.html}{\includegraphics{../../images/navegacion/descripcion.gif}}
\docLink{../../docs_curso/profesor.html}{\includegraphics{../../images/navegacion/profesor.gif}}
& \begin{tabular}{r}
{\color{darkgray}\small Funciones Cuadr�ticas} \\ \\ \\
\end{tabular}
\docLink{04_07.tex}{\includegraphics{../../images/navegacion/siguiente.gif}}
\\ \hline
\end{tabular}
}
%fin encabezado

%nombre capitulo
\begin{center}
\colorbox{azulc}{{\color{white} \large CAP�TULO 4}}  {\large
{\color{ azulc} FUNCIONES}}
\end{center}

\newline

%nombre leccion

\colorbox{naranja}{{\color{white} \normalsize  Lecci\'on 4.6. }}
{\normalsize {\color{naranja} Rectas Paralelas y Rectas Perpendiculares}}

\newline

La posici�n relativa de dos rectas en el plano cartesiano puede
expresarse mediante las pendientes. As� tenemos que dos rectas son
paralelas si y solo si tienen la misma pendiente.

\bigskip

\textbf{Teorema 4.6.1. } Dos rectas de pendientes $m_{1}$y $m_{2}$
son perpendiculares si y solo si $m_{1}m_{2}=-1$. Cualquier recta
vertical es perpendicular a cualquier recta horizontal.

\bigskip

\textit{Demostraci�n. } Consideremos dos rectas que se cortan en
un punto. Como cada una de ellas es paralela a una recta que pasa
por el origen $O$, basta considerar el caso en que el punto de
corte coincide con $O$.

\bigskip

\begin{center}
\begin{tabular}{ccc}
\includegraphics{imagenes/4_24.gif}  \\
\end{tabular}
\end{center}

\bigskip

Sean $y=m_{1}x$ y $y=m_{2}x$ las ecuaciones de las rectas. Sean
adem�s $x_{1}$ y $x_{2}$ reales diferentes de $0$. El punto
$P\left( x_{1},m_{1}x_{1}\right)$ pertenece a la primera recta y
el punto $Q\left( x_{2},m_{2}x_{2}\right)$ pertenece a la segunda.

\bigskip

$QOP$ es un �ngulo recto si y solo si

\bigskip

\[\left( d\left( O,P\right) \right)^{2}+\left( d\left( O,Q\right) \right)^{2}=\left( d\left( P,Q\right) \right)^{2}\]

\bigskip

esto es

\bigskip

\[x_{1}^{2}+\left( m_{1}x_{1}\right)^{2}+x_{2}^{2}+\left( m_{2}x_{2}\right)^{2}=\left( x_{2}-x_{1}\right)^{2}+\left(m_{2}x_{2}-m_{1}x_{1}\right)^{2}\]
\[0=-2x_{1}x_{2}-2m_{1}x_{1}m_{2}x_{2}=-2x_{1}x_{2}\left( 1+m_{1}m_{2}\right)\]

\bigskip

Puesto que $x_{1}x_{2}\neq 0$, esto equivale a

\bigskip

\[-1=m_{1}m_{2}.\]

\bigskip

\textbf{Ejemplo 4.9. } \quad \begin{enumerate} \item La recta que
pasa por los puntos $\left(1,5\right)$ y $\left(-2,3\right)$ tiene
pendiente

\bigskip

\[m=\frac{5-3}{1-\left( -2\right) }=\frac{2}{3}\]

\bigskip

Su ecuaci�n es

\bigskip

\[y-5=\frac{2}{3}\left( x-1\right)\]

\bigskip

esto es

\bigskip

\[\frac{2}{3}x-y=\frac{-13}{3}\]

\bigskip

la recta paralela a ella que pasa por el punto $\left(4,-1\right)$
tiene ecuaci�n

\bigskip

\[y-\left( -1\right) =\frac{2}{3}\left(x-4\right),\]

\bigskip

esto es,

\bigskip

\[\frac{2}{3}x-y=\frac{11}{3}\]

\bigskip

y la recta que es perpendicular a las anteriores y pasa por el
punto $\left( 4,-1\right)$ tiene pendiente $m=\frac{-3}{2}$ y
ecuaci�n

\bigskip

\[y-\left( -1\right) =-\frac{3}{2}\left( x-4\right),\]

\bigskip

es decir,

\bigskip

\[\frac{3}{2}x+y=5\]

\bigskip

\item Sean $P\left( x_{1,}y_{1} \right)$ y $Q\left(
x_{2\acute{,}}y_{2}\right)$. La mediatriz del segmento $PQ$ es la
recta perpendicular a �l, que pasa por su punto medio. Veamos su
ecuaci�n.

\bigskip

Supongamos $x_{1}\neq x_{2,\quad}y_{1}\neq y_{2}$ la pendiente del
segmento es $m=\frac{y_{2}-y_{1}}{x_{2}-x_{1}}$ y su punto medio
es $\left( \frac{x_{1}+x_{2}}{2},\frac{y_{1}+y_{2}}{2}\right)$.
Entonces la ecuaci�n de la mediatriz es

\bigskip

\[y-\frac{y_{2}+y_{1}}{2}=-\frac{x_{2}-x_{1}}{y_{2-}y_{1}}\left( x-\frac{x_{1}+x_{2}}{2}\right)\]

\bigskip

Una recta vertical tiene ecuaci�n $x=c$ donde $c$ es una constante
y no representa una funci�n.
\end{enumerate}

%Pie de p�gina
\newline

{\color{gray}
\begin{tabular}{@{\extracolsep{\fill}}lcr}
\hline \\
\docLink{04_05.tex}{\includegraphics{../../images/navegacion/anterior.gif}}
\begin{tabular}{l}
{\color{darkgray}\small Funciones Lineales} \\ \\ \\
\end{tabular} &
\docLink[_top]{../../index.html}{\includegraphics{../../images/navegacion/inicio.gif}}
\docLink{../../docs_curso/contenido.html}{\includegraphics{../../images/navegacion/contenido.gif}}
\docLink{../../docs_curso/descripcion.html}{\includegraphics{../../images/navegacion/descripcion.gif}}
\docLink{../../docs_curso/profesor.html}{\includegraphics{../../images/navegacion/profesor.gif}}
& \begin{tabular}{r}
{\color{darkgray}\small Funciones Cuadr�ticas} \\ \\ \\
\end{tabular}
\docLink{04_07.tex}{\includegraphics{../../images/navegacion/siguiente.gif}}
\end{tabular}
}

\end{quote}

\newline

\begin{flushright}
\includegraphics{../../images/interfaz/copyright.gif}
\end{flushright}
\end{document}

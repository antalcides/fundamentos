\documentclass[10pt]{article} %tipo documento y tipo letra

\definecolor{azulc}{cmyk}{0.72,0.58,0.42,0.20} % color titulo
\definecolor{naranja}{cmyk}{0.21,0.5,1,0.03} % color leccion

\def\eje{\centerline{\textbf{ EJERCICIOS.}}} % definiciones propias

\begin{document}

\begin{quote}

% inicio encabezado
{\color{gray}
\begin{tabular}{@{\extracolsep{\fill}}lcr}
\docLink{geometria4.tex}{\includegraphics{../../images/navegacion/anterior.gif}}
\begin{tabular}{l}
{\color {darkgray} {\small Pol�gonos}} \\ \\ \\
\end{tabular} &
\docLink[_top]{../../index.html}{\includegraphics{../../images/navegacion/inicio.gif}}
\docLink{../../docs_curso/contenido.html}{\includegraphics{../../images/navegacion/contenido.gif}}
\docLink{../../docs_curso/descripcion.html}{\includegraphics{../../images/navegacion/descripcion.gif}}
\docLink{../../docs_curso/profesor.html}{\includegraphics{../../images/navegacion/profesor.gif}}
& \begin{tabular}{r}
{\color {darkgray} {\small Tipos de Tri�ngulos}} \\ \\ \\
\end{tabular}
\docLink{geometria6.tex}{\includegraphics{../../images/navegacion/siguiente.gif}}
\\ \hline
\end{tabular}
}
%fin encabezado

%nombre capitulo
\begin{center}
\colorbox{azulc}{{\color{white} \large CAP�TULO 3}}  {\large
{\color{ azulc} MODULO DE GEOMETR�A}}
\end{center}

\newline

%nombre leccion

\colorbox{naranja}{{\color{white} \normalsize  Lecci\'on 3.5. }}
{\normalsize {\color{naranja} Regiones Poligonales}}

\newline

De la definici�n de pol�gono podemos concluir que todo pol�gono
est� contenido completamente en un plano. Dado un pol�gono se
distinguen entonces dos conjuntos en el plano: el interior del
pol�gono y el exterior. La uni�n de un pol�gono con su interior es
una regi�n poligonal.

\bigskip

\[
\begin{array}{ccc}
\includegraphics{/imagenes/region_poligonal1.gif} & \includegraphics{/imagenes/region_poligonal2.gif}
\end{array}
\]

\bigskip

Un pol�gono se dice convexo s� y s�lo si su correspondiente regi�n
poligonal es convexa (es decir si dados dos puntos cualesquiera en
la regi�n el segmento de recta que determinan, est� completamente
contenido en ella). Muchos de los pol�gonos con los que trabajamos
corrientemente son convexos.

\bigskip

\[
\begin{array}{ccc}
\includegraphics{/imagenes/no_convexo1.gif} & \includegraphics{/imagenes/no_convexo.gif}
\end{array}
\]

%Pie de p�gina
\newline

{\color{gray}
\begin{tabular}{@{\extracolsep{\fill}}lcr}
\hline \\
\docLink{geometria4.tex}{\includegraphics{../../images/navegacion/anterior.gif}}
\begin{tabular}{l}
{\color {darkgray} {\small Pol�gonos}} \\ \\ \\
\end{tabular} &
\docLink[_top]{../../index.html}{\includegraphics{../../images/navegacion/inicio.gif}}
\docLink{../../docs_curso/contenido.html}{\includegraphics{../../images/navegacion/contenido.gif}}
\docLink{../../docs_curso/descripcion.html}{\includegraphics{../../images/navegacion/descripcion.gif}}
\docLink{../../docs_curso/profesor.html}{\includegraphics{../../images/navegacion/profesor.gif}}
& \begin{tabular}{r}
{\color {darkgray} {\small Tipos de Tri�ngulos}} \\ \\ \\
\end{tabular}
\docLink{geometria6.tex}{\includegraphics{../../images/navegacion/siguiente.gif}}
\end{tabular}
}

\end{quote}

\newline

\begin{flushright}
\includegraphics{../../images/interfaz/copyright.gif}
\end{flushright}
\end{document}

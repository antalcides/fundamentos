
\documentclass[10pt]{article}
%%%%%%%%%%%%%%%%%%%%%%%%%%%%%%%%%%%%%%%%%%%%%%%%%%%%%%%%%%%%%%%%%%%%%%%%%%%%%%%%%%%%%%%%%%%%%%%%%%%%%%%%%%%%%%%%%%%%%%%%%%%%%%%%%%%%%%%%%%%%%%%%%%%%%%%%%%%%%%%%%%%%%%%%%%%%%%%%%%%%%%%%%%%%%%%%%%%%%%%%%%%%%%%%%%%%%%%%%%%%%%%%%%%%%%%%%%%%%%%%%%%%%%%%%%%%
\usepackage{graphicx}
\usepackage{amsmath}

\setcounter{MaxMatrixCols}{10}
%TCIDATA{OutputFilter=Latex.dll}
%TCIDATA{Version=5.00.0.2606}
%TCIDATA{<META NAME="SaveForMode" CONTENT="1">}
%TCIDATA{BibliographyScheme=Manual}
%TCIDATA{LastRevised=Friday, March 25, 2005 20:37:55}
%TCIDATA{<META NAME="GraphicsSave" CONTENT="32">}

\input{tcilatex}

\begin{document}


%TCIMACRO{%
%\HTMLButton{ENCABEZADO}{<table width="100&#37" height="100&#37" border="0" cellpadding="0" cellspacing="0">
%<tr>
%<td valign="top">
%<div align="center">
%<table width="100&#37" border="0" cellspacing="0" cellpadding="3">
%<tr> <!-- InstanceBeginEditable name="anterior" -->
%<td width="30&#37"> <div align="left"><a href="algebra11.html"><img src="../../images/navegacion/anterior.gif" width="35" height="35" border="0" align="absmiddle"></a>Fracciones</div></td>
%<!-- InstanceEndEditable -->
%<td width="40&#37"> <div align="center"><!-- #BeginLibraryItem "/Library/superiorbar.lbi" --><a href="../../index.html" target="_top"><img src="../../images/navegacion/inicio.gif" alt="Ir al Inicio del Curso" width="35" height="35" border="0"></a><a href="../../docs_curso/contenido.html" target="principal"><img src="../../images/navegacion/contenido.gif" alt="Tabla de contenido del curso" width="35" height="35" border="0"></a><a href="../../docs_curso/descripcion.html" target="principal"><img src="../../images/navegacion/descripcion.gif" alt="Descripci&oacute;n del curso" width="35" height="35" border="0"></a><a href="../../docs_curso/profesor.html" target="principal"><img src="../../images/navegacion/profesor.gif" alt="P&aacute;gina del profesor(a)" width="35" height="35" border="0"></a><!-- #EndLibraryItem --></div></td>
%<!-- InstanceBeginEditable name="siguiente" -->
%<td width="30&#37"> <div align="right">Desigualdades Cuadr�ticas<a href="algebra13.html"><img src="../../images/navegacion/siguiente.gif" width="35" height="35" border="0" align="absmiddle"></a></div></td>
%<!-- InstanceEndEditable --></tr>
%</table>
%</div></td>
%</tr>
%<tr>
%<td valign="top" bgcolor="#FFFFFF">
%<table width="98&#37" border="0" align="center" cellpadding="5" cellspacing="0">
%<tr> <!-- InstanceBeginEditable name="titulo" -->
%<td background="../../images/titulo_bg3.gif" bgcolor="#FFFFFF"> <div align="right" class="titulo">
%<object classid="clsid:D27CDB6E-AE6D-11cf-96B8-444553540000" codebase="http://download.macromedia.com/pub/shockwave/cabs/flash/swflash.cab#version=6,0,29,0" width="600" height="20">
%<param name="movie" value="../../flash/titulo.swf?titulo=CAPITULO 2: FUNDAMENTOS DE ALGEBRA">
%<param name="quality" value="high">
%<param name="wmode" value="transparent">
%<embed src="../../flash/titulo.swf?titulo=CAPITULO 2: FUNDAMENTOS DE ALGEBRA" width="600" height="20" quality="high" pluginspage="http://www.macromedia.com/go/getflashplayer" type="application/x-shockwave-flash" wmode="transparent"></embed></object>
%</div></td>
%<!-- InstanceEndEditable --></tr>
%<tr> <!-- InstanceBeginEditable name="contenido" -->
%<td valign="top">}}%

%TCIMACRO{%
%\HTMLButton{TITULOS1}{<br>
%<p>
%<table border="0" cellspacing="0" cellpadding="0">
%<tr>
%    <td bgcolor="#999999"><font color="#FFFFFF" size="2"><strong>&nbsp;Lecci�n 2.13.
%      </strong></font><strong><font color="#FFFFFF" size="4">&nbsp;</font></strong>
%    </td>
%<td><div align="left"> <font color="#666666" size="4"> &nbsp;&nbsp;<strong>
%        <font size="3">Desigualdades</font></strong></font>
%      </div></td>
%</tr>
%</table>
%</p>}}%

Una condici\'{o}n en $x$ es una expresi\'{o}n que contiene la variable $x$ y
se transforma en una proposici\'{o}n matem\'{a}tica, es decir en una afirmaci%
\'{o}n que es verdadera o falsa, cuando se sustituye $x$ por un elemento del
dominio en consideraci\'{o}n, en nuestro caso por un n\'{u}mero real.

El conjunto de elementos del dominio que hacen de la condici\'{o}n una
proposici\'{o}n verdadera, se llama el \textit{conjunto soluci\'{o}n} de la
condici\'{o}n.

La mayor\'{\i}a de las condiciones que se presentan en matem\'{a}ticas
tienen la forma de una ecuaci\'{o}n o de una desigualdad. En esta secci\'{o}%
n estudiaremos algunas desigualdades y sus soluciones.

Resolver una desigualdad es encontrar su conjunto soluci\'{o}n, es decir
encontrar todos los n\'{u}meros reales que la hacen verdadera. El
procedimiento para resolver desigualdades consiste en transformarlas en
desigualdades equivalentes, es decir desigualdades que tienen las mismas
soluciones, hasta que el conjunto soluci\'{o}n sea obvio. Las herramientas
para este trabajo son las propiedades del orden entre los n\'{u}meros reales
estudiadas en la secci\'{o}n 
%TCIMACRO{\HTMLButton{link 1.4.}{<a href="../cap1/reales4.tex">1.4.</a>}}%
. Por su uso tan frecuente nos permitimos recordar las siguientes:

\begin{itemize}
\item Si $x<y$ entonces $x+z<y+z$ para todo n\'{u}mero real $z$.

\item Si $x<y$ y $z>0$ entonces $xz<yz$ y $\dfrac{x}{z}<\dfrac{y}{z}$.

\item Si $x<y$ y $z<0$ entonces $xz>yz$ y $\dfrac{x}{z}>\dfrac{y}{z}$.
\end{itemize}

\textbf{Ejemplo 2.41.} Resolvamos la desigualdad

\begin{equation*}
2x+7\leq 5x-6
\end{equation*}

Las siguientes desigualdades son equivalentes:

\begin{align*}
2x+7 \leq 5x-6 \qquad &\text{Desigualdad dada} \\
-3x+7 \leq -6 \qquad &\text{Sumando $-5x$} \\
-3x \leq -13 \qquad &\text{Sumando $-7$} \\
x \geq \dfrac{13}{3} \qquad &\text{Multiplicando por $-\dfrac{1}{3}$} \\
\end{align*}

Por lo tanto, el conjunto soluci\'{o}n de la desigualdad es el intervalo $[%
\dfrac{13}{3},\infty )$, que se muestra en la figura siguiente

\begin{center}
%TCIMACRO{\HTMLButton{[IMAGEN] img/img1.gif}{<img src="img/img1.gif">}}%
\end{center}

\textbf{Ejemplo 2.42. } Resolvamos la desigualdad

\begin{equation*}
-5<4x+1<9
\end{equation*}

Aunque la desigualdad dada es equivalente a las dos desigualdades

\begin{equation*}
-5<4x+1 \quad \text{y} \quad 4x+1<9
\end{equation*}

las podemos resolver simult\'{a}neamente de la siguiente forma:

\begin{align*}
-5 <4x+1<9 \qquad &\text{ Desigualdad dada} \\
-6 <4x<8 \qquad &\text{Sumando $-1$} \\
-\dfrac{6}{4} <x<\dfrac{8}{4} \qquad &\text{Multiplicando por $\dfrac{1}{4}$}
\\
-\dfrac{3}{2} <x<2 \qquad &\text{Realizando las operaciones}
\end{align*}

Por lo tanto, el conjunto soluci\'{o}n de la desigualdad es el intervalo $%
\left(-\dfrac{3}{2},2\right)$.

\textbf{Ejemplo 2.43.}

Resolvamos la desigualdad

\begin{equation*}
x^{2}-10x+21>0
\end{equation*}

La desigualdad es equivalente a

\begin{equation*}
(x-7)(x-3)>0
\end{equation*}

El producto $(x-7)(x-3)$ puede cambiar de signo solo en $7$ o en $-3$, que
son los puntos donde $x-7=0$ o $x-3=0$. Estos puntos los podemos llamar
puntos de separaci\'{o}n y nos dividen la recta en tres intervalos

\begin{equation*}
(-\infty ,3), \qquad (3,7) \qquad \text{ y } \qquad (7,\infty)
\end{equation*}

En cada uno de estos intervalos $(x-7)(x-3)$ conserva el signo, es decir,
siempre es positivo o siempre es negativo. Para determinar el signo en cada
intervalo usamos un punto de prueba, elegido dentro del intervalo. Por
ejemplo si tomamos $x=0$ en el intervalo $(-\infty ,3)$ los valores de $(x-7)
$ y $(x-3)$ son ambos negativos y por lo tanto $(x-7)(x-3)>0$ en este
intervalo. Similarmente se procede con los otros intervalos. Los resultados
se pueden expresar en una tabla de signos como la siguiente

\begin{center}
\begin{tabular}{|l|c|c|c|}
\hline
Intervalo & $(-\infty ,3)$ & $(3,7)$ & $(7,\infty )$ \\ \hline
Signo de $(x-7)$ & $-$ & $-$ & $+$ \\ \hline
Signo de $(x-3)$ & $-$ & $+$ & $+$ \\ \hline
Signo de $(x-7)(x-3)$ & $+$ & $-$ & $+$ \\ \hline
\end{tabular}
\end{center}

donde el signo $(x-7)(x-3)$ se obtiene aplicando las reglas de los signos.

Por lo tanto, vemos que la soluci\'{o}n de la desigualdad es $%
(-\infty,3)\cup (7,\infty )$.

Una manera mas pr\'{a}ctica de resolver esta desigualdad es elaborando un
diagrama de signos, como se muestra a continuaci\'{o}n.

\begin{center}
%TCIMACRO{\HTMLButton{[IMAGEN] img/img2.gif}{<img src="img/img2.gif">}}%
\end{center}

En el diagrama, las l\'{\i}neas verticales corresponden a los puntos de
separaci\'{o}n y la recta horizontal es la recta real.

\textbf{Ejemplo 2.44. } Resolvamos la desigualdad

\begin{equation*}
(2x+3)(4-x)(x+5)\leq 0
\end{equation*}

Elaboramos un diagrama de signos. Primero obtenemos los puntos de separaci%
\'{o}n resolviendo las ecuaciones $2x+3=0$, $4-x=0$ y $x+5=0$. Los puntos de
separaci\'{o}n son $-\dfrac{3}{2}$, $4$ y $-5$.

Tenemos el siguiente diagrama

\begin{center}
%TCIMACRO{\HTMLButton{[IMAGEN] img/img3.gif}{<img src="img/img3.gif">}}%
\end{center}

Analizando el signo resultante, es decir el signo de $(2x+3)(4-x)(x+5)$,
vemos que la soluci\'{o}n de la desigualdad dada es $\left[-5,-\dfrac{3}{2}%
\right]\cup \lbrack 4,\infty )$.

\textbf{Ejemplo 2.45. } Resolvamos la desigualdad

\begin{equation*}
\dfrac{3x-1}{x+4}>2
\end{equation*}

La desigualdad es equivalente a cada una de las siguientes

\begin{align*}
\dfrac{3x-1}{x+4}-2 &>0 \\
\dfrac{(3x-1)-2(x+4)}{x+4} &>0 \\
\dfrac{x-9}{x+4} &>0
\end{align*}

Elaborando el diagrama de signos tenemos

\begin{center}
%TCIMACRO{\HTMLButton{[IMAGEN] img/img4.gif}{<img src="img/img4.gif">}}%
\end{center}

Por lo tanto la soluci\'{o}n de la desigualdad es $(-\infty,-4)\cup
(9,\infty )$.

%TCIMACRO{%
%\HTMLButton{HTML Field}{</td>
%<!-- InstanceEndEditable --></tr>
%<tr>
%<td height="25" background="../../images/titulo_bg3.gif">
%<div align="center"><font color="#FFFFFF">__</font> </div></td>
%</tr>
%</table></td>
%</tr>
%<tr>
%<td valign="bottom">
%<table width="100&#37" border="0" cellspacing="0" cellpadding="3">
%<tr> <!-- InstanceBeginEditable name="anterior" -->
%<td width="30&#37"> <div align="left"><a href="algebra11.html"><img src="../../images/navegacion/anterior.gif" width="35" height="35" border="0" align="absmiddle"></a>Fracciones</div></td>
%<!-- InstanceEndEditable -->
%<td width="40&#37"> <div align="center"><!-- #BeginLibraryItem "/Library/superiorbar.lbi" --><a href="../../index.html" target="_top"><img src="../../images/navegacion/inicio.gif" alt="Ir al Inicio del Curso" width="35" height="35" border="0"></a><a href="../../docs_curso/contenido.html" target="principal"><img src="../../images/navegacion/contenido.gif" alt="Tabla de contenido del curso" width="35" height="35" border="0"></a><a href="../../docs_curso/descripcion.html" target="principal"><img src="../../images/navegacion/descripcion.gif" alt="Descripci&oacute;n del curso" width="35" height="35" border="0"></a><a href="../../docs_curso/profesor.html" target="principal"><img src="../../images/navegacion/profesor.gif" alt="P&aacute;gina del profesor(a)" width="35" height="35" border="0"></a><!-- #EndLibraryItem --></div></td>
%<!-- InstanceBeginEditable name="siguiente" -->
%<td width="30&#37"> <div align="right">Desigualdades Cuadr�ticas<a href="algebra13.html"><img src="../../images/navegacion/siguiente.gif" width="35" height="35" border="0" align="absmiddle"></a></div></td>
%<!-- InstanceEndEditable --></tr>
%</table></td>
%</tr>
%<tr>
%<td valign="bottom">
%<div align="right"><img src="../../images/interfaz/copyright.gif" width="470" height="20" border="0" usemap="#Map"></div></td>
%</tr>
%</table>}}%

\end{document}

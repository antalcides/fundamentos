\documentclass[10pt]{article} %tipo documento y tipo letra

\definecolor{azulc}{cmyk}{0.72,0.58,0.42,0.20} % color titulo
\definecolor{naranja}{cmyk}{0.21,0.5,1,0.03} % color leccion

\def\eje{\centerline{\textbf{ EJERCICIOS.}}} % definiciones propias

\begin{document}

\begin{quote}

% inicio encabezado
{\color{gray}
\begin{tabular}{@{\extracolsep{\fill}}lcr}
\docLink{geometria9.tex}{\includegraphics{../../images/navegacion/anterior.gif}}
\begin{tabular}{l}
{\color {darkgray} {\small Propiedades de los \'{A}ngulos}} \\ \\ \\
\end{tabular} &
\docLink[_top]{../../index.html}{\includegraphics{../../images/navegacion/inicio.gif}}
\docLink{../../docs_curso/contenido.html}{\includegraphics{../../images/navegacion/contenido.gif}}
\docLink{../../docs_curso/descripcion.html}{\includegraphics{../../images/navegacion/descripcion.gif}}
\docLink{../../docs_curso/profesor.html}{\includegraphics{../../images/navegacion/profesor.gif}}
& \begin{tabular}{r}
{\color {darkgray} {\small \'{A}ngulos Tri�ngulo}} \\ \\ \\
\end{tabular}
\docLink{geometria12.tex}{\includegraphics{../../images/navegacion/siguiente.gif}}
\\ \hline
\end{tabular}
}
%fin encabezado

%nombre capitulo
\begin{center}
\colorbox{azulc}{{\color{white} \large CAP�TULO 3}}  {\large
{\color{ azulc} MODULO DE GEOMETR�A}}
\end{center}

\newline

%nombre leccion

\colorbox{naranja}{{\color{white} \normalsize  Lecci\'on 3.10. }}
{\normalsize {\color{naranja} Rectas Perpendiculares}}

\newline

\textbf{Definici�n 3.10.1. } Si dos rectas al cortarse forman
cuatro �ngulos iguales, se dice que son perpendiculares y los
�ngulos formados son rectos.

\bigskip

\[
\begin{array}{ccc}
\includegraphics{/imagenes/recta_perpend.gif}
\end{array}
\]

\bigskip

En la figura el $\angle ABC$ es recto, $m\perp n$, $\overline{AB}
\perp \overline{ BC}$.

\bigskip

Si dos rectas coplanares $l$ y $m$ son perpendiculares a la misma
recta, entonces son paralelas.

\bigskip

\[
\begin{array}{ccc}
\includegraphics{/imagenes/recta_perpend1.gif}
\end{array}
\]

\bigskip

En un plano si una recta es perpendicular a una de dos rectas
paralelas, entonces es perpendicular a la otra.

%Pie de p�gina
\newline

{\color{gray}
\begin{tabular}{@{\extracolsep{\fill}}lcr}
\hline \\
\docLink{geometria9.tex}{\includegraphics{../../images/navegacion/anterior.gif}}
\begin{tabular}{l}
{\color {darkgray} {\small Propiedades de los \'{A}ngulos}} \\ \\ \\
\end{tabular} &
\docLink[_top]{../../index.html}{\includegraphics{../../images/navegacion/inicio.gif}}
\docLink{../../docs_curso/contenido.html}{\includegraphics{../../images/navegacion/contenido.gif}}
\docLink{../../docs_curso/descripcion.html}{\includegraphics{../../images/navegacion/descripcion.gif}}
\docLink{../../docs_curso/profesor.html}{\includegraphics{../../images/navegacion/profesor.gif}}
& \begin{tabular}{r}
{\color {darkgray} {\small \'{A}ngulos Tri�ngulo}} \\ \\ \\
\end{tabular}
\docLink{geometria12.tex}{\includegraphics{../../images/navegacion/siguiente.gif}}
\end{tabular}
}

\end{quote}

\newline

\begin{flushright}
\includegraphics{../../images/interfaz/copyright.gif}
\end{flushright}
\end{document}

\documentclass[10pt]{article} %tipo documento y tipo letra

\definecolor{azulc}{cmyk}{0.72,0.58,0.42,0.20} % color titulo
\definecolor{naranja}{cmyk}{0.50,0.42,0.42,0.06} % color leccion

\def\Reales{\mathbb{R}}
\def\Naturales{\mathbb{N}}
\def\Enteros{\mathbb{Z}}
\def\Racionales{\mathbb{Q}}
\def\Irr{\mathbb{I}}
\def\contradiccion{($\rightarrow \leftarrow$)}


\begin{document}

\begin{quote}

% inicio encabezado
{\color{gray}
\begin{tabular}{@{\extracolsep{\fill}}lcr}
\docLink{taller2.tex}{\includegraphics{../../../images/navegacion/anterior.gif}}
\begin{tabular}{l}
{\color{darkgray}\small Taller 2 } \\ \\ \\
\end{tabular} &
\docLink[_top]{../../../index.html}{\includegraphics{../../../images/navegacion/inicio.gif}}
\docLink{../../../docs_curso/contenido.html}{\includegraphics{../../../images/navegacion/contenido.gif}}
\docLink{../../../docs_curso/descripcion.html}{\includegraphics{../../../images/navegacion/descripcion.gif}}
\docLink{../../../docs_curso/profesor.html}{\includegraphics{../../../images/navegacion/profesor.gif}}
& \begin{tabular}{r}
{\color{darkgray}\small Taller 4} \\ \\ \\
\end{tabular}
\docLink{taller4.tex}{\includegraphics{../../../images/navegacion/siguiente.gif}}
\\ \hline
\end{tabular}
}
%fin encabezado

%nombre capitulo
\begin{center}
\colorbox{azulc}{{\color{white} \large GEOMETR\'{I}A}}  {\large
{\color{ azulc} }}
\end{center}

\newline

%nombre leccion

\colorbox{naranja}{{\color{white} \normalsize  TALLER 3 }}
{\normalsize {\color{naranja} }}

\newline

\begin{enumerate}

\item En la figura que aparece a continuaci�n determinar �reas de
los tri�ngulos $EFH$, $FGH$, $EGH$.

\[
\includegraphics{imagenes/taller3_1.gif}
\]

\item Hallar el �rea del $\triangle XYZ$ en cada uno de los
siguientes casos.

\[
\begin{array}{cc}
\includegraphics{imagenes/taller3_2.gif} & \includegraphics{imagenes/taller3_3.gif}
\end{array}
\]


\item
\begin{enumerate}
\item Hallar �rea de los tri�ngulos $PQR$, $PRS$, $PQS$ que se
determinan en la figura.

\[
\includegraphics{imagenes/taller3_4.gif}
\]

\item Determinar el �rea, en unidades cuadradas, del cuadril�tero
$ABCD$.

\[
\includegraphics{imagenes/taller3_5.gif}
\]


\item
\begin{enumerate}
\item Trazar un tri�ngulo $ABC$ cualquiera, dibujar sus tres
alturas. Determinar el �rea del tri�ngulo en $cm^2$, midiendo la
longitud de cada uno de sus lados y la altura correspondiente a
ese lado. �Qu� valor tiene el �rea, en cada caso? \item En el
tri�ngulo $ABC$ de la figura se han trazado las alturas
$\overline{AW}$ y $\overline{CF}$. Si $AB=8$, $CF=6$ y  $AW=7$.
Determinar $CB$.

\[
\includegraphics{imagenes/taller3_6.gif}
\]

\end{enumerate}

\item
\begin{enumerate}
\item Determinar la longitud de la diagonal de un campo
rectangular cuyos lados miden $24$ y $70$ metros respectivamente.
\item Si se sabe que un cateto de un tri�ngulo rect�ngulo es el
doble del otro. Determinar su hipotenusa. �Cu�l es la raz�n entre
la hipotenusa y el cateto menor?
\end{enumerate}

\item Dados $\overline{PR}$ y $\overline{SQ}$ di�metros de la
circunferencia $C$

\[
\includegraphics{imagenes/taller3_7.gif}
\]


Discutir la validez de las siguientes afirmaciones

{\bf (1)} $\overline{OP} \cong \overline{OQ} \cong {OR} \cong
\overline{OS}$
\newline
{\bf (2)} $\angle POQ \cong \angle ROS$

\item
\begin{enumerate}
\item Trazar un segmento $\overline{AB}$ \item Construir un
circunferencia con radio $AB$ \item Construir una circunferencia
con di�metro $AB$
\end{enumerate}

\item Las dos circunferencias que aparecen en la figura son
conc�ntricas (es decir tienen el mismo centro)

\[
\includegraphics{imagenes/taller3_8.gif}
\]

Si el radio de la circunferencia mayor es cuatro veces el radio de
la circunferencia menor, �cu�l es el �rea de la regi�n sombreada?,
�cu�l es la raz�n entre las �reas del c�rculo menor y el mayor?,
�cu�l es la raz�n entre las longitudes de la circunferencia menor
y la mayor?

\item En la figura que presenta a continuaci�n se han cortado $8$
discos circulares de metal de una l�mina rectangular de $12cm $
por $24 cm$. La l�mina restante no se usa.

\[
\includegraphics{imagenes/taller3_9.gif}
\]

�Cu�l es el �rea de metal que no se usa?

\item Una circunferencia de di�metro $12$ unidades se ha inscrito
en el cuadrado $EFGH$, como se muestra en la figura.

\[
\includegraphics{imagenes/taller3_10.gif}
\]

�Cu�l es el �rea de la regi�n sombreada?

\item En la circunferencia de la figura, $\overline{BD}$ es un
di�metro, $OD=15$ y $m \angle AOD=20�$

\[
\includegraphics{imagenes/taller3_11.gif}
\]


\begin{enumerate}
\item Determinar la longitud de $\overline{AB}$. \item Determinar
�rea del sector determinado por $A$, $O$ y $D$.
\end{enumerate}

\end{enumerate}

\end{enumerate}

%Pie de p�gina
\newline

{\color{gray}
\begin{tabular}{@{\extracolsep{\fill}}lcr}
\hline \\
\docLink{taller2.tex}{\includegraphics{../../../images/navegacion/anterior.gif}}
\begin{tabular}{l}
{\color{darkgray}\small Taller 2 } \\ \\ \\
\end{tabular} &
\docLink[_top]{../../../index.html}{\includegraphics{../../../images/navegacion/inicio.gif}}
\docLink{../../../docs_curso/contenido.html}{\includegraphics{../../../images/navegacion/contenido.gif}}
\docLink{../../../docs_curso/descripcion.html}{\includegraphics{../../../images/navegacion/descripcion.gif}}
\docLink{../../../docs_curso/profesor.html}{\includegraphics{../../../images/navegacion/profesor.gif}}
& \begin{tabular}{r}
{\color{darkgray}\small Taller 4} \\ \\ \\
\end{tabular}
\docLink{taller4.tex}{\includegraphics{../../../images/navegacion/siguiente.gif}}
\end{tabular}
}

\end{quote}

\newline

\begin{flushright}
\includegraphics{../../../images/interfaz/copyright.gif}
\end{flushright}
\end{document}

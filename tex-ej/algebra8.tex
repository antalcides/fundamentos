\documentclass[10pt]{article} %tipo documento y tipo letra

\definecolor{azulc}{cmyk}{0.72,0.58,0.42,0.20} % color titulo
\definecolor{naranja}{cmyk}{0.21,0.5,1,0.03} % color leccion

\def\eje{\centerline{\textbf{ EJERCICIOS.}}} % definiciones propias

\begin{document}

\begin{quote}

% inicio encabezado
{\color{gray}
\begin{tabular}{@{\extracolsep{\fill}}lcr}
\docLink{algebra7.tex}{\includegraphics{../../images/navegacion/anterior.gif}}
\begin{tabular}{l}
{\color{darkgray}\small Suma y Resta} \\ \\ \\
\end{tabular} &
\docLink[_top]{../../index.html}{\includegraphics{../../images/navegacion/inicio.gif}}
\docLink{../../docs_curso/contenido.html}{\includegraphics{../../images/navegacion/contenido.gif}}
\docLink{../../docs_curso/descripcion.html}{\includegraphics{../../images/navegacion/descripcion.gif}}
\docLink{../../docs_curso/profesor.html}{\includegraphics{../../images/navegacion/profesor.gif}}
& \begin{tabular}{r}
{\color{darkgray}\small Divisi�n} \\ \\ \\
\end{tabular}
\docLink{algebra9.tex}{\includegraphics{../../images/navegacion/siguiente.gif}}
\\ \hline
\end{tabular}
}
%fin encabezado

%nombre capitulo
\begin{center}
\colorbox{azulc}{{\color{white} \large CAP�TULO 2}}  {\large
{\color{ azulc} FUNDAMENTOS DE ALGEBRA}}
\end{center}

\newline

%nombre leccion

\colorbox{naranja}{{\color{white} \normalsize  Lecci\'on 2.8. }}
{\normalsize {\color{naranja} Multiplicaci�n de Polinomios}}

\newline

La multiplicaci�n de polinomios esta basada en la aplicaci�n
repetida de la propiedad distributiva y la utilizaci�n de la ley
de los exponentes

\bigskip

\[a^{m}a^{n}=a^{m+n}\]

\bigskip

con $m$ y $n$ enteros positivos.

\bigskip

\textbf{Ejemplo 2.25. } \quad\begin{itemize} \item
$(5x^{2}y^{3})(-3x^{4}y^{2})=-15x^{6}y^{5}$ \item
\begin{align*}
(3x+2y)(2x^{2}-xy+y)&=3x(2x^{2}-xy+y)+2y(2x^{2}-xy+y)\\
&=(6x^{3}-3x^{2}y+3xy)+(4x^{2}y-2xy^{2}+2y^{2})\\
&=6x^{3}+x^{2}y+3xy-2xy^{2}+2y^{2}
\end{align*}

\bigskip

En la pr�ctica se procede como se indica en la siguiente
ilustraci�n

\bigskip

\begin{center}
\renewcommand{\tabcolsep}{0cm}
\begin{tabular}{ccccc}
$2x^2$ & $-xy$ & $+y$ & & \\
$3x$ & $+2y$ &&& \\ \cline{1-3} $6x^3$ & $-3x^2y$ & $+3xy$ & & \\
& $+4x^2$ & & $-2xy^2$ & $2y^2$\\ \hline $6x^3$ & $+x^2y$ & $+3xy$
& $-2xy^2$ & $+2y^2$
\end{tabular}
\end{center}

\bigskip

donde la tercera l�nea se obtiene multiplicando cada t�rmino de la
primera l�nea por $3x$ y la cuarta l�nea se obtiene multiplicando
cada t�rmino de la primera l�nea por $2y$. Finalmente sumando las
filas tercera y cuarta se obtiene el resultado que es la �ltima
fila.
\end{itemize}

\bigskip

Un caso particular importante es el de la multiplicaci�n de dos
polinomios en la misma variable. En este caso es conveniente
ordenar los polinomios de tal forma que el exponente de la
variable vaya decreciendo.

\bigskip

\textbf{Ejemplo 2.26. } Efectuemos el siguiente producto de
polinomios

\bigskip

\[(2x^{3}-3x^{2}+x-7)(x^{4}-6x^{2}+1)\]

\bigskip

Tenemos

\bigskip

\begin{center}
\renewcommand{\tabcolsep}{0cm}
\begin{tabular}{cccccccc}
$2x^3$ & $-3x^2$ & $+x$ & $-7$ &&&&\\
$x^4$ & $-6x^2$ & $+1$ &&&&&\\ \cline{1-4} $2x^7$ & $-3x^6$ &
$+x^5$ & $-7x^4$ &&&&\\
&& $-12x^5$ & $+18x^4$ & $-6x^3$ & $+42x^2$ &&\\
&&&& $+2x^3$ & $-3x^2$ & $+x$ & $-7$ \\ \hline $2x^7$ & $-3x^6$ &
$-11x^5$ & $+11x^4$ & $-4x^3$ & $+39x^2$ & $+x$ & $-7$
\end{tabular}
\end{center}

%Pie de p�gina
\newline

{\color{gray}
\begin{tabular}{@{\extracolsep{\fill}}lcr}
\hline \\
\docLink{algebra7.tex}{\includegraphics{../../images/navegacion/anterior.gif}}
\begin{tabular}{l}
{\color{darkgray}\small Suma y Resta} \\ \\ \\
\end{tabular} &
\docLink[_top]{../../index.html}{\includegraphics{../../images/navegacion/inicio.gif}}
\docLink{../../docs_curso/contenido.html}{\includegraphics{../../images/navegacion/contenido.gif}}
\docLink{../../docs_curso/descripcion.html}{\includegraphics{../../images/navegacion/descripcion.gif}}
\docLink{../../docs_curso/profesor.html}{\includegraphics{../../images/navegacion/profesor.gif}}
& \begin{tabular}{r}
{\color{darkgray}\small Divisi�n} \\ \\ \\
\end{tabular}
\docLink{algebra9.tex}{\includegraphics{../../images/navegacion/siguiente.gif}}
\end{tabular}
}

\end{quote}

\newline

\begin{flushright}
\includegraphics{../../images/interfaz/copyright.gif}
\end{flushright}
\end{document}

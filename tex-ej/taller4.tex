\documentclass[10pt]{article} %tipo documento y tipo letra

\definecolor{azulc}{cmyk}{0.72,0.58,0.42,0.20} % color titulo
\definecolor{naranja}{cmyk}{0.50,0.42,0.42,0.06} % color leccion

\def\Reales{\mathbb{R}}
\def\Naturales{\mathbb{N}}
\def\Enteros{\mathbb{Z}}
\def\Racionales{\mathbb{Q}}
\def\Irr{\mathbb{I}}
\def\contradiccion{($\rightarrow \leftarrow$)}


\begin{document}

\begin{quote}

% inicio encabezado
{\color{gray}
\begin{tabular}{@{\extracolsep{\fill}}lcr}
\docLink{taller3.tex}{\includegraphics{../../../images/navegacion/anterior.gif}}
\begin{tabular}{l}
{\color{darkgray}\small Taller 3 } \\ \\ \\
\end{tabular} &
\docLink[_top]{../../../index.html}{\includegraphics{../../../images/navegacion/inicio.gif}}
\docLink{../../../docs_curso/contenido.html}{\includegraphics{../../../images/navegacion/contenido.gif}}
\docLink{../../../docs_curso/descripcion.html}{\includegraphics{../../../images/navegacion/descripcion.gif}}
\docLink{../../../docs_curso/profesor.html}{\includegraphics{../../../images/navegacion/profesor.gif}}
&
\\ \hline
\end{tabular}
}
%fin encabezado

%nombre capitulo
\begin{center}
\colorbox{azulc}{{\color{white} \large GEOMETR\'{I}A}}  {\large
{\color{ azulc} }}
\end{center}

\newline

%nombre leccion

\colorbox{naranja}{{\color{white} \normalsize  TALLER 4 }}
{\normalsize {\color{naranja} }}

\newline
\begin{enumerate}
\item Si los tri�ngulos $OMN$ y $QMP$ de la figura son semejantes,

\[
\includegraphics{imagenes/taller4_1.gif}
\]


\begin{enumerate}
\item determinar la raz�n de semejanza. \item hallar la longitud
de $\overline{QP}$ \item comparar �ngulos correspondientes de los
dos tri�ngulos.
\end{enumerate}

\item �Son los pares de tri�ngulos que se muestran en la figura
semejantes? Explique porqu� si o por que no.

\[
\begin{array}{cc}
\includegraphics{imagenes/taller4_2.gif} & \includegraphics{imagenes/taller4_3.gif}
\end{array}
\]


\item
\begin{enumerate}
\item Explicar porqu� las parejas de tri�ngulos que aparecen en la
siguiente figura son semejantes.

\[
\begin{array}{cc}
\includegraphics{imagenes/taller4_4.gif} &
\includegraphics{imagenes/taller4_5.gif} \\
\includegraphics{imagenes/taller4_6.gif} &
\includegraphics{imagenes/taller4_7.gif} \\
\includegraphics{imagenes/taller4_8.gif} &
\includegraphics{imagenes/taller4_9.gif} \\
\end{array}
\]

\item Hallar raz�n de semejanza entre los lados de los tri�ngulos
correspondientes. \item Determinar longitudes de los lados
desconocidos y medidas de los �ngulos.
\end{enumerate}

\bigskip\item En cada una de las siguientes figuras se determinan
al menos dos tri�ngulos

\[
\begin{array}{cc}
\includegraphics{imagenes/taller4_10.gif} &
\includegraphics{imagenes/taller4_11.gif}\\
\includegraphics{imagenes/taller4_12.gif} &
\end{array}
\]

\begin{enumerate}
\item �Son estos tri�ngulos semejantes? \item Si son semejantes,
�qu� criterio permite determinar la semejanza?
\end{enumerate}

\item En la figura se tiene que: $WY=3VY$ y $XY=3YZ$


\[
\includegraphics{imagenes/taller4_13.gif}
\]

Argumentar porqu�, $\triangle WXY \cong \triangle VZY$

\item La altura $\overline{CD}$ del tri�ngulo $ABC$ divide la
hipotenusa en dos segmentos de longitudes $6$ y $9$ unidades.
Hallar $\overline{CD}$ y los dos catetos.

\[
\includegraphics{imagenes/taller4_14.gif}
\]

\item Analizar como cambia el volumen de una caja de dimensiones
$l$, $w$ y $h$ cuando se realizan los siguientes cambios sobre
ellas:
\begin{enumerate}
\item Una de las dimensiones se multiplica por cinco y la otras no
se cambian. \item Todas las dimensiones se cuadruplican. \item Una
de las dimensiones se incrementa en seis unidades y las otras no
se cambian. \item Las tres dimensiones se incrementan en una
unidad.
\end{enumerate}

\item Recuerde que el volumen $V$ de un cilindro de radio $r$ y
altura $h$ es $V= \pi r^2 h$ (�rea de la base por la altura)

\[
\includegraphics{imagenes/taller4_15.gif}
\]


\begin{enumerate}
\item Si un tanque cil�ndrico tiene $100$ pies de di�metro y $70$
pies de alto, �cu�l es el volumen del tanque? \item Compare los
vol�menes de dos cilindros que tengan la misma altura, pero el
radio del segundo sea la tercera parte del radio del primero.
\item Los dos s�lidos que aparecen en la figura tienen la misma
altura.

�Tienen ellos el mismo volumen?. Si no lo tienen, �cu�l de los
tiene mayor volumen?

\[
\begin{array}{cc}
\includegraphics{imagenes/taller4_16.gif} & \includegraphics{imagenes/taller4_17.gif}
\end{array}
\]

\end{enumerate}

\item Si sabemos que el volumen $V$ de un cono recto de altura $h$
y �rea de la base $B$ es $V= \frac{1}{3} Bh = \frac{1}{3} \pi r^2
h$

\[
\includegraphics{imagenes/taller4_18.gif}
\]

\begin{enumerate}
\item Si un cono tiene un volumen de $40cm^3$ y su altura es
$5cm$, �cu�l es el radio de su base? \item Si un cono y un
cilindro tienen bases y alturas iguales y el volumen del cilindro
es $V$, �cu�l es volumen del cono?
\end{enumerate}

\item En la figura se tiene que $BC=4$, $AB=12$ y $CG=3$
\begin{enumerate}
\item Calcular las �reas de los cuadril�teros, $ABCD$, $AEHD$,
$AEFB$. \item Determinar la longitud de $\overline{BG}$ y la
longitud de $\overline{BH}$. \item Determinar el �rea del
tri�ngulo $BGH$.
\end{enumerate}

\[
\includegraphics{imagenes/taller4_19.gif}
\]


\end{enumerate}
%Pie de p�gina
\newline

{\color{gray}
\begin{tabular}{@{\extracolsep{\fill}}lcr}
\hline \\
\docLink{taller3.tex}{\includegraphics{../../../images/navegacion/anterior.gif}}
\begin{tabular}{l}
{\color{darkgray}\small Taller 3 } \\ \\ \\
\end{tabular} &
\docLink[_top]{../../../index.html}{\includegraphics{../../../images/navegacion/inicio.gif}}
\docLink{../../../docs_curso/contenido.html}{\includegraphics{../../../images/navegacion/contenido.gif}}
\docLink{../../../docs_curso/descripcion.html}{\includegraphics{../../../images/navegacion/descripcion.gif}}
\docLink{../../../docs_curso/profesor.html}{\includegraphics{../../../images/navegacion/profesor.gif}}
&
\end{tabular}
}

\end{quote}

\newline

\begin{flushright}
\includegraphics{../../../images/interfaz/copyright.gif}
\end{flushright}
\end{document}

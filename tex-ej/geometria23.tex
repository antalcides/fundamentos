\documentclass[10pt]{article} %tipo documento y tipo letra

\definecolor{azulc}{cmyk}{0.72,0.58,0.42,0.20} % color titulo
\definecolor{naranja}{cmyk}{0.21,0.5,1,0.03} % color leccion

\def\eje{\centerline{\textbf{ EJERCICIOS.}}} % definiciones propias

\begin{document}

\begin{quote}

% inicio encabezado
{\color{gray}
\begin{tabular}{@{\extracolsep{\fill}}lcr}
\docLink{geometria22.tex}{\includegraphics{../../images/navegacion/anterior.gif}}
\begin{tabular}{l}
{\color {darkgray} {\small Geometr�a del Espacio}} \\ \\ \\
\end{tabular} &
\docLink[_top]{../../index.html}{\includegraphics{../../images/navegacion/inicio.gif}}
\docLink{../../docs_curso/contenido.html}{\includegraphics{../../images/navegacion/contenido.gif}}
\docLink{../../docs_curso/descripcion.html}{\includegraphics{../../images/navegacion/descripcion.gif}}
\docLink{../../docs_curso/profesor.html}{\includegraphics{../../images/navegacion/profesor.gif}}
& \begin{tabular}{r}
{\color {darkgray} {\small Cap. 4. Funciones}} \\ \\ \\
\end{tabular}
\docLink{../cap4/04_01.tex}{\includegraphics{../../images/navegacion/siguiente.gif}}
\\ \hline
\end{tabular}
}
%fin encabezado

%nombre capitulo
\begin{center}
\colorbox{azulc}{{\color{white} \large CAP�TULO 3}}  {\large
{\color{ azulc} MODULO DE GEOMETR�A}}
\end{center}

\newline

%nombre leccion

\colorbox{naranja}{{\color{white} \normalsize  Lecci\'on 3.19. }}
{\normalsize {\color{naranja} Volumen de un prisma}}

\newline

El volumen de un cuerpo es la medida del espacio que el cuerpo
ocupa.

\bigskip

Si cada uno de los lados de las caras de un cubo tiene una unidad
de longitud, podemos afirmar que cada cara tiene una unidad
cuadrada de �rea. Como el cubo tiene seis caras podemos decir
adem�s que el �rea de la superficie de este cubo es de 6 unidades
cuadradas; pero tambi�n podemos afirmar que este cubo tiene una
unidad c�bica de volumen $(1u^3)$. Por esta raz�n es llamado cubo
unidad. Usualmente el volumen es medido en unidades c�bicas.

\bigskip

\textbf{Ejemplo 3.13}

�Cu�l ser� entonces el volumen del s�lido que se construye como se
ilustra en la figura con cubos unidad? Las dimensiones de la base
son respectivamente: 12 y 7 unidades lineales y tiene 17 unidades
de altura.

\bigskip

\[
\includegraphics{imagenes/volumen_prisma.gif}
\]

\bigskip

En la capa de la base se han colocado $12\times 7$ cubos unidad y
hay 17 de estas capas. Es decir el volumen total se puede hallar
determinando el producto $12\times 7\times 17$, hay pues en total
1248 cubos unidad, el volumen es de 1428 unidades c�bicas.
\end{ejem}

Como en el caso del �rea de un rect�ngulo la idea anterior nos
permite intuir que el volumen de un paralelep�pedo rect�ngulo (una
caja) es igual al producto de sus tres dimensiones.

\bigskip

\[
\includegraphics{imagenes/volumen_prisma1.gif}
\]

\bigskip

Si las dimensiones de este paralelep�pedo rect�ngulo son $a$, $b$,
$c$ su volumen $V$ es,

\bigskip

\[ V = a\cdot b \cdot c \]

\bigskip

Esta expresi�n es equivalente a afirmar que si el �rea de la base
del paralelep�pedo es $A$ y la altura es $c$, el volumen $V$ es,

\bigskip

\[V = A \cdot c\]

\bigskip

Como caso especial, el volumen de un cubo es el cubo de la arista,
pues sus tres dimensiones son iguales. Si la arista es $a$, el
volumen $V$ es,

\bigskip

\[V = a^3\]

\bigskip

La idea anterior se puede generalizar al volumen de un
paralelep�pedo cualquiera (o de un prisma cualquiera).

\bigskip

\[
\includegraphics{imagenes/paralelepipido.gif}
\]

\bigskip

El volumen del paralelep�pedo de la figura es $V = AB \cdot MQ
\cdot NM$, donde $NM$ es la altura y $AB \cdot MQ$ es el �rea de
la base. De donde $V=A*h$, siendo $h$ la altura y $A$ el �rea de
la base.

\bigskip

\textbf{Ejemplo 3.14. } \quad \begin{enumerate} \item Si un cubo
tiene $50$ $cm^3$ de volumen. �Cu�l es la longitud de un lado?

\bigskip

\emph{Soluci�n}. Sea $s$ la longitud de un lado, como $V=s^3=50$,
se tiene que $s=\sqrt[3]{50}\approx 3.7cm$

bigskip

\item Si un cubo tiene de lado $a$ $cm$ y otro tiene de lado $4a$
$cm$, �cu�l es la raz�n entre los vol�menes de los dos cubos?

\bigskip

\emph{Soluci�n}. El volumen $V$ del primer cubo es $a^3$ y el
volumen $V_1$ del segundo cubo es $(4a)^3$ y la raz�n entre el
volumen $V$ y el volumen $V_1$ es
$\frac{a^3}{64a^3}=\frac{1}{64}$.

\bigskip

\item Si las dimensiones de una caja se incrementan en 2, 3 y 4
unidades. �Qu� sucede con el volumen de la caja?

\bigskip

\[
\includegraphics{imagenes/volumen_prisma2.gif}
\]

\bigskip

\emph{Soluci�n}. Si las dimensiones originales de la caja son $x$,
$y$, $z$, su volumen original es $x*y*z$. Si las nuevas
dimensiones son $l$, $w$ y $h$, tenemos que:

\bigskip

\[l = z + 4, \quad w = y + 3 \quad \text{y}\quad h = x + 2.\]

\bigskip

De donde el volumen $V$ de la nueva caja es:

\bigskip

\[V = l*w*h = (z + 4)(y + 3)(x + 2)\]

\bigskip

Efectuando la multiplicaci�n obtenemos que:

\bigskip

\[V = xyz + 3xz + 4xy + 12x + 2yz + 6z + 8 y + 24\]

\bigskip

El volumen se incrementa entonces en: $3xz + 4xy +12x + 2yz + 6z +
8y + 24$

\bigskip

Por ejemplo si las dimensiones de la caja original son $x=6$,
$y=6$, $z=10$, el volumen de la caja ser�a 480 y el de la nueva
caja ser�a $(6+2)\cdot(8+3)\cdot(10+4)=1232$. En ese caso se
incrementar�a en 752 unidades c�bicas.

\end{enumerate}

%Pie de p�gina
\newline

{\color{gray}
\begin{tabular}{@{\extracolsep{\fill}}lcr}
\hline \\
\docLink{geometria22.tex}{\includegraphics{../../images/navegacion/anterior.gif}}
\begin{tabular}{l}
{\color {darkgray} {\small Geometr�a del Espacio}} \\ \\ \\
\end{tabular} &
\docLink[_top]{../../index.html}{\includegraphics{../../images/navegacion/inicio.gif}}
\docLink{../../docs_curso/contenido.html}{\includegraphics{../../images/navegacion/contenido.gif}}
\docLink{../../docs_curso/descripcion.html}{\includegraphics{../../images/navegacion/descripcion.gif}}
\docLink{../../docs_curso/profesor.html}{\includegraphics{../../images/navegacion/profesor.gif}}
& \begin{tabular}{r}
{\color {darkgray} {\small Cap. 4. Funciones}} \\ \\ \\
\end{tabular}
\docLink{../cap4/04_01.tex}{\includegraphics{../../images/navegacion/siguiente.gif}}
\end{tabular}
}

\end{quote}

\newline

\begin{flushright}
\includegraphics{../../images/interfaz/copyright.gif}
\end{flushright}
\end{document}

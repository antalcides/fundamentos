\documentclass[10pt]{article} %tipo documento y tipo letra

\definecolor{azulc}{cmyk}{0.72,0.58,0.42,0.20} % color titulo
\definecolor{naranja}{cmyk}{0.21,0.5,1,0.03} % color leccion

\def\eje{\centerline{\textbf{ EJERCICIOS.}}} % definiciones propias

\begin{document}

\begin{quote}

% inicio encabezado
{\color{gray}
\begin{tabular}{@{\extracolsep{\fill}}lcr}
\docLink{trigo4.tex}{\includegraphics{../../images/navegacion/anterior.gif}}
\begin{tabular}{l}
{\color{darkgray}\small \'{A}ngulos Especiales} \\ \\ \\
\end{tabular} &
\docLink[_top]{../../index.html}{\includegraphics{../../images/navegacion/inicio.gif}}
\docLink{../../docs_curso/contenido.html}{\includegraphics{../../images/navegacion/contenido.gif}}
\docLink{../../docs_curso/descripcion.html}{\includegraphics{../../images/navegacion/descripcion.gif}}
\docLink{../../docs_curso/profesor.html}{\includegraphics{../../images/navegacion/profesor.gif}}
& \begin{tabular}{r}
{\color{darkgray}\small Algunas Propiedades} \\ \\ \\
\end{tabular}
\docLink{trigo7.tex}{\includegraphics{../../images/navegacion/siguiente.gif}}
\\ \hline
\end{tabular}
}
%fin encabezado

%nombre capitulo
\begin{center}
\colorbox{azulc}{{\color{white} \large CAP�TULO 5}}  {\large
{\color{ azulc} NOTAS DE TRIGONOMETRIA}}
\end{center}

\newline

%nombre leccion

\colorbox{naranja}{{\color{white} \normalsize  Lecci\'on 5.5. }}
{\normalsize {\color{naranja} Funciones Trigonom�tricas de N�meros Reales}}

\newline

En muchas aplicaciones se toma como dominio de las funciones
trigonom�tricas subconjuntos de n�meros reales.

\bigskip

\textbf{Definici�n 5.5.1. } Si $t$ es un n�mero real
positivo,negativo � cero.

\bigskip

Si $t$ es positivo y $f$ es una funci�n trigonom�trica que est�
definida en $t$, $f(t)$ es la misma imagen por $f$ del �ngulo
orientado positivamente que mide $t$ radianes.

\bigskip

Si $t$ es un real negativo, $f(t)$ es la misma imagen por $f$ del
�ngulo que mide $-t$ radianes y est� orientado negativamente.

\bigskip

Igualmente: $cos t$, es el coseno del �ngulo cuya medida es $t$
radianes, orientado positivamente si $t$ es positivo. Si $t$ es
negativo, $cos t$ es el coseno del �ngulo cuya medida es $-t$
radianes y est� orientado negativamente

\bigskip

Si el lado final de $t$ no est� en el eje $y$ , $t$ es real $tan
t$ se define de la misma forma como se definieron Seno y Coseno
teniendo en cuenta si $t$ es positivo � $t$ es negativo.

\bigskip

\textbf{Ejemplo 5.9. } $\tan 30$ es la tangente del �ngulo que
mide $30$ radianes. Observe que $\tan 30 \neq \tan 30^\circ$.

\bigskip

$sen \left(-3 \right)$ es el seno del �ngulo orientado
negativamente que mide $3$ radianes.

\bigskip

$\cos \left(-\frac{\pi}{4} \right)$ es el coseno del �ngulo que
mide $\frac{\pi}{4}$ radianes y que est� orientado negativamente.

\bigskip

\colorbox{naranja}{{\color{white} \normalsize 5.5.1. }}
{\normalsize {\color{naranja} Dominio de las funciones}}

\bigskip

Hemos visto anteriormente que no importa cual sea la medida u
orientaci�n del �ngulo, es posible encontrar los valores del seno
y del coseno; as� estas funciones pueden ser definidas para
cualquier n�mero real, lo que permite afirmar que el dominio de
las funciones seno y coseno es el conjunto de n�meros reales.

\bigskip

No sucede lo mismo con la funci�n tangente. Se puede calcular
$\tan t$ s�lo si el lado final del �ngulo $t$ no est� sobre el eje
$y$, esto implica que en el dominio de la funci�n tangente no
est�n:

\bigskip

\[\dfrac{\pi }{2};\dfrac{3\pi }{2};\dfrac{5\pi }{2};\ldots-\dfrac{\pi }{2};-\dfrac{3\pi }{2};-\dfrac{5\pi }{2};\ldots\]

\bigskip

Concluimos entonces que el dominio de la funci�n tangente es:

\bigskip

\[R-\left\{\left( 2n+1\right) \dfrac{\pi }{2}:\text{ n es
entero}\right\}\]

\bigskip

El dominio de las funciones secante, cosecante y cotangente
tampoco es el conjunto de todos los n�meros reales.

\bigskip

De acuerdo con sus definiciones:

\bigskip

El dominio de la secante es :

\bigskip

\[R-\left\{ \left( 2n+1\right) \dfrac{\pi }{2}:\text{ n es entero}\right\}\]

\bigskip

El dominio de la cosecante es :

\bigskip

\[R-\left\{ n\pi :\text{ n es entero}\right\} \]

\bigskip

El dominio de la cotangente es:
\[R-\left\{ n\pi :\text{ n es entero}\right\}\]

\bigskip

\textbf{Ejemplo 5.10. } $\tan 30$ es la tangente del �ngulo que
mide $30$ radianes

\bigskip

\[\tan 30\neq \tan 30^{\circ }\]

\bigskip

$\csc 180$ existe pues $180$ no es un m�ltiplo de $\pi ,$ pero
$\csc 180^{\circ }$ no existe.

\bigskip

$sen \left( -3\right) $ es el seno del �ngulo orientado
negativamente que mide 3 radianes

\bigskip

$\cos\left(-\dfrac{\pi }{4}\right) $ es el coseno del �ngulo que
mide $\frac{\pi }{4}$ radianes y que est� orientado negativamente.

%Pie de p�gina
\newline

{\color{gray}
\begin{tabular}{@{\extracolsep{\fill}}lcr}
\hline \\
\docLink{trigo4.tex}{\includegraphics{../../images/navegacion/anterior.gif}}
\begin{tabular}{l}
{\color{darkgray}\small \'{A}ngulos Especiales} \\ \\ \\
\end{tabular} &
\docLink[_top]{../../index.html}{\includegraphics{../../images/navegacion/inicio.gif}}
\docLink{../../docs_curso/contenido.html}{\includegraphics{../../images/navegacion/contenido.gif}}
\docLink{../../docs_curso/descripcion.html}{\includegraphics{../../images/navegacion/descripcion.gif}}
\docLink{../../docs_curso/profesor.html}{\includegraphics{../../images/navegacion/profesor.gif}}
& \begin{tabular}{r}
{\color{darkgray}\small Algunas Propiedades} \\ \\ \\
\end{tabular}
\docLink{trigo7.tex}{\includegraphics{../../images/navegacion/siguiente.gif}}
\end{tabular}
}

\end{quote}

\newline

\begin{flushright}
\includegraphics{../../images/interfaz/copyright.gif}
\end{flushright}
\end{document}

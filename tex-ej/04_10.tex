\documentclass[10pt]{article} %tipo documento y tipo letra

\definecolor{azulc}{cmyk}{0.72,0.58,0.42,0.20} % color titulo
\definecolor{naranja}{cmyk}{0.21,0.5,1,0.03} % color leccion

\def\eje{\centerline{\textbf{ EJERCICIOS.}}} % definiciones propias

\begin{document}

\begin{quote}

% inicio encabezado
{\color{gray}
\begin{tabular}{@{\extracolsep{\fill}}lcr}
\docLink{04_09.tex}{\includegraphics{../../images/navegacion/anterior.gif}}
\begin{tabular}{l}
{\color{darkgray}\small Logaritmos} \\ \\ \\
\end{tabular} &
\docLink[_top]{../../index.html}{\includegraphics{../../images/navegacion/inicio.gif}}
\docLink{../../docs_curso/contenido.html}{\includegraphics{../../images/navegacion/contenido.gif}}
\docLink{../../docs_curso/descripcion.html}{\includegraphics{../../images/navegacion/descripcion.gif}}
\docLink{../../docs_curso/profesor.html}{\includegraphics{../../images/navegacion/profesor.gif}}
& \begin{tabular}{r}
{\color{darkgray}\small Operaciones de Funciones} \\ \\ \\
\end{tabular}
\docLink{04_11.tex}{\includegraphics{../../images/navegacion/siguiente.gif}}
\\ \hline
\end{tabular}
}
%fin encabezado

%nombre capitulo
\begin{center}
\colorbox{azulc}{{\color{white} \large CAP�TULO 4}}  {\large
{\color{ azulc} FUNCIONES}}
\end{center}

\newline

%nombre leccion

\colorbox{naranja}{{\color{white} \normalsize  Lecci\'on 4.10. }}
{\normalsize {\color{naranja} Las funciones $e^x$ \ y Logaritmo Natural}}

\newline

Consideremos ahora esta situaci�n:

\bigskip

Mensualmente se invierte un capital $C$. Si la tasa de inter�s $i$
es anual y se expresa como decimal, la cantidad acumulada al mes,
capital mas inter�s, es $A\left( 1+\frac{i}{12}\right) $ y si se
invierte $k$ meses, la cantidad acumulada es
$A\left(1+\frac{i}{12}\right)^{k}$. M�s generalmente, si $n$ son
los per�odos de inversi�n durante el a�o y el capital se invierte
durante $t$ a�os, el capital acumulado ser� $A\left(
t\right)=C\left( 1+\frac{i}{n}\right)^{nt}$. Esta es la f�rmula de
inter�s compuesto y es un ejemplo ilustrativo del crecimiento
exponencial.

\bigskip

Supongamos que el n�mero $n$ de per�odos se incrementa cada vez
m�s. Para algunos enteros $n$, aproximando a la octava cifra
decimal, el comportamiento de la expresi�n
$\left(1+\frac{i}{n}\right)^{n}$ tomando $i=1$ est� descrito en la
siguiente tabla

\bigskip

\begin{center}
\begin{tabular}{llcll}
$n$ & $\left( 1+\frac{1}{n}\right)^{n}$ &  & $n$ & $\left(1+\frac{1}{n}\right)^{n}$ \\
$1$ & $2$ &  & $1000000$ & $2.71828047$ \\
$10$ & $2.59374246$ & & $10000000$ & $2.71828169$ \\
$100$ & $2.70481383$ &  & $100000000$ & $2.71828181$ \\
$1000$ & $2.71692393$ &  & $1000000000$ & $2.71828183$
\end{tabular}
\end{center}

\bigskip

El valor de la expresi�n $\left(1+\frac{1}{n}\right)^{n}$ tiende a
un n�mero irracional denotado por $e$, que aparece en matem�ticas
y en f�sica y que, con 14 cifras decimales es $e \approx
2.71828182849045$. Una aproximaci�n es $2.71828$.

\bigskip

\textbf{Definici�n 4.10.1. } La {\bf funci�n exponencial natural}
es la funci�n exponencial de base $e$, $f\left( x\right)=e^{x}$.

\bigskip

La funci�n exponencial natural como toda exponencial solo toma
valores positivos y, puesto que $e>1$, es una funci�n creciente.
La funci�n $e^{-x}$ es decreciente.

\bigskip

\begin{center}
\begin{array}{cc}
\includegraphics{/imagenes/4_30.gif} &
\includegraphics{/imagenes/4_31.gif} \\
$y=e^{x}$ & $y=e^{-x}$
\end{array}
\end{center}

\bigskip

\textbf{Definici�n 4.10.2. } Si $y$ es un n�mero real que puede
escribirse en la forma $y=e^{x}$ entonces el exponente $x$ se
llama el logaritmo natural de $y$ y se denota con $\ln y$.

\bigskip

As�, como vimos en general,

\bigskip

\begin{itemize}
\item $y=e^{x}\Leftrightarrow \ln y=x$ \item $\ln
\left(e^{x}\right)=x$, $x\in \mathbf{R}$ \item $e^{\ln y}=y$,
$y\in \mathbf{R}^{+}$
\end{itemize}

\bigskip

Tambi�n de la observaciones generales tenemos que la gr�fica de la
funci�n $g\left( x\right)=\ln x$ se puede obtener reflejando la
gr�fica de la funci�n $f\left(x\right)=e^{x}$ en la recta de
ecuaci�n $y=x$. Estas son las gr�ficas de las dos funciones y de
la recta.

\bigskip

\begin{center}
\includegraphics{/imagenes/4_32.gif}
\end{center}

\bigskip

La funci�n logaritmo natural es creciente; y satisface las
siguientes condiciones.

\bigskip

\begin{enumerate}
\item $\ln x+\ln y=\ln xy$ \item $\ln x-\ln y=\ln \frac{x}{y}$
\item $\ln x^{y}=y\ln x$ \item $\log _{a}x=\frac{\ln x}{\ln a}$
\end{enumerate}

\bigskip

\textbf{Ejemplo 4.13. } \quad \begin{enumerate} \item Usando
logaritmo natural, la igualdad $e^{2t}=4+x$ se puede expresar en
la forma $2t=\ln \left( 4+x\right)$ y la igualdad $e^{4}=a$ en la
forma $4=\ln a$.

\bigskip

\item $\ln \sqrt[3]{1+e}-\ln \sqrt[3]{e+e^{2}}=\ln
\left(\sqrt[3]{\frac{1+e}{e+e^{2}}}\right) =\ln
e^{-\frac{1}{3}}=-\frac{1}{3}\ln e=-\frac{1}{3}$
\end{enumerate}

%Pie de p�gina
\newline

{\color{gray}
\begin{tabular}{@{\extracolsep{\fill}}lcr}
\hline \\
\docLink{04_09.tex}{\includegraphics{../../images/navegacion/anterior.gif}}
\begin{tabular}{l}
{\color{darkgray}\small Logaritmos} \\ \\ \\
\end{tabular} &
\docLink[_top]{../../index.html}{\includegraphics{../../images/navegacion/inicio.gif}}
\docLink{../../docs_curso/contenido.html}{\includegraphics{../../images/navegacion/contenido.gif}}
\docLink{../../docs_curso/descripcion.html}{\includegraphics{../../images/navegacion/descripcion.gif}}
\docLink{../../docs_curso/profesor.html}{\includegraphics{../../images/navegacion/profesor.gif}}
& \begin{tabular}{r}
{\color{darkgray}\small Operaciones de Funciones} \\ \\ \\
\end{tabular}
\docLink{04_11.tex}{\includegraphics{../../images/navegacion/siguiente.gif}}
\end{tabular}
}

\end{quote}

\newline

\begin{flushright}
\includegraphics{../../images/interfaz/copyright.gif}
\end{flushright}
\end{document}

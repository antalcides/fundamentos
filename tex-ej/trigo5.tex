\documentclass[10pt]{article} %tipo documento y tipo letra

\definecolor{azulc}{cmyk}{0.72,0.58,0.42,0.20} % color titulo
\definecolor{naranja}{cmyk}{0.21,0.5,1,0.03} % color leccion

\def\eje{\centerline{\textbf{ EJERCICIOS.}}} % definiciones propias

\begin{document}

\begin{quote}

% inicio encabezado
{\color{gray}
\begin{tabular}{@{\extracolsep{\fill}}lcr}
\docLink{trigo3.tex}{\includegraphics{../../images/navegacion/anterior.gif}}
\begin{tabular}{l}
{\color{darkgray}\small Funciones Trigonom�tricas} \\ \\ \\
\end{tabular} &
\docLink[_top]{../../index.html}{\includegraphics{../../images/navegacion/inicio.gif}}
\docLink{../../docs_curso/contenido.html}{\includegraphics{../../images/navegacion/contenido.gif}}
\docLink{../../docs_curso/descripcion.html}{\includegraphics{../../images/navegacion/descripcion.gif}}
\docLink{../../docs_curso/profesor.html}{\includegraphics{../../images/navegacion/profesor.gif}}
& \begin{tabular}{r}
{\color{darkgray}\small N�meros Reales} \\ \\ \\
\end{tabular}
\docLink{trigo6.tex}{\includegraphics{../../images/navegacion/siguiente.gif}}
\\ \hline
\end{tabular}
}
%fin encabezado

%nombre capitulo
\begin{center}
\colorbox{azulc}{{\color{white} \large CAP�TULO 5}}  {\large
{\color{ azulc} NOTAS DE TRIGONOMETRIA}}
\end{center}

\newline

%nombre leccion

\colorbox{naranja}{{\color{white} \normalsize  Lecci\'on 5.4. }}
{\normalsize {\color{naranja}  FUNCIONES TRIGONOM\'{E}TRICAS DE
-$\theta $}}

\newline

$\theta $ y $-\theta $ son �ngulos con la misma medida pero
difieren en la orientaci�n, con lo cual : Si $\left( x,y\right)$
pertenece al lado final de $\theta \left( x,-y\right)$ pertenecer�
al lado final de $-\theta$. Observe las figuras:

\bigskip

\begin{center}
\includegraphics{img/img19.gif}
\includegraphics{img/img20.gif}
\end{center}

\bigskip

Como $\sqrt{x^{2}+y^{2}}=\sqrt{x^{2}+\left( -y\right)^{2}}$, la
distancia del origen a cualquiera de los puntos es igual. Llamando
$r$ a esta distancia, tenemos:

\bigskip

\begin{align*}
sen\theta = \frac{y}{r}; & \quad sen\left(-\theta \right)=
-\frac{y}{r} \\
\cos\theta = \frac{x}{r}; & \quad\cos\left(-\theta \right)=
\frac{x}{r}\\
\tan\theta =\frac{y}{x}; & \quad\tan\left(-\theta \right) =
-\frac{y}{x}
\end{align*}

\bigskip

Esto es:

\bigskip

\begin{align*}
sen\theta &=-sen\left(-\theta \right)\\
\cos\theta &=\cos\left(-\theta \right)\\
\tan\theta &=-\tan\left(-\theta \right)
\end{align*}

%Pie de p�gina
\newline

{\color{gray}
\begin{tabular}{@{\extracolsep{\fill}}lcr}
\hline \\
\docLink{trigo3.tex}{\includegraphics{../../images/navegacion/anterior.gif}}
\begin{tabular}{l}
{\color{darkgray}\small Funciones Trigonom�tricas} \\ \\ \\
\end{tabular} &
\docLink[_top]{../../index.html}{\includegraphics{../../images/navegacion/inicio.gif}}
\docLink{../../docs_curso/contenido.html}{\includegraphics{../../images/navegacion/contenido.gif}}
\docLink{../../docs_curso/descripcion.html}{\includegraphics{../../images/navegacion/descripcion.gif}}
\docLink{../../docs_curso/profesor.html}{\includegraphics{../../images/navegacion/profesor.gif}}
& \begin{tabular}{r}
{\color{darkgray}\small N�meros Reales} \\ \\ \\
\end{tabular}
\docLink{trigo6.tex}{\includegraphics{../../images/navegacion/siguiente.gif}}
\end{tabular}
}

\end{quote}

\newline

\begin{flushright}
\includegraphics{../../images/interfaz/copyright.gif}
\end{flushright}
\end{document}

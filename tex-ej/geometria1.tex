\documentclass[10pt]{article} %tipo documento y tipo letra

\definecolor{azulc}{cmyk}{0.72,0.58,0.42,0.20} % color titulo
\definecolor{naranja}{cmyk}{0.21,0.5,1,0.03} % color leccion

\def\eje{\centerline{\textbf{ EJERCICIOS.}}} % definiciones propias

\begin{document}

\begin{quote}

% inicio encabezado
{\color{gray}
\begin{tabular}{@{\extracolsep{\fill}}lcr}
\docLink{../cap2/algebra14.tex}{\includegraphics{../../images/navegacion/anterior.gif}}
\begin{tabular}{l}
{\color {darkgray} {\small Cap. 2 Desigualdades}} \\ \\ \\
\end{tabular} &
\docLink[_top]{../../index.html}{\includegraphics{../../images/navegacion/inicio.gif}}
\docLink{../../docs_curso/contenido.html}{\includegraphics{../../images/navegacion/contenido.gif}}
\docLink{../../docs_curso/descripcion.html}{\includegraphics{../../images/navegacion/descripcion.gif}}
\docLink{../../docs_curso/profesor.html}{\includegraphics{../../images/navegacion/profesor.gif}}
& \begin{tabular}{r}
{\color {darkgray} {\small Algo de Historia}} \\ \\ \\
\end{tabular}
\docLink{geometria2.tex}{\includegraphics{../../images/navegacion/siguiente.gif}}
\\ \hline
\end{tabular}
}
%fin encabezado

%nombre capitulo
\begin{center}
\colorbox{azulc}{{\color{white} \large CAP�TULO 3}}  {\large
{\color{ azulc} MODULO DE GEOMETR�A}}
\end{center}

\newline

%nombre leccion

\colorbox{naranja}{{\color{white} \normalsize  Lecci\'on 3.1. }}
{\normalsize {\color{naranja} Introducci�n}}

\newline

Desde sus or�genes como una herramienta para describir y medir
figuras, la geo\-me\-tr�a ha crecido en sus teor�as y m�todos con
los cuales se pueden construir y estudiar modelos tanto del mundo
f�sico, como de otros fen�menos del mundo real. El estudio de las
magnitudes, por ejemplo, constituye parte fundamental de la vida
cotidiana y es b�sico en las ciencias naturales. Continuamente nos
encontramos, adem�s, con representaciones planas de objetos
espaciales que aparecen en los dibujos y en las im�genes y estas
deben ser analizadas y usadas para construir objetos. La
geo\-me\-tr�a es una valiosa herramienta tanto para construir
representaciones visuales de conceptos y procedimientos de otros
dominios de las matem�ticas y de otras ciencias, como para
desarrollar pensamiento y comprensi�n.

\bigskip

La geo\-me\-tr�a es un punto de encuentro de la matem�tica como
teor�a y la matem�tica como fuente de modelos. Es en la actualidad
una herramienta tanto en las aplicaciones tradicionales como en
las innovativas: gr�ficos computarizados, procesamiento de
im�genes, patrones de reconocimiento, rob�tica e investigaci�n de
operaciones. Es pues una herramienta manipulativa, intuitiva,
deductiva y anal�tica.

\bigskip

Pero, �c�mo fueron realmente los inicios de esta importante rama
de la ma\-te\-m�\-ti\-ca?, rese�aremos a continuaci�n algunos
apartes de las primeras fases de su desarrollo.

%Pie de p�gina
\newline

{\color{gray}
\begin{tabular}{@{\extracolsep{\fill}}lcr}
\hline \\
\docLink{../cap2/algebra14.tex}{\includegraphics{../../images/navegacion/anterior.gif}}
\begin{tabular}{l}
{\color {darkgray} {\small Cap. 2 Desigualdades}} \\ \\ \\
\end{tabular} &
\docLink[_top]{../../index.html}{\includegraphics{../../images/navegacion/inicio.gif}}
\docLink{../../docs_curso/contenido.html}{\includegraphics{../../images/navegacion/contenido.gif}}
\docLink{../../docs_curso/descripcion.html}{\includegraphics{../../images/navegacion/descripcion.gif}}
\docLink{../../docs_curso/profesor.html}{\includegraphics{../../images/navegacion/profesor.gif}}
& \begin{tabular}{r}
{\color {darkgray} {\small Algo de Historia}} \\ \\ \\
\end{tabular}
\docLink{geometria2.tex}{\includegraphics{../../images/navegacion/siguiente.gif}}
\end{tabular}
}

\end{quote}

\newline

\begin{flushright}
\includegraphics{../../images/interfaz/copyright.gif}
\end{flushright}
\end{document}

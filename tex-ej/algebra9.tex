\documentclass[10pt]{article} %tipo documento y tipo letra

\definecolor{azulc}{cmyk}{0.72,0.58,0.42,0.20} % color titulo
\definecolor{naranja}{cmyk}{0.21,0.5,1,0.03} % color leccion

\def\eje{\centerline{\textbf{ EJERCICIOS.}}} % definiciones propias

\begin{document}

\begin{quote}

% inicio encabezado
{\color{gray}
\begin{tabular}{@{\extracolsep{\fill}}lcr}
\docLink{algebra8.tex}{\includegraphics{../../images/navegacion/anterior.gif}}
\begin{tabular}{l}
{\color{darkgray}\small Multiplicaci�n} \\ \\ \\
\end{tabular} &
\docLink[_top]{../../index.html}{\includegraphics{../../images/navegacion/inicio.gif}}
\docLink{../../docs_curso/contenido.html}{\includegraphics{../../images/navegacion/contenido.gif}}
\docLink{../../docs_curso/descripcion.html}{\includegraphics{../../images/navegacion/descripcion.gif}}
\docLink{../../docs_curso/profesor.html}{\includegraphics{../../images/navegacion/profesor.gif}}
& \begin{tabular}{r}
{\color{darkgray}\small Ra�ces de Polinomios} \\ \\ \\
\end{tabular}
\docLink{algebra92.tex}{\includegraphics{../../images/navegacion/siguiente.gif}}
\\ \hline
\end{tabular}
}
%fin encabezado

%nombre capitulo
\begin{center}
\colorbox{azulc}{{\color{white} \large CAP�TULO 2}}  {\large
{\color{ azulc} FUNDAMENTOS DE ALGEBRA}}
\end{center}

\newline

%nombre leccion

\colorbox{naranja}{{\color{white} \normalsize  Lecci\'on 2.9. }}
{\normalsize {\color{naranja} Divisi�n de Polinomios}}

\newline

En cuanto a la divisi�n tenemos el principio de la divisi�n
euclidiana de acuerdo con el cu�l, dados dos polinomios $p(x)$ y
$q(x)$ con $q(x)\neq 0$, al dividir $p(x)$ por $q(x)$ se obtienen
dos polinomios: un cociente $c(x)$ y un residuo $r(x)$ tales que

\bigskip

\begin{align*}
p(x)&=q(x)c(x)+r(x) \quad\text{y} \\
r(x)&=0\quad \text{o}\quad r(x)\neq 0\quad \text{y su grado es
menor que el grado de $q(x)$}
\end{align*}

\bigskip

Los polinomios $c(x)$ y $r(x)$ son �nicos. Estos polinomios se
obtienen mediante el proceso llamado divisi�n larga.

\bigskip

Consideremos por ejemplo,

\begin{align*}
p(x)&=4+3x-2x^{2}+3x^{3}+6x^{4}\\
q(x)&=1+x+4x^{2}
\end{align*}

\bigskip

O escritos en orden descendente de los exponentes de $x$,

\bigskip

\begin{align*}
p(x)&=6x^{4}+3x^{3}-2x^{2}+3x+4\\
q(x)&=4x^{2}+x+1
\end{align*}

\bigskip

Al dividir $p(x)$ por $q(x)$ obtenemos

\bigskip

\begin{center}
\renewcommand{\arraystretch}{2.2}
\renewcommand{\tabcolsep}{0cm}
\begin{tabular}{ccccccccc}
$6x^{4}$ & $+3x^{3}$ & $-2x^{2}$ & $+3x$ & $+4$ & \qquad\qquad &
$4x^{2}$ & $+x$ & $+1$\\ \cline{7-9} $-6x^{4}$ &
$-\dfrac{3}{2}x^{3}$ & $-\dfrac{3}{2}x^{2}$ &&&&
$\dfrac{3}{2}x^{2}$ & $+\dfrac{3}{8}x$ & $-\dfrac{31}{32}$\\
\cline{1-3} & $\dfrac{3}{2}x^{3}$ & $-\dfrac{7}{2}x^{2}$ &&&&&&\\
& $-\dfrac{3}{2}x^{3}$ & $-\dfrac{3}{8}x^{2}$ & $-\dfrac{3}{8}x$
&&&&& \\ \cline{2-4} && $-\dfrac{31}{8}x^{2}$ & $+\dfrac{21}{8}x$
&&&&& \\ &&
$\dfrac{31}{8}x^{2}$ & $+\dfrac{31}{8}x$ & $+\dfrac{31}{32}$ &&&& \\
\cline{3-5} &&& $\dfrac{115}{32}x$ & $+\dfrac{159}{32}$ &&&&
\end{tabular}
\end{center}

\bigskip

de donde el cociente es

\bigskip

\[c(x)=\frac{3}{2}x^{2}+\frac{3}{8}x-\frac{31}{32}\]

\bigskip

y el residuo es

\bigskip

\[r(x)=\frac{115}{32}x+\frac{159}{32}\]

\bigskip

tenemos efectivamente que

\bigskip

\begin{align*}
q(x)c(x)+r(x)&=\left( 4x^{2}+x+1\right) \left(
\frac{3}{2}x^{2}+\frac{3}{8}x-\frac{31}{32}\right) +\left(
\frac{115}{32}x+\frac{159}{32}\right)\\
&=\left( 6x^{4}+3x^{3}-2x^{2}-\frac{19}{32}x- \frac{31}{32}\right)
+\left( \frac{115}{32}x+\frac{159}{32}\right)\\
&=6x^{4}+3x^{3}-2x^{2}+3x+4=p(x)
\end{align*}

\bigskip

Veamos otros casos:

\bigskip

\begin{enumerate}
\item Sean

\begin{align*}
p(x)&=x^{2}+x+1\\
q(x)&=2x^{3}-x^{2}+x-1
\end{align*}

\bigskip

Tenemos la igualdad

\bigskip

\[x^{2}+x+1=\left( 2x^{3}-x^{2}+x-1\right) .0+\left(x^{2}+x+1\right)\]

\bigskip

es decir, el cociente es el polinomio $0$ y el residuo es el
polinomio $p(x)$.

\bigskip

Este ejemplo ilustra el caso en que $p(x)=0$ o el grado de $p(x)$
es menor que el grado de $q(x)$: como cociente basta tomar 0 y
como residuo el mismo polinomio que se est� dividiendo.

\item
\begin{align*}
p(x)&=5x^{4}+3x^{3}-x^{2}+3x+2\\
q(x)&=x-2
\end{align*}

\bigskip

\begin{center}
\renewcommand{\arraystretch}{1.5}
\renewcommand{\tabcolsep}{0cm}
\begin{tabular}{*{10}{c}}
$5x^{4}$ & $+3x^{3}$ & $-x^{2}$ & $+3x$ & $+2$ & \qquad\qquad &
$x$ & $-2$ &&\\ \cline{7-10} $-5x^{4}$ & $+10x^{3}$ &&&&& $5x^{3}$
& $+13x^{2}$ & $+25x$ & $+53$ \\ \cline{1-2} & $13x^{3}$ &&&&&&&& \\
& $-13x^{3}$ & $+26x^{2}$ &&&&&&& \\ \cline{2-3} && $25x^{2}$
&&&&&&& \\ && $-25x^{2}$ & $+50x$ &&&&&&\\ \cline{3-4} &&& $53x$
&&&&&&\\ &&& $-53x$ & $+106$ &&&&&\\ \cline{4-5} &&&& $108$ &&&&&
\end{tabular}
\end{center}

\bigskip

As�, el cociente es $c(x)=5x^{3}+13x^{2}+25x+53$ el residuo es
$r(x)=108$.

\bigskip

Tambi�n en este caso,

\bigskip

\begin{align*}
q(x)c(x)+r(x)&=(x-2)(5x^{3}+13x^{2}+25x+53)+108\\
&=5x^{4}+(13-2\times 5)x^{3}+(25-2\times 13)x^{2}\\
&\quad+(53-2\times 25)x+(108-2\times 53)\\
&=5x^{4}+3x^{3}-x^{2}+3x+2
\end{align*}

\bigskip

Observemos los dos �ltimos renglones: el coeficiente principal de
$p(x)$ , que es $5$, y el coeficiente principal del cociente
coinciden puesto que el coeficiente principal de $q(x)$ es 1. Los
dem�s coeficientes de $p(x)$ se obtienen al realizar las
operaciones $q(x)c(x)+r(x)$. As� $3$, el coeficiente de $x^{3}$ en
$p(x),$ es igual a $13$, que es el coeficiente de $x^{2}$ en
$c(x)$, menos 2 veces $5$, que es el coeficiente de $x^{3}$ de
$c(x).$ El coeficiente de $x^{2}$ en $p(x),$ esto es $-1,$es igual
a $25$, que es el coeficiente de $x$\ en $c(x)$, menos 2 veces
$13$ que es el coeficiente de $x^{2}$ en $c(x).$ El coeficiente de
$x$ en $p(x),$es decir $3$, es igual a $53$, que es el t�rmino
independiente de $c(x)$, menos 2 veces $25$, que es el coeficiente
de $x$ en $c(x)$. Finalmente, el t�rmino independiente de $p(x)$,
es decir $2$, es igual a $106$, que es el residuo $r(x)$, menos 2
veces $53$, que es el t�rmino independiente de $c(x)$.

\bigskip

Esto nos permite expresar los coeficientes del cociente as�:

\bigskip

El coeficiente principal, coeficiente de $x^{3}$, es igual al
coeficiente principal de $p(x)$.

\bigskip

El coeficiente de $x^{2}$ en $c(x)$ es igual al coeficiente de
$x^{3}$ en $p(x)$ m�s 2 veces el coeficiente de $x^{3}$ en $c(x)$.

\bigskip

El coeficiente de $x$ en $c(x)$ es igual al coeficiente de $x^{2}$
en $p(x)$ m�s 2 veces el coeficiente de $x^{2}$ en $c(x)$.

\bigskip

El t�rmino independiente de $c(x)$ es igual al coeficiente de $x$
en $p(x)$ m�s 2 veces el coeficiente de $x$ en $c(x)$.

\bigskip

El residuo $r(x)$ es igual al t�rmino independiente de $p(x)$ m�s
2 veces el t�rmino independiente de $c(x)$.

\end{enumerate}

\bigskip

Tambi�n en este caso podemos hacer un an�lisis m�s general: al
dividir un polinomio $p(x)$ de grado n por uno de grado 1 de la
forma $x-d$, se obtiene un cociente cuyo grado es $n-1$ y un
residuo que es 0 o un polinomio de grado 0, es decir, una
constante.

\bigskip

Si
\begin{align*}
p(x)&=a_{n}x^{n}+a_{n-1}x^{n-1}+\cdots+a_{0}\\
c(x)&=b_{n-1}x^{n-1}+b_{n-2}x^{n-2}+\cdots+b_{0}\qquad\text{y}\\
r(x)&=k
\end{align*}

\bigskip

la igualdad

\bigskip

\[p(x)=(x-d)c(x)+r(x)\]

\bigskip

es decir,

\bigskip

\begin{align*}
a_{n}x^{n}+a_{n-1}x^{n-1}+\cdots+a_{0}&=(x-d)(b_{n-1}x^{n-1}+\cdots+b_{0})+k\\
&=b_{n-1}x^{n}+(b_{n-2}-db_{n-1})x^{n-1}\\
&\quad+(b_{n-3}-db_{n-2})x^{n-2}+\cdots\\
&\quad+(b_{0}-db_{1})x+(k-db_{0})
\end{align*}

\bigskip

implica

\begin{align*}
a_{n}&=b_{n-1}\\
a_{n-1}&=b_{n-2}-db_{n-1}\\
a_{n-2}&=b_{n-3}-db_{n-2}\\
a_{1}&=b_{0}-db_{1}\\
a_{0}&=k-db_{0} \\
\intertext{es decir,}\\
b_{n-1}&=a_{n}\\
b_{n-2}&=a_{n-1}+db_{n-1}\\
b_{n-3}&=a_{n-2}+db_{n-2}\\
&\vdots\\
b_{0}&=a_{1}+db_{1}\\
k&=a_{0}+db_{0}\\
\end{align*}

\bigskip

As�, tomando una fila formada por todos los coeficientes de $p(x)$
en orden $a_{n}a_{n-1}\ldots a_{0}$ y sumando $0$ a $a_{n}$ y as�
sucesivamente a cada uno de los dem�s coeficientes
$a_{n-1},a_{n-2},\ldots,a_{0}$, el producto de $d$ por la �ltima
suma, se obtienen, en orden, los coeficientes del cociente y el
residuo.

\bigskip

\begin{center}
\renewcommand{\arraystretch}{1.5}
\begin{tabular}{*{6}{c}}
$a_{n}$ & $a_{n-1}$ & $a_{n-2}$ & $\cdots$ & $a_{1}$ & $a_{0}$\\
$0$ & $db_{n-1}$ & $db_{n-2}$ & $\cdots$ & $db_{1}$ & $db_{0}$\\
\hline $b_{n-1}$ & $b_{n-2}$ & $b_{n-3}$ & $\cdots$ & $b_{0}$ &
$k$
\end{tabular}
\end{center}

\bigskip

En esto consiste la \textbf{divisi�n sint�tica}.

\bigskip

As�, para obtener el cociente de dividir
$7x^{5}+4x^{3}+3x^{2}-x-6$ por $x+1$, tenemos en cuenta que
$x+1=x-(-1)$ procedemos as�:

\bigskip

\begin{center}
\renewcommand{\arraystretch}{1.5}
\begin{tabular}{*{6}{r}}
7 & 0 & $-4$ & 3 & $-1$ & $-6$\\
0 & $-7$ & 7 & $-3$ & 0 & 1\\ \hline 7 & $-7$ & 3 & 0 & $-1$ &
$-5$
\end{tabular}
\end{center}

\bigskip

El cociente es entonces $c(x)=7x^{4}-7x^{3}+3x^{2}-1$ y el residuo
es $r(x)=-5$.

\bigskip

\bigskip Volviendo a la discusi�n general acerca de la divisi�n, cuando, el
divisor es de forma $x-d$, el residuo es una contante $k$ y

\bigskip

\[p(x)=(x-d)c(x)+k\]

\bigskip

Entonces, al calcular $p(x)$ en $d$ (es decir al reemplazar $x$
por $d$ en el polinomio y realizar las operaciones indicadas),
obtenemos:

\bigskip

\begin{align*}
p(d)&=(d-d)c(d)+k\\
&=0+k=k
\end{align*}

\bigskip

As� hemos demostrado el \textbf{TEOREMA DEL RESIDUO}:

\bigskip

\textbf{Teorema 2.9.1. (Teorema del Residuo)} El residuo de
dividir un polinomio $p(x)$ por el polinomio $x-d$ es $p(d)$.

\bigskip

\textbf{Ejemplo 2.27. } \quad\begin{enumerate} \item Al evaluar el
polinomio
\[p(x)=2+3x-x^{2}+3x^{3}+5x^{4}\quad \text{en} \quad x=2\]

\bigskip

obtenemos

\bigskip

\begin{align*}
p(2)&=2+3\ast 2-2^{2}+3\ast 2^{3}+5-2^{4}\\
&=108
\end{align*}

\bigskip

y este es el residuo de dividir $p(x)$ por $x-2$ como fue visto
antes.

\bigskip

\item Podemos asegurar que el residuo de dividir
$p(x)=-6-x+3x^{2}+4x^{3}+7x^{5}$ por $x+1$ es $-5=p(-1)$. Este
resultado tambi�n fue hallado por divisi�n sint�tica antes.
\end{enumerate}

\bigskip

Como un caso particular del teorema del residuo aparece el
\textbf{TEOREMA DEL FACTOR:}

\bigskip

\textbf{Teorema 2.9.2. (Teorema del Factor)} El polinomio $x-d$ es
factor del polinomio $p(x)$ si y solo si $p(d)=0$.

\bigskip

\textit{Demostraci�n : } Como $p(x)=(x-d)c(x)+p(d)$, $x-d$ es
factor del $p(x)$ si y solo si $p(d)=0$.

\bigskip

A continuaci�n nos ocuparemos de este caso.

%Pie de p�gina
\newline

{\color{gray}
\begin{tabular}{@{\extracolsep{\fill}}lcr}
\hline \\
\docLink{algebra8.tex}{\includegraphics{../../images/navegacion/anterior.gif}}
\begin{tabular}{l}
{\color{darkgray}\small Multiplicaci�n} \\ \\ \\
\end{tabular} &
\docLink[_top]{../../index.html}{\includegraphics{../../images/navegacion/inicio.gif}}
\docLink{../../docs_curso/contenido.html}{\includegraphics{../../images/navegacion/contenido.gif}}
\docLink{../../docs_curso/descripcion.html}{\includegraphics{../../images/navegacion/descripcion.gif}}
\docLink{../../docs_curso/profesor.html}{\includegraphics{../../images/navegacion/profesor.gif}}
& \begin{tabular}{r}
{\color{darkgray}\small Ra�ces de Polinomios} \\ \\ \\
\end{tabular}
\docLink{algebra92.tex}{\includegraphics{../../images/navegacion/siguiente.gif}}
\end{tabular}
}

\end{quote}

\newline

\begin{flushright}
\includegraphics{../../images/interfaz/copyright.gif}
\end{flushright}
\end{document}

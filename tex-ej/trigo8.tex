\documentclass[10pt]{article} %tipo documento y tipo letra

\definecolor{azulc}{cmyk}{0.72,0.58,0.42,0.20} % color titulo
\definecolor{naranja}{cmyk}{0.21,0.5,1,0.03} % color leccion

\def\eje{\centerline{\textbf{ EJERCICIOS.}}} % definiciones propias

\begin{document}

\begin{quote}

% inicio encabezado
{\color{gray}
\begin{tabular}{@{\extracolsep{\fill}}lcr}
\docLink{trigo7.tex}{\includegraphics{../../images/navegacion/anterior.gif}}
\begin{tabular}{l}
{\color{darkgray}\small Algunas Propiedades} \\ \\ \\
\end{tabular} &
\docLink[_top]{../../index.html}{\includegraphics{../../images/navegacion/inicio.gif}}
\docLink{../../docs_curso/contenido.html}{\includegraphics{../../images/navegacion/contenido.gif}}
\docLink{../../docs_curso/descripcion.html}{\includegraphics{../../images/navegacion/descripcion.gif}}
\docLink{../../docs_curso/profesor.html}{\includegraphics{../../images/navegacion/profesor.gif}}
& \begin{tabular}{r}
{\color{darkgray}\small Expresiones con Seno y Coseno} \\ \\ \\
\end{tabular}
\docLink{trigo9.tex}{\includegraphics{../../images/navegacion/siguiente.gif}}
\\ \hline
\end{tabular}
}
%fin encabezado

%nombre capitulo
\begin{center}
\colorbox{azulc}{{\color{white} \large CAP�TULO 5}}  {\large
{\color{ azulc} NOTAS DE TRIGONOMETRIA}}
\end{center}

\newline

%nombre leccion

\colorbox{naranja}{{\color{white} \normalsize  Lecci\'on 5.7. }}
{\normalsize {\color{naranja} \'{A}ngulos de Referencia}}

\newline

Para el c�lculo de los valores de las funciones trigonom�tricas de
cualquier n�mero (o �ngulo), basta con conocer las que
corresponden a un n�mero que est� en el intervalo
$\left(0,\dfrac{\pi }{2}\right)$, (�ngulos agudos).

\bigskip

Para realizar este proceso se utiliza un �ngulo llamado �ngulo de
referencia.

\bigskip

\textbf{Definici�n 5.7.1. } Un �ngulo de referencia $\theta _{r}$
para $\theta $, es el �ngulo agudo que forman el lado final de
$\theta $ y el eje $x$.

\bigskip

\begin{center}
\includegraphics{img/angulo_ref1.gif}
\includegraphics{img/angulo_ref2.gif}
\end{center}

\bigskip

\begin{center}
\includegraphics{img/angulo_ref3.gif}
\includegraphics{img/angulo_ref4.gif}
\end{center}

\bigskip

Para calcular los valores de las funciones de un �ngulo no
cuadrantal $\left( \theta \right)$, usando los �ngulos de
referencia, se hallan las que corresponden al �ngulo de referencia
y se hace la relaci�n teniendo en cuenta el cuadrante al cual
pertenece el �ngulo dado.

\bigskip

\textbf{Ejemplo 5.13. } Si $\theta =135^{\circ }$

\bigskip

\begin{center}
\includegraphics{img/angulo_ref5.gif}
\end{center}

\bigskip

\begin{align*}
\theta _{r}&=180^{\circ }-135^{\circ }\\
\theta _{r}&=45^{\circ }
\end{align*}

\bigskip

\[sen 135^{\circ }=sen45^{\circ },\quad \cos 135^{\circ }=-\cos 45^{\circ }, \quad\tan 135^{\circ}=-\tan 45^{\circ}\]

\bigskip

\textbf{Ejemplo 5.14. } Si $\theta =\dfrac{7\pi }{6}$

\bigskip

\begin{center}
\includegraphics{img/angulo_ref6.gif}
\end{center}

\bigskip

\begin{align*}
\theta _{r}&=\dfrac{7\pi }{6}-\pi,\\
\theta _{r}&=\dfrac{\pi }{6}
\end{align*}

\bigskip

\[sen\dfrac{7\pi}{6}=-sen\dfrac{\pi}{6}, \quad\cos\dfrac{7\pi }{6}=-\cos\dfrac{\pi }{6}, \quad\tan\dfrac{7\pi }{6}=\tan\dfrac{\pi}{6}\]

\bigskip

\textbf{Ejemplo 5.15. } Si $t =3.5$

\bigskip

\begin{center}
\includegraphics{img/angulo_ref7.gif}
\end{center}

\bigskip

Como $\pi < 3.5 < \dfrac{3\pi }{2}$, $\theta _{r}=3.5-\pi $

\bigskip

\[sen t=-sen\left( 3.5-\pi \right), \quad \cos t=-\cos\left( 3.5-\pi \right), \quad \tan t =\tan\left(3.5-\pi \right)\]

\bigskip

\textbf{Ejemplo 5.16. } Si $t=\dfrac{5\pi }{3}$,

\bigskip

\begin{center}
\includegraphics{img/angulo_ref8.gif}
\end{center}

\bigskip

\begin{align*}
\theta _{r}&=2\pi -5\dfrac{\pi }{3}\\
\theta _{r}&=\dfrac{\pi }{3}
\end{align*}

\bigskip

\[sen t = -sen\dfrac{\pi }{3}, \quad \cos t = \cos\dfrac{\pi }{3}, \quad \tan t = -\tan\dfrac{\pi}{3}\]

%Pie de p�gina
\newline

{\color{gray}
\begin{tabular}{@{\extracolsep{\fill}}lcr}
\hline \\
\docLink{trigo7.tex}{\includegraphics{../../images/navegacion/anterior.gif}}
\begin{tabular}{l}
{\color{darkgray}\small Algunas Propiedades} \\ \\ \\
\end{tabular} &
\docLink[_top]{../../index.html}{\includegraphics{../../images/navegacion/inicio.gif}}
\docLink{../../docs_curso/contenido.html}{\includegraphics{../../images/navegacion/contenido.gif}}
\docLink{../../docs_curso/descripcion.html}{\includegraphics{../../images/navegacion/descripcion.gif}}
\docLink{../../docs_curso/profesor.html}{\includegraphics{../../images/navegacion/profesor.gif}}
& \begin{tabular}{r}
{\color{darkgray}\small Expresiones con Seno y Coseno} \\ \\ \\
\end{tabular}
\docLink{trigo9.tex}{\includegraphics{../../images/navegacion/siguiente.gif}}
\end{tabular}
}

\end{quote}

\newline

\begin{flushright}
\includegraphics{../../images/interfaz/copyright.gif}
\end{flushright}
\end{document}

\documentclass[10pt]{article} %tipo documento y tipo letra

\definecolor{azulc}{cmyk}{0.72,0.58,0.42,0.20} % color titulo
\definecolor{naranja}{cmyk}{0.21,0.5,1,0.03} % color leccion

\def\eje{\centerline{\textbf{ EJERCICIOS.}}} % definiciones propias

\begin{document}

\begin{quote}

% inicio encabezado
{\color{gray}
\begin{tabular}{@{\extracolsep{\fill}}lcr}
\docLink{trigo12.tex}{\includegraphics{../../images/navegacion/anterior.gif}}
\begin{tabular}{l}
{\color{darkgray}\small Ecuaciones Trigonom�tricas} \\ \\ \\
\end{tabular} &
\docLink[_top]{../../index.html}{\includegraphics{../../images/navegacion/inicio.gif}}
\docLink{../../docs_curso/contenido.html}{\includegraphics{../../images/navegacion/contenido.gif}}
\docLink{../../docs_curso/descripcion.html}{\includegraphics{../../images/navegacion/descripcion.gif}}
\docLink{../../docs_curso/profesor.html}{\includegraphics{../../images/navegacion/profesor.gif}}
& \begin{tabular}{r}
{\color{darkgray}\small \'{A}ngulos M�ltiples} \\ \\ \\
\end{tabular}
\docLink{trigo14.tex}{\includegraphics{../../images/navegacion/siguiente.gif}}
\\ \hline
\end{tabular}
}
%fin encabezado

%nombre capitulo
\begin{center}
\colorbox{azulc}{{\color{white} \large CAP�TULO 5}}  {\large
{\color{ azulc} NOTAS DE TRIGONOMETRIA}}
\end{center}

\newline

%nombre leccion

\colorbox{naranja}{{\color{white} \normalsize  Lecci\'on 5.12. }}
{\normalsize {\color{naranja} F�rmulas de Suma y Resta}}

\newline

En ocasiones se encuentran expresiones de la forma: $\cos
\left(\alpha +\beta \right)$, $sen\left( \alpha -\beta
\right)$,\ldots, y es importante poder escribirlas directamente en
t�rminos de $sen\alpha$, $sen\beta$, $\cos\alpha$, $\cos\beta$.

\bigskip

Mediante construcciones geom�tricas, la definici�n de distancia y
el uso de la identidades fundamentales se puede demostrar que

\bigskip

\[\cos \left( \alpha - \beta \right)=\cos \alpha \cos \beta + sen \alpha sen \beta,\]

\bigskip

y a partir de ella, determinar el valor del coseno de la suma y
otros resultados.

\bigskip

Las igualdades son v�lidas para cualquier tipo de �ngulos y su
medida puede estar dada en grados sexagesimales o en radianes.
As�:

\bigskip

La f�rmula para determinar el coseno de la suma se encuentra a
partir de la anterior, expresando a $s+t$ como $s-(-t)$:

\bigskip

\[\cos\left( s+t \right)= \cos\left(s - \left(-t\right)\right)=\cos s \cos \left( -t\right) + sen s sen \left(-t
\right)\]

\bigskip

Teniendo en cuenta que $\cos \left( -t \right) =\cos t$ y que $sen
\left( -t \right) =-sen t$:

\bigskip

\[\cos \left( s+t \right) =\cos s \cos t -sen s sen t\]

\bigskip

Usando estas identidades podemos hallar el seno y el coseno del
complemento de un �ngulo:

\bigskip

\[\cos\left( \dfrac{\pi }{2}-\alpha \right) =\cos\dfrac{\pi}{2}\cos \alpha +sen\dfrac{\pi}{2}sen\alpha.\]

\bigskip

Como $\cos\dfrac{\pi }{2}=0$ y $sen\dfrac{\pi }{2}=1$

\bigskip

\[\cos \left( \frac{\pi }{2}-\alpha \right) =sen \alpha\]

\bigskip

\begin{align*}
\cos \alpha  &= \cos \left( \frac{\pi }{2}-\left(
\frac{\pi}{2}-\alpha \right) \right)\\
\cos \alpha  &= \cos \frac{\pi }{2}\cos\left( \frac{\pi}{2}-\alpha
\right) + sen \frac{\pi}{2}sen\left(\frac{\pi}{2}-\alpha\right)\\
\cos \alpha  &= sen \left( \frac{\pi }{2}-\alpha \right)
\end{align*}

\bigskip

\textbf{Conclusi�n}: El seno de un �ngulo es el coseno de su
complemento y el coseno de un �ngulo es el seno de su complemento.

\bigskip

Haciendo uso de este resultado podemos encontrar el seno de la
suma y de la diferencia de dos �ngulos.

\bigskip

\[sen\left( \alpha +\beta \right) =\cos\left(\frac{\pi }{2}-\left( \alpha +\beta \right) \right) =\cos\left(\left( \frac{\pi }{2}-\alpha \right) -\beta\right).\]

\bigskip

Se utiliza la identidad para el coseno de la diferencia:

\bigskip

\[sen\left( \alpha +\beta \right) =\cos\left(\frac{\pi }{2}-\alpha \right) \cos\beta +sen\left( \frac{\pi}{2}-\alpha \right) sen\beta\]

\bigskip

Nuevamente a partir de la conclusi�n encontrada:

\bigskip

\bigskip

\[sen \left( \alpha +\beta \right) =sen\alpha\cos\beta +\cos\alpha sen\beta\]

\bigskip

Mediante un razonamiento similar al que se hizo para el caso del
coseno se puede hallar el seno de la diferencia:

\bigskip

\begin{align*}
sen \left( \alpha -\beta \right) &=sen\left( \alpha
+\left(-\beta \right) \right) \\
&=sen\alpha \cos\left( -\beta \right) +\cos\alpha
sen\left(-\beta \right)\\ \\
sen\left( \alpha -\beta \right)& =sen\alpha \cos\beta -sen\beta
\cos\alpha
\end{align*}

\bigskip

$\tan \left( \alpha \pm \beta \right) $ se obtiene haciendo uso de
la identidad $\tan t = \dfrac{sen t}{\cos t}$

\bigskip

\begin{align*}
\tan\left( \alpha +\beta \right) &=\dfrac{\tan\alpha +\tan\beta
}{1-\tan\alpha \tan\beta };\\
\tan\left( \alpha -\beta \right) &=\dfrac{\tan\alpha -\tan\beta
}{1+\tan\alpha \tan\beta }
\end{align*}

\bigskip

\textbf{Ejemplo 5.32. } Como $\dfrac{11\pi }{12}=\dfrac{2\pi
}{3}+\dfrac{\pi }{4}$. A partir de los valores de seno y coseno de
$\dfrac{2\pi }{3}$ y $\dfrac{\pi }{4}$, se puede hallar $sen
\dfrac{11\pi }{12}$, $\cos \dfrac{11\pi }{12}$ y $\tan \left(
\frac{\pi }{12}\right)$

\bigskip

\begin{align*}
sen\left( \dfrac{11\pi }{12}\right) &=sen\left( \dfrac{2\pi
}{3}+\dfrac{\pi }{4}\right)\\
&=sen\dfrac{2\pi}{3}\cos\dfrac{\pi}{4}+\cos\dfrac{2\pi}{3}sen\dfrac{\pi}{4}\\
\\
sen \left( \dfrac{11\pi }{12}\right)
&=\dfrac{\sqrt{3}}{2}\dfrac{\sqrt{2}}{2}+\left(-\dfrac{1}{2}\right)
\left( \dfrac{\sqrt{2}}{2}\right)\\
&=\dfrac{\sqrt{6}}{4}-\dfrac{\sqrt{2}}{4}
\end{align*}

\bigskip

\begin{align*}
\cos\dfrac{11\pi}{12}&=\cos\dfrac{2\pi}{3}\cos\dfrac{\pi}{4}-
sen\dfrac{2\pi}{3}sen\dfrac{\pi}{4}\\
&=\left( -\frac{1}{2}\right) \left( \frac{\sqrt{2}}{2}\right) -
\left( \frac{\sqrt{3}}{2}\right) \left(
\frac{\sqrt{2}}{2}\right)\\
&=\left( -\dfrac{\sqrt{2}}{4}\right)
-\left(\dfrac{\sqrt{6}}{4}\right)\\
&=-\dfrac{\left(\sqrt{2}+\sqrt{6}\right)}{4}\\
&=-\frac{\sqrt{2}}{4}\left(1+\sqrt{3}\right)
\end{align*}

\bigskip

\begin{align*}
\tan \left( \frac{11\pi }{12}\right) &=\dfrac{\tan\dfrac{2\pi}{3}
+\tan\dfrac{\pi}{4}}{1-\tan\dfrac{2\pi}{3}\tan\dfrac{\pi}{4}}\\
&=-\dfrac{\sqrt{3}+1}{ 1+\sqrt{3}}
\end{align*}

\bigskip

\textbf{Ejemplo 5.33. } Si $\alpha $ es un �ngulo en el primer
cuadrante con $\cos \alpha =\frac{4}{5}$ y $\beta $ es un �ngulo
en el segundo cuadrante con $sen \beta =\frac{12}{13}$. Eval�e
$sen\left( \alpha +\beta\right)$, $\cos \left( \alpha +\beta
\right)$ y $\tan\left(\alpha +\beta \right)$. Determine el
cuadrante de $\alpha +\beta$.

\bigskip

\emph{Soluci�n}:

\bigskip

$sen \alpha =\pm \sqrt{1-\frac{16}{25}}$. Como $\alpha $ es un
�ngulo de primer cuadrante:

\bigskip

\begin{align*}
sen \alpha &=\sqrt{1-\frac{16}{25}}=\frac{3}{5}\\
\tan \alpha &=\dfrac{\frac{3}{5}}{\frac{4}{5}}=\frac{3}{4}\\ \\
\cos^{2}\beta &=1-\frac{144}{169}=\frac{25}{169}\\
\cos \beta &=\pm \sqrt{\frac{25}{169}}=\pm \frac{5}{13}
\end{align*}

\bigskip

Como $\beta $ es un �ngulo de segundo cuadrante:

\bigskip

\begin{align*}
\cos\beta &=-\frac{5}{13}\\
\tan \beta &=\dfrac{\frac{12}{13}}{-\frac{5}{13}}=-\frac{12}{5}
\end{align*}

\bigskip

Entonces:

\bigskip

\begin{align*}
sen\left( \alpha +\beta \right) &=sen\alpha \cos\beta +sen\beta
\cos\alpha\\
&=\left( \frac{3}{5}\right) \left( -\frac{5}{13}\right) +\left(
\frac{12}{13} \right) \left( \frac{4}{5}\right)\\
sen\left( \alpha +\beta \right) &=\frac{33}{65}
\end{align*}

\bigskip

\begin{align*}
\cos\left( \alpha +\beta \right) &=\cos\alpha \cos\beta
-sen\alpha sen\beta\\
&=\left( \frac{4}{5}\right) \left( -\frac{5}{13}\right) -\left(
\frac{3}{5}\right) \left( \frac{12}{13}\right)\\
\cos\left( \alpha +\beta \right) &=-\frac{56}{65}
\end{align*}

\bigskip

\begin{align*}
\tan\left( \alpha +\beta \right) &=\dfrac{\tan\alpha +\tan\beta
}{1-\tan\alpha\tan\beta}\\
&=\dfrac{\frac{3}{4}-\frac{12}{5}}{1-\left( \frac{3}{4}\right)
\left( -\frac{12}{5}\right) }\\
\tan\left( \alpha +\beta \right) &=-\dfrac{33}{56}
\end{align*}

\bigskip

Como $sen\left(\alpha +\beta \right)$ es positivo y
$\cos\left(\alpha +\beta \right)$ es negativo, $\alpha +\beta$ es
un �ngulo de segundo cuadrante.

\bigskip

\textbf{Ejemplo 5.34. } Si $\alpha$ es un �ngulo de segundo
cuadrante y $\beta$ es un �ngulo de tercer cuadrante, con
$\cos\alpha=\cos\beta=-\frac{3}{7}$, calcule $sen \left( \alpha
-\beta \right)$ y $\cos\left(\alpha -\beta \right)$ y determine el
cuadrante para $\alpha -\beta $.

\bigskip

\emph{Soluci�n}:

\bigskip

$sen \alpha =\pm \sqrt{1-\frac{9}{49}}$, como $\alpha $ es un
�ngulo de segundo cuadrante:

\bigskip

\[sen \alpha =\sqrt{\frac{40}{49}}=\frac{2}{7}\sqrt{10}\]

\bigskip

Como $\beta $ es un �ngulo de tercer cuadrante:

\bigskip

\[sen\beta =-\frac{2}{7}\sqrt{10}\]

\bigskip

Entonces:

\bigskip

\begin{align*}
sen \left( \alpha -\beta \right) &=sen\alpha \cos\beta
-sen\beta \cos\alpha\\
&=\left( \frac{2}{7}\sqrt{10}\right) \left( -\frac{3}{7}\right)
-\left( \frac{ -2}{7}\sqrt{10}\right) \left( -\frac{3}{7}\right)\\
sen \left( \alpha -\beta \right) &=-\frac{12}{49}\sqrt{10}
\end{align*}

\bigskip

\begin{align*}
\cos \left( \alpha -\beta \right) &=\cos\alpha \cos\beta
+sen\alpha sen\beta\\
&=\left( -\frac{3}{7}\right) \left( -\frac{3}{7}\right)
+\left(\frac{2}{7} \sqrt{10}\right)
\left(-\frac{2}{7}\sqrt{10}\right)\\
\cos\left( \alpha -\beta \right)
&=\frac{9}{49}-\frac{4}{49}\sqrt{10}=\frac{1}{49}\left(
9-4\left(10\right) \right)\\
&=-\frac{31}{49} < 0
\end{align*}

\bigskip

Como $sen \left(\alpha -\beta \right)$ y $\cos \left( \alpha
-\beta \right)$ son negativos, $\alpha -\beta $ es un �ngulo de
tercer cuadrante.


%Pie de p�gina
\newline

{\color{gray}
\begin{tabular}{@{\extracolsep{\fill}}lcr}
\hline \\
\docLink{trigo12.tex}{\includegraphics{../../images/navegacion/anterior.gif}}
\begin{tabular}{l}
{\color{darkgray}\small Ecuaciones Trigonom�tricas} \\ \\ \\
\end{tabular} &
\docLink[_top]{../../index.html}{\includegraphics{../../images/navegacion/inicio.gif}}
\docLink{../../docs_curso/contenido.html}{\includegraphics{../../images/navegacion/contenido.gif}}
\docLink{../../docs_curso/descripcion.html}{\includegraphics{../../images/navegacion/descripcion.gif}}
\docLink{../../docs_curso/profesor.html}{\includegraphics{../../images/navegacion/profesor.gif}}
& \begin{tabular}{r}
{\color{darkgray}\small \'{A}ngulos M�ltiples} \\ \\ \\
\end{tabular}
\docLink{trigo14.tex}{\includegraphics{../../images/navegacion/siguiente.gif}}
\end{tabular}
}

\end{quote}

\newline

\begin{flushright}
\includegraphics{../../images/interfaz/copyright.gif}
\end{flushright}
\end{document}

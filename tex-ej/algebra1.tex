\documentclass[10pt]{article} %tipo documento y tipo letra

\definecolor{azulc}{cmyk}{0.72,0.58,0.42,0.20} % color titulo
\definecolor{naranja}{cmyk}{0.21,0.5,1,0.03} % color leccion

\def\eje{\centerline{\textbf{ EJERCICIOS.}}} % definiciones propias

\begin{document}

\begin{quote}

% inicio encabezado
{\color{gray}
\begin{tabular}{@{\extracolsep{\fill}}lcr}
\docLink{../cap1/reales12.tex}{\includegraphics{../../images/navegacion/anterior.gif}}
\begin{tabular}{l}
{\color{darkgray}\small Cap. 1. Ra�z Cuadrada de N�meros Negativos } \\ \\ \\
\end{tabular} &
\docLink[_top]{../../index.html}{\includegraphics{../../images/navegacion/inicio.gif}}
\docLink{../../docs_curso/contenido.html}{\includegraphics{../../images/navegacion/contenido.gif}}
\docLink{../../docs_curso/descripcion.html}{\includegraphics{../../images/navegacion/descripcion.gif}}
\docLink{../../docs_curso/profesor.html}{\includegraphics{../../images/navegacion/profesor.gif}}
& \begin{tabular}{r}
{\color{darkgray}\small Exponentes Positivos} \\ \\ \\
\end{tabular}
\docLink{algebra2.tex}{\includegraphics{../../images/navegacion/siguiente.gif}}
\\ \hline
\end{tabular}
}
%fin encabezado

%nombre capitulo
\begin{center}
\colorbox{azulc}{{\color{white} \large CAP�TULO 2}}  {\large
{\color{ azulc} FUNDAMENTOS DE ALGEBRA}}
\end{center}

\newline

%nombre leccion

\colorbox{naranja}{{\color{white} \normalsize  Lecci\'on 2.1. }}
{\normalsize {\color{naranja} Expresiones Algebraicas}}

\newline

Llamamos \textit{variable} a una letra o s�mbolo que representa
cualquier elemento de un conjunto en determinado. Llamamos
\textit{constante} a un elemento fijo del conjunto considerado. Si
tal conjunto es el conjunto $\mathbb{R}$ de los n�meros reales,
las variables y constantes representan n�meros reales.

\bigskip

Usualmente representamos las variables por las �ltimas letras del
alfabeto, $x,y,z,w$. En algunos casos representamos las constantes
por las primeras letras del alfabeto, $a,b,c,d$. En estos casos
los s�mbolos $a,b$, etc representan elementos fijos pero
arbitrarios del conjunto considerado.

\textbf{Ejemplo 2.1. } \quad
\begin{enumerate}
\item [a)] En la expresi�n

\[2x^{2}+4x-7\]

la letra $x$ es una variable y los n�meros $2,4$ y $7$ son
constantes.

\item [b)] En la expresi�n

\[ax+b\]

$x$ es una variable y $a$ y $b$ son constantes.
\end{enumerate}

\bigskip

Llamamos \textit{expresiones algebraicas} a las constantes, las
variables o las combinaciones de constantes y variables mediante
las operaciones de suma, resta, multiplicaci�n, divisi�n,
elevaci�n a potencias y extracci�n de ra�ces.

\bigskip

\textbf{Ejemplo 2.2. } Las siguientes son algunas expresiones
algebraicas:

\bigskip

\[5,\quad x^{3},\quad 3ay^{2},\quad (6xy-2y)5x^{2},\quad -3x^{2}y+7x+2y-8,\]
\[\dfrac{2x+3}{x^{2}-5x-6},\quad (2z^{-5}-3z^{-1})^{\frac{3}{5}},\quad
\dfrac{\sqrt{x+2y}}{x-\sqrt[3]{3x}}\]

\bigskip

En una expresi�n algebraica, cada una de las partes separadas por
medio de una suma o de una resta se llama un \textit{t�rmino}.

\bigskip

\textbf{Ejemplo 2.3. } En la expresi�n

\bigskip

\[5x^{2}y-\dfrac{2x+1}{3y-5}+\sqrt{xy-6}+7\]

\bigskip

los t�rminos son $5x^{2}y$, $\dfrac{2x+1}{3y-5}$, $\sqrt{xy-6}$ y
$7$.

\bigskip

\bigskip Observamos que expresiones tales como $\dfrac{2x+1}{3y-5}$ y
$\sqrt{xy-6}$ se consideran como un solo t�rmino.

\bigskip

Si un t�rmino consiste de un producto de dos o mas factores,
decimos que cada factor es el \textit{coeficiente} del producto de
los otros factores. Por ejemplo en el t�rmino $7x^{2}y$, $7$ es el
coeficiente de $x^{2}y$, $7x^{2}$ es el coeficiente de $y$, y $7y$
es el coeficiente de $ x^{2}$. Si un coeficiente es un n�mero, lo
llamamos el coeficiente num�rico. En el t�rmino anterior
$7x^{2}y$, $7$ es el coeficiente num�rico.

\bigskip

Si una expresi�n algebraica consiste de un solo t�rmino, la
llamamos un \textit{monomio}, si consiste de dos t�rminos la
llamamos un \textit{binomio}, si consiste de tres t�rminos la
llamamos un \textit{trinomio} y as� sucesivamente.

\bigskip

Vamos ahora a enfocar nuestra atenci�n sobre algunas de las
operaciones que se utilizan para formar expresiones algebraicas.

%Pie de p�gina
\newline

{\color{gray}
\begin{tabular}{@{\extracolsep{\fill}}lcr}
\hline \\
\docLink{../cap1/reales12.tex}{\includegraphics{../../images/navegacion/anterior.gif}}
\begin{tabular}{l}
{\color{darkgray}\small Cap. 1. Ra�z Cuadrada de N�meros Negativos } \\ \\ \\
\end{tabular} &
\docLink[_top]{../../index.html}{\includegraphics{../../images/navegacion/inicio.gif}}
\docLink{../../docs_curso/contenido.html}{\includegraphics{../../images/navegacion/contenido.gif}}
\docLink{../../docs_curso/descripcion.html}{\includegraphics{../../images/navegacion/descripcion.gif}}
\docLink{../../docs_curso/profesor.html}{\includegraphics{../../images/navegacion/profesor.gif}}
& \begin{tabular}{r}
{\color{darkgray}\small Exponentes Positivos} \\ \\ \\
\end{tabular}
\docLink{algebra2.tex}{\includegraphics{../../images/navegacion/siguiente.gif}}
\end{tabular}
}

\end{quote}

\newline

\begin{flushright}
\includegraphics{../../images/interfaz/copyright.gif}
\end{flushright}
\end{document}

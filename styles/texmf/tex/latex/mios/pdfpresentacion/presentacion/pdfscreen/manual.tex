\documentclass[a4paper]{article}
\usepackage{xspace,colortbl}
\usepackage[print,panelright,code,paneltoc,sectionbreak]{pdfscreen}
\begin{screen}
 \margins{.65in}{.65in}{.65in}{.65in}
 \screensize{6.25in}{8in}
 \changeoverlay
 \paneloverlay{but.pdf}
 \overlay{logo.pdf}
 \def\pfill{\vskip6pt}
\end{screen}

\renewcommand\floatpagefraction{1}
\renewcommand\textfraction{0}
\def\pdfscreen{\texttt{\small\color{section1}pdfscreen}\xspace}

\begin{print}
\notesname{Notes:}
\makeatletter
\def\@seccntformat#1{\llap{\scshape\color{section\thesection@level}
     \csname the#1\endcsname.\hspace*{6pt}}}
\makeatother
\end{print}

\begin{document}

\begin{screen}
\title{\color{section0}\Huge Manual}
\end{screen}

\begin{print}
\title{\Huge\texttt{pdfscreen.sty} --- Manual}
\end{print}

\author{\color{section1}\Large C.~V.~Radhakrishnan\\
        {\small\href{mailto:cvr@river-valley.com}
        {\color{section1}\texttt{cvr@river-valley.com}}}}
\maketitle
\begin{screen}
\vfill

\end{screen}

\begin{abstract}
\noindent \pdfscreen package helps to redesign the pdf output of your normal
documents fit to be read in a computer monitor while retaining the
freedom to format it for conventional printing. This has been brought
about by redefining the margins and page height/\allowbreak width and
related dimensions to fit into that of the computer screen. By
changing the options to \verb+print+ you can switch the package to
format the document in the conventional way as your class file
dictates.
\end{abstract}

\begin{print}
\tableofcontents
\end{print}
\begin{screen}

\vfill
\end{screen}

\section{Setup}\label{setup}

An elaborate manual is not needed for using  \pdfscreen, since
it is nothing but an extension of the \verb+hyperref.sty+ of
\href{mailto:sebastian.rahtz@oucs.ac.uk}{Sebastian Rahtz}.
The primary aim of the package is to change the dimensions of the width
and height of the page so as to provide an ideal dimension
that is fit for screen viewing rather than printing. As such, all those
dimensions that control the page shape are redefined to result the
desired screen size. The preamble portion requires the package loading
command as given below:
\begin{decl}
 |\usepackage|\Oarg{screen,panelleft}\Arg{pdfscreen}
\end{decl}
There is no need to specify the
\verb+\usepackage{hyperref}+ with its options, since \verb+hyperref+ is
loaded by the \pdfscreen. Unlike previous verions you can load
|hyperref.sty| prior to \pdfscreen with necessary options if you like,
if so \pdfscreen will not reload |hyperref|. The default backend driver
for \pdfscreen is |pdftex|. However, you can specify your backend
driver as an option.

It will be nicer if \pdfscreen is loaded as the last package
in the preamble so as to avoid further redefinition of commands that
are used by \pdfscreen.


\subsection{Options}\label{options}
The following options are available:

\begin{enumerate}

\item \verb+screen+ -- generates the screen version

\item \verb+print+  -- generates the print that looks like your
\verb+dvi+

\item \verb+panelleft+ -- navigation panel in the left
side

\item \verb+panelright+ -- navigation panel in the right
side

\item \verb+nopanel+ -- suppresses the panel

\item \verb+paneltoc+ -- table of contents in the panel. With this
option invoked, please do not use |\tableofcontents| command in the
document and |paneltoc| stops as soon as |\tableofcontents| command is
encountered.

\item |sectionbreak| -- will introduce pagebreak before a section.

\item |code| -- provides commands that can be used to list |verbatim|
like listing of program code as found in the \LaTeX{} documentation.

\begin{verbatim}
 \begin{decl}\\
  |\usepackage|\oarg{options}\arg{package}\\
  |\screensize|\Arg{6.25in}\Arg{8in}
 \end{decl}
\end{verbatim}

\begin{decl}
|\usepackage|\oarg{options}\arg{package}\\
|\screensize|\Arg{6.25in}\Arg{8in}
\end{decl}

\item \textbf{Backend drivers}: |dvips|, |dvipsone|, \dots, |vtex| can be
specified as an optional backend driver. |pdftex| is the default.

\item \textbf{Color schemes}: There are six color schemes --
bluelace, blue, gray, orange, palegreen and chocolate -- available for
panel and buttons that you can give as an option to the  package.
Default is blue.

\item \textbf{Foreign language support}: Not all the foreign languages
are supported. Only 15 European languages are supported at the moment.
However, all the language names as you give in the \verb+babel+
package can be given here as an option. If the language is not
supported, the package will default to English. I may request users to
send me the translation of the button text in the navigation panel in
your language, if it is not supported.

\item \verb+nocfg+ -- an option to suppress the configuration file (see
\autoref{cfg}), if you don't want to use its specifications.

\end{enumerate}

\subsection{Other parameters to be passed}\label{parameters}
Few more parameters are to be passed on to the \pdfscreen to make it more
functional. They are:

\begin{decl}\verb+\emblema+\arg{graphic file name}\end{decl} the name of the graphic file that appears on
the navigation panel.

\begin{decl}\verb+\urlid+\arg{URL name}\end{decl} The |home page|
button in the navigation panel will be linked to the \textsc{url}.

\begin{decl}\verb+\screensize+\arg{height}\arg{width}\end{decl} 
This command will facilitate to specify the screen dimensions of the
pdf output. No default screen dimensions are available, and therefore
the user has to specify it explicitly. There are no restrictions on the
screen dimensions. Unlike previous versions, the user is free to choose
any dimension. The default width of the panel is 15\% of the width of
the screen.

\begin{decl}
 |\margins|\arg{left}\arg{right}\arg{top}\arg{bottom}
\end{decl}
This command will set the margins of the document.
There are no default values for margins and you will have to specify it
explicitly in the document preamble.

With |\margins| and |\screensize| explicitly given in the document
preamble, \pdfscreen now obeys whatever screen size and
margins the user has specified. This change is brought consequent to
the bug report of \href{dpstory@uakron.edu}{D.~P.~Story}.

\subsection{Typical preamble}
A typical document preamble is given below (with which this document is
formatted):

\begin{decl}
|\documentclass[a4paper,11pt]{article}|\\
|\usepackage{xspace,colortbl}|\\
| |\\
|\usepackage|\Oarg{screen,panelleft,gray,paneltoc}\Arg{pdfscreen}\\
|\margins|\Arg{.75in}\Arg{.75in}\Arg{.75in}\Arg{.75in}\\
|\screensize|\Arg{6.25in}\Arg{8in}\\
|\overlay|\Arg{lightsteelblue.pdf}\\
| |\\
|\begin{document}|
\end{decl}

\subsection{Packages needed to run \pdfscreen}
The following packages are needed for smooth compilation (grab the
latest from \textsc{ctan}:

\begin{enumerate}
\item |hyperref.sty|
\item |comment.sty|
\item |truncate.sty|
\item |graphicx.sty|
\item |color.sty|
\item |colortbl.sty|
\item |calc.sty|
\item |amssymb.sty|
\item |amsbsy.sty|
\item |shortvrb.sty|
\item |fancybox.sty|

\end{enumerate}

\section{Navigation Panel}
The design of the  navigation panel is left entirely to the imagination
of the user. One can create a panel as per his taste. A
newer command, \verb+\panel+ has been added and a default
panel is also supplied, which is nothing but a box with navigation
buttons vertically arranged. There is also a command:

\begin{decl}
  |\addButton|\arg{length}\arg{button text string}
\end{decl}
which you can use to generate buttons of your choice. Here is an
example of how to produce a Next Page button:
\begin{decl}
  |\Acrobatmenu{NextPage}{\addButton|\Arg{1.25in}\Arg{Next Page}|}|
\end{decl}
This will generate the following navigation button
\begin{center}
  \Acrobatmenu{NextPage}{\addButton{1.25in}{Next Page}}
\end{center}
and clicking this will take you to the next page. In the same way, you
can build buttons with images too, for which the command,
\begin{decl}
  |\imageButton|\arg{width}\arg{height}\arg{graphic file name}
\end{decl}
will be useful.
\begin{center}
\href{http://www.tug.org}{\imageButton{.75in}{!}{tex}}
\end{center}

The \textsc{tug} image button is generated by:
\begin{verbatim}
  \href{http://www.tug.org}{\imageButton{.5in}{!}{tex.png}}
\end{verbatim}

Clicking this button will take you to \url{http://www.tug.org}, the
\TeX{} Users Group web site. 

The navigation panel can be positioned at user's will either to the
left side or right side. To see different types of output click the
buttons in \autoref{fig2}.

\begin{figure}\footnotesize\fboxsep 10pt
\colorbox{panelbackground}{%
\begin{minipage}{\linewidth-20pt}
\color{orange}
\begin{flushleft}
\def\pba#1{\begin{minipage}[c][18pt][c]{75pt}#1\end{minipage}}
\def\pbb#1{$\Rightarrow$\fboxsep3pt\colorbox{white}
     {\begin{minipage}[c]{\linewidth-100pt}
       \footnotesize\color{section1}#1
      \end{minipage}}}
\definecolor{buttonbackground}{rgb}{.902,.902,.980}
\definecolor{buttonshadow}{rgb}{.412,.412,.412}
\pba{\href{left.pdf}{\addButton{1in}{Left Panel}}}
\pbb{{\tt\string\usepackage}\Oarg{screen,panelleft}\Arg{pdfscreen}}\\[2pt]
\pba{\href{right.pdf}{\addButton{1in}{Right Panel}}}
\pbb{{\tt\string\usepackage}\Oarg{screen,panelright}\Arg{pdfscreen}}\\[2pt]
\pba{\href{nopanel.pdf}{\addButton{1in}{No Panel}}}
\pbb{{\tt\string\usepackage}\Oarg{screen,nopanel}\Arg{pdfscreen}}\\[2pt]
\pba{\href{portrait.pdf}{\addButton{1in}{Portrait}}}
\pbb{Change screen size with the command:\hfill\break
    {\tt\string\screensize}\arg{height}\arg{width}}\\[2pt]
\pba{\href{square.pdf}{\addButton{1in}{Square}}}
\pbb{Change screen size with the command:\hfill\break
    {\tt\string\screensize\arg{height}\arg{width}} command}\\[2pt]
\pba{\href{widepanel.pdf}{\addButton{1in}{Wide Panel}}}
\pbb{{\tt\string\panelwidth}=\textcolor{red}{\it\textless dimension\textgreater}}\\[2pt]
\pba{\href{print.pdf}{\addButton{1in}{Print version}}}
\pbb{{\tt\string\usepackage}\Oarg{print}\Arg{pdfscreen}}
\end{flushleft}
\end{minipage}}
\caption{Different types of panel positioning\label{fig2}}
\end{figure}


Width of the panel can be changed by explicitly giving in the preamble
of the document like \verb+\panelwidth=<dimension>+. 
The default is 15\% of the screen width, however, there is a minimum
value of 1in in case 15\% of the screen width goes below 1in.

You can define your own panel which is nothing but a vertical box with
whatever stuff you want to fit into. The navigation panel of this
document is made up of the following code:

\begin{verbatim}
  \panelwidth=1.3in
  \def\panel{\colorbox{panelbackground}
   {\begin{minipage}[t][\paperheight][b]{\panelwidth}
     \centering\null\vspace*{12pt}
    \includegraphics[width=.75in]{univ}\par\vfill
    \href{\@urlid}{\addButton{.85in}{\@Panelhomepagename}}\par\vfill
    \Acrobatmenu{FirstPage}{\addButton{.85in}
      {\FBlack\@Paneltitlepagename}}\par\vfill
    \Acrobatmenu{FirstPage}{\addButton{.2in}
      {\FBlack\scalebox{.8}[1.4]{\btl\btl}}}\hspace{-3pt}
    \Acrobatmenu{PrevPage}{\addButton{.2in}
      {\FBlack\scalebox{.8}[1.4]{\btl}}}\hspace{-3pt}
    \Acrobatmenu{NextPage}{\addButton{.2in}
      {\LBlack\scalebox{.8}[1.4]{\rtl}}}\hspace{-3pt}
    \Acrobatmenu{LastPage}{\addButton{.2in}
      {\LBlack\scalebox{.8}[1.4]{\rtl\rtl}}}\par\vfill
    \Acrobatmenu{GoBack}{\addButton{.85in}
      {\@Panelgobackname}}\par\vfill
    \Acrobatmenu{FullScreen}{\addButton{.85in}{Full Screen}}\par\vfill
    \Acrobatmenu{Close}{\addButton{.85in}{\@Panelclosename}}\par\vfill
    \Acrobatmenu{Quit}{\addButton{.85in}{\@Panelquitname}}\par
    \null\vspace*{12pt}
    \end{minipage}}}
\end{verbatim}

\section{Other facilities}
\subsection{Background}

The background of the screen area can be overlayed with a graphic file
with the command |\overlay|\arg{graphic file name}. Alternatively, you
can specify a background color by saying |\backgroundcolor|\arg{color}
where \verb+color+ is a predefined color with the commands provided by
the \verb+color.sty+.

The background of the panel can also be provided by a graphic file with
the command |\paneloverlay|\arg{graphic file}. If you do not specify
an overlay graphic for the panel, the panelbackground color will take
effect. You can redefine the panelbackground color to your choice, if
you find the default color distasteful, or else you can specify the
same in the \verb+pdfscreen.cfg+ file.
|\overlayempty| and |\paneloverlayempty| commands help you to
suppress the overlays at any stage.

A new command |\changeoverlay| has been introduced and a series of
small pdf files, each with less than 2 KB file size are offered, so
that you can have different overlays with the change of each section
unit and it will reset on every tenth section.

You can create your own overlays and modify the |\change| command
with your overlay files.

\subsection{Bottom buttons}

Bottom footer menu can be invoked with |\bottombuttons| and can be
closed with |\nobottombuttons|. So also in the academic interest,
|\topbuttons| and |\notopbuttons| are also available. You can
have both the buttons in the same page, though it is bizarre looking.

\subsection{Table of contents in the panel}

The package option \verb+paneltoc+ will allow you to have the table
of contents in the navigation panel. In an article, only section
headings are shown in the panel toc. Users are requested to use this
option with caution, since it is prone to blow up the list and verbatim
environments spanning multiple pages. However, a manual intervention
with a \verb+\clearpage+ command at the appropriate location can
ease you to a certain extent.

\subsection{Configuration file}\label{cfg}

You can keep a configuration file called \verb+pdfscreen.cfg+ in which
you can provide your own translation of button
text if your language is unsupported by the package, newer color
schemes if you dislike the schemes offered, your
\textsc{url} id, affiliation and division names, date
argument, graphic file name of your logo/\allowbreak emblem. I would
suggest to use a configuration file, in which you can give all your
site specific requirements, that has another advantage of eliminating
the clustered look at the preamble of the document. A typical
configuration file is supplied with the package.

\section{Slides}

A slide environment is available which can be entered as
\begin{verbatim}
 \begin{slide}
 .
 .
 .
 slide material
 .
 .
 .
 \end{slide}
 \end{verbatim}
This is a box spanning the width and height of the text area, within
which the material will be vertically centered. 

\subsection{Fonts}
All the font attributes have been redefined to make them larger than
the usual size inside the slide environment. However, if you want to
revert to the original size, you will have to add the word \verb+real+
before the font size command, i.e., for \verb+\normalsize+, use
\verb+\realnormalsize+; for \verb+\large+ it is \verb+\reallarge+ and
so forth.

\subsection{Post-processing}

The postprocessor \emph{viz.}, \verb+PPower4+ can be applied to the pdf
generated with this package, so that incremental additions to the pages
are possible. \verb+PPower4+ is available at
\href{ftp://ftp.dante.de/support/PPower4}{\textsc{ctan}}. You may need
Java Virtual Machine running in your system to work with
\verb+PPower4+. I have not tried the \TeX{}Power package by
\href{mailto:Stephan.Lehmke@cs.uni-dortmund.de}{Stephan Lehmke}
with \verb+pdfscreen+, but I would recommend this for effecting
incremental builds.

\begin{print}
With |print| option invoked, the slide environment will print as a
boxed minipage with another ovalbox adjacent to the slide box. This
will prove useful for the audience to record any notes/queries during
presentation which they can discuss with the speaker at the end of his
session. The `Notes' that appear in the query box can be redefined to
the string you like in your language by the command:
\begin{decl}
|\def\notesname|\arg{your string}
\end{decl}
\end{print}

\noindent{\begin{slide}
\color{section0}
\begin{center}
\Large\itshape The objectives of the slide option are:
\end{center}
\sffamily
\rightmargin\leftmargin
\begin{itemize}
\rightmargin\leftmargin
\itshape
\item to devise a method for easier technical presentation.

\item to help the mix of mathematical formulae with text and graphics
which the present day \textsc{wysiwyg} tools fail to accomplish.

\item to exploit the platform independence of \TeX{} so that
presentation documents become portable.

\item to offer the freedom and possibilities of using various
backgrounds and other embellishments that a user can imagine to have in
his presentation.

\end{itemize}

\end{slide}}

\section{Page Transition}
\begin{itemize}
\item You can exploit the page transition facilities in the Acrobat.
Specify your choice by using the command |\pagedissolve|\arg{option}.

\item A list of page dissolve options and keys are given in \autoref{tab1}.

\end{itemize}

The page dissolve options are taken from the well known book, \emph{Web
Publishing with Acrobat/\textsc{pdf}} by Thomas Merz will largely help to know
the options for \verb+\pagedissolve+ function. 

\definecolor{gray9}{rgb}{1,.894,.769}
\subsubsection*{Keys for page transitions}
\def\dash{\noalign{\vskip1.5pt}\hline\noalign{\vskip1.5pt}}
\begin{screen}
\begin{table}[b]
\sffamily\itshape\footnotesize
\setlength\arrayrulewidth{0pt}
\begin{tabular}{@{}p{.15\linewidth}p{.8\linewidth}@{}}
\rowcolor{section1}Key           & Explanation\\\dash
\rowcolor{gray9} /Split          & Two lines sweep across the screen to reveal
                                   the new page similar to opening a curtain.\\\dash
\rowcolor{buttondisable} /Blinds & Similar to /Split, but with several lines
                                   resembling  ``Venetian blinds''\\\dash
\rowcolor{gray9} /Box            & A box enlarges from the center of the old
                                   page to reveal the new one.\\\dash
\rowcolor{buttondisable} /Wipe   & A single line ``wipes'' across the old page
                                   to reveal the new  one.\\\dash
\rowcolor{gray9} /Dissolve       & The old page ``dissolves'' to reveal the
                                   new one.\\\dash
\rowcolor{buttondisable} /Glitter& Similar to /Dissolve, except the effect
                                   sweeps from one edge to another.\\\dash
\rowcolor{gray9} /R (Replace)    & The old page is simply replaced with
                                   the new one without any special effect.
                                   This is the default. 
\end{tabular}
\caption{Keys for page transition\label{tab1}}
\end{table}
\end{screen}

\begin{print}
\begin{table}
\sffamily\itshape\footnotesize
\setlength\arrayrulewidth{.1pt}
\begin{tabular}{@{}p{.15\linewidth}p{.8\linewidth}@{}}
\dash
 Key            & Explanation\\\dash
 /Split         & Two lines sweep across the screen to reveal
                  the new page similar to opening a curtain.\\\dash
 /Blinds        & Similar to /Split, but with several lines
                  resembling  ``Venetian blinds''\\\dash
 /Box           & A box enlarges from the center of the old
                  page to reveal the new one.\\\dash
 /Wipe          & A single line ``wipes'' across the old page
                  to reveal the new  one.\\\dash
 /Dissolve      & The old page ``dissolves'' to reveal the
                  new one.\\\dash
 /Glitter& Similar to /Dissolve, except the effect
                  sweeps from one edge to another.\\\dash
 /R (Replace)   & The old page is simply replaced with
                  the new one without any special effect.
                  This is the default. \\\dash
\end{tabular}

\caption{Keys for page transition\label{tab1}}
\end{table}
\end{print}

For some of the transitions, additional parameters may be specified. The
code given below results in a split effect with the lines moving
horizontally (/H) from the inner parts of the page to the outer parts
(/O). The duration of the effect is two seconds (/D):
\begin{verbatim}
         /S /Split /D 2 /Dm /H /M /O 
\end{verbatim}

All supported parameters for page dissolve, along with
the kind of transition on which the parameters may be applied are given
in  \autoref{tab2}.

\begin{screen}
\begin{table}
\sffamily\itshape\footnotesize
\setlength\arrayrulewidth{0pt}
\begin{tabular}{@{}p{.15\linewidth}p{.8\linewidth}@{}}
\rowcolor{section1}Key           & Explanation\\ \dash
\rowcolor{gray9} /D              & Duration of the transition effect in
                                   seconds (applies to all effects)\\ \dash
\rowcolor{buttondisable} /Di\hfill\break
                (Direction)      & Direction of the movement (multiples of
                                   90$^\circ$ only). Values increase in a
                                   counterclockwise fashion, 0$^\circ$
                                   points to the  right (for /Wipe and
                                   /Glitter).\\\dash 
\rowcolor{gray9} /Dm\hfill\break
                   (Dimension)   & Possible values are /H or /V for a
                                   horizontal or  vertical effect,
                                   respectively (for /Split and
                                   /Blinds).\\\dash  
\rowcolor{buttondisable} /M \hfill\break
                  (Motion)       & Specifies whether the effect is
                                   performed from  the center out or
                                   the edges in. Possible values  are
                                   /I for in and /O for out (for /Split
                                   and /Box).  
\end{tabular}

\caption{Additional parameters for page transitions\label{tab2}}
\end{table}
\end{screen}

\begin{print}
\begin{table}
\sffamily\itshape\footnotesize
\setlength\arrayrulewidth{.10pt}
\begin{tabular}{@{}p{.15\linewidth}p{.8\linewidth}@{}}
\dash
Key           & Explanation\\ \dash
/D            & Duration of the transition effect in
                                   seconds (applies to all effects)\\ \dash
/Di\hfill\break
 (Direction)  & Direction of the movement (multiples of
              90$^\circ$ only). Values increase in a
              counterclockwise fashion, 0$^\circ$
              points to the  right (for /Wipe and
              /Glitter).\\\dash 
 /Dm\hfill\break
  (Dimension) & Possible values are /H or /V for a
              horizontal or  vertical effect,
              respectively (for /Split and
              /Blinds).\\\dash  
 /M \hfill\break
  (Motion)    & Specifies whether the effect is
              performed from  the center out or
              the edges in. Possible values  are
              /I for in and /O for out (for /Split
              and /Box).  \\
\dash
\end{tabular}

\caption{Additional parameters for page transitions\label{tab2}}
\end{table}
\end{print}

\section{Bugs and TODO}

\begin{enumerate}
\item Enumerated and itemized lists spanning across pages create a
nasty |missing \item| error. \href{mailto:dpstory@uakron.edu}{D.~P.~Story}
has found out the reason to be the centering command, |\begin{center}|
\dots |\end{center}| code inside the |\panel| macros. His suggestion to
change this to |\centering| has made a dramatic effect thus eliminating
the |missing \item| error. However, if |paneltoc| option is invoked,
there is a slight shift in the spacing between table of contents
entries in the panel toc is oberved. Still a bug \dots

\item |verbatim| environment when spanned across pages (or if a page
break occurs amidst a verbatim environment) also had similar bug,
resulting in the panel toc items appearing in verbatim mode. A fix has
been made again by \href{mailto:dpstory@uakron.edu}{D.~P.~Story} by
putting the verbatim in a vertical box and spliting with |\vsplit| at
the pagebreak. This solved the above problem, but a new problem of loss
of colors and font attributes  in the second box material after split
has errupted.

\end{enumerate}

\section{Acknowledgements}

The development of this package was funded by the University of
Auckland, New Zealand (\href{mailto:j.hillas@auckland.ac.nz}{John
Hillas} of Department of Economics made available the funds, to whom I
owe much gratitude). I owe much thanks to
\href{mailto:sebastian.rahtz@oucs.ac.uk}{Sebastian Rahtz} for his
patient replies to my incessant queries. The design of the side
navigation panel is due to \href{mailto:kaveh@focal.demon.co.uk}{Kaveh
Bazargan} of Focal Image Ltd., London and it was at his goading that I
started writing this package.

The current version (v1.3) is a complete rewrite and therefore
it may be possible that the documents compiled with older versions may
break down, though every effort has been made to keep the downward
compatibility and the old commands and options are still kept in order
not to blow up your documents which runs smoothly with previous
versions.

I firmly believe that the code demands further optimization and may
have bugs that will show up during exhaustive usage. I request you to
send me the bug reports as and when they exhibit their ugly faces.

My special thanks are for all those users who deploy
\verb+pdfscreen.sty+ for their document preparation, and for their kind
words of appreciation.

\pdfscreen works well with the much used package, \verb+exerquiz.sty+
of \href{mailto:dpstory@uakron.edu}{Donald P.~Story}
(\url{http://www.math.uakron.edu/~dpstory/pdf_demos.html}) without
clashes. Feedback on the usage of \TeX{}Power with \pdfscreen is
requested from users.

My mail id for contact is \href{mailto:cvr@river-valley.com}%
{\tt cvr@river-valley.com}.

\end{document}


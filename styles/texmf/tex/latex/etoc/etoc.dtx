% -*- coding: iso-latin-1; -*-
%<*ins>
\def\pkgname{etoc}
\def\pkgdate{2013/03/07}
\def\pkgversion{v1.07f}
\def\pkgdescription{Completely customisable TOCs (jfB)}
%</ins>
%%
%% Package `etoc' by Jean-Francois Burnol
%% Copyright (C) 2012, 2013 by Jean-Francois Burnol
%%
%
% The copyright statement at the top of this file applies to `etoc.dtx'
% and to its derived files.
% 
%     This work may be distributed and/or modified under the
%     conditions of the LaTeX Project Public License, either
%     version 1.3c of this license or (at your option) any later
%     version. This version of this license is in 
%          http://www.latex-project.org/lppl/lppl-1-3c.txt
%     and the latest version of this license is in
%          http://www.latex-project.org/lppl.txt
%     and version 1.3 or later is part of all distributions of
%     LaTeX version 2005/12/01 or later. 
%
% The author of this work is Jean-Francois Burnol <jfbu at free dot fr>. 
% This work has the LPPL maintenance status `author-maintained'.
% 
%  Installation:
%  ============
%
% `latex etoc.dtx' --> this extracts the style file `etoc.sty'
%
% `latex etoc.dtx' (twice more) finishes producing the documentation
%
%  Move `etoc.sty' to a suitable location within the TeX installation:
%          etoc.sty -> <TDS>/tex/latex/etoc/etoc.sty
%
%  The generated auxiliary files may be discarded.
% 
%<*none>
\def\lasttimestamp{Time-stamp: <14-05-2013 16:34:59 CEST jfb>}
\def\docdate{2013/05/14}
\def\striptimestamp#1 <#2 #3 #4 #5>{#2 at #3 #4}
\edef\dtxtimestamp{\expandafter\striptimestamp\lasttimestamp}
\ProvidesFile{\pkgname.dtx}
 [`\pkgname' source and documentation (\dtxtimestamp)]
\begingroup
\input docstrip.tex
\askforoverwritefalse
\def\pkgpreamble{\defaultpreamble^^J\MetaPrefix^^J%
\string\ProvidesPackage{\pkgname}^^J%
\space[\pkgdate\space\pkgversion\space\pkgdescription]}
\generate{\nopreamble
\file{\pkgname.ins}{\from{\pkgname.dtx}{ins}}
\usepreamble\pkgpreamble
\file{\pkgname.sty}{\from{\pkgname.dtx}{package}}}
\endgroup
\iffalse
%</none>
%<*ins>
%-------------------------------------------------------------------------------
%% This file `etoc.ins' is provided for compatibility with TeX distributions
%% expecting to find it for installation of `etoc.sty'.
%%
%% As usual `latex etoc.ins' produces `etoc.sty' from the source `etoc.dtx'
%% (an already existing etoc.sty in the same repertory will be overwritten)
%% 
%% Move `etoc.sty' to a suitable location:
%% etoc.sty -> <TDS>/tex/latex/etoc/etoc.sty
%%
%% The generated auxiliary files may be discarded. See `etoc.dtx' for the
%% statements of copyright and conditions of distribution or modification.
%%
\input docstrip.tex
\askforoverwritefalse
\def\pkgpreamble{\defaultpreamble^^J\MetaPrefix^^J%
\string\ProvidesPackage{\pkgname}^^J%
\space[\pkgdate\space\pkgversion\space\pkgdescription]}
\generate{\usepreamble\pkgpreamble
\file{\pkgname.sty}{\from{\pkgname.dtx}{package}}}
\endbatchfile
%-------------------------------------------------------------------------------
%</ins>
%<*none>
\fi
\documentclass[a4paper,11pt,abstract]{scrdoc}
%%\OnlyDescription

\usepackage[T1]{fontenc}
\usepackage[latin1]{inputenc}
\usepackage[hscale=0.66,vscale=0.75]{geometry}
\pagestyle{headings}

\makeatletter
% 2013/03/07. 
% This macro allows to conveniently center a line inside a paragraph and still
% use there in \verb or other commands changing catcodes.
% A proposito, \centerline uses \hsize and not \linewidth!!!
    \def\@centeredline {\hbox to \linewidth
                        \bgroup \hss \bgroup
                        \aftergroup\centeredline@ }
    \newcommand*\centeredline {%
      \ifhmode
        \\\relax
        \def\centeredline@{\hss\egroup\hskip\z@skip}%
      \else
        \def\centeredline@{\hss\egroup}%
      \fi
      \afterassignment\@centeredline
      \let\next=}

   \let\original@check@percent\check@percent
   \let\check@percent\relax
\makeatother

\def\MacroFont{\ttfamily\small\baselineskip12pt\relax}

\usepackage{txfonts} 

% m'occuper de la taille de txtt:
% au passage je supprime le \hyphenchar \font\m@ne


% ATTENTION ATTENTION ATTENTION Un changement de taille comme \small semble
% d�clencher (peut-�tre) la relecture de t1txtt.fd et quoi qu'il en soit le
% \hyphenchar est REMIS � -1 !

\DeclareFontFamily{T1}{txtt}{}
\DeclareFontShape{T1}{txtt}{m}{n}{	%medium
     <->s*[.96] t1xtt%
}{}
\DeclareFontShape{T1}{txtt}{m}{sc}{	%cap & small cap
     <->s*[.96] t1xttsc%
}{}
\DeclareFontShape{T1}{txtt}{m}{sl}{	%slanted
     <->s*[.96] t1xttsl%
}{}
\DeclareFontShape{T1}{txtt}{m}{it}{	%italic
     <->ssub * txtt/m/sl%
}{}
\DeclareFontShape{T1}{txtt}{m}{ui}{   	%unslanted italic
     <->ssub * txtt/m/sl%
}{}
\DeclareFontShape{T1}{txtt}{bx}{n}{	%bold extended
     <->t1xbtt%
}{}
\DeclareFontShape{T1}{txtt}{bx}{sc}{	%bold extended cap & small cap
     <->t1xbttsc%
}{}
\DeclareFontShape{T1}{txtt}{bx}{sl}{	%bold extended slanted
     <->t1xbttsl%
}{}
\DeclareFontShape{T1}{txtt}{bx}{it}{	%bold extended italic
     <->ssub * txtt/bx/sl%
}{}
\DeclareFontShape{T1}{txtt}{bx}{ui}{  	%bold extended unslanted italic
     <->ssub * txtt/bx/sl%
}{}
\DeclareFontShape{T1}{txtt}{b}{n}{	%bold
     <->ssub * txtt/bx/n%
}{}
\DeclareFontShape{T1}{txtt}{b}{sc}{	%bold cap & small cap
     <->ssub * txtt/bx/sc%
}{}
\DeclareFontShape{T1}{txtt}{b}{sl}{	%bold slanted
     <->ssub * txtt/bx/sl%
}{}
\DeclareFontShape{T1}{txtt}{b}{it}{   	%bold italic
     <->ssub * txtt/bx/it%
}{}
\DeclareFontShape{T1}{txtt}{b}{ui}{   	%bold unslanted italic
     <->ssub * txtt/bx/ui%
}{}


\usepackage{xspace}
\usepackage{xcolor}
\usepackage{graphicx}
\usepackage{enumitem}

\definecolor{joli}{RGB}{225,95,0}
\definecolor{JOLI}{RGB}{225,95,0}
\definecolor{BLUE}{RGB}{0,0,255}
\colorlet{niceone}{green!35!blue!75}

\usepackage[english]{babel}
\AtBeginDocument{
  \renewcommand\partname{Part}
  \addto\captionsenglish{\renewcommand\partname{Part}}
}

% ok il faut ceci pour dvipdfmx:
% \def\pgfsysdriver{pgfsys-dvipdfm.def}

\usepackage{tikz}
\usetikzlibrary{trees}    % added for "cyclic" grow function 2013/03/02
\usepackage{tikz-qtree}

\usepackage[% dvipdfmx,
pdfencoding=pdfdoc,bookmarks=true]{hyperref}

\hypersetup{%
linktoc=all,%      why is the important stuff
bookmarksdepth=3,% in a *hard* to find README?
breaklinks=true,%
hidelinks,%
pdfauthor={Jean-Fran\c cois Burnol},%
pdftitle={The etoc package},%
pdfsubject={LaTeX, table of contents},%
pdfkeywords={LaTeX, table of contents},%
pdfstartview=FitH,%
pdfpagemode=UseOutlines}

% for use when \usepackage{hyperref} is commented out
% \makeatletter
% \let\texorpdfstring\@firstoftwo
% \def\hyperref[#1]#2{{#2}}
% \def\href#1#2{{#2}}
% \makeatother


\usepackage{etoc}  % executes \RequirePackage{multicol}
%--------
% add-ons for the new section `surprising uses of etoc' (2013/01/24)
\newcounter{visibletoc}
\renewcommand{\etocaftertitlehook}
   {\stepcounter{visibletoc}\etoctoccontentsline{visibletoc}{\thevisibletoc}}
\etocsetlevel{visibletoc}{6}
%--------
% add-ons for the `fancy TOC' (2013/01/25)
    \newcounter{dummypart}
    \newcounter{dummychapter}
    \newcounter{dummysection}
    \etocsetlevel{dummypart}{6}
    \etocsetlevel{dummychapter}{6}
    \etocsetlevel{dummysection}{6}
%--------
% for "Another compatibility mode"  (2013/02/21):
\makeatletter
\newcommand*{\MyQuasiStandardTOC}[1]{%
  \begingroup
  \let\savedsectionline\l@section
  \let\savedsubsectionline\l@subsection 
  \etocsetstyle{section}{}
    {\ifnum\etocthenumber=3
      \etocsetstyle{subsection}
        {\par\nopagebreak\begingroup
         \leftskip1.5em \rightskip\@tocrmarg
         \parfillskip \@flushglue 
         \parindent 0pt
         \normalfont\normalsize\rmfamily\itshape
         % \columnsep1em
         % \begin{minipage}{\dimexpr\linewidth-\leftskip-\rightskip\relax}%
         % \begin{multicols}{2}%
         \etocskipfirstprefix}
        {\allowbreak\,--\,}
        {\etocname\ \textup{(\etocnumber)}}
        % {.\par\end{multicols}\end{minipage}\par\endgroup}%
        {.\par\endgroup}%
      \else
        \etocsetstyle{subsection}{}{}
        {\savedsubsectionline{\numberline{\etocnumber}\etocname}{\etocpage}}{}%
      \fi} 
    {\savedsectionline
      {\numberline{{\color{cyan}\etocthenumber}}\etocname}{\etocpage}}
    {}%
  \etocstandarddisplaystyle
  \setcounter{tocdepth}{2}
  \tableofcontents #1
  \endgroup}
\makeatother
%--------
% 2 mars \thispartstats macro:
% 7 mars je retire la phrase Click on the names or numbers to get confirmation!

\newsavebox\firstnamei
\newsavebox\firstnumberi
\newsavebox\lastnamei
\newsavebox\lastnumberi
\newsavebox\firstnameii
\newsavebox\firstnumberii
\newsavebox\lastnameii
\newsavebox\lastnumberii
\newcounter{mycounti}
\newcounter{mycountii}

\newcommand*{\thispartstatsauxi}{}
\newcommand*{\thispartstatsauxii}{}
\newcommand*{\oldtocdepth}{}

\newcommand*{\thispartstats}{%
  \edef\oldtocdepth{\arabic{tocdepth}}%
  \setcounter{tocdepth}{2}%
  \setcounter{mycounti}{0}%
  \setcounter{mycountii}{0}%
  \def\thispartstatsauxi{%
         \sbox{\firstnamei}{\color{cyan}\etocname}%
         \sbox{\firstnumberi}{\color{cyan}\etocnumber}%
         \def\thispartstatsauxi{}}%
  \def\thispartstatsauxii{%
         \sbox{\firstnameii}{\color{cyan}\etocname}%
         \sbox{\firstnumberii}{\color{cyan}\etocnumber}%
         \def\thispartstatsauxii{}}%
  \begingroup
  \etocsetstyle{subsection}
    {}
    {}
    {\thispartstatsauxii
     \stepcounter{mycountii}%
     \sbox{\lastnameii}{\color{teal}\etocname}%
     \sbox{\lastnumberii}{\color{teal}\etocnumber}}
    {}
  \etocsetstyle{section}
    {}
    {}
    {\thispartstatsauxi
     \stepcounter{mycounti}%
     \sbox{\lastnamei}{\color{teal}\etocname}%
     \sbox{\lastnumberi}{\color{teal}\etocnumber}}
    {Here are some statistics for this part: it contains \arabic{mycounti}
      section\ifnum\value{mycounti}>1 s\fi{} and
      \arabic{mycountii} subsection\ifnum\value{mycountii}>1 s\fi. The name of
	 the first section is 
     \unhbox\firstnamei{} and the corresponding number is 
     \unhbox\firstnumberi. The name of the last section is 
     \unhbox\lastnamei{} and its number is \unhbox\lastnumberi. The name of
	 the first subsection is 
     \unhbox\firstnameii{} and the corresponding number is 
     \unhbox\firstnumberii. The name of the last subsection is 
     \unhbox\lastnameii{} and its number is \unhbox\lastnumberii.}% 
  \etocinline % don't do a \par automatically
  \etocsettocstyle {}{}
  \localtableofcontents
  \endgroup
  \setcounter{tocdepth}{\oldtocdepth}%
}

% \makeatletter
% \newcommand*{\restoretocdepth}{}%
% \newcommand*{\thispartstats}[1][{part}{section}]{%
% \edef\restoretocdepth{\noexpand\setcounter{tocdepth}{\arabic{tocdepth}}}%
% \setcounter{tocdepth}{\csname Etoc@\@secondoftwo#1@@\endcsname}%
% \begingroup
% \etocsetstyle{\@secondoftwo#1}
%     {\setcounter{mycount}{0}%
%      \def\thispartstatsaux{%
%          \sbox\firstname{\color{cyan}\etocname}%
%          \sbox\firstnumber{\color{cyan}\etocnumber}%
%          \def\thispartstatsaux{}}}
%     {}
%     {\thispartstatsaux
%      \stepcounter{mycount}%
%      \sbox\lastname{\color{teal}\etocname}%
%      \sbox\lastnumber{\color{teal}\etocnumber}}
%     {this \@firstoftwo#1 has \arabic{mycount}
%       \@secondoftwo#1\ifnum\c@mycount>1 s\fi. The name of
% 	   its first is 
%      \unhbox\firstname{} and the corresponding number is 
%      \unhbox\firstnumber. The name of the last \@secondoftwo#1 is 
%      \unhbox\lastname{} and its number is \unhbox\lastnumber. 
%      Click on the names or numbers to get confirmation!}% 
% \etocsettocstyle{Here are some statistics for this \@firstoftwo#1: }{\par}%
% %\etocinline
% \localtableofcontents
% \endgroup
% \restoretocdepth}
% \makeatother



\newcommand\toc{\csa{table\-of\-contents}\xspace}
\newcommand\localtoc{\csa{local\-table\-of\-contents}\xspace}
\newcommand\tocb{\csb{table\-of\-contents}\xspace}
\newcommand\localtocb{\csb{local\-table\-of\-contents}\xspace}

\newcommand\etoc{%
\texorpdfstring{{\color{joli}\ttfamily\bfseries etoc}}{etoc}\xspace}

\DeclareRobustCommand\csa[1]{{\ttfamily\hyphenchar\font45 \char`\\ #1}}
\DeclareRobustCommand\csb[1]
{{\color{blue}\ttfamily\hyphenchar\font45 \char`\\ #1}}

\newcommand\cshyp[1]{\texorpdfstring{\csa{#1}}{\textbackslash #1}}
\newcommand\csbhyp[1]{\texorpdfstring{\csb{#1}}{\textbackslash #1}}

\newcommand\lowast{\raisebox{-.25\height}{*}}
\newcommand\starit[1]{\csa{#1\lowast}}
\newcommand\staritb[1]{\csb{#1\lowast}}


\hyphenation{toc-depth sec-num-depth etoc-framed-style etoc-ruled-style}
\hyphenation{etoc-toc-style-with-marks etoc-ruled etoc-framed}
\hyphenation{etoc-stan-dard-display-style}
\hyphenation{etoc-the-page etoc-the-name etoc-the-num-ber etoc-if-num-bered}
\hyphenation{etoc-link etoc-stan-dard-lines etoc-toc-lines}
\hyphenation{etoc-in-ner-top-sep}
\hyphenation{etoc-the-linked-name etoc-the-linked-num-ber etoc-the-linked-page}
\hyphenation{etoc-the-link}
\hyphenation{etoc-skip-first-pre-fix etoc-the-page
  etoc-ruled-style etoc-mul-ti-col-style etoc-default-lines}

\frenchspacing

\renewcommand\familydefault\sfdefault

\setcounter{secnumdepth}{4}

\begin{document}

\rmfamily
\thispagestyle{empty}

\begin{center}
  {\normalfont\Large The \etoc package}\\
\textsc{Jean-Fran�ois Burnol}\par
  \footnotesize \ttfamily 
  jfbu (at) free (dot) fr\\
  Package version: \pkgversion\ (\pkgdate)\\
  Documentation generated from the source file\\
  with timestamp ``\dtxtimestamp''\par
\end{center}

\begin{abstract}
  The \etoc package gives to the user complete control on how
  the entries of the table of contents should be constituted
  from the \emph{name}, \emph{number}, and \emph{page number}
  of each sectioning unit. This goes via the definition of
  \emph{line styles} for each sectioning level used in the
  document. The package provides its own custom line styles.
  Simpler ones are given as examples in the documentation. The
  simplest usage will be to take advantage of the layout facilities
  of packages dealing with list environments.

  The \csa{tableofcontents} command may be used arbitrarily
  many times and it has a variant \csa{localtableofcontents}
  which prints tables of contents `local' to the current
  surrounding document unit. An extension of the
  \csa{label}/\csa{ref} syntax allows to reproduce (with another
  layout) a TOC defined somewhere else in the document.

  The formatting inherited (and possibly customized by other
  packages) from the document class may also be used in
  \emph{compatibility mode}. Regarding the \emph{global toc
    display}, \etoc provides pre-defined styles based on a
  multi-column format, optionally with a frame or a ruled
  title.

  As the assignment of levels to the sectioning units can be
  changed at any time, \etoc can be used in a quite general
  manner to also create custom ``lists of'', additionally to
  the tables of contents related to the document sectioning
  units. No auxiliary file is used apart from the usual |.toc|
  file.
\end{abstract}


\section*{Foreword}

Popular packages dealing with TOCs include |tocloft|,
|titletoc| and |minitoc|. Why another one? well, initially I
started \etoc for my own use, and only later found out about
the above mentioned packages ... 

As is well explained in the |tocloft| package documentation,
the standard \LaTeX{} layout for the Table of Contents is
buried in the class definitions. In particular, most of the
lengths therein are hardcoded, and the only way for the user
to change them is to recopy the class definitions into the
document and then change them to obtain what is desired
(within suitable \csa{makeatletter} and \csa{makeatother}).
The more reasonable alternative is to use a dedicated package such as
|tocloft| or to use another flexible document class.

However, although now things are hopefully not hard-coded, one
still has to go through the package or class interface. This
means one has to memorize a (possibly large) number of macros
which will serve only to this task, and one will always be
constrained to customizing one initially given layout.

The spirit of \etoc is something else. The user will deal with
the \emph{name}, the \emph{number} and the \emph{page number}
corresponding to each document sectional division (and
found in a line of the |.toc| file) in a completely arbitrary
manner: they are made available via the \csb{etocname},
\csb{etocnumber}, and \csb{etocpage} commands. 

\etoc is
compatible with the |article|, |book|, |report|, |scrartcl|,
|scrbook|, |scrreprt| and |memoir| classes. 

% Preliminary
% testing is positive, but the author not being familiar with the
% latter classes,\footnote{latter=all but |article|...} he does
% not have a stock of documents which could have been used to
% test all their bells and whistles in the presence of
% \etoc. \par


\begingroup
\renewcommand\lowast{{\normalsize\raisebox{-.4\height}{*}}}

\section*{\bfseries\small Change history}

\setlength{\columnsep}{\etoccolumnsep}

\makeatletter

\def\jfverbatim{\@beginparpenalty \predisplaypenalty 
  %\leftskip \z@
  \parindent \z@ 
  \parfillskip \@flushglue 
  \parskip \tw@\p@ plus 1fil\relax
  \let \do \@makeother \dospecials
  \catcode`\~\z@ 
  \footnotesize\normalfont\baselineskip10pt\relax
  \frenchspacing \obeyspaces \jf@xverbatim }


\begingroup 
\catcode `|=0 \catcode `[= 1 \catcode`]=2 
\catcode `\{=12 \catcode `\}=12 \catcode`\\=12 
|long|gdef|jf@xverbatim#1\end{jfverbatim}[#1|end[jfverbatim]]
|endgroup

\def\endjfverbatim{} 

\newdimen\jfverbadim

\everypar{\leftskip\jfverbadim\bgroup
          \def\par{\egroup\jfverbadim\z@\@@par}%
          \def\jfverbaspace#1{\ifcat\noexpand#1\noexpand~\unskip\else\ \fi#1}}

\begingroup\obeyspaces\def\x{\endgroup%
\def {\ifvmode\advance\jfverbadim.5em\relax\else\expandafter\jfverbaspace\fi}}\x

\makeatother

\begin{multicols}{2}
\begin{jfverbatim}
v1.07f [2013/03/07]

   new macros \etocthelinkedname, \etocthelinkednumber,
   \etocthelinkedpage, and \etocthelink.


v1.07e [2013/03/01]

   improvements in the package own line styles with
   regards to penalties and vertical spaces. 

   addition to the documentation of an example of
   a tree-like table of contents (uses tikz).

   more such examples added 2013/03/03.


v1.07d [2013/02/24]

   minor code improvements and new documentation 
   section "Another compatibility mode".


v1.07b [2013/02/02]

   removal of the \xspace from the macros \etocname, 
   \etocnumber, \etocpage.

   additional examples in the documentation.

   
v1.07 [2013/01/29]

   new commands:

      \etocthename, \etocthenumber, \etocthepage, \etoclink, 

      \etoctoccontentsline, \etoctoccontentsline~lowast

      \etocnopar, \etocaftercontentshook

   modified command: \etocmulticolstyle

   new documentation section "Surprising uses of etoc" which
   explains how to do "Lists of arbitrary things", in
   addition to the tables of contents.


v1.06 [2012/12/07]

   the standard macros \l@section etc... are modified only
   during the calls to \tableofcontents; they can thus be
   customized as will by the user (with the help of a
   package like tocloft) and this will be taken into account
   by etoc for the TOCs typeset in compatibility mode.


v1.05 [2012/12/01]

   \localtableofcontents replaces \tableofcontents~lowast (for
   compatibility with the memoir class).

   compatibility with KOMA-script and memoir document
   classes.


v1.04 [2012/11/24]

   a (possibly local) table of contents can be labeled:
 
       \tableofcontents \label{toc:1}

   and reproduced elsewhere in the document (with a possibly
   completely different layout):
 
       \tableofcontents \ref{toc:1}

   
v1.02 [2012/11/18]

   initial version.

\end{jfverbatim}
\end{multicols}
\endgroup  

\setcounter{tocdepth}{3}
\etoctoclines
\etocmarkboth\contentsname
\etocmulticolstyle[1]
     {\noindent\bfseries\Large
      \leaders\hrule height1pt\hfill
      \MakeUppercase{Table of Contents}}
\begingroup
  \etocsetlevel{subsection}{3}
  \etocsetlevel{subsubsection}{6}

\tableofcontents \label{toc:main}


\endgroup

\clearpage

\part{Overview}\label{part:one}

\thispartstats

\setcounter{tocdepth}{-3}
\localtableofcontents \label{toc:partone}
\setcounter{tocdepth}{3}

\section{Initial motivation: nested lists}

The initial impetus was to feed nested list environments
with the data consisting of the \emph{name}
(\csb{etocname}), \emph{number} (\csb{etocnumber}), and
\emph{page number} (\csb{etocpage}) as
recorded\footnote{the \emph{number} has to be disentangled
  from the \emph{name}, and in case |hyperref| is present, the
  hyperlink has to be redistributed around each of them.} in the
|.toc| file.
% \footnote{I got started about this through
%   reading this question on the
%   \TeX{} Stackexchange site:
%   \href{http://tex.stackexchange.com/questions/79682/can-i-get-a-list-of-all-sections-as-a-simple-enumerate-list}{tex.stackexchange.com/questions/79682}}
For example, typesetting the line corresponding to the first
sub-section in a given section would open a list environment
which would be closed only when a section, chapter, or part
line entry in the |.toc| file would be encountered. \etoc
allows to do this very easily and the opening and closing
may be for example |\begin{enumerate}| and |\end{enumerate}|
pairs, will all the customizing allowed by packages such as
|enumitem|.

\subsection{Limitations intrinsic to the use of environments}
\label{ssec:limitations}

There is a first limitation to this method: the |.toc| file
may contain other commands, such as language changing
commands, which do not expect to see their scope limited in
this way inside a group (\LaTeX's environments create groups).
Therefore the package own line styles
(illustrated by the main table of contents in this document)
do not make use of environments to avoid that problem.

A second limitation is that one may nest at most 4 levels of
enumerate environments, and 4 levels of itemize environment. I
tried alternating them and did succeed to nest 6 levels (and
not 8 alas . . . \footnote{this is
  surely a well-known issue which I did not at all
  investigate any further.}). With
\csb{etocnumber} as the optional parameter to \csa{item}:
\csa{item}|[\etocnumber]| one may transform the itemize into an
enumerated list... anyway, 4 levels of sectional divisions in
a TOC are generally sufficient, and again using |enumerate|
environments is only a possibility provided by \etoc, it is by
no means mandatory to use them in the line styles
specifications.

We will give in this manual a simple-minded example of nested
use of |enumerate| environments. More sophisticated examples
would use more sophisticated |enumitem| options. One may say
then that again the user has to memorize some customizing!
indeed, but the syntax and option names to memorize are in no
way related only to matters of tables of contents, hence an
economy of use of the poor brain.

The built-in default ``line styles'' provided by the package
do not make use of environments.

\section{Line styles and toc display style}

A distinction shall be made between the \emph{line styles},
\emph{i.e.} the way the name, number and page numbers are
used at each level, and the \emph{toc display style} (for
lack of a better name) which tells how the title should be
set, whether an entry in the |.toc| file should be made,
whether the contents should be typeset with multiple columns,
etc... the latter is governed by the command
\csa{etocsettocstyle} (or some higher-level commands) and the
former by the command \csa{etocsetstyle}.

\subsection{\csbhyp{etocsettocstyle} for the toc display}

The low-level \csb{etocsettocstyle} command allows to decide
what should be done before and after the line entries of the
TOC are typeset, and in particular how the title should be
printed. It has two arguments, the first one is executed
before the TOC contents (typically it will print
``Contents'' and define suitable head-marks) and the second is
executed after the TOC contents.

\etoc provides four (customizable) higher level toc styles:
\csb{etocmulticolstyle}, \csb{etoctocstyle},
\csb{etocruledstyle}, and \csb{etocframedstyle}. All use the
|multicol| package with a default of two columns
(single-column mode is of course allowed).

These commands must be followed either by \toc or \localtoc.

\subsection{\csbhyp{etocsetstyle} for the line styles}

The command to inform \etoc of what to do with \csa{etocname},
\csa{etocnumber}, and \csa{etocpage} is called
\csb{etocsetstyle}. It has five mandatory arguments. The first
one is the name of the sectional unit: a priori known names
are |book|, |part|, |chapter|, |section|, |subsection|,
|subsubsection|, |paragraph|, and |subparagraph|. The four
other arguments say: 1) \emph{what to do when this level is first
encountered, down from a more general one,} then 2) \& 3) (two
arguments, a `prefix' and a `contents') \emph{what to do when a new
entry of that type is found,} and 4) \emph{the last argument is the
code to execute when a division unit of higher importance is
again hit upon.}

\subsection{Compatibility mode}\label{ssub:compat}

Both for the ``line styles'' and the ``toc display style'', it is
possible to switch into a compatibility mode which uses the
defaults from the
document class.\footnote{for the ``toc display style''
  \etoc checks if it knows the class, and if not defaults to
  the |article| class layout.} This is activated by:\\
\makebox[5.2cm][l]{\csb{etocstandardlines}}
 |% `line entries' as without \usepackage{etoc}|\\
\makebox[5.2cm][l]{\csb{etocstandarddisplaystyle}}
 |% `toc display'  as without \usepackage{etoc}|


If the command \csa{etocsetstyle} has not been used in the
preamble the package will be at |\begin{document}| in this
  compatibility mode: hence just adding \csa{usepackage\{etoc\}}
  should hopefully not change anything to the look of a
  previously existing document, under the |article|, |book|,
  |report|, |scrartcl|, |scrbook|, |scrreprt| and |memoir|
  classes.

  Any use of \csa{etocsetstyle} in the preamble or body of the
  document turns off the compatibility mode for line styles
  (but not for the global display style; for this one needs to
  use the command \csa{etocsettocstyle}).

To exit after \csa{etocstandardlines} from compatibility mode
one uses the command \csb{etoctoclines}, which re-activates
the latest line styles as defined by \csa{etocsetstyle} (if
their scope was not limited to a group or environment). The
command \csb{etocdefaultlines} resets the line styles to be
the package initial default ones.


\section{Arbitrarily many TOCs, and local ones too}

\etoc allows arbitrarily many \csa{tableofcontents} commands in
your document. The line styles and the toc display style may of
course be changed in-between. Furthermore 
\csa{localtableofcontents} will print local tables of
contents\footnote{Up to version |1.04| we called this
  \starit{tableofcontents}, but for reasons of compatibility
  with the |memoir| class, we have decided to drop this
  usage.}: \emph{i.e.} all sections and sub-units inside a
given chapter, or all subsubsections and lower inside a given
subsection, etc . . .

\subsection{Labeling and reusing elsewhere}

\etoc allows the labeling of a TOC with \csa{label\{toc:A\}}
and will redisplay it elsewhere when told
\csa{tableofcontents}\csa{ref\{toc:A\}}. The actual layout
(title inclusive) used for the cloned TOC will be decided
locally. The line styles and toc display style (including the
title) will be the current ones and the current value of the
|tocdepth| counter is obeyed. As an example here is the table
of contents of Part \ref{part:custom}:

\begingroup

\etocstandardlines
\renewcommand{\etocbkgcolorcmd}{\color{green!5}}
\fboxsep1ex

\etocframed[1]{\fbox{\makebox[.5\linewidth]{\etocfontminusone 
\hyperref[toc:c]{I am from far away}}}}
\label{toc:d}
\ref{toc:c}

\endgroup

We actually did something like:
\centeredline{\toc\space\csa{label}|{toc:d}|\csa{ref}|{toc:c}|} Hence
\hyperref[toc:d]{\color{niceone}the present location}
can itself now be referred to via |\ref{toc:d}|: it gives
the id of this TOC\footnote{\emph{i.e}
{\color{niceone}\ref{toc:d}}, there was an invisible
TOC with id 2 and another one whose identity is left to the reader's sagacity.}
  in the sequence of 
document TOCs, and will be a link if package |hyperref| is
used.

However one should not use elsewhere \toc|\ref{toc:d}|. Due to
the way \etoc implements the cloning, the doubly cloned TOC
will be typeset as a full table of contents. So to clone
again, one should use the original: \toc|\ref{toc:c}|.

\subsection{The hyperref option \emph{bookmarksdepth}}
\label{ssec:bookmarksdepth}

When modifying the counter |tocdepth| for the purposes of
multiple uses of \toc or \localtoc, one should consider that
package |hyperref| by default takes into account the
\emph{current} value of the |tocdepth| counter to decide
whether the final |pdf| output will contain a bookmark
corresponding to the used sectioning command. Thus, one 
often will have to reset |tocdepth| to its previous value
immediately after the display of the table of contents. 

Or, there is the \emph{bookmarksdepth=n} option of package
|hyperref|, with \emph{n} the desired document bookmarks
depth, which can be numeric or the name of a level known to
|hyperref|. The present document passed |bookmarksdepth=3| as
option to |hyperref|, so as to not have to reset |tocdepth|
each time its value was changed.

\subsection{On manually adding layout commands to the
  \texorpdfstring{\texttt{.toc}}{.toc} file}

When displaying in that way many tables of contents in the same
document one should of course beware of the impact of adding
manually things
to the |.toc| file. For example, inserting
\csa{addtocontents}|{toc}{\string\clearpage}|
just
before a \csa{part} to fix the problem when some part entry (in
the table of contents) is isolated at the bottom of one page,
will cause problems with multiple TOCs: this \csa{clearpage}
will be executed by \etoc each time a \toc or \localtoc
command is encountered! The more prudent thing is to have
issued rather: {\csa{addtocontents}|{toc}{\string\myclearpage}|,}
to have a |\let\myclearpage\relax| at the top level of the
document and  to use where needed something like:
\begin{verbatim}
\let\myclearpage\clearpage
\tableofcontents
\let\myclearpage\relax
\end{verbatim}
The |memoir| class has the command \csa{settocdepth} which
writes a \csa{changetoc\-depth} command inside the |.toc| file.
This will impact the typesetting by \etoc of \emph{all} tables
of contents, with possibly unexpected results: imagine the
document has \csa{settocdepth}|{chapter}| at some point to
avoid having the sections from subsequent chapters be listed
in the main table of contents. Then a local table of contents
in one of these chapters will print a title but will be
without any entry. A solution is to do \csa{begingroup}
|\renewcommand*\changetocdepth[1]{}| \csa{localtableofcontents}
\csa{endgroup}, and to set the desired level for the local
table of contents with the other |memoir| command \csa{maxtocdepth}.
\footnote{The |memoir| class allows multiple calls to the \toc
  command, so these issues already arise there,
  independently of \etoc, see page 170 of the |memoir|
  manual.}

\subsection{Shuffling the levels with \csbhyp{etocsetlevel}}

The intrinsic levels manipulated by \etoc are numeric: from
|-2| (which corresponds to |book| in the |memoir| class) down
to |5| (|subparagraph|). But the assignment of a numeric level
to a given name can be modified at any time with the command
\csa{etocsetlevel}\marg{level\_name}\marg{n}. In conjunction
with the use of the \LaTeX{} |tocdepth| counter, this has
powerful applications: \meta{level\_name} does not have to
coincide with an actual document sectioning command, and \etoc
can be used to print arbitrary ``lists of things'', using no
other auxiliary file than the |.toc| file. This is
explained further in the section \ref{sec:surprising}.

\section{Two examples}

Here is a simple example of use of the package
functionalities. Immediately after the |\part{Overview}|
line in the source file we inserted:
\begin{verbatim}
    \setcounter{tocdepth}{-3}
    \localtableofcontents \label{toc:partone}
    \setcounter{tocdepth}{3}
\end{verbatim}
The |tocdepth| having been  set to -3, nothing at all was typeset, as
\etoc 
cancels printing even the heading of the TOC if the |tocdepth| is -3 or
less (and it is even ``-2 or less'', except for the |memoir| class).
We will then display here this TOC with:
\centeredline{|\tableofcontents \ref{toc:partone}|}

But first let us set up some line styles. We choose a style for sections and
sub-sections which would be suitable for, respectively, sections and
sub-sections in an average length memoir. The line style specifications have
some redundancy for clarity, and do not care about what to do at possible page
breaks. Also, they do not worry about potential multi-column use.

\begin{verbatim}
\begingroup % we start a group to keep the style changes local
\newlength{\tocleftmargin}    \setlength{\tocleftmargin}{5cm}
\newlength{\tocrightmargin}   \setlength{\tocrightmargin}{1cm}

\etocsetstyle{section}              % will pretend to be a Chapter
{\addvspace{1ex}\parfillskip0pt
 \leftskip\tocleftmargin            % (already done in title)
 \rightskip\the\tocrightmargin plus 1fil
 \parindent0pt\color{cyan}}         % (already done)
{\bfseries\LARGE\upshape\addvspace{1ex}\leavevmode}
{\llap{Chapter\hspace{.5em}{\etocnumber}\hspace{.75cm}}\etocname
 \hfill\makebox[-\tocrightmargin][l]{\makebox[0pt]{\etocpage}}\par}
{}

\etocsetstyle{subsection}           % will pretend to be a Section
{}
{\mdseries\large\addvspace{.5ex}\leavevmode}
{\llap{\etocnumber\hspace{.75cm}}\textit{\etocname}%
 \hfill\makebox[-\tocrightmargin][l]{\makebox[0pt]{\etocpage}}\par}
{}

\etocsettocstyle{\color{cyan}\parindent0pt \leftskip\tocleftmargin
  \leavevmode\leaders\hrule height 1pt\hfill\ 
  \huge\textit{My Beautiful Thesis}\par}{\bigskip}

% \color{cyan}\parindent0pt and \leftskip\tocleftmargin were repeated
% in the <start> code of the ``section style'', for clarity of code.

\tableofcontents \ref{toc:partone}
\endgroup
\end{verbatim}

\begingroup
\newlength{\tocleftmargin}
\setlength{\tocleftmargin}{5cm}
\newlength{\tocrightmargin}
\setlength{\tocrightmargin}{1cm}

\etocsetstyle{section}
{\addvspace{1ex}\parfillskip0pt
 \leftskip\tocleftmargin
 \rightskip\the\tocrightmargin plus 1fil
 \parindent0pt\color{cyan}}
{\bfseries\LARGE\upshape\addvspace{1ex}\leavevmode}
{\llap{Chapter\hspace{.5em}{\etocnumber}\hspace{.75cm}}\etocname
 \hfill\makebox[-\tocrightmargin][l]{\makebox[0pt]{\etocpage}}\par}
{}

\etocsetstyle{subsection}
{}
{\mdseries\large\addvspace{.5ex}\leavevmode}
{\llap{\etocnumber\hspace{.75cm}}\textit{\etocname}%
 \hfill\makebox[-\tocrightmargin][l]{\makebox[0pt]{\etocpage}}\par}
{}

\etocsettocstyle{\color{cyan}\parindent0pt\leftskip\tocleftmargin
  \leavevmode\leaders\hrule height 1pt\hfill\ 
  \huge\textit{My Beautiful Thesis}\par}{\bigskip}

\tableofcontents \ref{toc:partone}
\endgroup

As one can see, the code uses the basic commands from
\TeX/\LaTeX{} for paragraph layouts and an efficient mix of
plain \TeX{} and \LaTeX{} syntaxes.

Users not so knowledgeable in the \TeX{} syntax (for cause of having
been exposed only to \LaTeX{} ``introductions'')
have the possibility explained earlier to use nested
|enumerate| environments, of course this means mastering
another syntax.

However, using as here the \TeX{} primitive |\par|,\footnote{\csa{par} is
  redefined by \LaTeX{} but this is of no immediate
  relevance here.} and basic skip registers
|\leftskip|, |\rightskip|, \csa{parfillskip}, ... is by far the
surest way to completely understand and master what happens.
Any user of \LaTeX{} should learn their
signification.

\subsection{Again the table of contents of this part}\label{ssec:again}

Let us now try out some more sophisticated line styles. The
display will use the \csa{etocframedstyle} package command, which
requires that the produced table of contents fits on a single
page. With all the recent additions to the documentation there is
not enough room here, so we wrap it up in a figure environment and
it will show \hyperref[toc:floating]{\color{niceone}on the next page.}


\setcounter{tocdepth}{3}

\begin{figure*}[th!]\centering
\colorlet{subsecnum}{black}
\colorlet{secbackground}{green!30}
\colorlet{tocbackground}{red!20!green!20}

\renewcommand{\etocbkgcolorcmd}{\color{tocbackground}}
\renewcommand{\etocleftrulecolorcmd}{\color{tocbackground}}
\renewcommand{\etocrightrulecolorcmd}{\color{tocbackground}}
\renewcommand{\etocbottomrulecolorcmd}{\color{tocbackground}}
\renewcommand{\etoctoprulecolorcmd}{\color{tocbackground}}

\renewcommand{\etocleftrule}{\vrule width 3cm}
\renewcommand{\etocrightrule}{\vrule width 1cm}
\renewcommand{\etocbottomrule}{\hrule height 12pt}
\renewcommand{\etoctoprule}{\hrule height 12pt}

\renewcommand{\etocinnertopsep}{0pt}
\renewcommand{\etocinnerbottomsep}{0pt}
\renewcommand{\etocinnerleftsep}{0pt}
\renewcommand{\etocinnerrightsep}{0pt}

\newcommand\shiftedwhiterule[2]{%
    \hbox to \linewidth{\color{white}%
    \hskip#1\leaders\vrule height1pt\hfil}\nointerlineskip
\vskip#2}

\etocsetstyle{subsubsection}{\etocskipfirstprefix}
{\shiftedwhiterule{\leftskip}{6pt}}
{\sffamily\footnotesize
\leftskip2.3cm\hangindent1cm\rightskip.5cm\noindent
\hbox to 1cm{\color{subsecnum}\etocnumber\hss}%
\color{black}\etocname\leaders\hbox to .2cm{\hss.}\hfill
\rlap{\hbox to .5cm{\hss\etocpage\hskip.1cm}}\par
\nointerlineskip\vskip3pt}
{}

\etocsetstyle{subsection}{\etocskipfirstprefix}
{\shiftedwhiterule{1.5cm}{6pt}}
{\sffamily\small
\leftskip1.5cm\hangindent.8cm\rightskip.5cm\noindent
\hbox to .75cm{\color{subsecnum}\etocnumber\hss}%
\color{black}\etocname\leaders\hbox to .2cm{\hss.}\hfill
\rlap{\hbox to .5cm{\hss\etocpage\hskip.1cm}}\par
\nointerlineskip\vskip3pt}
{}

\newcommand{\coloredstuff}[2]{%
            \leftskip0pt\rightskip0pt\parskip0pt
            \fboxsep0pt % \colorbox uses \fboxsep also when no frame!
       \noindent\colorbox{secbackground}
               {\parbox{\linewidth}{%
                    \vskip5pt
                    {\noindent\color{#1}#2\par\nointerlineskip}
                    \vskip3pt}}%
       \par\nointerlineskip}

\etocsetstyle{section}{\coloredstuff{blue}
     {\hfil \bfseries\large Contents of Part One\hfil}}
{\vskip3pt\sffamily\small}
{\coloredstuff{blue}{\kern1cm\hbox to .5cm{\etocnumber\hss}%
    \etocname\hfill
{\hbox to .5cm{\hss\etocpage\hskip.1cm}}}%
 \vskip6pt}
{}


% attention car \ref{toc:floating} fera r�f�rence au compteur de
% toutes les toc, pas visibletoc. 

\etocframedstyle[1]{}
\tableofcontents \label{toc:floating} \ref{toc:partone}
\vspace{-\baselineskip}
\centeredline{|\tableofcontents \ref{toc:partone}| 
(\emph{cf.} \autoref{ssec:again} and \hyperref[toc:clone]{\color{niceone}this other toc})}
\end{figure*}

The actual design is not pre-built in \etoc; it uses its `framed' style with a
background color. The frame borders have been set to have the same color as the
one serving as background for the entire thing. This design (with other colors)
is in use also for \hyperref[toc:clone]{\color{niceone}this other toc}, and the
reader is referred to \hyperref[toc:clone]{\color{niceone}its location} for the
coding used.

\section{Surprising uses of \etoc}
\label{sec:surprising}

% This variant uses macros rather than boxes, which allows greater flexibility.

\makeatletter
\newcommand*\firstsubname       {}
\newcommand*\lastsubname        {}
\newcommand*\firstsubnumber     {}
\newcommand*\lastsubnumber      {}
\newcommand*\thissectionstatsaux{}

\newcommand*{\thissectionstats}{%
  \edef\oldtocdepth{\arabic{tocdepth}}%
  \setcounter{tocdepth}{2}%
  \setcounter{mycounti}{0}%
  \def\thissectionstatsaux{% ou plus simple si on ne veut pas le lien.
         \let\firstsubname\etocthelinkedname
         \let\firstsubnumber\etocthelinkednumber
         \def\thissectionstatsaux{}}
  \begingroup
  \etocsetstyle{subsection} {} {}
    {\thissectionstatsaux
     \stepcounter{mycounti}%
     \let\lastsubname\etocthelinkedname
     \let\lastsubnumber\etocthelinkednumber }
    {Here are some statistics for this section. It contains \arabic{mycounti}
      subsections. The name of 
	 its first is 
     \emph{\color{cyan}\firstsubname{}} and the corresponding number is 
     {\color{cyan}\firstsubnumber}. The name of the last subsection is 
     \emph{\color{teal}\lastsubname{}} and its number is 
     {\color{teal}\lastsubnumber}.}% 
   \etocsettocstyle {}{}
   \etocinline
   \localtableofcontents
  \endgroup
  \setcounter{tocdepth}{\oldtocdepth}%
}
\makeatother

% more general version:
% \makeatletter
% \newcommand*\firstsubname{}
% \newcommand*\lastsubname{}
% \newcommand*\firstsubnumber{}
% \newcommand*\lastsubnumber{}
% \renewcommand*{\thispartstats}[1][section]{%
% \edef\oldtocdepth{\arabic{tocdepth}}%
% \setcounter{tocdepth}{\csname Etoc@#1@@\endcsname}%
% \begingroup
% \etocsetstyle{book}{}{}{}{}%
% \etocsetstyle{part}{}{}{}{}%
% \etocsetstyle{chapter}{}{}{}{}%
% \etocsetstyle{section}{}{}{}{}%
% \etocsetstyle{subsection}{}{}{}{}%
% \etocsetstyle{subsubsection}{}{}{}{}%
% \etocsetstyle{paragraph}{}{}{}{}%
% \etocsetstyle{subparagraph}{}{}{}{}%
% \setcounter{mycounti}{0}%
% \def\thispartstatsaux{% ou plus simple si on ne veut pas le lien.
%        \toks@\expandafter{\etocthename}%
%        \edef\firstsubname{\noexpand\hyperlink{\Hy@tocdestname}{\the\toks@}}%
%        \toks@\expandafter{\etocthenumber}%
%        \edef\firstsubnumber{\noexpand\hyperlink{\Hy@tocdestname}{\the\toks@}}%
%        \def\thispartstatsaux{}}
% \etocsetstyle{#1}
%     {}
%     {}
%     {\thispartstatsaux
%      \stepcounter{mycounti}%
%      \toks@\expandafter{\etocthename}%
%      \edef\lastsubname{\noexpand\hyperlink{\Hy@tocdestname}{\the\toks@}}%
%      \toks@\expandafter{\etocthenumber}%
%      \edef\lastsubnumber{\noexpand\hyperlink{\Hy@tocdestname}{\the\toks@}}%
%     }
%     {}% 
% \etocsettocstyle{}
% {\arabic{mycounti}
%       #1\ifnum\c@mycounti>1 s\fi. The name of
% 	its first is 
%      \emph{\color{cyan}\firstsubname{}} and the corresponding number is 
%      {\color{cyan}\firstsubnumber}. The name of the last #1 is 
%      \emph{\color{teal}\lastsubname{}} and its number is 
%      {\color{teal}\lastsubnumber}. 
%      Click on the names or numbers to get confirmation!}%
% \etocinline
% Here are some statistics for this unit: it has\localtableofcontents\par
% \endgroup
% \setcounter{tocdepth}{\oldtocdepth}}
% \makeatother

\thissectionstats

\subsection{The TOC of TOCs}

\begingroup % \endgroup just after the \tableofcontents command
\etocinline
\etocsetlevel{part}{1}
\etocsetlevel{chapter}{1}
\etocsetlevel{visibletoc}{0}
\etocsetstyle{visibletoc}
    {\etocskipfirstprefix}
    {, }
    {{\color{niceone}\etocname}}
    {}
\etocsettocstyle{}{}
\setcounter{tocdepth}{0}


Here is the numbered and linked list of all tables of contents which are
displayed within this document: 
\tableofcontents\endgroup. And to obtain
it here we just wrote:\footnote{click the `\protect\thevisibletoc'
  in the list to get confirmation... and click on the previous `\number
  \numexpr\csname c@visibletoc\endcsname-1\relax' for a surprise...
\\ The \hyperref[toc:floating]{\color{niceone}floating TOC} \number
  \numexpr\csname c@visibletoc\endcsname-2\relax\ is responsible for
  some numbering madness here: the numbers are listed
  in the order the TOCs appear in the document; but the numbering
  itself is from the order of the TOCs in the \emph{source} of
  this document... }\par\smallskip 
{\leftskip1cm\rightskip2cm
  \ttfamily\small\baselineskip11pt \noindent Here is the
  numbered and linked list of all tables of contents which are
  displayed within this
  document:~\string\tableofcontents.\par}

  The preparatory work was the following. First, we defined a
  counter |visibletoc| whose vocation is to get incremented at
  each displayed toc. \etoc has its own private counter but it
  counts all TOCs, even those not displayed because the |tocdepth|
  value was |-2| or |-3|.

  We could have added manually |\refstepcounter{visibletoc}|
  and |\label| commands at all suitable locations in the
  document source, and we would then have used here |\ref|
  commands, but this imposes heavy manual editing of the source.

  There is a much better way: there is a hook
  \csb{etocaftertitlehook} and we told it to increment the
  |visibletoc| counter and to write a line to the |.toc| file,
  in a manner analogous to what sectioning commands such as
  |chapter|, |section|, or |subsection| do. As \etoc
  increments its own private counter even before typesetting
  the title of a table of contents, this provides (most of the
  time) a better link destination than any counter manipulated
  from inside \csa{etocaftertitlehook} (for which the link
  would target the area just after the title). So, rather than
  including |\refstepcounter{visibletoc}| inside
  \csa{etocaftertitlehook}, we just put there
  |\stepcounter{visibletoc}| followed by the command
  \csb{etoctoccontentsline}|{visibletoc}{\thevisibletoc}|. This \etoc
  command \csb{etoctoccontentsline}\marg{level\_name}\marg{name} has the same
  effect as:
  \centeredline{|    |%
    |\addcontentsline{toc}|\marg{level\_name}\marg{name}}
  but its usefulness is to circumvent\footnote{using \csa{addtocontents} rather
    than \csa{addcontentsline}} the patching for automatic creation
  of bookmarks done to \csa{addcontentsline} by the |hyperref| package, 
  as pdf bookmarks don't make 
  much sense here (and would elicit a complaint
  of |hyperref| that the bookmark level is `unknown').

  The package provides a starred variant
  \staritb{etoctoccontentsline}, which will allow the creation
  of bookmarks and has a third mandatory argument which is the
  Level to be used by these bookmarks.

  Finally, the preamble of the document did
  |\etocsetlevel{visibletoc}{6}|. The level |6| (or anything
  with a higher number) is ignored, even if |tocdepth| has
  value |10| for example; this is independently of whether \etoc
  uses the document class default line styles or its own line
  styles, or the ones defined by the user with the
  \csa{etocsetstyle} command. So there is no need to worry that
  something could go wrong. 

  Then, only here we have set |\etocsetlevel{visibletoc}{0}|. And to display
  only this kind of entries we assign temporarily to |part| and |chapter| level
  |1| (or anything higher than zero) and set |tocdepth| to the value |0|. We
  also did \csa{etocsetstyle\{visibletoc\}\{\string\etocskipfirstprefix\}\{,
    \}\{\string\etocname\}\{\}} which defines an inline display with the comma
  as separator. Finally, as \etoc issues |\par| automatically by default just
  before typesetting a table of contents, we used the command \csb{etocinline}
  (also known as \cs{etocnopar}) which turns off this behavior.

Here are the implementation details:

\begingroup
\begin{verbatim}
. . . in the preamble:
\newcounter{visibletoc}
\renewcommand{\etocaftertitlehook}
   {\stepcounter{visibletoc}\etoctoccontentsline{visibletoc}{\thevisibletoc}}
\etocsetlevel{visibletoc}{6}
. . .
\begin{document}
. . . document body
\subsection{Surprising uses of etoc}
\begingroup
    \etocinline
    \etocsetlevel{part}{1}
    \etocsetlevel{chapter}{1}
    \etocsetlevel{visibletoc}{0}
    \etocsetstyle{visibletoc}
        {\etocskipfirstprefix}{, }{{\color{niceone}\etocname}}{}
    \etocsettocstyle{}{}     % don't set any title, rules or frame or multicol!
    \setcounter{tocdepth}{0} % display only the `visibletoc' entries from .toc

Here is the numbered and linked list of all tables of contents which are
displayed within this document: \tableofcontents.
\endgroup
\end{verbatim}

\endgroup

After |\etocsetstyle{visibletoc}{..}{..}{..}{..}|, all future TOCs (not in
compatibility mode) will use the
defined style for level |0| (which is normally the level for
chapters). To keep these changes strictly local the simplest
manner is to put everything inside a group.

Subsection \ref{subsec:interverting} gives another use of the
shuffling of levels. 


\subsection{Arbitrary ``Lists Of...''}


This idea of interverting the levels is very powerful and allows
to let \etoc display lists of arbitrary things contained in the
document. All of that still
using nothing else than the |.toc| file! 
Example: imagine a document with dozens of
exercises, perhaps defined as |\newtheorem{exercise}{}[section]|. Let
us explain how to instruct \etoc to display an hyperlinked list
of all these exercises. For this we put in the preamble:
\begin{verbatim}
 \newtheorem{exerci}{}[section] 
    % the exercice number will be recoverable via \etocname: v--here--v
 \newcommand*{\exercisetotoc}{\etoctoccontentsline{exercise}{\theexerci}}
 \newenvironment{exercise}{\begin{exerci}\exercisetotoc}{\end{exerci}}
 \etocsetlevel{exercise}{6}
\end{verbatim}
In this way, \csa{etocname} will give the exercise number (but \csa{etocnumber}
will be empty). Had we used instead
\begin{verbatim}
 \newcommand*{\exercisetotoc}
    {\etoctoccontentsline{exercise}{\protect\numberline{\theexerci}}}
\end{verbatim}
the exercise number would then have been available via \csa{etocnumber}, and
\csa{etocname} would have been empty. It doesn't matter which one of the two
methods is used. The \etoc command \csa{etoctoccontentsline}|{..}{..}| is
provided as a substitute to \csa{addcon\-tentsline}|{toc}{..}{..}|: this is to
avoid the patching which is done by |hyperref| to \csa{addcontentsline} in its
process of creation of bookmarks. If one wants to authorize |hyperref| to create
bookmarks at a specific level \meta{n}, one can use (here with \meta{n}$=$|2|)
the starred variant \starit{etoctoccontentsline} which has an additional
argument:
\begin{verbatim}
\newcommand{\exercisetotoc}{\etoctoccontentsline*{exercise}{\theexerci}{2}}
\end{verbatim}

This example originates with question |94766| on the
\TeX-StackExchange site. The counter |exerci| is already
incremented by the |exerci| theorem environment, and
provides the correct destination for the link added by package
|hyperref|. The command \csa{exercisetotoc} adds for each exercise a line
to the |.toc| file, corresponding to a fictitious document
unit with name `|exercise|'. A
four-column list, including the sections, can then be
typeset with the following code:
\begin{verbatim}
  \setcounter{tocdepth}{2}     % sections are at level 1 and will show up
\begingroup
  \etocsetlevel{exercise}{2}   % but:
  \etocsetlevel{chapter}{3}    %     no chapters
  \etocsetlevel{subsection}{3} %     no subsections
  \etocsetlevel{part}{3}       %     no parts
  \etocsetstyle{exercise}{}{}  % \etocname = exercise number
    {\noindent\etocname\strut\leaders\etoctoclineleaders\hfill\etocpage\par}
    {\pagebreak[2]\vskip\baselineskip}
  \etocsetstyle{section}{}{}
    {\noindent\strut{\bfseries\large\etocnumber\hskip.5em\etocname}\par
     \nopagebreak[3]}{}
  \etocruledstyle[4]{\Large\bfseries List of the exercises}
  \setlength{\columnseprule}{.4pt}
  \tableofcontents
\endgroup
\end{verbatim}


In the above, recall that \LaTeX{} counters are global. The
current |tocdepth| value is |2|, and if not reset it will prevent
|hyperref| to assign bookmarks to sub-subsections (level |3|).
The global |hyperref| option \emph{bookmarksdepth} can be
used to avoid having to systematically reset |tocdepth| after
having changed it.


\subsection{A TOC with a fancy layout}

Another question (numbered |61297|) on the \TeX{} StackExchange
site was about using \LaTeX{} to obtain a table of contents where
the sections from a given chapter would be represented by a number
range (like 18--22 for a given chapter, 42--49 for another one ...
of course to be inserted automatically in the TOC). Here is
the result of my effort at using \etoc for this specific problem. How
this was done will be found on the above cited site.


%\clearpage

\begingroup

    \etocsetlevel{dummypart}{-1}
    \etocsetlevel{dummychapter}{0}
    \etocsetlevel{dummysection}{1}
    \etocsetlevel{part}{2}
    \etocsetlevel{chapter}{2}
    \etocsetlevel{section}{2}

    \newif\ifextraidone
    \newif\ifextraiidone

    \etocsetstyle{dummypart}
    {}
    {}
    {\begin{center}\Large\bfseries PART \etocnumber
    \ifextraiidone\\\etocname.\fi\end{center}
    \ifextraidone\else
    \noindent\hskip.7\linewidth
        \hbox to .2\linewidth
% f�vrier (ou janvier?) 2013
% je modifie ici en ajoutant un \hss, pour compenser la plus petite
% taille de txrm. Ok.
        {\hss\small\textsc{\bfseries Sections.\hss Page.}}\par\fi
    \extraidonetrue
    }
    {}

    \newcommand\mytocleaders{\hbox to .125\linewidth{...\hss}}
    \etocsetstyle{dummychapter}
    {}
    {\par\noindent\etocifnumbered
        {\makebox[.15\linewidth][r]{\bfseries\etocnumber.}}
        {\hspace*{.15\linewidth}}}
    {\hbox to .65\linewidth
        {\hspace{1em}\etocname\leaders\mytocleaders\hss}%
     \hskip-.1\linewidth
     \hbox to .2\linewidth{\hspace{1.5em}...\hss
                    \etocifnumbered
                        {\etocpage}
                        {\ifextraiidone\else\etocpage\fi}}%
     \hskip-.2\linewidth
    }
    {}

    \newbox\forsectionnumbers
    \makeatletter
% 21 f�vrier 2013: je rajoute \color@begingroup et \color@endgroup
    \etocsetstyle{dummysection}
    {\setbox\forsectionnumbers=\hbox to .1\linewidth
        \bgroup\color@begingroup\hss\etocskipfirstprefix}
    {\@gobble}
    {\etocnumber---}
    {\etocnumber\color@endgroup\egroup
    % for reasons I do not quite understand, in some pdf viewers the dots
    % do not completely disappear if here \fboxsep0pt is used.
    % (probl�mes li�s � l'anti-aliasing, sur Mac OS X, Skim, Preview...)
    % (j'ai maintenu ici la m�thode de mon post sur stackexchange, mais
    %  sans doute je pourrais r�fl�chir � une autre fa�on �vitant d'avoir �
    %  � effacer �) 
        \fboxsep=.5pt\colorbox{white}{\box\forsectionnumbers}\par
        \ifextraiidone\else
            \begin{center}\bfseries Concord.\end{center}
            \noindent\makebox[.15\linewidth][r]{\textsc{\bfseries Lesson.}}\par
        \fi\extraiidonetrue
    }
    \makeatother

    \renewcommand{\etocinnertopsep}{0pt}

    \etocruledstyle[1]{%
        \parbox{\linewidth}{%
            \centering
            \textsc{\bfseries\LARGE\MakeUppercase{Table of Contents}}\\
            \rule{.2\linewidth}{2pt}}%
    }
        
    \setcounter{tocdepth}{1}

    \tableofcontents
   

\makeatletter
    \def\adddummysection {\stepcounter{dummysection}%
      \addtocontents {toc}{\protect\contentsline 
          {dummysection}{\protect\numberline{\thedummysection}}%
                   {\the\dummypage }%
     \ifEtoc@hyperref{\@currentHref }\fi}}
 
    \def\adddummychapter #1{\stepcounter{dummychapter}%
      \advance\dummypage2
      \addtocontents {toc}{\protect\contentsline 
          {dummychapter}{\protect\numberline{\thedummychapter}#1}%
                   {\the\dummypage }%
     \ifEtoc@hyperref{\@currentHref }\fi}}


    \def\adddummychapno #1{\advance\dummypage2
      \addtocontents {toc}{\protect\contentsline 
          {dummychapter}{#1}{\the\dummypage }%
      \ifEtoc@hyperref{\@currentHref }\fi}}

    \def\adddummypart #1{\stepcounter{dummypart}%
       \advance\dummypage4
       \addtocontents {toc}{\protect\contentsline 
          {dummypart}{\protect\numberline{\Roman{dummypart}}#1}%
                     {\the\dummypage }%
       \ifEtoc@hyperref{\@currentHref }\fi}}
     
\makeatother
    
    \newcount\dummypage \dummypage-1

    \adddummypart {}
    

    \adddummychapno {Introductory}

    \newcount\tempcount

    \loop\advance\tempcount by 1
    \adddummysection
    \ifnum\tempcount<8
    \repeat

    \adddummychapter {Concord of Subject and Verb}

    \loop\advance\tempcount by 1
    \adddummysection
    \ifnum\tempcount<17
    \repeat

    \adddummychapter {Concord of Substantive and Adjective}

    \loop\advance\tempcount by 1
    \adddummysection
    \ifnum\tempcount<22
    \repeat

    \adddummychapno {Concord of Relative and its Antecedent}

    \loop\advance\tempcount by 1
    \adddummysection
    \ifnum\tempcount<25
    \repeat

    \adddummypart {Government}

    \adddummychapter {The Accusative Case}

    \adddummychapno {General uses}

    \loop\advance\tempcount by 1
    \adddummysection
    \ifnum\tempcount<30
    \repeat

    \adddummychapno {Particular uses}

    \loop\advance\tempcount by 1
    \adddummysection
    \ifnum\tempcount<37
    \repeat

    \adddummychapter {Verbs governing two Accusatives}

    \loop\advance\tempcount by 1
    \adddummysection
    \ifnum\tempcount<41
    \repeat

    \adddummychapter {The Causal}

    \loop\advance\tempcount by 1
    \adddummysection
    \ifnum\tempcount<49
    \repeat

    \adddummychapter {The Instrumental Case}

    \adddummychapno {General uses}

    \loop\advance\tempcount by 1
    \adddummysection
    \ifnum\tempcount<54
    \repeat

    \adddummychapno {Particular uses}

    \loop\advance\tempcount by 1
    \adddummysection
    \ifnum\tempcount<59
    \repeat

    \adddummychapter {The Dative Case}

    \adddummychapno {General uses}

    \loop\advance\tempcount by 1
    \adddummysection
    \ifnum\tempcount<65
    \repeat

    \adddummychapno {Particular uses}

    \loop\advance\tempcount by 1
    \adddummysection
    \ifnum\tempcount<71
    \repeat

    \adddummychapter {The Ablative Case}

    \adddummychapno {General uses}

    \loop\advance\tempcount by 1
    \adddummysection
    \ifnum\tempcount<75
    \repeat

    \adddummychapno {Particular uses}

    \loop\advance\tempcount by 1
    \adddummysection
    \ifnum\tempcount<86
    \repeat

    \adddummychapter {The Locative Case}

    \adddummychapno {General uses}

    \loop\advance\tempcount by 1
    \adddummysection
    \ifnum\tempcount<92
    \repeat

    \adddummychapno {Particular uses}

    \loop\advance\tempcount by 1
    \adddummysection
    \ifnum\tempcount<100
    \repeat
\endgroup
This is not an image inclusion, the TOC is produced from its original
|tex| source inserted in this document after replacement of 
|part|, |chapter| or |section| with |dummypart|, |dummychapter| and
|dummysection| (and there is also a dummy page count).  We
copied the line styles used in the original and
displayed the table of contents following:
\begin{verbatim}
    \etocsetlevel{dummypart}  {-1} \etocsetlevel{part}   {2}
    \etocsetlevel{dummychapter}{0} \etocsetlevel{chapter}{2}
    \etocsetlevel{dummysection}{1} \etocsetlevel{section}{2}
                    \setcounter{tocdepth}{1}
\end{verbatim}
Each chapter displays the numbers of only the first and the last sections it
  contains. A technique for doing this is explained in the
  \autoref{ssec:statistics}.  


\subsection{Another compatibility mode}

As explained in the section \ref{ssub:compat}, the commands
\csa{etocstandardlines} and \csa{etocstandarddisplaystyle} tell \etoc to,
essentially, act as an observer. The document class layout for the table of
contents is then perfectly obeyed. There is no way to customize this standard
layout (change fonts, margins, vertical spacings, etc...) from within the
package. For this, use some package dedicated to this task; because \etoc either
is (temporarily perhaps) in compatibility mode with no customization on its part
possible, or the user has specified the layout in \csa{etocsetstyle} commands
(and \csa{etocsettocstyle}) and is (supposedly...) in complete control.

Well, there is actually an alternative. It is possible to use the
\csa{etocsetstyle} commands to recreate an artificial compatibility mode, in
order to achieve effects like the following, all things being otherwise equal to
the document class defaults:
\begin{enumerate}[noitemsep]
\item get the |hyperref| link to encapsulate only the names, but not the numbers
  of each entry of the table of contents,
\item use the document class style for chapters and sections, but modify it only
  for subsections,
\item do either of the above only for some portions of the table of contents.
\end{enumerate}

Here is how to proceed. One puts in the preamble:
\begingroup
\def\MacroFont{\small\ttfamily\baselineskip11pt\relax}
\begin{verbatim}
\makeatletter
\newcommand{\MyStandardTOC}{%
  \begingroup
  \let\savedpartline\l@part
  \let\savedchapterline\l@chapter  %% remove if article/scrartcl class
  \let\savedsectionline\l@section
  \let\savedsubsectionline\l@subsection 
  % and so on if \subsubsection, etc... is used
  % 
  % for the book or article classes:
  \etocsetstyle{part}{}{}
    {\savedpartline{\etocnumber\hspace{1em}\etocname}{\etocpage}}{}%
  % for the scrbook or scrartcl classes:
  \etocsetstyle{part}{}{}
    {\savedpartline{\numberline{\etocnumber}\etocname}{\etocpage}}{}%
  % identical in book/article/scrbook/scrartcl classes:
  \etocsetstyle{chapter}{}{}   %%% only for book and scrbook
    {\savedchapterline{\numberline{\etocnumber}\etocname}{\etocpage}}{}%
  \etocsetstyle{section}{}{} 
    {\savedsectionline{\numberline{\etocnumber}\etocname}{\etocpage}}{}%
  \etocsetstyle{subsection}{}{}
    {\savedsubsectionline{\numberline{\etocnumber}\etocname}{\etocpage}}{}%
  % etc... if further sectioning units are used
  % (see the text for what to do with the memoir class)
  \etocstandarddisplaystyle % this is for the title, page-marks, etc...
  \tableofcontents     
  \endgroup}
\makeatother
\end{verbatim}
Of course if the document has only one table of contents then there is no need
to put the commands inside a macro, or even inside a group.\footnote{and if
moreover one just wants to keep the same layout as in the default, one may
question why using \etoc... there is \emph{one} good reason: numbers and names
are separately \texttt{hyperref} links, whereas normally there is only one link
holding both the number and the name corresponding to one toc entry.} With these
commands 
\etoc will construct a TOC completely identical to what would have been done by
one of the document class: |article|, |book|, |scrartcl|, |scrbook|. \footnote{For the
|memoir| class, one needs a bit more: each of the command \csa{booknumberline},
\csa{partnumberline} and \csa{chapternumberline} will have to be saved with a
\csa{let}, and, one then specifies:
\centeredline{%
\csa{etocsetstyle\{chapter\}\{\}\{\}\{\string\savedchapterline\{%
\string\savedchapternumberline}}
\hbox to \linewidth{\hfil\hfil \ttfamily 
        \{\csa{etocnumber}\}\csa{etocname}\}\{\csa{etocpage}\}\}\{\}\hfil} 
(and analogously for |part|, respectively |book|).}
The number and the name of each entry are each separately an |hyperref| link, as
is always the case with \etoc, when not in compatibility mode. Replacing
\csa{etocnumber} with \csa{etocthenumber} will give a TOC where the numbers are
not links anymore, but the names still are. Or one may decide to use \csa{etocthename}
and keep an hyperlinked number with
\csa{etocnumber}. 

Here is a subtler example where one only marginally modifies the
sections (adding color to the number and removing the |hyperref| link) and keeps
the subsections as in the default, \emph{except} for those of one specific
section, for which the layout is completely modified:
\MyQuasiStandardTOC{\ref{toc:partone}}
\bigskip
This example only has sections and subsections, and the code used in \csa{MyStandardTOC} was:
\begin{verbatim}
\etocsetstyle{section}{}
  {\ifnum\etocthenumber=3
      \etocsetstyle{subsection}
        {\par\nopagebreak\begingroup
         \leftskip1.5em \rightskip\@tocrmarg \parfillskip\@flushglue 
         \parindent 0pt \normalfont\normalsize\rmfamily\itshape
         % \columnsep1em
         % \begin{minipage}{\dimexpr\linewidth-\leftskip-\rightskip\relax}%
         % \begin{multicols}{2}%
         \etocskipfirstprefix}
        {\allowbreak\,--\,}
        {\etocname\ \textup{(\etocnumber)}}
        {.\par\endgroup}%
        % {.\par\end{multicols}\end{minipage}\par\endgroup}%
    \else
      \etocsetstyle{subsection}
       {}{}
       {\savedsubsectionline{\numberline{\etocnumber}\etocname}{\etocpage}}
       {}%
    \fi} 
  {\savedsectionline{\numberline{{\color{cyan}\etocthenumber}}\etocname}{\etocpage}}
  {}%
\end{verbatim}
\endgroup

Notice the page head-mark added by this standard TOC. Sections and subsections
are printed exactly as in the default (except for the subsections of one
specific user-chosen section and except for the color of the section numbers),
with no need to specify explicitely any length, font or other formatting
instructions. But we had to examine the |scrartcl| sources to determine what to
use for \csa{leftskip} and \csa{rightskip} for our customized subsection
entries.

Also, a fancier layout has been commented out.

\subsection{The TOC as a tree}\label{tocastree}

Using |tikz|\footnote{\url{http://ctan.org/pkg/pgf}} and the package
|tikz-qtree|\footnote{\url{http://ctan.org/pkg/tikz-qtree}} we shall display the
table of contents of this part as a tree. The technique (this whole subsection
perhaps should have double dangerous-bend signs) is to use the \etoc modified
command \csa{tableofcontents} not for typesetting, but to prepare a
macro, or rather here a token list with name \csa{treetok}, with all the
instructions to be executed later. Putting \csa{etocnumber} or \csa{etocname}
commands in \csa{treetok} would be of no use: to which number or name would they
refer to, in such a delayed execution? Rather the \emph{contents} of
\csa{etocthenumber} or \csa{etocthename} are added with suitable decoration to
\csa{treetok}: the \etoc line styles are modified to expand once
\csa{etocthename} (for example) at the time of the execution and then add the
outcome to \csa{treetok} for later execution in a |tikzpicture|.

\newtoks\treetok
\newtoks\tmptok

\newcommand*\qtreenode{}

\newcommand*\appendtotok[2]{% #1=token list, #2=macro, expands once #2
  #1\expandafter\expandafter\expandafter
    {\expandafter\the\expandafter #1#2}}


% 7 mars: j'utilise \etocthelinkedname d�fini � partir de 1.07f
% mais bien s�r je sais que �a ne marche pas (sauf dans l'arbre pour le lien de
% la section du bas) dans un QTree. Mais �a marche parfaitement dans un arbre
% TikZ natif.
\newcommand*\PrepareSectionNode{%
  \tmptok {\centering\bfseries}%
  \appendtotok\tmptok\etocthelinkedname
  \edef\qtreenode{ [.{\noexpand\parbox{2cm}{\the\tmptok}}}%
}

% 7 mars: j'utilise maintenant \etocthelinkedname
\newcommand*{\PrepareSubsectionNode}{%
  \tmptok {\raggedright}%  j'ai essay� aussi avec \sloppy
  \appendtotok\tmptok\etocthelinkedname
  \edef\qtreenode{ [.{\noexpand\parbox{6cm}{\the\tmptok}}}%
}

\etocsetstyle{section}
  {\etocskipfirstprefix}
  {\appendtotok\treetok{ ]}}
  {\PrepareSectionNode \appendtotok\treetok\qtreenode}
  {\appendtotok\treetok{ ]}}

\etocsetstyle{subsection}
  {\etocskipfirstprefix}
  {\appendtotok\treetok{ ]}}
  {\PrepareSubsectionNode \appendtotok\treetok\qtreenode}
  {\appendtotok\treetok{ ]}}

\etocsettocstyle
    {\treetok{\Tree [.{\hyperref[part:one]{Overview}}}}
    {\global\appendtotok\treetok{ ]}}

\setcounter{tocdepth}{2}
\smallskip
\tableofcontents \ref{toc:partone}

{\centering
  \begin{tikzpicture}[grow'=right]
  \tikzset{sibling distance=1ex,
         level 1/.style={level distance=3cm},
         level 2/.style={level distance=6cm},
         every level 0 node/.style={draw,fill=cyan!5,inner sep=6pt},
         every level 1 node/.style={draw,circle,thick,fill=blue!5},
         every level 2 node/.style={draw,thick,fill=red!5},   
         edge from parent/.style= {%
             draw, thick, color=teal,
             edge from parent path=
                {[dashed](\tikzparentnode.east) -- (\tikzchildnode.west)}}
         } 
  \the\treetok
  \end{tikzpicture}
\par
}
\smallskip

We told \etoc to insert the |hyperref|
links inside the argument to the
  \csa{Tree} command (from package |tikz-qtree|).%
\footnote{this uses the macro \csa{etocthelinkedname} introduced with
package version |1.07f|. The same effect was achieved in earlier versions
in a more complicated 
manner.} But here in
  our use of |tikz-qtree| we have been confronted with the problem that the
  hyperlinks, even if correctly added to the \csa{treetok} token list, appear
  in the final document in wrong locations (except for the bottom section and
the bottom subsection nodes).
\footnote{|dvipdfmx| complains:\quad
|** WARNING ** Annotation out of page boundary.|\quad
|Current page's MediaBox: [0 0 595.276 841.89]|\quad
|Annotation: [94.3997 1200.33 146.512 1209.98]|. 
I read on the internet that there is some problem with |pgf| with regards to
hyperlinks, but I don't know if this is the same issue (as this is my first
ever use of TikZ). [2013/03/01]\\
Update [2013/03/03]: it \emph{does} work when using the regular TikZ syntax
for trees, so perhaps there is here some interaction with |tikz-qtree|. See next section.}

Here is the code used, the whole thing being with double dangerous-bend
signs, I shall not comment much the actual \LaTeX{} programming used
therein:

\begingroup
\def\MacroFont{\small\ttfamily\baselineskip10pt\relax}
\begin{verbatim}
\newtoks\treetok  \newtoks\tmptok  \newcommand*\qtreenode{}
\newcommand*{\appendtotok}[2]{% #1=token list, #2=macro, expands once #2
  #1\expandafter\expandafter\expandafter
    {\expandafter\the\expandafter #1#2}}

\newcommand*{\PrepareSectionNode}{%
  \tmptok {\centering\bfseries}%
  \appendtotok\tmptok\etocthelinkedname
  \edef\qtreenode{ [.{\noexpand\parbox{2cm}{\the\tmptok}}}%
}
\newcommand*{\PrepareSubsectionNode}{%
  \tmptok {\raggedright}%
  \appendtotok\tmptok\etocthelinkedname
  \edef\qtreenode{ [.{\noexpand\parbox{6cm}{\the\tmptok}}}% 
}
%%%% only sections and subsections:
\etocsetstyle{section}
  {\etocskipfirstprefix}
  {\appendtotok\treetok{ ]}}
  {\PrepareSectionNode \appendtotok\treetok\qtreenode}
  {\appendtotok\treetok{ ]}}
\etocsetstyle{subsection}
  {\etocskipfirstprefix}
  {\appendtotok\treetok{ ]}}
  {\PrepareSubsectionNode \appendtotok\treetok\qtreenode}
  {\appendtotok\treetok{ ]}}

\etocsettocstyle
    {\treetok{\Tree [.Overview}}
    {\global\appendtotok\treetok{ ]}}
% See the tikz-qtree (David Chiang) documentation for the Qtree syntax of
% Alexis Dimitriadis (the spaces at various places before the square
% brackets are important).

% The whole effect of \tableofcontents will be to fill the token list
% \treetok according to the syntax expected by tikz-qtree. The opening
% portion of \etocsettocstyle initializes \treetok, and the closing portion
% adds a final square bracket. A \global is needed as etoc always creates a
% group when typesetting a TOC.

% This QTree syntax has the advantage to use square brackets and not braces,
% which eases things quite a bit in a TeX context. The next section explains
% how to create a token list with the original TikZ syntax for trees.

\tableofcontents \ref{toc:partone} 
% Time now to display the tree (see tikz-qtree documentation)
\begin{tikzpicture}[grow'=right]
\tikzset{sibling distance=1ex,
         level 1/.style={level distance=4cm},
         level 2/.style={level distance=6cm},
         every level 0 node/.style={draw,fill=cyan!5,inner sep=6pt},
         every level 1 node/.style={circle,draw,thick,fill=blue!5},
         every level 2 node/.style={draw,thick,fill=red!5},   
         edge from parent/.style= {%
             draw, thick, color=teal,
             edge from parent path=
                {[dashed](\tikzparentnode.east) -- (\tikzchildnode.west)}}
         }
\the\treetok
\end{tikzpicture}
\end{verbatim}
\endgroup


\subsection{The TOC as a molecule}\label{ssec:molecule}

It is also possible to construct a TOC tree obeying the TikZ syntax for trees:
this is a more complicated task for the \etoc line styles for reasons related to
the importance of braces in \TeX{}, they need, when filling up the token list to
be always balanced at each step (unfortunately \textsc{Leslie Lamport}'s book
has no mention whatsoever of token lists, and \LaTeX{} does not really expect
users to know about them; this whole section is thus for advanced users).

The simplest strategy is to allocate a token list (or use a macro) for each
level used: we may need a \csa{parttok}, a \csa{chaptertok}, a
\csa{sectiontok} and a \csa{subsectiontok}, to help in the task of filling
up the total \csa{treetok}. As we are interested here in the table of
contents of this (or another) document part, only a \csa{sectiontok} and a
\csa{subsectiontok} will be needed.

And the nice thing is that now, our hyperlinks do work. As mentioned already we
can not use the commands \csa{etocname}, etc\dots, for delayed execution. We can
on the other hand store the current values of \csa{etocthename}, etc\dots, but
don't get then the |hyperref| links. Earlier versions of this documentation
explained how to do things with |\hyperlink| and |\Hy@tocdestname|, but since
|1.07f|, the package provides \csa{etocthelinkedname}, etc\dots, specifically
for 
this kind of task.

\begingroup
\def\MacroFont{\small\ttfamily\baselineskip10pt\relax}
\begin{verbatim}
% \newtoks\treetok
\newtoks\sectiontok \newtoks\subsectiontok \newcommand*{\treenode}{}

\newcommand*{\appendchildtree}[2]{% token list t1 becomes: t1 child {t2}
   \edef\tmp{\the#1 child {\the#2}}%
   #1\expandafter{\tmp}%
}
\newcommand*{\preparetreenode}{%
  \tmptok\expandafter{\etocthelinkednumber}% expanded one time (mandatory)
  \edef\treenode{node {\the\tmptok}}%
}

\setcounter{tocdepth}{2}
\etocsetstyle{section}
  {\etocskipfirstprefix}
  {\appendchildtree\treetok\sectiontok}
  {\preparetreenode \sectiontok\expandafter{\treenode}}
  {\appendchildtree\treetok\sectiontok}

\etocsetstyle{subsection}
  {\etocskipfirstprefix}
  {\appendchildtree\sectiontok\subsectiontok}
  {\preparetreenode \subsectiontok\expandafter{\treenode}}
  {\appendchildtree\sectiontok\subsectiontok}

\etocsettocstyle
  {\treetok{\node {\hyperref[part:one]{Overview}}}}
  {\global\appendtotok\treetok{ ;}}

\tableofcontents \ref{toc:partone}
\end{verbatim}


At this stage, \cs{treetok} has been prepared, and a TikZ picture can be
created (it uses the TikZ library |trees| for its ``cyclic'' grow method):
\begin{verbatim}
\begin{center}   
   \begin{tikzpicture}
              [grow cyclic,
               level 1/.style={level distance=4cm,sibling angle=60},
               level 2/.style={level distance=2cm,sibling angle=60},
               every node/.style={ball color=red,circle,text=cyan},
               edge from parent path={[dashed,very thick,color=cyan]
                           (\tikzparentnode) --(\tikzchildnode)}] 
     \the\treetok
   \end{tikzpicture}
\end{center}
\end{verbatim}
\endgroup


\newtoks\sectiontok
\newtoks\subsectiontok
\newcommand*\treenode {}

\newcommand*{\appendchildtree}[2]{% token list t1 becomes: t1 child {t2}
   \edef\tmp{\the#1 child {\the#2}}%
   #1\expandafter{\tmp}%
}

\newcommand*{\preparetreenode}{%
  \tmptok\expandafter{\etocthelinkednumber}%
  \edef\treenode{node {\the\tmptok}}%
}

\etocsetstyle{section}
  {\etocskipfirstprefix}
  {\appendchildtree\treetok\sectiontok}
  {\preparetreenode
   \sectiontok\expandafter{\treenode}}
  {\appendchildtree\treetok\sectiontok}

\etocsetstyle{subsection}
  {\etocskipfirstprefix}
  {\appendchildtree\sectiontok\subsectiontok}
  {\preparetreenode
   \subsectiontok\expandafter{\treenode}}
  {\appendchildtree\sectiontok\subsectiontok}

\setcounter{tocdepth}{2}

\etocsettocstyle
  {\treetok{\node {\hyperref[part:one]{Overview}}}}
  {\global\appendtotok\treetok{ ;}}

\tableofcontents \ref{toc:partone}

\begin{center}  
   \begin{tikzpicture}
              [grow cyclic,
               level 1/.style={level distance=4cm,sibling angle=60},
               level 2/.style={level distance=2cm,sibling angle=60},
               every node/.style={ball color=red,circle,text=cyan},
               edge from parent path={[dashed,very thick,color=cyan]
                           (\tikzparentnode) --(\tikzchildnode)}] 
     \the\treetok
   \end{tikzpicture}
\end{center}

This is nice, and especially so as this ``molecule TOC'' is fully hyperlinked!
I don't know why it goes wrong for the trees
constructed with the \cs{Tree} command of package |tikz-qtree|, contrarily to
those coded as here with the original ``child'' TikZ syntax.


\etocsettocstyle
  {\treetok{\node {\autoref{part:globalcmds}}}}
  {\global\appendtotok\treetok{ ;}}

\tableofcontents \ref{toc:globalcmds}

\noindent  
\parbox{4cm}{\begin{tikzpicture}
              [grow cyclic,
               level 1/.style={level distance=2.5cm,sibling angle=60},
               level 2/.style={level distance=1cm,sibling angle=50},
               every node/.style={ball color=red,circle,text=cyan},
               edge from parent path={[very thick,color=cyan]
                       (\tikzparentnode) --(\tikzchildnode)}] 
     \the\treetok
   \end{tikzpicture}}%
\begin{minipage}{\dimexpr\linewidth-4cm\relax}
  On the side, the (fully hyperlinked) table of contents of
  \autoref{part:globalcmds}.
  \def\MacroFont{\small\ttfamily\hyphenchar\font-1 \baselineskip10pt\relax}%
\begin{verbatim}
\etocsettocstyle
  {\treetok{\node {\autoref{part:globalcmds}}}}
  {\global\appendtotok\treetok{ ;}}
\tableofcontents \ref{toc:globalcmds}
\noindent  
\parbox{4cm}{%
   \begin{tikzpicture}
      [grow cyclic,
       level 1/.style={level distance=2.5cm,sibling angle=60},
       level 2/.style={level distance=1cm,sibling angle=50},
       every node/.style={ball color=red,circle,text=cyan},
       edge from parent path={[very thick,color=cyan]
               (\tikzparentnode) --(\tikzchildnode)}] 
     \the\treetok
   \end{tikzpicture}}%
...
\end{verbatim}
\end{minipage}

\part{Package commands for line styles}

\thispartstats

\setcounter{tocdepth}{3}

\etocsetstyle{section}
{\begin{enumerate}[leftmargin=.75cm, label=\etocifnumbered
      {{\fboxrule1pt\fcolorbox{green}{white}{\etocnumber}}}{}]}
% Dec 7, 2012. I hit upon a problem whose origin I found was
% with xcolor, as xcolor modifies \fbox
% (try \section{\fbox{A}} with and without xcolor). The fix
% was to \protect the \fbox used in the |enumitem| label.
% Strangely enough, with |hyperref| active, the problem did
% not show up. Anyhow I now use \fcolorbox reather than \fbox
% and there is no problem anymore. Also I don't have now to
% use \normalcolor which already needed protection.
{\normalsize\bfseries\rmfamily\item}
{\etocname{} (page \etocpage)}
{\end{enumerate}}

\etocsetstyle{subsection}
{\begin{enumerate}[leftmargin=0cm, label=\etocnumber]}
{\normalfont \item}
{\etocname{} (p.~\etocpage)}
{\end{enumerate}}

\etocsetstyle{subsubsection}
{\par\nobreak\begingroup\normalfont\footnotesize\itshape\etocskipfirstprefix}
{\allowbreak\,--\,}
{\etocname}
{.\hfil\par\endgroup\pagebreak[3]}

% 27 janvier 22:30
% je d�finis les macros (non prot�g�es) 
% \etocthename, \etocthenumber, \etocthepage

\etocruledstyle[1]{\etocfontminusone\color{green}%
     \fboxrule1pt\fboxsep1ex
     \framebox[\linewidth]
              {\normalcolor\hss Contents of this second part\hss}}
\localtableofcontents \label{toc:a}

\section{The \csbhyp{etocsetstyle} command}


\subsection{The \csbhyp{etocname} and \csbhyp{etocpage} commands}

Let us explain how \etoc was used to produce the table of
contents displayed at the beginning of this second part.
This
is a local table of contents, and we used the command \localtoc.


We shall distinguish between the \emph{line styles} and the
\emph{toc display style}. The line styles were (essentially)
obtained in the following manner:%
\footnote{the present document has
  {\ttfamily\string\renewcommand\string{%
      \string\familydefault\string}\string{\string\sfdefault\string}}
  in its preamble, hence \csa{normalfont} switches to the
  |sans| typeface; so in the section line-style, I wrote
  \csa{rmfamily} instead.}

\begingroup\small
\begin{verbatim}
\etocsetstyle{section}
{\begin{enumerate}}
{\normalsize\bfseries\rmfamily\item}
{\etocname{} (page \etocpage)}
{\end{enumerate}}

\etocsetstyle{subsection}
{\begin{enumerate}}
{\normalfont\item}
{\etocname{} (p.~\etocpage)}
{\end{enumerate}}

\etocsetstyle{subsubsection}
{\par\nobreak\begingroup\normalfont
        \footnotesize\itshape\etocskipfirstprefix}
{\allowbreak\,--\,}
{\etocname}
{.\hfil\par\endgroup\pagebreak[3]}
\end{verbatim}
\endgroup

These provisory style definitions rely on the automatic
numbering generated by the |enumerate| environments but it is
much better to use the further command \csb{etocnumber} inside
the item label, which gives the real thing. The improved
definitions will thus be explained later.

Each \csb{etocsetstyle} command has five mandatory arguments:\\
\hbox to \linewidth
{\hfil\csb{etocsetstyle}\color{blue}\marg{levelname}%
  \marg{start}\marg{prefix}\marg{contents}\marg{finish}\hfil}
The initially recognized \meta{levelname}'s are the sectioning
levels of the standard document classes: from \emph{part} (or
\emph{book} which is used by the |memoir| class) down to
\emph{subparagraph}.

The \meta{start} code is executed when a toc entry of that level is
encountered and the previous one was at a higher level. The
\meta{finish} code is executed when one again encounters a higher
level toc entry. In the mean-time all entries for that level are
typeset by executing first the \meta{prefix} code and then the
\meta{contents} code. 


The (robust) commands \csb{etocname}, \csb{etocnumber} and \csb{etocpage} are
provided 
for use inside the \marg{prefix} and \marg{contents} parts of the
\csa{etocsetstyle} specification. They represent of course, the name, number, and
page number of the corresponding toc
entry.\footnote{\protect\fbox{%
\parbox{.9\textwidth}{up to version
\texttt{1.07a} the package put an \csa{xspace} in each of
\csa{etocname}, \csa{etocnumber}, and \csa{etocpage}, but this
wasn't such a great idea and has now been removed.}}} If package |hyperref| is
active in the document and has added hyperlinks to the TOC data, then these
links are kept in the commands \csa{etocname}, \csa{etocnumber} and
\csa{etocpage} (this last one will have a link only if |hyperref| was passed option \emph{linktoc=all}.)

\subsection{The \csbhyp{etocskipfirstprefix} command}

The chosen |subsubsection| style also uses the command
\csb{etocskipfirstprefix}, which, if present, \emph{must} be
the very last one in the \emph{start} code. It instructs to not
use for the first item the specified ``prefix'' code.

With this style, one would have 
to be imaginative to design something then for paragraph and
subparagraph entries! perhaps as superscripts? Well, usually
one does not need paragraphs and subparagraphs numbered and
listed in the TOC, so our putative user here chose a design
where no provision is made for them and added the definitive:
\begin{verbatim}
\etocsetstyle{paragraph}{}{}{}{}
\etocsetstyle{subparagraph}{}{}{}{}
\end{verbatim}
This is also the situation with the default package line styles!

\subsection{The \csbhyp{etocnumber} command}
So far, our specifications would use the numbering generated
by the |enumerate| environments, but of course we generally want
the actual numbers as found in the |.toc| file. This is
available via the \csb{etocnumber} command. To get the labels
in the |enumerate| list to use it we can proceed with the
syntax {\ttfamily label=\char32} from the package |enumitem|:
\begin{verbatim}
\etocsetstyle{section}
{\begin{enumerate}[label=\etocnumber]}
{\normalsize\bfseries\rmfamily\item}
{\etocname{} (page \etocpage)}
{\end{enumerate}}
\end{verbatim}
Rather than just \csa{etocnumber} we then used something like
|\fbox{\etocnumber}|. Note that \csa{etocnumber} is a robust
command which explains why it can be used inside the label specification
without needing an added |\protect|.


\subsubsection{The \csbhyp{etocifnumbered} switch}

The \csa{fbox} would give an unaesthetic result in the case of
an unnumbered section (which ended up in the table of
contents via an \csa{addcontentsline} command).\footnote{as
  seen we use \csa{fcolorbox} rather than \csa{fbox}. Due
  to some redefinition made by package |xcolor|, had we used
  \csa{fbox} (and not used |hyperref|) we would have needed
  \csa{protect}\csa{fbox}.}

The \csa{etocifnumbered}\marg{A}\marg{B} command executes
\meta{A} if the number exists, and \meta{B} if not. So we use
it in the code which was finally chosen for the |section| level:
\begin{verbatim}
\etocsetstyle{section}
{\begin{enumerate}[leftmargin=.75cm, label=\etocifnumbered
      {{\fboxrule1pt\fcolorbox{green}{white}{\etocnumber}}}{}]}
{\normalsize\bfseries\rmfamily\item}
{\etocname{} (page \etocpage)}
{\end{enumerate}}

\etocsetstyle{subsection}
{\begin{enumerate}[leftmargin=0cm, label=\etocnumber]}
{\normalfont \item}
{\etocname{} (p.~\etocpage)}
{\end{enumerate}}
\end{verbatim}

If we had changed only the |section| level, and not the
|subsection| level, an error on compilation would have occurred
because the package style for subsections expects to start `in
vertical mode'. An additional \csa{par} token in the
\meta{contents} part of the |section| level would have fixed
this: |{...(page \etocpage)\par}|.

\subsection{The \csbhyp{etocthename}, \csbhyp{etocthenumber}, and
\csbhyp{etocthepage} commands}

It is sometimes desirable to have access to the name, number
and page number without the hyperref link data: something
similar to the starred variant of the \csa{ref} command, when
package |hyperref| is used. For example one may wish to use
the unit or page number in some kind of numeric context, or
change its formatting. This is provided by the |\etocthe...|
commands.

These commands are not protected, so in moving argument
contexts (for example in a label specification) they should be
preceded by |\protect|.

\subsection{The \csbhyp{etoclink} command}

The command \csb{etoclink}\marg{linkname} can be used in the line style
specifications in a manner analogous to the argument-less
commands \csa{etocname}, \csa{etocnumber} and \csa{etocpage}. It
creates a link (if such a link was added by |hyperref| to the
|.toc| file entry) whose destination is the corresponding document
unit and whose name is the given argument. 
Hence |\etoclink{\etocthename}| is like the original
|\etocname|.
Notice that if |hyperref| was not instructed to put a link in the page number
(via 
its option \emph{linktoc=all}) then \etoc's \csa{etocpage} contains no link
either, but |\etoclink{\etocthepage}| does. 


The command \csa{etoclink} is robust.

\subsection{The \csbhyp{etocthelinkedname}, \csbhyp{etocthelinkednumber},
\csbhyp{etocthelinkedpage} and \csbhyp{etocthelink} commands.}

This is for advanced uses by advanced users. 

The \csa{etocthename} macro has
been mentioned before; using it in instructions such as
|\global\let\lastone\etocthename| in an \etoc line style will
define the macro |\lastone| to expand to the last name seen at the
corresponding level. But no facilities was previously available to
do the same with the link data.

The package now provides \csa{etocthelinkedname} to do the similar
thing, with the link data included. It was used in this
documentation when doing the
\hyperref[tocastree]{examples with trees}.

Also provided with the similar goal: \csa{etocthelinkednumber},
\csa{etocthelinkedpage} (which contains a link only if |hyperref|
added one to the page number) and \csa{etocthelink}\marg{linkname}
which allows to make a link with an arbitrary name.

All these commands are fragile.


\section{The \csbhyp{etocsetlevel} command}

As already explained in the section \ref{sec:surprising}, one
can inform \etoc of a level to associate to a given sectioning
command with \csa{etocsetlevel}. For example:
\begin{verbatim}
\etocsetlevel{cell}{0}
\etocsetlevel{molecule}{1}
\etocsetlevel{atom}{2}
\etocsetlevel{nucleus}{3}
\end{verbatim}
Of course, in compatibility mode, it will be assumed here that the
macros |\l@cell|, |\l@molecule|, ..., pre-exist. If no table
of contents is typeset in compatibility mode, then all that
matters is that the line styles have been set. If for example
|section| is at level |1|, then there is no need to do a
\csa{etocsetstyle}|{molecule}| if \csa{etocsetstyle}|{section}|
has already been done (and it has been done by the package
itself in its definition of its own line styles).
 
The accepted levels run from |-2| to |6| inclusive. Anything else is
mapped to |6|, which is a dummy level, never displayed. The package
does:
\begin{verbatim}
\etocsetlevel{book}{-2}
\etocsetlevel{part}{-1}
\etocsetlevel{chapter}{0}
\etocsetlevel{section}{1}
\etocsetlevel{subsection}{2}
\etocsetlevel{subsubsection}{3}
\etocsetlevel{paragraph}{4}
\etocsetlevel{subparagraph}{5}
\end{verbatim}
\etoc own custom styles are activated by \csa{etocdefaultlines}. They
are illustrated by the main table of contents of this
document. 

These level assignments can be modified at anytime: see the
section \ref{sec:surprising} for various applications of this technique. As one
further example, let's mention here that the \hyperref[toc:main]{main table of
contents} of this document was typeset following these instructions:
\begin{verbatim}
\setcounter{tocdepth}{3}
\etocdefaultlines % use the package default line styles. At this early stage in
                  % the document they had not yet been modified by \etocsetstyle
                  % commands, so \etoctoclines could have been used, too.
\etocmarkboth\contentsname
\etocmulticolstyle[1]                   % one-column display
    {\noindent\bfseries\Large
     \leaders\hrule height1pt\hfill
     \MakeUppercase{Table of Contents}}
\begingroup                             % use a group to limit the scope of the
  \etocsetlevel{subsection}{3}          %   subsection level change.
  \etocsetlevel{subsubsection}{4}       % anything > tocdepth=3.
  \tableofcontents \label{toc:main}
\endgroup
\end{verbatim}
In this way, the subsections used the style originally designed for
subsubsections, the subsubsections were not printed. Without this modification,
the appearance would have been very different: the package line styles were
targeted to be employed in documents with many many sub-sub-sections, in a
two-column layout, giving thus a more compact output that what is achieved by
the default \LaTeX{} table of contents. But here, we have few sub-sub-sections
and it is more interesting to drop them and print in a visually different manner
sections and subsections. 


\section{Scope of commands added to 
         the \texorpdfstring{\texttt{.toc}}{.toc} file}

\begingroup % pour \small et \MacroFont en particulier.
\small

\subsection{Testing the scope}

Let us switch to the color red, and also add this command to the |.toc| file:

\color{red!50}
\addtocontents{toc}{\string\color{red!50}}

\def\MacroFont{\footnotesize\ttfamily}
\begin{verbatim}
   \color{red!50}                             % changing text color
   \addtocontents{toc}{\string\color{red!50}} % and also in the .toc file
\end{verbatim}


\subsection{This is a (pale) red subsection for illustrative purposes}

Actually, this title here was printed black, due to the way the |scrartcl| class
works (it would have been red in the |article| class), but we are more
interested in how it looks in the tables of contents: it does appear red in the
\hyperref[toc:main]{\color{niceone}{main table of contents}} at the beginning of
this document, and also in the \hyperref[toc:a]{\color{niceone}{table of
contents for this part}}. Both entries obey as expected the
|\color{red!50}| command inserted in the |.toc| file.

But let us now close this subsection and start a section.


\section{Am I also red?}

The question is about how it appears in the tables of contents: the answer is
that, yes it is red in the \hyperref[toc:main]{\color{niceone}main TOC}, and no
it is not red in the \hyperref[toc:a]{\color{niceone}local TOC for this part}.
The reason is that the \meta{finish} code for the subsection level closed a
group, as it used |\end{enumerate}|.

This illustrates the discussion from \autoref{ssec:limitations}. 

The default package line styles do not contain group opening and closing
instructions: the influence of a command added to the |.toc| file will propagate
until cancelled by another explicit such command inserted in the |.toc|
file.

\begin{verbatim}
    \normalcolor
    \addtocontents{toc}{\string\normalcolor}
\end{verbatim}
\endgroup

\normalcolor
\addtocontents{toc}{\string\normalcolor}

Back to black. Note that this scope problem arises in real life in a
multi-lingual document, as the |babel| package writes to the |.toc| file the
language changes occurring in the document. 


%\clearpage


\part{Package commands for toc display styles}
\label{part:globalcmds}


\thispartstats


\begin{verbatim}
\setcounter{tocdepth}{-3}
\localtableofcontents \label{toc:globalcmds}
\end{verbatim}

\setcounter{tocdepth}{-3}
\localtableofcontents \label{toc:globalcmds}

\section{Specifying the toc display style}


The \emph{toc display} style says whether the TOC appears with
multiple columns or just one, whether the title is typeset as
in the |article| or |book| class, or should be centered above
the entries, with rules on its sides, or if the entire TOC
should be put in a frame. For example, to opt for a ruled
heading and single column layout, one issues commands of the
following type:
\begin{verbatim}
\etocruledstyle[1]{Title}
    \tableofcontents (or \localtableofcontents)
shortcuts:
    \etocruled[1]{Title}  (or  \etoclocalruled[1]{Title})
\end{verbatim}

\subsection{The command \csbhyp{etocruledstyle}}

The general format of \csa{etocruledstyle} is:\\
\centerline{\color{blue}\csa{etocruledstyle}%
\oarg{number of columns}\marg{title of the toc}}
\noindent Note that the title is horizontal material, if it
does not fit on one line it should be put in a \csa{parbox} of
a given width. We did this and even enclosed the parboxes in
\csa{fbox}es to get frames around them. For the example with
the standard formatting we did not use an \csa{fbox} and got
rid of the horizontal rules via:
\begin{verbatim}
\renewcommand{\etoctoprule}{\hrule height 0pt}
\end{verbatim}
The green frame for the heading of the table of contents at
the \hyperref[toc:a]{%
  \color{niceone} start of the second part of this
  document} was obtained with:
\begin{verbatim}
\etocruledstyle[1]{\etocfontminusone\color{green}%
     \fboxrule1pt\fboxsep1ex
     \framebox[\linewidth]
              {\normalcolor\hss Contents of this second part\hss}}
\end{verbatim}


\subsection{The command \csbhyp{etocmulticolstyle}}

This is also a command with one optional and one mandatory argument:\\
\centerline{\color{blue}\csa{etocmulticolstyle}%
\oarg{number\_of\_columns}\marg{heading}}
The \meta{number\_of\_columns} can go from 1 to 10 (it
defaults to 2, and from 2 on is passed to a |multicols|
environment). The \meta{heading} should be some `vertical' material like:
\begin{center}
  \meta{heading} = {\ttfamily\string\section\lowast}\marg{title}
\end{center}
[\emph{New with 1.07}] An explicit |\par| not being accepted in the
\meta{heading} argument (this is actually a restriction
originating in the
|multicols| environment), an implicit one is automatically
added by \etoc at the end of the argument, as in this
example which shows how the main table of contents of this document was
configured:
\begin{verbatim}
\etocmulticolstyle{\noindent\bfseries\Large
                   \leaders\hrule height1pt\hfill
                   \MakeUppercase{Table of Contents}}
\end{verbatim}
After \csa{etocmulticolstyle} all future \csa{tableofcontents} will use the
specified style, if not changed in-between. A shortcut for
just one table of contents and 
not affecting the styles of later TOCs is:\\
\centerline{\csa{etocmulticol}\oarg{number\_of\_columns}\marg{heading}}
And there is also
\csa{etoclocalmulticol}\oarg{number\_of\_columns}\marg{heading}.


\subsubsection{The command \csbhyp{etoctocstyle}}

\centeredline{\color{blue}\csa{etoctocstyle}\oarg{kind}%
\marg{number\_of\_columns}\marg{title}}
\centeredline{=
\csa{etocmulticolstyle}|[|{\itshape number\_of\_columns}|]|%
|{\kind*{|\itshape title\upshape|}}|}
where |kind| is one of |chapter|, |section|, . . . and defaults
to |chapter| or |section| depending on the document class.

\paragraph{\texorpdfstring{\csb{etoc\-toc\-style\-with\-marks}}{etoctocstylewithmarks}}%
\leavevmode\unskip{\color{blue}\oarg{kind}\marg{number\_of\_columns}\marg{title}\marg{mark}}
\centeredline
{=\csa{etocmulticolstyle}|[|{\itshape number\_of\_columns}|]|%
|{\kind*{|\itshape title \ttfamily\upshape\string\markboth%
|{\MakeUppercase{|{\rmfamily\itshape mark}|}}}}|} 
where |kind| is one of |chapter|, |section|, ... The
actual display of the marks depends on the settings of the
page style. There is variant \csb{etoctocstylewithmarksnouc}
which does not uppercase.


\paragraph{Do we really want paragraph entries in the TOC?}

\paragraph{really?}

\subsection{The command  \csbhyp{etocruled}}

As a shortcut to set the style with \csb{etocruledstyle} and
then issue a
\csa{tableofcontents}, all inside a group
so that  future table of contents will not be
affected, there is:\\
\centerline{\csb{etocruled}\oarg{number\_of\_columns}\marg{title}}
And the local form will be \csb{etoclocalruled}.

\subsection{The commands \csbhyp{etocframedstyle} and \csbhyp{etocframed}}

Same mechanism:\\
\centerline{\color{blue}\csa{etocframedstyle}%
  \oarg{number\_of\_columns}\marg{title}}
and the accompanying shortcut:\\
\centerline{\csb{etocframed}\oarg{number\_of\_columns}\marg{title}}
Here the entire table of contents is framed, hence this can
only work if it fits on a page. Note that the title itself is
not framed, if one wants a frame one should set it up
inside the \meta{title} argument to \csa{etocframedstyle} or
\csa{etocframed}. There is also \csa{etoc\-localframedstyle} and
\csa{etoclocalframed}.

\subsection{Headings, titles, \csbhyp{etocinnertopsep}}

[\emph{modified in v1.07}] There is a slight difference
between \csa{etocmulticolstyle} and \csa{etocruledstyle} or
\csa{etocframedstyle}. For \csa{etocmulticolstyle} the mandatory
\meta{heading} argument can be something like:
\starit{section}%
|{|\emph{Table of Contents}|}|.  On the contrary  \csa{etocruledstyle} and
\csa{etocframedstyle} expect an argument
``in LR mode'' (to use the terminology from the \emph{LaTeX, a
document preparation system}). This means that multiline
contents arguments to \csa{etocruledstyle} or
\csa{etocframedstyle} must be enclosed in something like a \csa{parbox}.

[\emph{new in v1.07}] The command \csa{etocmulticolstyle} now
also accepts horizontal mode material in its mandatory
argument \meta{heading}: it internally automatically adds a
closing |\par|. So one can use for example
|\etocmulticolstyle{Hello World}|. Speaking of |\par|, there
is a |multicols| aspect which has nothing to do with \etoc,
the input on the left creates a compilation error:
\begingroup\renewcommand\columnseprule{.4pt}
\begin{multicols}{2}
\begin{verbatim}
\begin{multicols}{2}[hello\par world]
someone here?
\end{multicols}
\end{verbatim}
\columnbreak
\begin{verbatim}
\let\oldpar\par
\begin{multicols}{2}[hello\oldpar world]
  at least me.
\end{multicols}
\end{verbatim}
\end{multicols}
\endgroup
But the version on the right does not (it disguises |\par| so as to be
acceptable).  \etoc provides
\csb{etocoldpar} as a substitute for |\par| (it does
|\let\etocoldpar\par| just before the |multicols|
environment and automatically adds it to close the heading, 
before the vertical skip of value
\csa{etocinnertopsep}).\footnote{this command \csa{etocoldpar} (=
working \csa{par} in the argument to \csa{etocmulticolstyle}) is
not related to the switch \csa{etocinline} whose purpose is to
tell \etoc not to do a \csa{par} before the table of
contents.} The command \csb{etocoldpar} can also be used
explicitely in the mandatory argument to
\csa{etocmulticolstyle}.


An important dimension used by all three of \csa{etocmulticolstyle},
\csa{etocruledstyle} and \csa{etocframedstyle} is
\csb{etocinnertopsep}. It gives the amount of separation
between the heading and the start of the contents. Its default
value is |2ex| and it is changed by
|\renewcommand*{\etocinnertopsep}|\marg{new\_value}, not with
|\setlength|. 


\subsection{The command \csbhyp{etocsettocstyle}}

This is a command with two mandatory arguments:\\
\centerline{\color{blue}\csa{etocsettocstyle}\marg{before\_toc}%
\marg{after\_toc}}
The \marg{before\_toc} part is responsible for typesetting the
heading, for example it can be something like
\starit{section}|{\contentsname}|. 

Generally speaking this heading
should leave \TeX{} in vertical mode when the actual
typesetting of the contents will start: the line styles (either
from the standard classes or the package default line styles)
expect to start in `vertical mode'.

It can also contain
instructions to mark the page headings.  Or
it could check (book class) to see if two-column mode is on,
and switch to one-column style, and the \meta{after\_toc} part
would then reenact the two-column mode.

The previously described commands \csa{etocmulticolstyle},
\csa{etocruledstyle}, and \csa{etocframedstyle} actually call
\csa{etocsettocstyle} as a lower-level routine, and start a
|multicols| environment in \marg{before\_toc} to close it in
\marg{after\_toc}. 

\subsection{The compatibility mode \csbhyp{etocstandarddisplaystyle}}

\etoc will then emulate what the document class would have
done regarding the global display style of the table of
contents, in its absence. All customizing from inside the
class should be obeyed, too.

\section{Starred variants of the \csbhyp{tableofcontents}
  etc... commands}

The \toc, \localtoc, \csa{etocmulticol}, etc... have starred
variants (the star must be before the other arguments). For
all but the |memoir| class, they are like the original. For
the |memoir| class, the original prints an entry in the |.toc|
file, as is the usage for the original \toc command in that
class, whereas the starred variants do not, as is the habit in
that class.

As soon as one starts using local table of contents one
discovers that the default |memoir| thing which is to create a
|chapter| entry for each TOC is not convenient. The command
\csb{etocmemoirtoctotocfmt}\marg{kind}\marg{name} will change
the format (\meta{kind} is |chapter|, |section|, |subsection|... and
\meta{name} can be for example \csa{contentsname}.) The initial set-up
is with |chapter| and |\contentsname|. 

The format of the actual heading of the TOC should also be set
appropriately (for example with \csa{etoctocstyle}), to use the
identical division unit as in the first argument to
\csa{etocmemoirtoctotocfmt}. 

A weird situation arises when one has two successive
\csa{localtableofcontents} (obviously this is not a truly real
life situation), just after a \csa{part} for example. The first
one creates (if the default has not been modified as indicated
above) a Chapter heading which is
written to the |.toc|. Then the second one thinks to be local
to this chapter . . . and as a result it displays nothing. The
fix is to define the second one to be a clone of the first
one.

Independently of the situation with the |memoir| class there is
generally speaking a hook macro called
\csb{etocaftertitlehook} which is inhibited by using the
starred variants of the displaying commands. Except for the
|memoir| class, this hook is initially defined to do nothing.
There is also \csb{etocaftercontentshook},  similarly
defined to do nothing. They can be used for some special
effects.

\section{Table of contents for this part}


\subsection{Testing the compatibility mode}

As a third example we now print the local table of contents
for this part. First we will test the compatibility mode.\footnote{the
present document uses the |scrartcl| class, and we check here that
the \etoc compatibility mode does respect the customizing done via the
class commands.}   The original was invisibly defined with a label at the
beginning of this part \ref{part:globalcmds}.
\begin{verbatim}
\KOMAoptions{toc=left}
\etocstandarddisplaystyle % necessary for the display to obey toc=left
\etocstandardlines
\setcounter{tocdepth}{3}
\tableofcontents \ref{toc:globalcmds}
\end{verbatim}
\KOMAoptions{toc=left}
\etocstandarddisplaystyle
\etocstandardlines
\setcounter{tocdepth}{3}
\tableofcontents \ref{toc:globalcmds}

\subsection{A framed display}

We now opt for a ``framed'' style, using the package default
line styles and some colors added. 

% \clearpage

\begin{verbatim}
\etocdefaultlines
\begingroup % we use a group to limit the scope of the next commands
\renewcommand{\etoccolumnsep}{2em}
\renewcommand{\etocinnerleftsep}{1.5em}
\renewcommand{\etocinnerrightsep}{1.5em}
% specify a background color for the toc contents
\renewcommand{\etocbkgcolorcmd}{\color{yellow!10}}
%\renewcommand{\etocbkgcolorcmd}{\relax}
% set up the top and bottom rules 
\renewcommand{\etoctoprule}{\hrule height 1pt}
\renewcommand{\etoctoprulecolorcmd}{\color{red!25}}
\renewcommand{\etocbottomrule}{\hrule height 1pt}
\renewcommand{\etocbottomrulecolorcmd}{\color{red!25}}
% set up the left and right rules
\renewcommand{\etocleftrule}{\vrule width 5pt}
\renewcommand{\etocrightrule}{\vrule width 5pt}
\renewcommand{\etocleftrulecolorcmd}{\color{red!25}}
\renewcommand{\etocrightrulecolorcmd}{\color{red!25}}
% use \fcolorbox to set up a colored frame for the title
\fboxrule1pt
\etocframedstyle{\normalsize\rmfamily\itshape
  \fcolorbox{red}{white}{\parbox{.8\linewidth}{\centering
      This is a table of contents \`a la \etoc, but just for
      the sections and subsections in this part. As it is put
      in a frame, it has to be small enough to fit on the 
      current page. It has the label |toc:b|.}}}
% set up a label for future (or earlier...) reference
\setcounter{tocdepth}{3}
\tableofcontents \label{toc:b} \ref{toc:globalcmds}
\endgroup
\end{verbatim}


\etocdefaultlines
\begingroup 
\renewcommand{\etoccolumnsep}{2em}
\renewcommand{\etocinnerleftsep}{1.5em}
\renewcommand{\etocinnerrightsep}{1.5em}
% specify a background color for the toc contents
\renewcommand{\etocbkgcolorcmd}{\color{yellow!10}}
% set up the top and bottom rules 
\renewcommand{\etoctoprule}{\hrule height 1pt}
\renewcommand{\etoctoprulecolorcmd}{\color{red!25}}
\renewcommand{\etocbottomrule}{\hrule height 1pt}
\renewcommand{\etocbottomrulecolorcmd}{\color{red!25}}
% set up the left and right rules
\renewcommand{\etocleftrule}{\vrule width 5pt}
\renewcommand{\etocrightrule}{\vrule width 5pt}
\renewcommand{\etocleftrulecolorcmd}{\color{red!25}}
\renewcommand{\etocrightrulecolorcmd}{\color{red!25}}
% use \fcolorbox to set up a colored frame for the title
\fboxrule1pt
\etocframedstyle{\normalsize\rmfamily\itshape
  \fcolorbox{red}{white}{\parbox{.8\linewidth}{\centering
      This is a table of contents \`a la \etoc, but just for
      the sections and subsections in this part. As it is put
      in a frame, it has to be small enough to fit on the 
      current page. It has the label |toc:b|.}}}
% set up a label for future (or earlier...) reference
\tableofcontents \label{toc:b} \ref{toc:globalcmds}
\endgroup

\subsection{A (crazy) inline display}

Let us finally make some crazy inline display of the table of
contents of this entire document. We will typeset the subsections as
footnotes... This kind of style is suitable for a hyperlinked
document, not for print! 


%\clearpage

\begingroup
\setcounter{tocdepth}{3}
\newsavebox{\forsubsections}
\etocsetstyle{part}{\etocskipfirstprefix}{. \upshape}{\bfseries\etocname:~~}{}
\etocsetstyle{section}{\itshape\etocskipfirstprefix}{, }{\mdseries\etocname}{}
\etocsetstyle{subsection}
   {\begin{lrbox}{\forsubsections}\upshape\etocskipfirstprefix}
   {; }
   {\etocname}
   {\end{lrbox}\footnote{\unhbox\forsubsections.}}
\etocsetstyle{subsubsection}{ (\itshape\etocskipfirstprefix}
  {, }{\etocname}{\/\upshape)}
\etocsettocstyle{Here is the inline table of contents. }{.\par}
\tableofcontents
\endgroup

Here is the code which has been used:
\begin{verbatim}
\begingroup
\setcounter{tocdepth}{3}
\newsavebox{\forsubsections}
\etocsetstyle{part}{\etocskipfirstprefix}{. \upshape}{\bfseries\etocname:~~}{}
\etocsetstyle{section}{\itshape\etocskipfirstprefix}{, }{\mdseries\etocname}{}
\etocsetstyle{subsection}
   {\begin{lrbox}{\forsubsections}\upshape\etocskipfirstprefix}
   {; }
   {\etocname}
   {\end{lrbox}\footnote{\unhbox\forsubsections.}}
\etocsetstyle{subsubsection}{ (\itshape\etocskipfirstprefix}
  {, }{\etocname}{\/\upshape)}
\etocsettocstyle{Here is the inline table of contents. }{.\par}
\tableofcontents
\endgroup
\end{verbatim}

% \clearpage

\part{Using and customizing \etoc}
\label{part:custom}

\thispartstats

\etocdefaultlines
\renewcommand{\etoctoprule}{\hrule height2pt depth0pt}
\renewcommand{\etoctoprulecolorcmd}{\color{red}}

\etocruledstyle{\normalfont\normalsize\rmfamily\fboxrule1pt\color{red}%
  \fbox{\parbox{.8\linewidth}{\centering\normalcolor This is a table of
      contents for the sections and
      subsections in this part. It carries the label |toc:c|}}} 

\localtableofcontents \label{toc:c}


\section{Summary of the main style commands}

\subsection{Setting up local styles}

\hbox{\color{green}\fboxrule1pt\fboxsep1em
\setbox0\hbox{\csa{etocthename}, \csa{etocthenumber}, \csa{etocthepage}, \csa{etoclink}\marg{linkname}}%
\framebox[\linewidth][c]
{\vbox{\hsize\wd0\normalcolor\noindent
\csa{etocsetstyle}\marg{levelname}%
  \marg{start}\marg{prefix}\marg{contents}\marg{finish}\\
\csa{etocname}, \csa{etocnumber}, \csa{etocpage},
\csa{etocifnumbered}\marg{A}\marg{B}\\
\csa{etocthename}, \csa{etocthenumber}, \csa{etocthepage}, \csa{etoclink}\marg{linkname}
}}}

\subsection{Setting up toc display styles}

\medskip
\hbox{\color{green}\fboxrule1pt\fboxsep1em
\setbox0\hbox{\csa{etoctocstylewithmarksnouc}\oarg{kind}%
\marg{number\_of\_columns}\marg{title}\marg{mark}}%
\framebox[\linewidth][c]
{\vbox{\hsize\wd0
\normalcolor\noindent
\csa{etocmulticolstyle}\oarg{number\_of\_columns}\marg{heading}\\
\csa{etoctocstyle}\oarg{kind}\marg{number\_of\_columns}\marg{title}\\
\csa{etoctocstylewithmarks}\oarg{kind}\marg{number\_of\_columns}%
\marg{title}\marg{mark}\\
\csa{etoctocstylewithmarksnouc}\oarg{kind}\marg{number\_of\_columns}%
\marg{title}\marg{mark}\\
\csa{etocruledstyle}\oarg{number\_of\_columns}\marg{title}\\
\csa{etocframedstyle}\oarg{number\_of\_columns}\marg{title}\\
\csa{etocsettocstyle}\marg{before\_toc}\marg{after\_toc}}}}

\subsection{Displaying tables of contents}


\medskip \hbox{\color{green}\fboxrule1pt\fboxsep1em
  \setbox0\hbox{\csa{etocname}, \csa{etocnumber},
    \csa{etocpage}, \csa{etocifnumbered}\marg{A}\marg{B}}%
  \framebox[\linewidth][c]
  {\vbox{\hsize\wd0\normalcolor\noindent
      \toc\\
      \localtoc\\
      \csa{etocmulticol}\oarg{number\_of\_columns}\marg{heading}\\
      \csa{etoclocalmulticol}\oarg{number\_of\_columns}\marg{heading}\\
      \csa{etocruled}\oarg{number\_of\_columns}\marg{title}\\
      \csa{etoclocalruled}\oarg{number\_of\_columns}\marg{title}\\
      \csa{etocframed}\oarg{number\_of\_columns}\marg{title}\\
      \csa{etoclocalframed}\oarg{number\_of\_columns}\marg{title}\\
      \hbox{}{\itshape\ttfamily\ \ \ \  and their starred variants}
   }}}

\subsection{Labels and references}

The commands (starred or not) to actually display the table of contents
can be followed with optional labels or references:\par

\medskip \hbox{\color{green}\fboxrule1pt\fboxsep1em
  \setbox0\hbox{\csa{etocname}, \csa{etocnumber},
    \csa{etocpage}, \csa{etocifnumbered}\marg{A}\marg{B}}%
  \framebox[\linewidth][c]
  {\vbox{\hsize\wd0\normalcolor\noindent
      \toc \csa{label}|\{toc:here\}|\\ 
      \toc \csa{ref}|\{toc:far\}| \\
      \toc \csa{label}|\{toc:here\}| \csa{ref}|\{toc:far\}| \\
      \localtoc \csa{label}|\{toc:here\}|\\
      \hbox{}{\itshape\ttfamily\ \ \ \ similarly with\ }%
      \csa{etocmulticol}{\itshape\ttfamily\ etc . . . }
   }}}


\medskip The commands for local tables of
contents do \emph{not} react to a \csa{ref} following them. 

When
re-displaying another toc, only its contents are transferred:
both the line styles and the toc display style are the ones
currently defined, not the ones from the cloned toc.


\subsection{The package default line styles: \csbhyp{etocdefaultlines}}


Activating the use of the package default line styles\footnote{they were written
  at a very early stage in the development of the package, and version |1.07e|
  has, among other things, modified the previous unsatisfactory use of penalties
  and vertical spacing commands. This will cause differences to
  documents having been compiled with earlier versions, apologies for that.} is
done via \csa{etocdefaultlines}, or \csa{etoctoclines} if these styles have not
been modified with \csa{etocsetstyle}. Sections and sub-sections are printed in
essentially the same manner, except that the leading for sub-sections is a bit
smaller (with document classes lacking a \csa{chapter} command, the sections are
printed in bold typeface; this is the case in the present document).
Sub-sub-sections are printed inline, in one paragraph, with no numbers or page
numbers. This style was designed and tested with documents having lots of
sub-sub-sections, and should be used on a two-column layout: it provides (only
in that situation with many sub-sub-sections) a more compact presentation than
what is achieved by the \LaTeX{} default.\footnote{and there will never be a
  Part or Chapter entry alone at the bottom of a column or page (except if it
  has no sub-unit).} On the other hand, used with a one-column layout, and with
few sub-sub-sections, the style is a bit more spread out vertically than the
\LaTeX{} default, sub-sections are not visually much different from sections
(especially for document classes with a \csa{chapter} command), so the result is
less hierarchical in appearance than in the \LaTeX{} default.

In this document, for the \hyperref[toc:main]{main table of contents}, we did 
|\etocsetlevel{subsection}{3}| hence the sub-sections were printed with the
sub-sub-section inline style. 

Let us, to the contrary, typeset now this main table of contents as if the
document had been done with a class having the \csa{chapter} command: we will
print sections as chapters, and subsections as sections. We use
\csa{etocsetlevel} for that, and also we need to change the font style of
``sections'' (which in truth are our subsections) to use not the bold but the
medium series; we modify the \csa{etocfontone} command for that.


\begin{verbatim}
\etocruledstyle[2]{\normalfont\normalsize\rmfamily\itshape
  \fbox{\parbox{.6\linewidth}{
      \leftskip 0pt plus .5fil
      \rightskip 0pt plus -.5fil
      \parfillskip 0pt plus 1fil This is the global table of
      contents on two columns, using \etoc default line styles, but with
      sections 
      being printed as chapters, and subsections as sections.
      }}}
\etocdefaultlines
\setcounter{tocdepth}{1}
\begingroup
\etocsetlevel{section}{0}
\etocsetlevel{subsection}{1}
\renewcommand*{\etocfontone}{\normalfont \normalsize}
\tableofcontents
\endgroup
\end{verbatim}

\etocruledstyle[2]{\normalfont\normalsize\rmfamily\itshape
  \fbox{\parbox{.6\linewidth}{
      \leftskip 0pt plus .5fil
      \rightskip 0pt plus -.5fil
      \parfillskip 0pt plus 1fil This is the global table of
      contents on two columns, using \etoc default line styles, but with
      sections 
      being printed as chapters, and subsections as sections.
      }}}

\etocdefaultlines
\setcounter{tocdepth}{1}

\begingroup
\etocsetlevel{section}{0}
\etocsetlevel{subsection}{1}
\renewcommand*{\etocfontone}{\normalfont \normalsize}

\tableofcontents
\endgroup

\subsection{One more example TOC layout}


I got motivated by a
question\footnote{\href{http://tex.stackexchange.com/questions/83184/how-to-change-style-and-color-of-table-of-content}{tex.stackexchange.com/questions/83184}}
I saw on the \TeX{} StackExchange site. I copied the color RGB
specifications from an answer which had been provided to the
question. The \csa{etocframedstyle} puts the title on the top
rule in a centered position. This is not very convenient for
this example so we included the title as part of the
\meta{start} code at section level, to get it \emph{inside}
the frame.

\begingroup\def\MacroFont{\footnotesize\ttfamily}%

\begin{verbatim}
\setcounter{tocdepth}{3}
\begingroup
\definecolor{subsecnum}{RGB}{13,151,225}
\definecolor{secbackground}{RGB}{0,177,235}
\definecolor{tocbackground}{RGB}{212,237,252}

\renewcommand{\etocbkgcolorcmd}{\color{tocbackground}}
\renewcommand{\etocleftrulecolorcmd}{\color{tocbackground}}
\renewcommand{\etocrightrulecolorcmd}{\color{tocbackground}}
\renewcommand{\etocbottomrulecolorcmd}{\color{tocbackground}}
\renewcommand{\etoctoprulecolorcmd}{\color{tocbackground}}

\renewcommand{\etocleftrule}{\vrule width 1cm}
\renewcommand{\etocrightrule}{\vrule width .5cm}
\renewcommand{\etocbottomrule}{\hrule height 12pt}
\renewcommand{\etoctoprule}{\hrule height 12pt}

\renewcommand{\etocinnertopsep}{0pt}
\renewcommand{\etocinnerbottomsep}{0pt}
\renewcommand{\etocinnerleftsep}{0pt}
\renewcommand{\etocinnerrightsep}{0pt}

\newcommand\shiftedwhiterule[2]{%
    \hbox to \linewidth{\color{white}%
    \hskip#1\leaders\vrule height1pt\hfil}\nointerlineskip\vskip#2}

\etocsetstyle{subsubsection}{\etocskipfirstprefix}
  {\shiftedwhiterule{\leftskip}{6pt}}
  {\sffamily\footnotesize
     \leftskip2.5cm\hangindent1cm\rightskip1cm\noindent
     \hbox to 1cm{\color{subsecnum}\etocnumber\hss}%
     \color{black}\etocname\leaders\hbox to .2cm{\hss.}\hfill
     \rlap{\hbox to 1cm{\hss\etocpage\hskip.2cm}}\par
     \nointerlineskip\vskip3pt}
  {}

\etocsetstyle{subsection}{\etocskipfirstprefix}
  {\shiftedwhiterule{1.5cm}{6pt}}
  {\sffamily\small
     \leftskip1.5cm\hangindent1cm\rightskip1cm\noindent
     \hbox to 1cm{\color{subsecnum}\etocnumber\hss}%
     \color{black}\etocname\leaders\hbox to .2cm{\hss.}\hfill
     \rlap{\hbox to 1cm{\hss\etocpage\hskip.2cm}}\par
     \nointerlineskip\vskip6pt}
  {}

\newcommand{\coloredstuff}[2]{%
            \leftskip0pt\rightskip0pt\parskip0pt
            \fboxsep0pt % \colorbox uses \fboxsep also when no frame!
       \noindent\colorbox{secbackground}
               {\parbox{\linewidth}{%
                    \vskip5pt
                    {\noindent\color{#1}#2\par\nointerlineskip}
                    \vskip3pt}}%
       \par\nointerlineskip}

\etocsetstyle{section}
    {\coloredstuff{white}
     {\hfil \hyperref[toc:b]{\bfseries\large I am a twin of  
     that other TOC (click me!)}\hfil}}
    {\vskip3pt\sffamily\small}
    {\coloredstuff{white}{\hbox to 1.5cm{\hss\etocnumber\hskip.2cm}%
     \etocname\hfill\hbox{\etocpage\hskip.2cm}}\vskip6pt}
    {}

\etocframedstyle[1]{}
\tableofcontents \label{toc:clone} \ref{toc:globalcmds}
\endgroup
\end{verbatim}
\endgroup


The coding is a bit involved\footnote{and reveals the author's
preference for the \TeX{} syntax...} as it does not use any additional
package. Also, it was written at some early stage and I have not revised it
since. 

A better solution would be to use some package to set
up a background color possibly extending accross pages, as the
framed style (which we used to get this background
color) can only deal with material short enough to fit on one
page.

Regarding colors, generally speaking all color commands inside
\etoc are initially defined to do nothing, and the choice to
use or not colors is left to the user.

\clearpage
\setcounter{tocdepth}{3}

\begingroup
\definecolor{subsecnum}{RGB}{13,151,225}
\definecolor{secbackground}{RGB}{0,177,235}
\definecolor{tocbackground}{RGB}{212,237,252}

\renewcommand{\etocbkgcolorcmd}{\color{tocbackground}}
\renewcommand{\etocleftrulecolorcmd}{\color{tocbackground}}
\renewcommand{\etocrightrulecolorcmd}{\color{tocbackground}}
\renewcommand{\etocbottomrulecolorcmd}{\color{tocbackground}}
\renewcommand{\etoctoprulecolorcmd}{\color{tocbackground}}

\renewcommand{\etocleftrule}{\vrule width 1cm}
\renewcommand{\etocrightrule}{\vrule width .5cm}
\renewcommand{\etocbottomrule}{\hrule height 12pt}
\renewcommand{\etoctoprule}{\hrule height 12pt}

\renewcommand{\etocinnertopsep}{0pt}
\renewcommand{\etocinnerbottomsep}{0pt}
\renewcommand{\etocinnerleftsep}{0pt}
\renewcommand{\etocinnerrightsep}{0pt}

\newcommand\shiftedwhiterule[2]{%
    \hbox to \linewidth{\color{white}%
    \hskip#1\leaders\vrule height1pt\hfil}\nointerlineskip
\vskip#2}

\etocsetstyle{subsubsection}{\etocskipfirstprefix}
{\shiftedwhiterule{\leftskip}{6pt}}
{\sffamily\footnotesize
\leftskip2.5cm\hangindent1cm\rightskip1cm\noindent
\hbox to 1cm{\color{subsecnum}\etocnumber\hss}%
\color{black}\etocname\leaders\hbox to .2cm{\hss.}\hfill
\rlap{\hbox to 1cm{\hss\etocpage\hskip.2cm}}\par
\nointerlineskip\vskip3pt}
{}

\etocsetstyle{subsection}{\etocskipfirstprefix}
{\shiftedwhiterule{1.5cm}{6pt}}
{\sffamily\small
\leftskip1.5cm\hangindent1cm\rightskip1cm\noindent
\hbox to 1cm{\color{subsecnum}\etocnumber\hss}%
\color{black}\etocname\leaders\hbox to .2cm{\hss.}\hfill
\rlap{\hbox to 1cm{\hss\etocpage\hskip.2cm}}\par
\nointerlineskip\vskip6pt}
{}

\newcommand{\coloredstuff}[2]{%
            \leftskip0pt\rightskip0pt\parskip0pt
            \fboxsep0pt % \colorbox uses \fboxsep also when no frame!
       \noindent\colorbox{secbackground}
               {\parbox{\linewidth}{%
                    \vskip5pt
                    {\noindent\color{#1}#2\par\nointerlineskip}
                    \vskip3pt}}%
       \par\nointerlineskip}

\etocsetstyle{section}{\coloredstuff{white}
     {\hfil \hyperref[toc:b]{\bfseries\large I am a twin of  
     that other TOC (click me!)}\hfil}}
{\vskip3pt\sffamily\small}
{\coloredstuff{white}{\hbox to 1.5cm{\hss\etocnumber\hskip.2cm}%
    \etocname\hfill\hbox{\etocpage\hskip.2cm}}\vskip6pt}
{}

\etocframedstyle[1]{}

\tableofcontents \label{toc:clone} \ref{toc:globalcmds}

\endgroup


\section{Customizing \etoc}



\subsection{Customizing  the \etoc pre-defined line styles}


We will simply list the relevant commands as defined in the
package. Customizing them goes through suitable
\csa{renewcommand}s:
\begin{verbatim}
\newcommand*\etocfontminustwo{\normalfont \LARGE \bfseries}
\newcommand*\etocfontminusone{\normalfont \large \bfseries}
\newcommand*\etocfontzero{\normalfont \large \bfseries}
\newcommand*\etocfontone{\normalfont \normalsize \bfseries}
\newcommand*\etocfonttwo{\normalfont \normalsize}
\newcommand*\etocfontthree{\normalfont \footnotesize}

\newcommand*\etocsepminustwo{4ex plus .5ex minus .5ex}
\newcommand*\etocsepminusone{4ex plus .5ex minus .5ex}
\newcommand*\etocsepzero{2.5ex plus .4ex minus .4ex}
\newcommand*\etocsepone{1.5ex plus .3ex minus .3ex}
%%\newcommand*\etocseptwo{1ex plus .15ex minus .15ex} % modified in 1.07e
\newcommand*\etocseptwo{.5ex plus .1ex minus .1ex}
\newcommand*\etocsepthree{.25ex plus .05ex minus .05ex}

\newcommand*\etocminustwoleftmargin{1.5em plus 0.5fil}
\newcommand*\etocminustworightmargin{1.5em plus -0.5fil}
\newcommand*\etocminusoneleftmargin{1em}
\newcommand*\etocminusonerightmargin{1em}

\newcommand*\etocbaselinespreadminustwo{1}
\newcommand*\etocbaselinespreadminusone{1}
\newcommand*\etocbaselinespreadzero{1}
\newcommand*\etocbaselinespreadone{1}
\newcommand*\etocbaselinespreadtwo{1}
\newcommand*\etocbaselinespreadthree{.9}
\newcommand*\etoctoclineleaders
    {\hbox{\normalfont\normalsize\hbox to 2ex {\hss.\hss}}}
\newcommand*\etocabbrevpagename{p.~}
\newcommand*\etocpartname{\partname}  % utilisateurs de frenchb: attention
                                      % car donne "partie" sans majuscule.
\newcommand*\etocbookname{Book} % to be modified according to language
\end{verbatim}

No customizing of the standard line styles is possible from
within \etoc. As already explained, when
\csa{etocstandardlines} has been issued, the package just makes
itself very discrete and acts only at the global level, and
the TOC entries are (hopefully) formatted as would have
happened in the absence of \etoc.\footnote{with the
  KOMA-script classes, we noticed that
  \csa{etocstandarddisplaystyle} was apparently needed for the
  KOMA options |toc=left| to be active at the level of the line entries.}  


The \csa{etocstandardlines} compatibility mode will work also with
sectioning commands made known to \etoc via \csa{etocsetlevel},
under the condition of course that these sectioning commands
are accompanied with all the relevant definitions for
typesetting toc entries in the \LaTeX{} default manner
(existence of the macros \csa{l@something} . . .).

Using the command \csa{etocsetstyle}, be it in the preamble or
in the body of the document, has the secondary effect of
switching off the compatibility mode.

\subsection{Customizing the toc display styles}


Again we list the relevant macros, what they do should be
legible from their names. Note that \csa{renewcommand}'s and
not \csa{setlength}'s have to be used for what appear to be
lengths, and that color commands are not just color
specifications, they must include \csa{color}, and are canceled
by re-defining them to do \csa{relax}.
\begin{verbatim}
\newcommand*\etocabovetocskip{3.5ex plus 1ex minus .2ex} 
\newcommand*\etocbelowtocskip{3.5ex plus 1ex minus .2ex}

\newcommand*\etoccolumnsep{2em}
\newcommand*\etocmulticolsep{0ex}
\newcommand*\etocmulticolpretolerance{-1}
\newcommand*\etocmulticoltolerance{200}
\newcommand*\etocdefaultnbcol{2}
\newcommand*\etocinnertopsep{2ex}
\newcommand*\etoctoprule{\hrule}
\newcommand*\etoctoprulecolorcmd{\relax}

% for the framed style only:
\newcommand*\etocinnerleftsep{2em}
\newcommand*\etocinnerrightsep{2em}
\newcommand*\etocinnerbottomsep{3.5ex}

\newcommand*\etocleftrule{\vrule}
\newcommand*\etocrightrule{\vrule}
\newcommand*\etocbottomrule{\hrule}
\newcommand*\etocleftrulecolorcmd{\relax}
\newcommand*\etocrightrulecolorcmd{\relax}
\newcommand*\etocbottomrulecolorcmd{\relax}

\newcommand*\etocbkgcolorcmd{\relax}

% hooks
\newcommand\etocframedmphook{\relax}
\end{verbatim}

The \csa{etocframedmphook} is positioned immediately
after the beginning of a minipage environment where the
contents of the framed TOC are typeset. 

The \csa{...colorcmd} things are initially set to be
\csa{relax}, so there is no need to do \csa{usepackage}|{color}|
if the document does not use colors. If the scope of a change
to a color command such as \csa{etocbkgcolorcmd} has not
been limited to a group and one then wishes to let it again be
\csa{relax} one must use a \csa{renewcommand} and not
\csa{let}\csa{etocbkgcolorcmd}\csa{relax}.

Regarding the dimensions of the top rule they can be specified
in |ex|'s or |em|'s as in this example:
\centeredline{|\renewcommand{\etoctoprule}{\hrule height 1ex}|} 
The package code is done in such a manner
that it is the font size in instance at the end of typesetting
the title argument to \csa{etocruledtoc} or
\csa{etocframedtoc} which will be used for the meaning of the
`1ex'. Of course also the other rule commands can have their
dimensions in font relative units, but their values are
decided on the basis of the font in effect just before the
table of contents.

The top and bottom rules do not have to be rules and can be
horizontal \emph{leaders} (of a specified height) in the general
\TeX{} sense. However the left and right rules are not
used as (horizontal) leaders but as objects of a given specified
width. Note that \emph{only} the Plain \TeX{} syntax for rules is
accepted here. 


% \clearpage
\part{Tips}

\thispartstats

\section{... and tricks}

\subsection{Hacking  framed parboxes}


\begin{verbatim}
\renewcommand\etoctoprule{\hrule height 2pt depth 2pt}
\etocruled{\color{green}\fboxrule2pt\fboxsep1ex
               \fbox{\raisebox{-\fontdimen22\textfont2}
                    {\color{blue}\parbox{.5\linewidth}
                       {\normalfont This text is perfectly centered
                        vertically with respect to the
                        surrounding horizontal rules.}}}}
\ref{toc:globalcmds}
\end{verbatim}

\renewcommand\etoctoprule{\hrule height 2pt depth 2pt}

\etocruled{\color{green}\fboxrule2pt\fboxsep1ex
              \fbox{\raisebox{-\fontdimen22\textfont2}
                    {\color{blue}\parbox{.5\linewidth}
                       {\normalfont This text is perfectly centered
                        vertically with respect to the
                        surrounding horizontal rules.}}}}
              \ref{toc:globalcmds}


\subsection{Interverting the levels}
\label{subsec:interverting}

Let us display and count all subsections occurring in this document (see
section \ref{sec:surprising} for other uses of this technique): 
\begin{verbatim}
\setcounter{tocdepth}{2}
\begingroup
\etocsetlevel{part}{3}
\etocsetlevel{section}{3}
\etocsetstyle{subsection}
    {\small\begin{enumerate}[itemsep=0pt,label=,leftmargin=0pt]}
    {\normalfont\bfseries\item}
    {\roman{enumi}. \mdseries\etocname{} (\etocnumber, p.~\etocpage)}
    {\end{enumerate}}
\renewcommand{\etoccolumnsep}{2.75em}
\renewcommand{\columnseprule}{1pt}
\etocmarkbothnouc{List of all subsections}
\etocmulticol[3]{\subsection{All subsections of this document}}
\endgroup
\end{verbatim}

\setcounter{tocdepth}{2}

\begingroup
\etocdefaultlines
\etocstandarddisplaystyle
\etocsetlevel{book}{3}
\etocsetlevel{part}{3}
\etocsetlevel{chapter}{3}
\etocsetlevel{section}{3}
\etocsetstyle{subsection}{\small
\begin{enumerate}[itemsep=0pt,label=,leftmargin=0pt]}
{\normalfont\bfseries\item}
{\roman{enumi}. \mdseries\etocname{} (\etocnumber, p.~\etocpage)}
{\end{enumerate}}

\renewcommand{\etoccolumnsep}{2.75em}
\renewcommand{\columnseprule}{1pt}

\etocmarkbothnouc{List of all subsections}

\etocmulticol[3]{\subsection{All subsections of this document}}

\endgroup

\setcounter{tocdepth}{3}


\subsection{Displaying statistics}\label{ssec:statistics}

Each part of this document starts with a paragraph telling how many sections and
subsections it has. Well, each one of this paragraph is a table of contents! We
designed a macro \csa{thispartstats} to do that. It uses ``storage'' boxes to
keep the information about the first and last section or subsection. Using boxes
is the simplest manner to encapsulate the |hyperref| link for later use (whether
there is one or none). However, one cannot modify then the font or the color
(using the \TeX{} primitive \csa{setbox} rather than the \LaTeX{} \csa{sbox}
would allow to change the color of the un-boxed saved box). If such a need
arises, one must switch from boxes to macros, and store the |hyperref| data for
later use as was done in the code presented in the \autoref{ssec:molecule}. We
did this for the first paragraph of the \autoref{sec:surprising}.

But first, the coding of \csa{thispartstats}:\par
\begingroup
\def\MacroFont{\footnotesize\ttfamily\relax}
\begin{verbatim}
\newsavebox\firstnamei  \newsavebox\firstnumberi
\newsavebox\lastnamei   \newsavebox\lastnumberi
\newsavebox\firstnameii \newsavebox\firstnumberii
\newsavebox\lastnameii  \newsavebox\lastnumberii
\newcounter{mycounti}   \newcounter{mycountii}
\newcommand*{\thispartstatsauxi}{} \newcommand*{\thispartstatsauxii}{}
\newcommand*{\oldtocdepth}{}
\newcommand*{\thispartstats}{%
  \edef\oldtocdepth{\arabic{tocdepth}}%
  \setcounter{tocdepth}{2}%
  \setcounter{mycounti}{0}%
  \setcounter{mycountii}{0}%
  \def\thispartstatsauxi{%
         \sbox{\firstnamei}{\color{cyan}\etocname}%
         \sbox{\firstnumberi}{\color{cyan}\etocnumber}%
         \def\thispartstatsauxi{}}%
  \def\thispartstatsauxii{%
         \sbox{\firstnameii}{\color{cyan}\etocname}%
         \sbox{\firstnumberii}{\color{cyan}\etocnumber}%
         \def\thispartstatsauxii{}}%
  \begingroup
  \etocsetstyle{subsection} {} {}
    {\thispartstatsauxii
     \stepcounter{mycountii}%
     \sbox{\lastnameii}{\color{teal}\etocname}%
     \sbox{\lastnumberii}{\color{teal}\etocnumber}} {}%
  \etocsetstyle{section} {} {}
    {\thispartstatsauxi
     \stepcounter{mycounti}%
     \sbox{\lastnamei}{\color{teal}\etocname}%
     \sbox{\lastnumberi}{\color{teal}\etocnumber}}
    {Here are some statistics for this part: it contains \arabic{mycounti}
    section\ifnum\value{mycounti}>1 s\fi{} and \arabic{mycountii}
    subsection\ifnum\value{mycountii}>1 s\fi. The name of the first section is
    \unhbox\firstnamei{} and the corresponding number is \unhbox\firstnumberi. 
    The name of the last section is \unhbox\lastnamei{} and its number is
    \unhbox\lastnumberi. The name of the first subsection is \unhbox\firstnameii{}
    and the corresponding number is \unhbox\firstnumberii. The name of the last
    subsection is \unhbox\lastnameii{} and its number is \unhbox\lastnumberii.}%
  \etocinline % don't do a \par automatically (but this is not used, actually).
  \etocsettocstyle {}{}
  \localtableofcontents  % to be used at the top level of a Part.
  \endgroup
  \setcounter{tocdepth}{\oldtocdepth}%
}
\end{verbatim}
%\endgroup

And now, the variant  which was used for
  \autoref{sec:surprising}, with macros rather than boxes: 
\begin{verbatim}
\makeatletter
\newcommand*\firstsubname   {}  \newcommand*\lastsubname    {}
\newcommand*\firstsubnumber {}  \newcommand*\lastsubnumber  {}
\newcommand*\thissectionstatsaux{}

\newcommand*{\thissectionstats}{%
  \edef\oldtocdepth{\arabic{tocdepth}}%
  \setcounter{tocdepth}{2}%
  \setcounter{mycounti}{0}%
  \def\thissectionstatsaux{% ou plus simple si on ne veut pas le lien.
         \let\firstsubname\etocthelinkedname
         \let\firstsubnumber\etocthelinkednumber
         \def\thissectionstatsaux{}}
  \begingroup
  \etocsetstyle{subsection} {} {}
    {\thissectionstatsaux
     \stepcounter{mycounti}%
     \let\lastsubname\etocthelinkedname
     \let\lastsubnumber\etocthelinkednumber }
    {Here are some statistics for this section. It contains \arabic{mycounti}
     subsections. The name of its first is \emph{\color{cyan}\firstsubname{}}
     and the corresponding number is {\color{cyan}\firstsubnumber}. The name of
     the last subsection is \emph{\color{teal}\lastsubname{}} and its number is 
     {\color{teal}\lastsubnumber}.}%
   \etocsettocstyle {}{}
   \etocinline
   \localtableofcontents
  \endgroup
  \setcounter{tocdepth}{\oldtocdepth}%
}
\makeatother
\end{verbatim}
\endgroup

\subsection{Compatibility with other packages}

\etoc loads the package |multicol|.\footnote{\protect\fbox{up to version
\texttt{1.07a} it also used package \texttt{xspace}, but this
has been removed from later versions.}} It is
|hyperref| aware and hopefully |hyperref| compatible! It
doesn't matter whether \etoc or |hyperref| is loaded first.

The contents of the |.toc| file (if it exists) are read into
memory by \etoc once, at the time of \csa{usepackage\{etoc\}}.
The |.toc| file will be opened for write operations only at
the time of the first TOC displaying command. 

\etoc can not really cohabit with packages modifying the \csa{tableofcontents}
  command: some sort of truce can be achieved if \etoc is loaded last, hence is
  the winner. 

When a \localtoc is inserted by the user in the document, a
line containing an \etoc inner command and an identification
number is added to the |.toc| file. The correct local table of
contents will be displayed only on the next |latex| run.

\etoc expects the document sectioning units to write their
data into the file having extension |.toc|, in the form of
lines containing the \csa{contentsline} command and its
arguments. The macros \csa{etocname}, \csa{etocnumber}, and
\csa{etocpage} contain the |hyperref| links, if present (note
that the \emph{linktoc=all} option of |hyperref| tells it to
put a link also in the page number corresponding to a given
toc entry). For example, the tables of contents of the present document are
all fully linked.\footnote{except the \hyperref[tocastree]{Qtree TOC} ...}

It is possible to customize (using package |tocloft| for
example) throughout the document the macros \csa{l@section},
\csa{l@subsection} ... and the effect will be seen in the next
table of contents typeset by \etoc in compatibility mode.

It is possible to use simultaneously \etoc and 
{\color{niceone}|tableof|}\footnote{\url{http://www.ctan.org/pkg/tableof}}. For the
advanced uses such as what is done in the \autoref{ssec:molecule} it is
  important 
to know that |tableof| adds one level of grouping inside the |.toc| file itself.
So when one needs to make some information \csa{global}, one can not wait to be
at the level of the second argument of \csa{etocsettocstyle}, as
|tableof| will already have closed the group then. The \csa{global} things must
be done at the latest in the \meta{finish} part of the top (or last) sectioning
  level used. 
This only applies of course to \csa{tableofcontents} or
  \csa{localtableofcontents} following the 
  \csa{nexttocwithtags}|{required}||{forbidden}| command from |tableof|.

And when the commands
  \csa{tableof} or \csa{tablenotof} of package 
  |tableof| are used,
  they typeset the table of contents according to the document class defaults:
  to benefit from the \etoc styles, it is mandatory to use either
  \csa{tableofcontents}, or \csa{localtableofcontents} or one of the other \etoc
  commands, and |tableof (v1.1)| will influence the outcome only if 
  \csa{nexttocwithtags}|{required}||{forbidden}| was added before the table of
  contents typesetting command.

\subsection{\TeX nical matters}

The \csa{etocname}, \csa{etocnumber}, \csa{etocpage} commands are protected against
premature expansion. They contain suitable |hyperref| links if package
|hyperref| is loaded and active for the TOC. The commands \csa{etoclink} and
\csa{etocifnumbered} are also protected against premature expansion.

On the other hand \csa{etocthename}, \csa{etocthenumber},
\csa{etocthepage} do not represent |hyperref| links, and are
\emph{not} protected against expansion.

The commands such as \csa{etocsetstyle}, \csa{etocsetlevel},
\csa{etocsettocstyle}, \csa{etocmulticolstyle},
\csa{etocruledstyle}, \csa{etocframedstyle} obey \LaTeX{}'s
groups. All TOCs are typeset inside groups.

\subsection{Errors and catastrophes}

  After using \csa{etocsetstyle} for one level, the remaining
  uncustomized levels use the \etoc default styles (those
  which are activated by \csa{etocdefaultlines}). One has to
  make sure that all levels needed for the next table of
  contents are mutually compatible: in particular the \etoc
  default styles expect to start in ``vertical mode''.

  When using multiple \toc commands in a document, one should beware from adding
  typesetting instructions directly in the |.toc| file, as they will be executed
  by \etoc for all TOCs: even for a \localtoc it doesn't matter if that
  instruction seems to concern material outside of its scope, it will get
  executed nevertheless. If absolutely necessary (but this should never be)
  these instructions should be done in such a way that they can be activated or
  deactivated easily from the document source, as need be.

  As is usual with toc and labels, after each change, one has
  to run latex a certain number of times to let the produced
  document get its final appearance (at least twice).\par

\bigskip

This is the documentation as of \docdate, printed from the source file with the
time stamp ``\dtxtimestamp''. The package version is \pkgversion, of
  \pkgdate. See the source for copyright and license information.\par 


\part{The code}

This source file |etoc.dtx| produces 
|etoc.sty| when one does |latex etoc.dtx| or |pdflatex etoc.dtx| 
(an |etoc.ins| file is also produced, for
distributions expecting it for installation). Two more runs
are necessary to finish producing the documentation. The
|etoc.sty| file should be moved to a suitable location within
the \TeX{} installation.

\section{Implementation}

\makeatletter
\StopEventually{\check@checksum\end{document}}
\let\check@percent\original@check@percent
\makeatother

Writing-up source code comments is hopefully for a future
release. 

% I don't want to have to type at this location (far from the top of the
% file) explicitly the package version or version date, as it is
% inconvenient to have to remember to do this when updating the package.
% Also, I prefer not to add macros to the |.sty| file giveing the package
% date, name, or version. So I cut out the following from the real macrocode
% environment (leaving out the \init@crossref.)
\makeatletter
\begingroup
\topsep\MacrocodeTopsep
\trivlist\parskip\z@\item[]
\macro@font
\leftskip\@totalleftmargin  \advance\leftskip\MacroIndent
\rightskip\z@  \parindent\z@  \parfillskip\@flushglue
\global\@newlistfalse \global\@minipagefalse
\ifcodeline@index
  \everypar{\global\advance\c@CodelineNo\@ne
  \llap{\theCodelineNo\ \hskip\@totalleftmargin}}%
\fi
\string\ProvidesPackage\string{\pkgname\string}\par
\noindent\space [\pkgdate\space\pkgversion\space\pkgdescription]\par
\nointerlineskip
\global\@inlabelfalse
\endtrivlist
\endgroup
\makeatother

% The catcode hackery next is to avoid to have <*package> to be listed
% in the commented source code...
% (c) 2012/11/19 jf burnol ;-)


\MakePercentIgnore

%
% \catcode`\<=0 \catcode`\>=11 \catcode`\*=11 \catcode`\/=11
% \let</none>\relax
% \def<*package>{\catcode`\<=12 \catcode`\>=12 \catcode`\*=12 \catcode`\/=12}
%
%</none>
%<*package>
%    \begin{macrocode}
\NeedsTeXFormat{LaTeX2e}
\RequirePackage{multicol}
%% \RequirePackage{xspace} %% REMOVED (1.07b)
\DeclareOption*{\PackageWarning{etoc}{Option `\CurrentOption' is unknown.}}
\ProcessOptions\relax
%    \end{macrocode}
% placeholder for comments
%    \begin{macrocode}
\newtoks\Etoc@toctoks
\def\Etoc@par{\par}
\newcommand*{\etocinline}{\def\Etoc@par{}}
\let\etocnopar\etocinline
\newif\ifEtoc@jj  % book
\newif\ifEtoc@j   % part  
\newif\ifEtoc@    % chapter
\newif\ifEtoc@i   % section
\newif\ifEtoc@ii  % subsection
\newif\ifEtoc@iii % subsubsection
\newif\ifEtoc@iv  % paragraph
\newif\ifEtoc@v   % subparagraph
\newif\ifEtoc@number
\newif\ifEtoc@hyperref
\newif\ifEtoc@parskip   % 1.07d
\newif\ifEtoc@tocwithid
\newif\ifEtoc@standard
\newif\ifEtoc@part 
%    \end{macrocode}
% placeholder for comments
%    \begin{macrocode}
\newif\ifEtoc@localtoc
\newif\ifEtoc@skipthisone
\newif\ifEtoc@stoptoc
\newif\ifEtoc@notactive
\newcounter{etoc@tocid}
\newif\ifEtoc@mustclosegroup
\def\etoc@{\etoc@} % Delimiter in \Etoc@getnb etc.. macros. This def added 1.07f
%    \end{macrocode}
% placeholder for comments
%    \begin{macrocode}
\@ifclassloaded{memoir}{\def\Etoc@minf{-\thr@@}}{\def\Etoc@minf{-\tw@}}
\def\Etoc@@minustwo@@{-\tw@}
\let\Etoc@@minusone@@\m@ne
\chardef\Etoc@@zero@@ 0
\let\Etoc@@one@@ \@ne
\let\Etoc@@two@@ \tw@
\let\Etoc@@three@@ \thr@@
\chardef\Etoc@@four@@ 4
\chardef\Etoc@@five@@ 5
\chardef\Etoc@@six@@ 6
\let\Etoc@localtop\Etoc@@minustwo@@
\def\Etoc@@minustwo@{minustwo}
\def\Etoc@@minusone@{minusone}
\def\Etoc@@zero@{zero}
\def\Etoc@@one@{one}
\def\Etoc@@two@{two}
\def\Etoc@@three@{three}
\def\Etoc@@four@{four}
\def\Etoc@@five@{five}
%\def\Etoc@@six@{six}
%    \end{macrocode}
% placeholder for comments
%    \begin{macrocode}
\def\Etoc@levellist{}
\def\Etoc@newlevel#1{% 
    \def\Etoc@levellist@elt{\noexpand\Etoc@levellist@elt\noexpand}%
    \edef\Etoc@levellist{\Etoc@levellist\Etoc@levellist@elt#1}}
\def\etocsetlevel#1#2{%
 \expandafter\Etoc@newlevel\csname l@#1\endcsname
 \ifcase#2\relax
      \expandafter\let \csname Etoc@#1@@\endcsname\Etoc@@zero@@
      \expandafter\let \csname Etoc@#1@\endcsname\Etoc@@zero@
   \or
      \expandafter\let \csname Etoc@#1@@\endcsname\Etoc@@one@@
      \expandafter\let \csname Etoc@#1@\endcsname\Etoc@@one@
   \or
      \expandafter\let \csname Etoc@#1@@\endcsname\Etoc@@two@@
      \expandafter\let \csname Etoc@#1@\endcsname\Etoc@@two@
   \or
      \expandafter\let \csname Etoc@#1@@\endcsname\Etoc@@three@@
      \expandafter\let \csname Etoc@#1@\endcsname\Etoc@@three@
   \or
      \expandafter\let \csname Etoc@#1@@\endcsname\Etoc@@four@@
      \expandafter\let \csname Etoc@#1@\endcsname\Etoc@@four@
   \or
      \expandafter\let \csname Etoc@#1@@\endcsname\Etoc@@five@@
      \expandafter\let \csname Etoc@#1@\endcsname\Etoc@@five@
   \or
      \expandafter\let \csname Etoc@#1@@\endcsname\Etoc@@six@@
   \else
   \ifnum#2=\m@ne
      \expandafter\let \csname Etoc@#1@@\endcsname\Etoc@@minusone@@
      \expandafter\let \csname Etoc@#1@\endcsname\Etoc@@minusone@
   \else
   \ifnum#2=-\tw@
      \expandafter\let \csname Etoc@#1@@\endcsname\Etoc@@minustwo@@
      \expandafter\let \csname Etoc@#1@\endcsname\Etoc@@minustwo@
   \else
       \PackageWarning{etoc}
         {unexpected value `#2' in \string\etocsetlevel.^^J%
          Should be -2,-1, 0, 1, 2, 3, 4, 5, or 6. Set to 6 (=ignored)}%
       \expandafter\let\csname Etoc@#1@@\endcsname\Etoc@@six@@
\fi\fi\fi}
\etocsetlevel{book}{-2}
\etocsetlevel{part}{-1}
\etocsetlevel{chapter}{0}
\etocsetlevel{section}{1}
\etocsetlevel{subsection}{2}
\etocsetlevel{subsubsection}{3}
\etocsetlevel{paragraph}{4}
\etocsetlevel{subparagraph}{5}
%    \end{macrocode}
% placeholder for comments
%    \begin{macrocode}
\def\Etoc@setflags #1{%
  \ifcase #1\relax
      \global\Etoc@vfalse
      \global\Etoc@ivfalse
      \global\Etoc@iiifalse
      \global\Etoc@iifalse
      \global\Etoc@ifalse
      \global\Etoc@true
  \or
      \global\Etoc@vfalse
      \global\Etoc@ivfalse
      \global\Etoc@iiifalse
      \global\Etoc@iifalse
      \global\Etoc@itrue
  \or
      \global\Etoc@vfalse
      \global\Etoc@ivfalse
      \global\Etoc@iiifalse
      \global\Etoc@iitrue
  \or
      \global\Etoc@vfalse
      \global\Etoc@ivfalse
      \global\Etoc@iiitrue
  \or
      \global\Etoc@vfalse
      \global\Etoc@ivtrue
  \or
      \global\Etoc@vtrue
  \else
    \ifnum#1=\m@ne
      \global\Etoc@vfalse
      \global\Etoc@ivfalse
      \global\Etoc@iiifalse
      \global\Etoc@iifalse
      \global\Etoc@ifalse
      \global\Etoc@false
      \global\Etoc@jtrue
    \else
      \global\Etoc@vfalse
      \global\Etoc@ivfalse
      \global\Etoc@iiifalse
      \global\Etoc@iifalse
      \global\Etoc@ifalse
      \global\Etoc@false
      \global\Etoc@jfalse
      \global\Etoc@jjtrue
    \fi
  \fi}
%    \end{macrocode}
% placeholder for comments
%    \begin{macrocode}
\AtBeginDocument{%
\@ifpackageloaded{parskip}{\Etoc@parskiptrue}{}%
\@ifpackageloaded{hyperref}{\Etoc@hyperreftrue
                            \def\Etoc@et@hop#1#2#3#4#5{#1{#3}{#4}{#5}#2}%
                            \long\def\Etoc@gobblesixorfive#1#2#3#4#5#6{}}
                           {\def\Etoc@et@hop#1#2#3#4{#1{#3}{#4}#2}%
                            \long\def\Etoc@gobblesixorfive#1#2#3#4#5{}}%
}
%    \end{macrocode}
% placeholder for comments
%    \begin{macrocode}
\def\Etoc@swa#1{%
        \Etoc@et@hop 
          {\Etoc@savedcontentsline{#1}}
          {\Etoc@prefix\Etoc@contents}}
\def\Etoc@swb#1{%
        \Etoc@et@hop 
          {\Etoc@savedcontentsline{#1}}
          {\Etoc@contents}}
\let\etocskipfirstprefix\@thirdofthree 
%    \end{macrocode}
% placeholder for comments
%    \begin{macrocode}
\def\Etoc@etoccontentsline#1{%
  \global\expandafter\let\expandafter\Etoc@tmp\csname Etoc@#1@@\endcsname
  \Etoc@skipthisonefalse
  \let\Etoc@next\Etoc@gobblesixorfive
  \ifnum\Etoc@tmp=\Etoc@@six@@
      \Etoc@skipthisonetrue
  \else
    \ifEtoc@localtoc
      \let\Etoc@prenext\relax
      \ifEtoc@stoptoc 
        \Etoc@skipthisonetrue 
      \fi
      \ifnum\Etoc@tmp<\Etoc@localtop 
        \def\Etoc@prenext{\global\Etoc@stoptoctrue}%
        \Etoc@skipthisonetrue
      \fi
      \ifEtoc@notactive
        \def\Etoc@prenext{\Etoc@setflags{\Etoc@tmp}}%
        \Etoc@skipthisonetrue
      \fi
      \Etoc@prenext
    \fi
  \fi
  \ifnum\c@tocdepth<\Etoc@tmp\relax\else
  \ifEtoc@skipthisone\else
  \global\let\Etoc@next\relax  
  \ifcase\Etoc@tmp 
      \ifEtoc@v \Etoc@end@five\fi
      \ifEtoc@iv \Etoc@end@four\fi
      \ifEtoc@iii \Etoc@end@three\fi
      \ifEtoc@ii \Etoc@end@two\fi
      \ifEtoc@i \Etoc@end@one\fi
      \ifEtoc@ \else \def\Etoc@next{\Etoc@begin@zero}\fi
      \def\Etoc@contents{\Etoc@contents@zero}%
      \def\Etoc@prefix{\Etoc@prefix@zero}%
  \or
      \ifEtoc@v \Etoc@end@five\fi
      \ifEtoc@iv \Etoc@end@four\fi
      \ifEtoc@iii \Etoc@end@three\fi
      \ifEtoc@ii \Etoc@end@two\fi
      \ifEtoc@i \else \def\Etoc@next{\Etoc@begin@one}\fi
      \def\Etoc@contents{\Etoc@contents@one}%
      \def\Etoc@prefix{\Etoc@prefix@one}%
  \or
      \ifEtoc@v \Etoc@end@five\fi
      \ifEtoc@iv \Etoc@end@four\fi
      \ifEtoc@iii \Etoc@end@three\fi
      \ifEtoc@ii \else \def\Etoc@next{\Etoc@begin@two}\fi
      \def\Etoc@contents{\Etoc@contents@two}%
      \def\Etoc@prefix{\Etoc@prefix@two}%
  \or
      \ifEtoc@v \Etoc@end@five\fi
      \ifEtoc@iv \Etoc@end@four\fi
      \ifEtoc@iii \else \def\Etoc@next{\Etoc@begin@three}\fi
      \def\Etoc@contents{\Etoc@contents@three}%
      \def\Etoc@prefix{\Etoc@prefix@three}%
  \or 
      \ifEtoc@v \Etoc@end@five\fi 
      \ifEtoc@iv \else \def\Etoc@next{\Etoc@begin@four}\fi
      \def\Etoc@contents{\Etoc@contents@four}%
      \def\Etoc@prefix{\Etoc@prefix@four}%
  \or 
      \ifEtoc@v \else \def\Etoc@next{\Etoc@begin@five}\fi
      \def\Etoc@contents{\Etoc@contents@five}%
      \def\Etoc@prefix{\Etoc@prefix@five}%
  \else
    \ifnum\Etoc@tmp=\m@ne
      \ifEtoc@v \Etoc@end@five\fi
      \ifEtoc@iv \Etoc@end@four\fi
      \ifEtoc@iii \Etoc@end@three\fi
      \ifEtoc@ii \Etoc@end@two\fi
      \ifEtoc@i \Etoc@end@one\fi
      \ifEtoc@ \Etoc@end@zero\fi
      \ifEtoc@j \else \def\Etoc@next{\Etoc@begin@minusone}\fi
      \def\Etoc@contents{\Etoc@contents@minusone}%
      \def\Etoc@prefix{\Etoc@prefix@minusone}%
    \else
      \ifEtoc@v \Etoc@end@five\fi
      \ifEtoc@iv \Etoc@end@four\fi
      \ifEtoc@iii \Etoc@end@three\fi
      \ifEtoc@ii \Etoc@end@two\fi
      \ifEtoc@i \Etoc@end@one\fi
      \ifEtoc@ \Etoc@end@zero\fi
      \ifEtoc@j \Etoc@end@minusone\fi
      \ifEtoc@jj \else \def\Etoc@next{\Etoc@begin@minustwo}\fi
      \def\Etoc@contents{\Etoc@contents@minustwo}%
      \def\Etoc@prefix{\Etoc@prefix@minustwo}%
      \fi
  \fi
    \ifnum\Etoc@tmp=\m@ne\Etoc@parttrue\else\Etoc@partfalse\fi
    \Etoc@setflags{\Etoc@tmp}%
  \fi\fi
  \Etoc@next
  \@firstoftwo{\Etoc@swa{#1}}{\Etoc@swb{#1}}}
%    \end{macrocode}
% |[2013/03/07]|:
%
% Up to |1.06| \etoc defined only \csa{etocname}, \csa{etocnumber} and
% \csa{etocpage}. The |hyperref| added data is recycled in the simplest manner,
% prefixing it with \csa{leavevmode}. The included \csa{Hy@tocdestname} is left
% unexpanded. Due to the initial (enormously incredible and opposite to the
% credo of separating the content from the form) \LaTeX{} flaw of mixing (Parts
% having their own format, not to mention other classes) in the |.toc| file the
% number and the name we have to spend some time with delimited macros to
% dis-entangle this, and reconstruct the possible |hyperref| data. Note that if
% the page number is not hyperlinked, \csa{etocpage} does \emph{not} add the
% link found possibly in the name.
%
% Then |1.07| added \csa{etocthename}, \csa{etocthenumber}, \csa{etocthepage}
% which are left fragile and do not have the links data, and
% \csa{etoclink}\marg{linkname} which is robust and reconstructs an arbitrarily
% named link. A need (for things like building up a token list to be used in a
% |tikzpicture|) arose later to have some form of the link which could be saved
% by a simple command like one can do |\global\let\lastname\etocthename|, and
% avoid having to manipulate \csa{Hy@tocdestname}. So |1.07f| adds
% \csa{etocthelinkedname}, \csa{etocthelinkednumber}, \csa{etocthelinkedpage},
% \csa{etocthelink}: they use \csa{hyperlink} with an expanded
% \csa{Hy@tocdestname}. 
%
% One could now define \csa{etocname}, etc ... to be the robust versions of
% \csa{etocthelinkedname}, etc ..., but the original definitions are kept by
% sentimentalism. |1.07f| also adds \csa{leavevmode} to \csa{etoclink} which
% should have been done earlier, as it was included in \csa{etocname} etc...
%
% attention,  \csa{@namedef}|{A}{B}| and not \csa{@namedef}|{A} {B}| !!
% on the other hand this gives a simple way to insert a space as the first
% token in the paramaters. For \csa{Etoc@again} (which appears later in the
% code), a \csa{@firstofone} construct is however the simplest of all.
%    \begin{macrocode}
\def\Etoc@lxyz #1#2{% 
     \@namedef {etoclink }{\leavevmode}%        fall-back
     \def\etocthelink         {}%               fall-back
     \@namedef {etocname }{\leavevmode #1}% fall-back (perhaps linked)
     \def\etocthename         {#1}%  (if link, will be removed later)
     \def\etocthelinkedname   {#1}%  will probably get redefined 
     \@namedef {etocpage }{\leavevmode #2}%      (perhaps linked)
   \Etoc@getthepage #2\etoc@ % defines also \etocthelinkedpage (and \etoclink)
   \Etoc@getnb #1\relax\relax\etoc@  % gets number *and* name, and \etoclink
   \ifEtoc@number\else               
      \ifEtoc@part 
        \Etoc@getit #1\hspace\relax\etoc@   % additional job for parts
      \fi
   \fi}
%    \end{macrocode}
% placeholder for comments
%    \begin{macrocode}
\def\Etoc@getthepage #1{%
  \let\Etoc@next\Etoc@getthepage@nohyp
  \ifEtoc@hyperref\ifx #1\hyper@linkstart
    \let\Etoc@next\Etoc@getthepage@hyp
  \fi\fi
  \Etoc@next #1%
}
\def\Etoc@getthepage@nohyp #1\etoc@ {%
    \def\etocthepage       {#1}%
    \def\etocthelinkedpage {#1}%
}
\def\Etoc@getthepage@hyp #1#2#3#4#5\etoc@ {%
  \@namedef{etoclink } ##1{\leavevmode #1{#2}{#3}{##1}#5}%
  \edef\etocthelink ##1{\noexpand\hyperlink {#3}{##1}}%
  \def\etocthepage {#4}%
  \toks@ {#4}%
  \edef\etocthelinkedpage {\noexpand\hyperlink {#3}{\the\toks@}}%
}
%
\def\Etoc@getnb #1{%
  \let\Etoc@next\Etoc@getnb@nohyp
  \ifEtoc@hyperref\ifx #1\hyper@linkstart
    \let\Etoc@next\Etoc@getnb@hyp
  \fi\fi
  \Etoc@next #1%
}
%
\def\Etoc@getit #1{%
  \let\Etoc@next\Etoc@getit@nohyp
  \ifEtoc@hyperref\ifx #1\hyper@linkstart
    \let\Etoc@next\Etoc@getit@hyp
  \fi\fi
  \Etoc@next #1%
}
%    \end{macrocode}
% placeholder for comments
%    \begin{macrocode}
\def\Etoc@getnb@nohyp #1#2#3\etoc@ {%
    \def\Etoc@getname ##1\relax\relax\etoc@ {%
      \@namedef {etocname }{\leavevmode ##1}%
      \def\etocthename       {##1}%
      \def\etocthelinkedname {##1}%
     }%
  \ifx #1\numberline
    \@namedef {etocnumber }{\leavevmode #2}%
    \def\etocthenumber       {#2}%
    \def\etocthelinkednumber {#2}%
    \Etoc@numbertrue
    \Etoc@getname #3\etoc@
  \else % then \etocthename and \etocthelinkedname already defined
    \@namedef {etocnumber }{\leavevmode}%
    \def\etocthenumber       {}%
    \def\etocthelinkednumber {}%
    \Etoc@numberfalse
  \fi
}
%    \end{macrocode}
% placeholder for comments
%    \begin{macrocode}
\def\Etoc@getnb@hyp #1#2#3#4#5#6\etoc@ {%
         \def\Etoc@getname ##1\relax\relax\etoc@ {%
          \@namedef {etocname }{\leavevmode #1{#2}{#3}{##1}#5}%
          \def\etocthename {##1}%
          \toks@           {##1}%
          \edef\etocthelinkedname {\noexpand\hyperlink {#3}{\the\toks@}}%
         }%
         \def\Etoc@getnbr ##1##2##3\etoc@ {%
          \ifx ##1\numberline
            \@namedef {etocnumber }{\leavevmode #1{#2}{#3}{##2}#5}%
            \def\etocthenumber {##2}%
            \toks@             {##2}%
            \edef\etocthelinkednumber {\noexpand\hyperlink {#3}{\the\toks@}}%
            \Etoc@numbertrue
            \Etoc@getname ##3\etoc@
          \else
            \@namedef {etocnumber }{\leavevmode}%
            \def\etocthenumber       {}%
            \def\etocthelinkednumber {}%
            \Etoc@numberfalse
            \def\etocthename {#4}%
            \toks@           {#4}%
            \edef\etocthelinkedname {\noexpand\hyperlink {#3}{\the\toks@}}%
          \fi
         }%
  \@namedef {etoclink }##1{\leavevmode #1{#2}{#3}{##1}#5}%
  \edef\etocthelink ##1{\noexpand\hyperlink {#3}{##1}}%
  \Etoc@getnbr #4\relax\relax\etoc@
}
%    \end{macrocode}
% placeholder for comments
%    \begin{macrocode}
\def\Etoc@getit@nohyp #1\hspace#2#3\etoc@ {%
    \def\Etoc@getname ##1\hspace\relax\etoc@ {%
       \@namedef {etocname }{\leavevmode ##1}%
       \def\etocthename       {##1}%
       \def\etocthelinkedname {##1}%
    }%
  \ifx\relax#2\else 
        \@namedef {etocnumber }{\leavevmode #1}%
        \def\etocthenumber       {#1}%
        \def\etocthelinkednumber {#1}%
        \Etoc@numbertrue
        \Etoc@getname #3\etoc@
  \fi
}
%    \end{macrocode}
% placeholder for comments
%    \begin{macrocode}
\def\Etoc@getit@hyp #1#2#3#4#5#6\etoc@ {%
    \def\Etoc@getname ##1\hspace\relax\etoc@ {%
       \@namedef {etocname }{\leavevmode #1{#2}{#3}{##1}#5}%
       \def\etocthename {##1}%
       \toks@           {##1}%
       \edef\etocthelinkedname {\noexpand\hyperlink {#3}{\the\toks@}}%
    }%
    \def\Etoc@getnbr ##1\hspace##2##3\etoc@ {%
       \ifx\relax##2\else
         \@namedef {etocnumber }{\leavevmode #1{#2}{#3}{##1}#5}%
         \def\etocthenumber {##1}%
         \toks@             {##1}%
         \edef\etocthelinkednumber {\noexpand\hyperlink {#3}{\the\toks@}}%
         \Etoc@numbertrue
         \Etoc@getname ##3\etoc@
       \fi
    }%
  \Etoc@getnbr #4\hspace\relax\etoc@
}
%    \end{macrocode}
% placeholder for comments
%    \begin{macrocode}
\newcommand*\etocthename   {}
\newcommand*\etocthenumber {}
\newcommand*\etocthepage   {}
\newcommand*\etocthelinkedname   {}
\newcommand*\etocthelinkednumber {}
\newcommand*\etocthelinkedpage   {}
\newcommand*\etocthelink   {}
\DeclareRobustCommand*{\etocname}  {}
\DeclareRobustCommand*{\etocnumber}{}
\DeclareRobustCommand*{\etocpage}  {}
\DeclareRobustCommand*{\etoclink}  {}
\DeclareRobustCommand*{\etocifnumbered}
   {\ifEtoc@number\expandafter\@firstoftwo\else\expandafter\@secondoftwo\fi}
%    \end{macrocode}
% placeholder for comments
%    \begin{macrocode}
\def\Etoc@readtoc#1{%
  \ifeof #1
     \let\Etoc@nextread\@gobble
     \global\Etoc@toctoks=\expandafter{\the\Etoc@toctoks}%
  \else
     \let\Etoc@nextread\Etoc@readtoc
     \read #1 to \Etoc@buffer
     \Etoc@toctoks=\expandafter\expandafter\expandafter
       {\expandafter\the\expandafter\Etoc@toctoks\Etoc@buffer}%
  \fi
  \Etoc@nextread{#1}%
}
\IfFileExists{\jobname .toc}
    {{\endlinechar=-1 
      \makeatletter
      \newread\Etoc@tf
      \openin\Etoc@tf\@filef@und
      \Etoc@readtoc\Etoc@tf
      \closein\Etoc@tf}}
    {\typeout{No file \jobname .toc.}}
%    \end{macrocode}
% placeholder for comments
%    \begin{macrocode}
\def\Etoc@openouttoc{% formerly \Etoc@starttoc
%% 1.07d: parskip and \@nobreakfalse stuff moved to \Etoc@tableofcontents
  \ifEtoc@hyperref 
    \ifx\hyper@last\@undefined
    \IfFileExists{\jobname .toc}
      {\Hy@WarningNoLine 
         {old toc file detected, not used; run LaTeX again (cheers from etoc)}%
       \global\Etoc@toctoks={}%
      }
      {}%
    \fi
  \fi
  \if@filesw 
   \newwrite \tf@toc 
   \immediate \openout \tf@toc \jobname .toc\relax 
  \fi 
  \gdef\Etoc@openouttoc{}% 1.07d, rather than using a boolean
}
%    \end{macrocode}
% placeholder for comments
%    \begin{macrocode}
\def\Etoc@toctoc{%
   \global\Etoc@vfalse
   \global\Etoc@ivfalse
   \global\Etoc@iiifalse
   \global\Etoc@iifalse
   \global\Etoc@ifalse
   \global\Etoc@false
   \global\Etoc@jfalse
   \global\Etoc@jjfalse
\ifEtoc@standard
    \etoc@setstyle{@minustwo}{}{}{}{}%
    \etoc@setstyle{@minusone}{}{}{}{}%
    \etoc@setstyle{@zero}{}{}{}{}%
    \etoc@setstyle{@one}{}{}{}{}%
    \etoc@setstyle{@two}{}{}{}{}%
    \etoc@setstyle{@three}{}{}{}{}%
    \etoc@setstyle{@four}{}{}{}{}%
    \etoc@setstyle{@five}{}{}{}{}%
\else
    \def\Etoc@levellist@elt##1{\let##1\Etoc@lxyz}%
    \Etoc@levellist
    \let\booknumberline\numberline
    \let\partnumberline\numberline
    \let\chapternumberline\numberline
\fi
\the\Etoc@toctoks
\ifEtoc@notactive\else 
  \ifEtoc@v   \Etoc@end@five\fi
  \ifEtoc@iv  \Etoc@end@four\fi
  \ifEtoc@iii \Etoc@end@three\fi
  \ifEtoc@ii  \Etoc@end@two\fi
  \ifEtoc@i   \Etoc@end@one\fi
  \ifEtoc@    \Etoc@end@zero\fi
  \ifEtoc@j   \Etoc@end@minusone\fi
  \ifEtoc@jj  \Etoc@end@minustwo\fi
\fi}
%    \end{macrocode}
% placeholder for comments
%    \begin{macrocode}
\def\etoc@@startlocaltoc#1#2{%
\let\Etoc@next\relax
    \ifEtoc@localtoc 
    \ifEtoc@notactive
        \ifnum #1=#2\relax
          \ifEtoc@jj  \let\Etoc@localtop\Etoc@@minusone@@ \fi
          \ifEtoc@j   \let\Etoc@localtop\Etoc@@zero@@  \fi
          \ifEtoc@    \let\Etoc@localtop\Etoc@@one@@   \fi
          \ifEtoc@i   \let\Etoc@localtop\Etoc@@two@@   \fi
          \ifEtoc@ii  \let\Etoc@localtop\Etoc@@three@@ \fi
          \ifEtoc@iii \let\Etoc@localtop\Etoc@@four@@  \fi
          \ifEtoc@iv  \let\Etoc@localtop\Etoc@@five@@  \fi
          \ifEtoc@v   \let\Etoc@localtop\Etoc@@six@@   \fi
          \def\Etoc@next{\global\Etoc@notactivefalse
                      \global\Etoc@vfalse
                      \global\Etoc@ivfalse
                      \global\Etoc@iiifalse
                      \global\Etoc@iifalse
                      \global\Etoc@ifalse
                      \global\Etoc@false
                      \global\Etoc@jfalse
                      \global\Etoc@jjfalse}%
        \fi
    \fi\fi
\Etoc@next}
\let\etoc@startlocaltoc\@gobble
%    \end{macrocode}
% placeholder for comments
%    \begin{macrocode}
\def\Etoc@localtableofcontents#1{%
    \edef\Etoc@tmp{#1}%
    \ifnum\Etoc@tmp<\@ne
    \PackageWarning{etoc}
       {Unknown toc id: run LaTeX to get references right}% 
     \leavevmode --unknown etoc ref: run latex again--\par
    \let\Etoc@next\@gobble\else\let\Etoc@next\@firstofone\fi
    \Etoc@next
    {\edef\etoc@startlocaltoc##1{%
         \noexpand\etoc@@startlocaltoc{##1}{#1}}
    \Etoc@localtoctrue
    \let\Etoc@localtop\Etoc@@minustwo@@
    \global\Etoc@stoptocfalse
    \global\Etoc@notactivetrue
    \Etoc@tableofcontents}%
    \endgroup\ifEtoc@mustclosegroup\endgroup\fi}
%    \end{macrocode}
% |[2013/03/07]|: I discover a \csa{@namedef} trick to construct the
% \csa{Etoc@again} space delimited macro:\\ 
% |    \@namedef {Etoc@again} {...stuff...}|\\
%  Original version was (copied from analogous stuff in |source2e|):\\
% |    {\def\1{\Etoc@again}\expandafter\gdef\1 {...stuff...}}|\\
% and in the end (now that I think about it) I simply use |\@firstofone|.\par
%    \begin{macrocode}
\def\Etoc@getrefno #1#2\etoc@ {#1}
\def\Etoc@getref #1{\@ifundefined{r@#1}{0}{\expandafter\expandafter\expandafter
          \Etoc@getrefno\csname r@#1\endcsname\relax\etoc@}}
\def\Etoc@ref#1{\Etoc@localtableofcontents{\Etoc@getref{#1}}}
\def\Etoc@label#1{\label{#1}\futurelet\Etoc@nexttoken\Etoc@t@bleofcontents}
\@firstofone{\def\Etoc@again} {\futurelet\Etoc@nexttoken\Etoc@t@bleofcontents}
%    \end{macrocode}
% placeholder for comments
%    \begin{macrocode}
\def\Etoc@t@bleofcontents{%
 \ifx\Etoc@nexttoken\label
    \def\Etoc@next{\expandafter\Etoc@label\@gobble}\else
 \ifx\Etoc@nexttoken\@sptoken
    \let\Etoc@next\Etoc@again\else
 \ifEtoc@tocwithid
    \def\Etoc@next{\Etoc@localtableofcontents{\c@etoc@tocid}}%
 \else
    \ifx\Etoc@nexttoken\ref
       \def\Etoc@next{\expandafter\Etoc@ref\@gobble}%
    \else
       \def\Etoc@next{\Etoc@localtocfalse
                      \global\Etoc@notactivefalse
                      \Etoc@tableofcontents
                      \endgroup
                      \ifEtoc@mustclosegroup\endgroup\fi}%
    \fi
 \fi\fi\fi\Etoc@next}
%    \end{macrocode}
% placeholder for comments
%    \begin{macrocode}
\def\table@fcontents{%
    \refstepcounter{etoc@tocid}% 
    \Etoc@tocwithidfalse
    \futurelet\Etoc@nexttoken\Etoc@t@bleofcontents}
\def\localtable@fcontents{%
    \refstepcounter{etoc@tocid}%
    \addtocontents{toc}
       {\string\etoc@startlocaltoc{\arabic{etoc@tocid}}}%
    \Etoc@tocwithidtrue
    \futurelet\Etoc@nexttoken\Etoc@t@bleofcontents}
%    \end{macrocode}
% placeholder for comments
%    \begin{macrocode}
\newcommand*\etocaftertitlehook{}
\newcommand*\etocaftercontentshook{}
\renewcommand*\tableofcontents{%
   \Etoc@openouttoc
   \Etoc@par
   \begingroup % closed in \Etoc@t@bleofcontents or \Etoc@localtableofcontents
     \def\etoc@startlocaltoc##1{\etoc@@startlocaltoc{##1}{\c@etoc@tocid}}%
     \@ifstar
     {\def\Etoc@aftertitlehook{}\table@fcontents}
     {\let\Etoc@aftertitlehook\etocaftertitlehook\table@fcontents}}
\newcommand*\localtableofcontents{%
   \Etoc@openouttoc
   \Etoc@par
   \begingroup % closed in \Etoc@t@bleofcontents or \Etoc@localtableofcontents
     \@ifstar
     {\def\Etoc@aftertitlehook{}\localtable@fcontents}
     {\let\Etoc@aftertitlehook\etocaftertitlehook\localtable@fcontents}}
%    \end{macrocode}
% placeholder for comments
%    \begin{macrocode}
\newcommand\etocsettocstyle[2]{%
\def\Etoc@tableofcontents
{\ifnum\c@tocdepth>\Etoc@minf
   \let\Etoc@@next\@firstofone\else
   \let\Etoc@@next\@gobble
\fi
\Etoc@@next{#1\ifEtoc@parskip\parskip\z@skip\fi %1.07d
              \Etoc@aftertitlehook
              \let\Etoc@savedcontentsline\contentsline
              \let\contentsline\Etoc@etoccontentsline
              \Etoc@toctoc
              \let\Etoc@@next\relax
                \ifEtoc@tocwithid\else
                \ifEtoc@localtoc
                \ifEtoc@notactive
                  \def\Etoc@@next{\Etoc@localtocfalse
                           \global\Etoc@notactivefalse
                           \Etoc@toctoc}%
                \fi\fi\fi
              \Etoc@@next
              \etocaftercontentshook
              #2\@nobreakfalse}}}  % 1.07d: \@nobreakfalse moved here
%    \end{macrocode}
% placeholder for comments
%    \begin{macrocode}
\newcommand*\etocsetstyle{\Etoc@standardfalse\etoc@setstyle}
\long\def\etoc@setstyle#1#2#3#4#5{%
\long\expandafter\def
     \csname Etoc@begin@\csname Etoc@#1@\endcsname\endcsname {#2}%
\long\expandafter\def
     \csname Etoc@prefix@\csname Etoc@#1@\endcsname\endcsname {#3}%
\long\expandafter\def
     \csname Etoc@contents@\csname Etoc@#1@\endcsname\endcsname {#4}%
\long\expandafter\def
     \csname Etoc@end@\csname Etoc@#1@\endcsname\endcsname {#5}}
%    \end{macrocode}
% placeholder for comments
%    \begin{macrocode}
\newcommand*\etocfontminustwo {\normalfont \LARGE \bfseries}
\newcommand*\etocfontminusone {\normalfont \large \bfseries}
\newcommand*\etocfontzero     {\normalfont \large \bfseries}
\newcommand*\etocfontone      {\normalfont \normalsize \bfseries}
\newcommand*\etocfonttwo      {\normalfont \normalsize}
\newcommand*\etocfontthree    {\normalfont \footnotesize}
%    \end{macrocode}
% placeholder for comments
%    \begin{macrocode}
\newcommand*\etocsepminustwo  {4ex \@plus .5ex \@minus .5ex}
\newcommand*\etocsepminusone  {4ex \@plus .5ex \@minus .5ex}
\newcommand*\etocsepzero      {2.5ex \@plus .4ex \@minus .4ex}
\newcommand*\etocsepone       {1.5ex \@plus .3ex \@minus .3ex}
%%\newcommand*\etocseptwo{1ex \@plus .15ex \@minus .15ex} % modified in 1.07e
\newcommand*\etocseptwo       {.5ex \@plus .1ex \@minus .1ex}
\newcommand*\etocsepthree     {.25ex \@plus .05ex \@minus .05ex}
%    \end{macrocode}
% placeholder for comments
%    \begin{macrocode}
\newcommand*\etocbaselinespreadminustwo {1}
\newcommand*\etocbaselinespreadminusone {1}
\newcommand*\etocbaselinespreadzero     {1}
\newcommand*\etocbaselinespreadone      {1}
\newcommand*\etocbaselinespreadtwo      {1}
\newcommand*\etocbaselinespreadthree    {.9}
%    \end{macrocode}
% placeholder for comments
%    \begin{macrocode}
\newcommand*\etocminustwoleftmargin  {1.5em plus 0.5fil}
\newcommand*\etocminustworightmargin {1.5em plus -0.5fil}
\newcommand*\etocminusoneleftmargin  {1em}
\newcommand*\etocminusonerightmargin {1em}
\newcommand*\etoctoclineleaders 
        {\hbox{\normalfont\normalsize\hb@xt@2ex {\hss.\hss}}}
\newcommand*\etocabbrevpagename {p.~}
\newcommand*\etocpartname       {\partname}
\newcommand*\etocbookname       {Book}
%    \end{macrocode}
% placeholder for comments
% The macro \csa{etocdefaultlines} was initially called \csa{etoctoclines}. Now
% \csa{etoctoclines} just does \csa{Etoc@standardfalse}.
%    \begin{macrocode}
\def\etocdefaultlines{\Etoc@standardfalse
%    \end{macrocode}
% Version |1.07e| has rewritten entirely the stuff related to penalties and
% \csa{addvspace}, as this was not satisfactory in the earlier versions, which
% were written at a early stage in the development of the package.
%    \begin{macrocode}
%% `book' in memoir class:
\etoc@setstyle{@minustwo}
  {\addpenalty\@M\etocskipfirstprefix}
  {\addpenalty\@secpenalty}
  {\begingroup 
   \etocfontminustwo
   \addvspace{\etocsepminustwo}%
   \parindent \z@ 
   \leftskip  \etocminustwoleftmargin
   \rightskip \etocminustworightmargin
   \parfillskip \@flushglue
   \vbox{\etocifnumbered{\etocbookname\enspace\etocnumber:\quad}{}\etocname
        \baselineskip\etocbaselinespreadminustwo\baselineskip
        \par}%
   \addpenalty\@M\addvspace{\etocsepminusone}%
   \endgroup}
  {}%
%% `part':
\etoc@setstyle{@minusone}
  {\addpenalty\@M\etocskipfirstprefix}
  {\addpenalty\@secpenalty}
  {\begingroup 
   \etocfontminusone
   \addvspace{\etocsepminusone}%
   \parindent \z@ 
   \leftskip  \etocminusoneleftmargin
   \rightskip \etocminusonerightmargin
   \parfillskip \@flushglue
   \vbox{\etocifnumbered{\etocpartname\enspace\etocnumber.\quad}{}\etocname
         \baselineskip\etocbaselinespreadminusone\baselineskip
         \par}%
   \addpenalty\@M\addvspace{\etocsepzero}%
   \endgroup}
  {}%
%% `chapter':
\etoc@setstyle{@zero}
  {\addpenalty\@M\etocskipfirstprefix}
  {\addpenalty\@itempenalty}
  {\begingroup 
   \etocfontzero
   \addvspace{\etocsepzero}%
   \parindent \z@ \parfillskip \@flushglue 
   \vbox{\etocifnumbered{\etocnumber.\enspace}{}\etocname
         \baselineskip\etocbaselinespreadzero\baselineskip
         \par}%
   \endgroup}
  {\addpenalty{-\@highpenalty}\addvspace{\etocsepminusone}}%
%% `section':
\etoc@setstyle{@one}
  {\addpenalty\@M\etocskipfirstprefix}
  {\addpenalty\@itempenalty}
  {\begingroup
   \etocfontone
   \addvspace{\etocsepone}%
   \parindent \z@ \parfillskip \z@ 
   \setbox\z@\vbox{\parfillskip\@flushglue
                   \etocname\par
                   \setbox\tw@\lastbox
                   \global\setbox\@ne\hbox{\unhbox\tw@\ }}%
   \dimen\z@=\wd\@ne
   \setbox\z@=\etoctoclineleaders
   \advance\dimen\z@\wd\z@
   \etocifnumbered
     {\setbox\tw@\hbox{\etocnumber, \etocabbrevpagename\etocpage}}
     {\setbox\tw@\hbox{\etocabbrevpagename\etocpage}}%
   \advance\dimen\z@\wd\tw@
   \ifdim\dimen\z@ < \linewidth
       \vbox{\etocname~%
             \leaders\box\z@\hfil\box\tw@
             \baselineskip\etocbaselinespreadone\baselineskip
             \par}
   \else
       \vbox{\etocname~%
             \leaders\copy\z@\hfil\break 
             \hbox{}\leaders\box\z@\hfil\box\tw@
             \baselineskip\etocbaselinespreadone\baselineskip
             \par}
   \fi
   \endgroup}
  {\addpenalty\@secpenalty\addvspace{\etocsepzero}}%
%% `subsection':
\etoc@setstyle{@two}
  {\addpenalty\@medpenalty\etocskipfirstprefix} 
  {\addpenalty\@itempenalty}
  {\begingroup
   \etocfonttwo
   \addvspace{\etocseptwo}%
   \parindent \z@ \parfillskip \z@
   \setbox\z@\vbox{\parfillskip\@flushglue
                   \etocname\par\setbox\tw@\lastbox
                   \global\setbox\@ne\hbox{\unhbox\tw@}}%
   \dimen\z@=\wd\@ne
   \setbox\z@=\etoctoclineleaders
   \advance\dimen\z@\wd\z@
   \etocifnumbered
     {\setbox\tw@\hbox{\etocnumber, \etocabbrevpagename\etocpage}}
     {\setbox\tw@\hbox{\etocabbrevpagename\etocpage}}%
   \advance\dimen\z@\wd\tw@
   \ifdim\dimen\z@ < \linewidth
       \vbox{\etocname~%
             \leaders\box\z@\hfil\box\tw@
             \baselineskip\etocbaselinespreadtwo\baselineskip
             \par}
   \else
       \vbox{\etocname~%
             \leaders\copy\z@\hfil\break 
             \hbox{}\leaders\box\z@\hfil\box\tw@
             \baselineskip\etocbaselinespreadtwo\baselineskip
             \par}
   \fi
   \endgroup}
  {\addpenalty\@secpenalty\addvspace{\etocsepone}}%
%% `subsubsection':
\etoc@setstyle{@three}
  {\addpenalty\@M
   \etocfontthree
   \vspace{\etocsepthree}%
   \noindent
   \etocskipfirstprefix}
  {\allowbreak\,--\,}
  {\etocname}
  {.\hfil
    \begingroup
     \baselineskip\etocbaselinespreadthree\baselineskip
     \par
    \endgroup
   \addpenalty{-\@highpenalty}}%
%    \end{macrocode}
% placeholder for comments
%    \begin{macrocode}
\etoc@setstyle{@four}{}{}{}{}%
\etoc@setstyle{@five}{}{}{}{}%
}
%    \end{macrocode}
% placeholder for comments
%    \begin{macrocode}
\newcommand*\etocabovetocskip{3.5ex \@plus 1ex \@minus .2ex} 
\newcommand*\etocbelowtocskip{3.5ex \@plus 1ex \@minus .2ex}
\newcommand*\etoccolumnsep{2em}
\newcommand*\etocmulticolsep{0ex}
\newcommand*\etocmulticolpretolerance{-1}
\newcommand*\etocmulticoltolerance{200}
\newcommand*\etocdefaultnbcol{2}
\newcommand*\etocinnertopsep{2ex}
%    \end{macrocode}
% placeholder for comments
%    \begin{macrocode}
\newcommand\etocmulticolstyle[2][\etocdefaultnbcol]{%
\etocsettocstyle
   {\let\etocoldpar\par
    \addvspace{\etocabovetocskip}%
    \ifnum #1>\@ne\let\Etoc@next\@firstoftwo
         \else \let\Etoc@next\@secondoftwo\fi
    \Etoc@next{%
    \multicolpretolerance\etocmulticolpretolerance
    \multicoltolerance\etocmulticoltolerance
    \setlength{\columnsep}{\etoccolumnsep}%
    \setlength{\multicolsep}{\etocmulticolsep}%
    \begin{multicols}{#1}[#2\etocoldpar\addvspace{\etocinnertopsep}]}
% 2013/01/29: erroneous \etocsepminusone at last replaced by \etocinnertopsep
% and definition of \etocoldpar added as multicols chokes on \par as part of #2
    {#2\par\addvspace{\etocinnertopsep}%
       \pretolerance\etocmulticolpretolerance
       \tolerance\etocmulticoltolerance}}
   {\ifnum #1>\@ne\let\Etoc@next\@firstofone
         \else \let\Etoc@next\@gobble\fi
    \Etoc@next{\end{multicols}}%
    \addvspace{\etocbelowtocskip}}}
%    \end{macrocode}
% placeholder for comments
%    \begin{macrocode}
\newcommand*\etocinnerbottomsep{3.5ex}
\newcommand*\etocinnerleftsep{2em}
\newcommand*\etocinnerrightsep{2em}
\newcommand*\etoctoprule{\hrule}
\newcommand*\etocleftrule{\vrule}
\newcommand*\etocrightrule{\vrule}
\newcommand*\etocbottomrule{\hrule}
\newcommand*\etoctoprulecolorcmd{\relax}
\newcommand*\etocbottomrulecolorcmd{\relax}
\newcommand*\etocleftrulecolorcmd{\relax}
\newcommand*\etocrightrulecolorcmd{\relax}
%    \end{macrocode}
% placeholder for comments
%    \begin{macrocode}
\def\etoc@ruledheading #1{%
   \hb@xt@\linewidth{\color@begingroup
          \hss #1\hss\hskip-\linewidth
          \etoctoprulecolorcmd\leaders\etoctoprule\hss
          \phantom{#1}%
          \leaders\etoctoprule\hss\color@endgroup}%
          \nointerlineskip\vskip\etocinnertopsep}
%    \end{macrocode}
% placeholder for comments
%    \begin{macrocode}
\newcommand*\etocruledstyle[2][\etocdefaultnbcol]{%
\etocsettocstyle
   {\addvspace{\etocabovetocskip}%
    \ifnum #1>\@ne\let\Etoc@next\@firstoftwo
         \else \let\Etoc@next\@secondoftwo\fi
    \Etoc@next
       {\multicolpretolerance\etocmulticolpretolerance
        \multicoltolerance\etocmulticoltolerance
        \setlength{\columnsep}{\etoccolumnsep}%
        \setlength{\multicolsep}{\etocmulticolsep}%
        \begin{multicols}{#1}[\etoc@ruledheading{#2}]}
       {\etoc@ruledheading{#2}\nobreak
         \pretolerance\etocmulticolpretolerance
         \tolerance\etocmulticoltolerance}} 
   {\ifnum #1>\@ne\let\Etoc@next\@firstofone
         \else \let\Etoc@next\@gobble\fi
    \Etoc@next{\end{multicols}}%
    \addvspace{\etocbelowtocskip}}}
%    \end{macrocode}
% placeholder for comments
%    \begin{macrocode}
\newcommand\etocframedmphook{\relax}
\newcommand*\etocbkgcolorcmd{\relax}
\def\Etoc@relax{\relax}
\newbox\etoc@framed@titlebox
\newbox\etoc@framed@contentsbox
\newcommand*\etocframedstyle[2][\etocdefaultnbcol]{%
\etocsettocstyle{%
    \addvspace{\etocabovetocskip}%
    \sbox\z@{#2}%
    \dimen\z@\dp\z@
        \ifdim\wd\z@<\linewidth \dp\z@\z@ \else \dimen\z@\z@ \fi
    \setbox\etoc@framed@titlebox=\hb@xt@\linewidth{\color@begingroup
        \hss 
        \ifx\etocbkgcolorcmd\Etoc@relax\else 
            \sbox\tw@{\color{white}%
            \vrule\@width\wd\z@\@height\ht\z@\@depth\dimen\z@}%
            \ifdim\wd\z@<\linewidth \dp\tw@\z@\fi
            \box\tw@
            \hskip-\wd\z@
        \fi
        \copy\z@
        \hss
        \hskip-\linewidth
        \etoctoprulecolorcmd\leaders\etoctoprule\hss%
        \hskip\wd\z@
        \etoctoprulecolorcmd\leaders\etoctoprule\hss\color@endgroup}%
    \setbox\z@\hbox{\etocleftrule\etocrightrule}%
    \dimen\tw@\linewidth\advance\dimen\tw@-\wd\z@
        \advance\dimen\tw@-\etocinnerleftsep
        \advance\dimen\tw@-\etocinnerrightsep
    \setbox\etoc@framed@contentsbox=\vbox\bgroup
        \hsize\dimen\tw@
        \kern\dimen\z@
        \vskip\etocinnertopsep
        \hbox\bgroup
        \begin{minipage}{\hsize}%
        \etocframedmphook
    \ifnum #1>\@ne\let\Etoc@next\@firstoftwo
    \else \let\Etoc@next\@secondoftwo\fi
        \Etoc@next
        {\multicolpretolerance\etocmulticolpretolerance
        \multicoltolerance\etocmulticoltolerance
        \setlength{\columnsep}{\etoccolumnsep}%
        \setlength{\multicolsep}{\etocmulticolsep}%
        \begin{multicols}{#1}}  
        {\pretolerance\etocmulticolpretolerance
         \tolerance\etocmulticoltolerance}}
     {\ifnum #1>\@ne\let\Etoc@next\@firstofone
         \else \let\Etoc@next\@gobble\fi
    \Etoc@next{\end{multicols}\unskip}%
    \end{minipage}%
    \egroup
    \vskip\etocinnerbottomsep
    \egroup
    \vbox{\hsize\linewidth
        \ifx\etocbkgcolorcmd\Etoc@relax\else
            \kern\ht\etoc@framed@titlebox
            \kern\dp\etoc@framed@titlebox
            \hb@xt@\linewidth{\color@begingroup
            \etocleftrulecolorcmd\etocleftrule
            \etocbkgcolorcmd
            \leaders\vrule 
                   \@height\ht\etoc@framed@contentsbox 
                   \@depth\dp\etoc@framed@contentsbox 
            \hss
            \etocrightrulecolorcmd\etocrightrule
            \color@endgroup}\nointerlineskip
            \vskip-\dp\etoc@framed@contentsbox
            \vskip-\ht\etoc@framed@contentsbox
            \vskip-\dp\etoc@framed@titlebox
            \vskip-\ht\etoc@framed@titlebox
        \fi
    \box\etoc@framed@titlebox\nointerlineskip
    \hb@xt@\linewidth{\color@begingroup
    {\etocleftrulecolorcmd\etocleftrule}%
    \hss\box\etoc@framed@contentsbox\hss
    \etocrightrulecolorcmd\etocrightrule\color@endgroup}
    \nointerlineskip
    \vskip\ht\etoc@framed@contentsbox
    \vskip\dp\etoc@framed@contentsbox
    \hb@xt@\linewidth{\color@begingroup\etocbottomrulecolorcmd
          \leaders\etocbottomrule\hss\color@endgroup}}
    \addvspace{\etocbelowtocskip}}}
%    \end{macrocode}
% placeholder for comments
%    \begin{macrocode}
\newcommand\etoc@multicoltoc[2][\etocdefaultnbcol]{%
    \etocmulticolstyle[#1]{#2}%
    \tableofcontents}
\newcommand\etoc@multicoltoci[2][\etocdefaultnbcol]{%
    \etocmulticolstyle[#1]{#2}%
    \tableofcontents*}
\newcommand\etoc@local@multicoltoc[2][\etocdefaultnbcol]{%
    \etocmulticolstyle[#1]{#2}%
    \localtableofcontents}
\newcommand\etoc@local@multicoltoci[2][\etocdefaultnbcol]{%
    \etocmulticolstyle[#1]{#2}%
    \localtableofcontents*}
%    \end{macrocode}
% placeholder for comments
%    \begin{macrocode}
\newcommand*\etoc@ruledtoc[2][\etocdefaultnbcol]{%
    \etocruledstyle[#1]{#2}%
    \tableofcontents}
\newcommand*\etoc@ruledtoci[2][\etocdefaultnbcol]{%
    \etocruledstyle[#1]{#2}%
    \tableofcontents*}
\newcommand*\etoc@local@ruledtoc[2][\etocdefaultnbcol]{%
    \etocruledstyle[#1]{#2}%
    \localtableofcontents}
\newcommand*\etoc@local@ruledtoci[2][\etocdefaultnbcol]{%
    \etocruledstyle[#1]{#2}%
    \localtableofcontents*}
%    \end{macrocode}
% placeholder for comments
%    \begin{macrocode}
\newcommand*\etoc@framedtoc[2][\etocdefaultnbcol]{%
    \etocframedstyle[#1]{#2}%
    \tableofcontents}
\newcommand*\etoc@framedtoci[2][\etocdefaultnbcol]{%
    \etocframedstyle[#1]{#2}%
    \tableofcontents*}
\newcommand*\etoc@local@framedtoc[2][\etocdefaultnbcol]{%
    \etocframedstyle[#1]{#2}%
    \localtableofcontents}
\newcommand*\etoc@local@framedtoci[2][\etocdefaultnbcol]{%
    \etocframedstyle[#1]{#2}%
    \localtableofcontents*}
%    \end{macrocode}
% placeholder for comments
%    \begin{macrocode}
\def\etocmulticol{\begingroup
    \Etoc@mustclosegrouptrue
    \@ifstar
    {\etoc@multicoltoci}
    {\etoc@multicoltoc}}
\def\etocruled{\begingroup
    \Etoc@mustclosegrouptrue
    \@ifstar
    {\etoc@ruledtoci}
    {\etoc@ruledtoc}}
\def\etocframed{\begingroup
    \Etoc@mustclosegrouptrue
    \@ifstar
    {\etoc@framedtoci}
    {\etoc@framedtoc}}
\def\etoclocalmulticol{\begingroup
    \Etoc@mustclosegrouptrue
    \@ifstar
    {\etoc@local@multicoltoci}
    {\etoc@local@multicoltoc}}
\def\etoclocalruled{\begingroup
    \Etoc@mustclosegrouptrue
    \@ifstar
    {\etoc@local@ruledtoci}
    {\etoc@local@ruledtoc}}
\def\etoclocalframed{\begingroup
    \Etoc@mustclosegrouptrue
    \@ifstar
    {\etoc@local@framedtoci}
    {\etoc@local@framedtoc}}
%    \end{macrocode}
% placeholder for comments
%    \begin{macrocode}
\def\etocarticlestyle{%
    \etocsettocstyle
    {\section *{\contentsname 
                \@mkboth {\MakeUppercase \contentsname}
                         {\MakeUppercase \contentsname}}}
    {}}
\def\etocarticlestylenomarks{%
    \etocsettocstyle
    {\section *{\contentsname}}
    {}}
%    \end{macrocode}
% placeholder for comments
%    \begin{macrocode}
\def\etocbookstyle{%
    \etocsettocstyle
    {\if@twocolumn \@restonecoltrue \onecolumn \else \@restonecolfalse \fi 
     \chapter *{\contentsname 
                \@mkboth {\MakeUppercase \contentsname}
                         {\MakeUppercase \contentsname}}}
    {\if@restonecol \twocolumn \fi}}
\def\etocbookstylenomarks{%
    \etocsettocstyle
    {\if@twocolumn \@restonecoltrue \onecolumn \else \@restonecolfalse \fi 
     \chapter *{\contentsname}}
    {\if@restonecol \twocolumn \fi}}
\let\etocreportstyle\etocbookstyle
\let\etocreportstylenomarks\etocbookstylenomarks
\def\etocmemoirtoctotocfmt #1#2{%
    \def\Etoc@addsuitablecontentsline{\addcontentsline {toc}{#1}{#2}}%
    \renewcommand*\etocaftertitlehook{%
      \ifmem@em@starred@listof
      \else\phantomsection\aftergroup\Etoc@addsuitablecontentsline\fi}}
\def\etocmemoirstyle{%
    \etocsettocstyle
        {\ensureonecol \par \begingroup \@nameuse {@tocmaketitle}
         \Etoc@aftertitlehook\let\Etoc@aftertitlehook\relax
         \parskip \cftparskip \@nameuse {cfttocbeforelisthook}}
        {\@nameuse {cfttocafterlisthook}\endgroup\restorefromonecol}}
%    \end{macrocode}
% placeholder for comments
%    \begin{macrocode}
\def\etocscrartclstyle{%
    \etocsettocstyle
        {\let\if@dynlist\if@tocleft
         \iftocfeature {toc}{onecolumn}
             {\iftocfeature {toc}{leveldown}
              {}
              {\if@twocolumn \aftergroup \twocolumn \onecolumn \fi }}
             {}%
         \tocbasic@listhead {\listoftocname}%
         \begingroup \expandafter \expandafter \expandafter 
         \endgroup \expandafter 
         \ifx 
             \csname microtypesetup\endcsname \relax 
         \else 
             \iftocfeature {toc}{noprotrusion}{}
                 {\microtypesetup {protrusion=false}%
                  \PackageInfo {tocbasic}%
                  {character protrusion at toc deactivated}}%
         \fi
         \setlength {\parskip }{\z@ }%
         \setlength {\parindent }{\z@ }%
         \setlength {\parfillskip }{\z@ \@plus 1fil}% 
         \csname tocbasic@@before@hook\endcsname
         \csname tb@toc@before@hook\endcsname}
        {\csname tb@toc@after@hook\endcsname 
         \csname tocbasic@@after@hook\endcsname}}
\let\etocscrbookstyle\etocscrartclstyle
\let\etocscrreprtstyle\etocscrartclstyle
%    \end{macrocode}
% placeholder for comments
%    \begin{macrocode}
\newcommand*\etocstandarddisplaystyle{\etocarticlestyle}
\newcommand*\etocmarkboth[1]{%
    \@mkboth{\MakeUppercase{#1}}{\MakeUppercase{#1}}}
\newcommand*\etocmarkbothnouc[1]{\@mkboth{#1}{#1}}
\newcommand\etoctocstyle[3][section]{\etocmulticolstyle[#2]%
    {\csname #1\endcsname *{#3}}}
\newcommand\etoctocstylewithmarks[4][section]{\etocmulticolstyle[#2]%
    {\csname #1\endcsname *{#3\etocmarkboth{#4}}}}
\newcommand\etoctocstylewithmarksnouc[4][section]{\etocmulticolstyle[#2]%
    {\csname #1\endcsname *{#3\etocmarkbothnouc{#4}}}}
%    \end{macrocode}
% placeholder for comments
%    \begin{macrocode}
\def\Etoc@redefetocstyle#1{%
    \renewcommand\etoctocstylewithmarks[4][#1]
    {\etocmulticolstyle[##2]%
          {\csname ##1\endcsname *{##3\etocmarkboth{##4}}}}
    \renewcommand\etoctocstylewithmarksnouc[4][#1]
    {\etocmulticolstyle[##2]%
          {\csname ##1\endcsname *{##3\etocmarkbothnouc{##4}}}}
    \renewcommand\etoctocstyle[3][#1]{%
     \etocmulticolstyle[##2]{\csname ##1\endcsname *{##3}}}}
\@ifclassloaded{scrartcl}
    {\renewcommand*\etocstandarddisplaystyle{\etocscrartclstyle}}{}
\@ifclassloaded{book}
    {\renewcommand*\etocfontone{\normalfont\normalsize}
     \renewcommand*\etocstandarddisplaystyle{\etocbookstyle}
     \Etoc@redefetocstyle{chapter}}{}
\@ifclassloaded{report}
    {\renewcommand*\etocfontone{\normalfont\normalsize}
     \renewcommand*\etocstandarddisplaystyle{\etocreportstyle}
     \Etoc@redefetocstyle{chapter}}{}
\@ifclassloaded{scrbook}
    {\renewcommand*\etocfontone{\normalfont\normalsize}
     \renewcommand*\etocstandarddisplaystyle{\etocscrbookstyle}
     \Etoc@redefetocstyle{chapter}}{}
\@ifclassloaded{scrreprt}
    {\renewcommand*\etocfontone{\normalfont\normalsize}
     \renewcommand*\etocstandarddisplaystyle{\etocscrreprtstyle}
     \Etoc@redefetocstyle{chapter}}{}
\@ifclassloaded{memoir}
    {\renewcommand*\etocfontone{\normalfont\normalsize}
     \etocmemoirtoctotocfmt{chapter}{\contentsname}%
     \renewcommand*\etocstandarddisplaystyle{\etocmemoirstyle}
     \Etoc@redefetocstyle{chapter}}{}
%    \end{macrocode}
% placeholder for comments
%    \begin{macrocode}
\def\Etoc@addtocontents #1#2%
    {\ifEtoc@hyperref
       \addtocontents {toc}{\protect\contentsline 
                {#1}{#2}%
                {\thepage }{\@currentHref }}%
     \else
       \addtocontents {toc}{\protect\contentsline 
          {#1}{#2}{\thepage }}%
     \fi}
\def\Etoc@addcontentsline@ #1#2#3%
    {\@namedef{toclevel@#1}{#3}%
     \addcontentsline {toc}{#1}{#2}}
\DeclareRobustCommand*{\etoctoccontentsline}
    {\@ifstar{\Etoc@addcontentsline@}{\Etoc@addtocontents}}
\newcommand*\etocstandardlines{\Etoc@standardtrue}
\newcommand*\etoctoclines{\Etoc@standardfalse} % 1.07b
\etocdefaultlines  % for initialization
\etocstandardlines 
\etocstandarddisplaystyle
\endinput
%    \end{macrocode}
% \MakePercentComment
\CharacterTable
 {Upper-case    \A\B\C\D\E\F\G\H\I\J\K\L\M\N\O\P\Q\R\S\T\U\V\W\X\Y\Z
  Lower-case    \a\b\c\d\e\f\g\h\i\j\k\l\m\n\o\p\q\r\s\t\u\v\w\x\y\z
  Digits        \0\1\2\3\4\5\6\7\8\9
  Exclamation   \!     Double quote  \"     Hash (number) \#
  Dollar        \$     Percent       \%     Ampersand     \&
  Acute accent  \'     Left paren    \(     Right paren   \)
  Asterisk      \*     Plus          \+     Comma         \,
  Minus         \-     Point         \.     Solidus       \/
  Colon         \:     Semicolon     \;     Less than     \<
  Equals        \=     Greater than  \>     Question mark \?
  Commercial at \@     Left bracket  \[     Backslash     \\
  Right bracket \]     Circumflex    \^     Underscore    \_
  Grave accent  \`     Left brace    \{     Vertical bar  \|
  Right brace   \}     Tilde         \~}

\CheckSum{2622}

\Finale
%%
%% End of file `etoc.dtx'.

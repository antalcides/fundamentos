%% LyX 2.0.4 created this file.  For more info, see http://www.lyx.org/.
%% Do not edit unless you really know what you are doing.
\documentclass[oneside,svgnames,x11names,x11names,HTML]{book}
\usepackage[utf8x]{inputenc}
\pagestyle{headings}
\setcounter{secnumdepth}{3}
\setcounter{tocdepth}{3}
\usepackage{float}
\usepackage{amstext}

\makeatletter

%%%%%%%%%%%%%%%%%%%%%%%%%%%%%% LyX specific LaTeX commands.
%% Because html converters don't know tabularnewline
\providecommand{\tabularnewline}{\\}

%%%%%%%%%%%%%%%%%%%%%%%%%%%%%% User specified LaTeX commands.
\usepackage{marco}
\usepackage{lettrine}
%\usepackage{keyval}% http://ctan.org/pkg/keyval
%\usepackage{environ}% http://ctan.org/pkg/environ
\usetikzlibrary{calc,trees,positioning,arrows,chains,shapes.geometric,
decorations.pathreplacing,decorations.pathmorphing,shapes,%
matrix,shapes.symbols,plotmarks,decorations.markings,shadows}
\usepackage{theoremref} %refereciar teoremas Por ejemplo: \thlabel{foobar}
\usepackage{marginnote}
\usepackage{fancybox}
\usepackage{hhline}
\usepackage{multirow} 
\usepackage{colortbl}%
\usepackage{pifont} 
\usepackage{eurosym} %para el Euro
%\usepackage{xltxtra}%para logos de la familia TeX
%\usepackage{mathspec}
\usepackage{metalogo}
%\usepackage{tabular}
%\usepackage{amsmath}
\usepackage{lipsum-es}
%\usepackage{xparse}% para declarar comandos
\usepackage{amssymb}
%\usepackage[thref,thmmarks,framed, amsthm]{ntheorem}
\usepackage{comment}
\usepackage[explicit]{titlesec}
\usepackage{emptypage}%pagina en blanco al final de capitulo
%%%%%%%%%%%%%%%%%%%%%%%%%%%%%%
%\documentclass[openany,svgnames,x11names]{book}
\usepackage{titletoc}
\usepackage{fancyhdr}
\usepackage{pagecolor}
\usepackage[spanish]{layout}
\usepackage{ucs}%codificacion vieja
\usepackage[utf8x]{inputenc}
%\usepackage[latin1]{inputenc}
%\usepackage{showframe}% traza layout en cada pagina
%%%======================================revisar
\usepackage{pifont}
\usetikzlibrary{shapes,snakes,positioning}
\pgfdeclarelayer{background}
\pgfdeclarelayer{foreground}
\pgfsetlayers{background,main,foreground}
\usepackage{wallpaper}
\usetikzlibrary{calc}
\usetikzlibrary{arrows}
\usepackage{epstopdf}
\usepackage{floatflt}
\usepackage{pgfplots}
\usepackage{setspace}
%%%%%%%%%%%%%%%%
\usepackage{extarrows}
%\usepackage{fixltx2e}
\usepackage{everyshi}% http://ctan.org/pkg/everyshi
%%%%%%%%%%%%%%%%%
\usepackage{amsmath,amssymb}
\usepackage{float}
\makeatletter
\let\@tmp\@xfloat     
\usepackage{fixltx2e}
\let\@xfloat\@tmp                    
\makeatother
%\usepackage{booktabs}
\usepackage{courier}
\usepackage{units}
\usepackage{url}
\usepackage{mathpazo}
\usepackage{amsfonts}
\usepackage{fancyvrb}
\usepackage{enumerate}
\usepackage{ifthen}
\usepackage{cancel}
\usepackage{layout}
\usepackage{footnote}
\usepackage{etex}%si pgfplot tiene problemas con new dim
\usepackage[frame,letter,cam]{crop}
\usepackage{microtype,soul,filecontents}
\usepackage{bbding}
\usepackage{lettrine,caption,multicol}
\usepackage{soul}
\usepackage{palatino}
\usepackage{calligra}
\usepackage[T1]{fontenc}
\usepackage[listings,theorems]{tcolorbox}
\usepackage{filecontents,ragged2e}
%\usepackage{floatflt}
\usepackage[makeindex]{imakeidx}
\usepackage{lmodern}
\usepackage{etoolbox}
\usepackage{tabularx}
%\usepackage{minitoc}
\usepackage{etoc}
%%%%%%%%%%%%%%%%%%%%%%%%%%%%%
\usepackage[paperheight=27.9cm,%
paperwidth=20.6cm,%
%centering,%
%textheight=22.9cm,
left=2.5cm,%
right=2.5cm,%
top=2.5cm,%
bottom=2cm,%
headheight=1cm,%
headsep=20pt,%
footskip=1cm,%
marginparsep=20pt,%
pdftex=false,%
letterpaper%
]{geometry}
%\usepackage[paperheight=25cm,%
%paperwidth=17cm,%
%centering,%
%left=1.5cm,%
%right=2cm,%
%top=2.5cm,%
%bottom=1.5cm,%
%headheight=0.5cm,%
%headsep=10pt,%
%%footskip=1cm,%
%marginparsep=20pt,
%margin=2cm,
%pdftex=false
%]{geometry}
%\usepackage[frame,center,letter,pdflatex]{crop}
%\usepackage{amsthm}
%\usepackage[framed, amsthm]{ntheorem}
\usepackage{listings}
\definecolor{lightgrey}{rgb}{0.9,0.9,0.9}
\definecolor{darkgreen}{rgb}{0,0.6,0}
\usepackage{fourier-orns}
% % % % % % % % % % % % % % % % % % % % % % % % % % %
% % % %preamble de framed % % %
%\usepackage{etex}
\usepackage{lmodern}
\usepackage{textcomp}
\usepackage{array}
\usepackage{booktabs}
\usepackage{microtype}
\usepackage{subcaption}
\newcommand*{\mail}[1]{\href{mailto:#1}{\texttt{#1}}}
\newcommand*{\pkg}[1]{\textsf{#1}}
\newcommand*{\cs}[1]{\texttt{\textbackslash#1}}
\makeatletter
\newcommand*{\cmd}[1]{\cs{\expandafter\@gobble\string#1}}
\makeatother
\newcommand*{\env}[1]{\texttt{#1}}
\newcommand*{\opt}[1]{\texttt{#1}}
\newcommand*{\meta}[1]{\textlangle\textsl{#1}\textrangle}
\newcommand*{\marg}[1]{\texttt{\{}\meta{#1}\texttt{\}}}

% % % % % % % % % % % % % % % % %framed
\def\indexname{\'Indice}
\def\contentsname{ CONTENIDO}
\def\listfigurename{Tabla de figuras}
\def\bibname{Bibliograf\'{\i}a}
\def\tablename{Tabla}
\def\proofname{Demostraci\'on}
\def\appendixname{Ap\'endice}
\def\chaptername{ Cap\'{\i}tulo}
\def\figurename{Figura}
%%%%%%%%%%%%%%%%%%%%%%%%%%%%%%%%% definicion de colores ================)
\definecolor{est1}{RGB}{0,177,235}
\definecolor{est2}{RGB}{0,119,158}
\definecolor{est3}{RGB}{235,137,0}
\definecolor{est4}{RGB}{158,66,0}
\definecolor{est5}{RGB}{20,20,20}
\definecolor{est6}{RGB}{235,235,235}
\definecolor{naranja1}{rgb}{1,0.5,0}
\definecolor{naranja2}{RGB}{255,127,0}
\definecolor{naranja3}{cmyk}{0,0.5,1,0}
\definecolor{naranja4}{HTML}{FF7F00}
%rgb
\definecolor{rojo}{rgb}{1,0,0}
\definecolor{verde}{rgb}{0,1,0}
\definecolor{azul}{rgb}{0,0,1}
%cmyk
\definecolor{blanco}{cmyk}{0,0,0,0}
\definecolor{cian}{cmyk}{1,0,0,0}
\definecolor{magenta}{cmyk}{0,1,0,0}
\definecolor{amarillo}{cmyk}{0,0,1,0}
\definecolor{negro}{cmyk}{0,0,0,1}
\definecolor{theblue}{rgb}{0.02,0.04,0.48}
\definecolor{thered}{rgb}{0.65,0.04,0.07}
\definecolor{thegreen}{rgb}{0.06,0.44,0.08}
\definecolor{thegrey}{gray}{0.5}
\definecolor{theshade}{gray}{0.94}
\definecolor{theframe}{gray}{0.75}
\definecolor{burl}{rgb}{0.27,0.22,0.20}
\definecolor{caper}{rgb}{0.36,0.46,0.23}
\definecolor{rhodo}{rgb}{0.58,0.63,0.45}
\definecolor{wood}{rgb}{0.61,0.51,0.43}
\definecolor{mesh}{rgb}{0.97,0.93,0.81}
\definecolor{wood}{rgb}{0.61,0.51,0.43}
\definecolor{warningColor}{named}{Red3}
\definecolor{doc}{RGB}{0,60,110}
\definecolor{boxheadcol}{gray}{.6}
\definecolor{boxcol}{gray}{.9}
\definecolor[named]{PowderBlue}{HTML}{B0E0E6}
\definecolor[named]{MidnightBlue}{HTML}{191970}
\definecolor{bl}{rgb}{0,0.2,0.8}
\definecolor{shcolor}{HTML}{FDEDD0}
\definecolor[named]{GreenTea}{HTML}{CAE8A2}
\definecolor[named]{MilkTea}{HTML}{C5A16F}
\definecolor[named]{SaddleBrown}{HTML}{8B4513}
\definecolor{FrameColor}{rgb}{0.25,0.25,1.0}
\definecolor{TitleColor}{rgb}{1.0,1.0,1.0}
\definecolor{TFFrameColor}{HTML}{CAE8A2}
\definecolor{TFTitleColor}{HTML}{C5A16F}
\definecolor{secnum}{RGB}{13,151,225}
\definecolor{ptcbackground}{RGB}{212,237,252}
\definecolor{ptctitle}{RGB}{0,177,235}
\definecolor{shadecolor}{RGB}{212,237,252}
\definecolor{visgreen}{rgb}{0.733, 0.776, 0}
\definecolor{myBGcolor}{HTML}{F6F0D6}
\definecolor[named]{PowderBlue}{HTML}{B0E0E6}
\definecolor[named]{MidnightBlue}{HTML}{191970}
\definecolor{mybrown}{RGB}{128,64,0}
\definecolor{lightgrey}{rgb}{0.9,0.9,0.9}
\definecolor{darkgreen}{rgb}{0,0.6,0}
\definecolor{Tan}{cmyk}{0.14,0.42,0.56,0}
%%%%%%%%%%%%%%%%
%%%%% Definicion de listing===========
\usepackage{caption}
\DeclareCaptionFont{white}{\color{white}}
\DeclareCaptionFormat{listing}{\colorbox{gray}{\parbox{\dimexpr\textwidth-2\fboxsep\relax}{C\'odigo \thesection .\ 
\thesource\ #3}}}
\captionsetup[source]{format=listing,labelfont=white,textfont=white, singlelinecheck=false, margin=0pt, font={bf,footnotesize}}
\newcounter{source}[section]
\lstnewenvironment{source}[2][]
{\refstepcounter{source}
\captionsetup{options=source}
\lstset{%
basicstyle=\tiny\ttfamily\bf,language={[LaTeX]TeX},caption=#1,label=#2,  
numbersep=5mm, numbers=left, numberstyle=\tiny, % number style
breaklines=true,framexleftmargin=10mm, xleftmargin=10mm,
backgroundcolor=\color{ptcbackground!60},frameround=fttt,escapeinside=??,
rulecolor=\color{ptctitle},
morekeywords={% Give key words here                                         % keywords
    maketitle},
keywordstyle=\color[rgb]{0,0,1},                    % keywords
        commentstyle=\color[rgb]{0.133,0.545,0.133},    % comments
        stringstyle=\color[rgb]{0.627,0.126,0.941}  % strings
%columns=fullflexible   
}
        }
{}
%%%%================= definicion del capitulo ======================)
 \newcommand*\chapterlabel{}
\titleformat{\chapter}
  {\gdef\chapterlabel{}
   \normalfont\sffamily\Huge\bfseries\scshape}
  {\gdef\chapterlabel{\thechapter\ }}{0pt}
  {\begin{tikzpicture}[remember picture,overlay]
    \node[yshift=-3cm] at (current page.north west)
      {\begin{tikzpicture}[remember picture, overlay]
        \draw[fill=ptcbackground!60,draw=ptcbackground!60] (0,0) rectangle
          (\paperwidth,3cm);
          \draw[ultra thick,fill=ptctitle,draw=ptctitle](0,0) -- (current page.east |- 0,0 );
          \draw [ptctitle,fill=ptctitle, ultra thick] (0.5,0) circle [radius=0.1];
         \draw[ptctitle,fill=ptctitle, ultra thick] (21,0) circle [radius=0.1];
        \node[anchor=east,xshift=.9\paperwidth,rectangle,
              rounded corners=20pt,inner sep=11pt,
              fill=ptctitle,draw=ptctitle]
              {\color{white} \chapterlabel\protect#1};
%               \draw[fill=green] (current page.north west) rectangle (current page.south east);
       \end{tikzpicture}        
      }; 
      \begin{pgfonlayer}{background}
%          \path (-1.4cm,2.8cm) node (tl) {};
%          \path (2.3cm, -8.4cm) node (br) {};
          \path[fill=ptcbackground] (current page.north west) rectangle (current page.south east);
      \end{pgfonlayer}
   \end{tikzpicture}  
     \vspace{20pt}
  }
\titlespacing*{\chapter}{0pt}{50pt}{-60pt}

%%%======================================== TOC
\preto{\frontmatter}{\pagecolor{ptcbackground}}{}{}
\preto{\mainmatter}{\pagecolor{myBGcolor}}{}{}
\preto{\backmatter}{\pagestyle{empty}\pagecolor{myBGcolor}}{}{}{}
%\patchcmd{\backmatter}{\pagecolor{myBGcolor}\pagestyle{empty}}{}{}{}
%\preto{\tableofcontents}{\begin{snugshade*}}{}{}
%\appto{\tableofcontents}{\end{snugshade*}}{}{}
%\patchcmd{\tableofcontents}{\contentsname}{\color{ptctitle}\contentsname}{}{}


    %%%%%%%%%%%%%%%%%%%%%%%%%%%%%%%%%%
 \setcounter{tocdepth}{2}
    \titlecontents{subsection}
  [5.8em]{\sffamily}
  {\color{secnum}\contentslabel{2.3em}\normalcolor}{}
  {\titlerule*[1000pc]{.}\contentspage\\\hspace*{-5.8em}\vspace*{2pt}%
    \color{ptctitle}\rule{\dimexpr\textwidth-15.5pt\relax}{1pt}}

    %%%%%%%%%%%%%%%%%%%%%%%%%%%%%%%
    \titlecontents{section}
  [4em]{\sffamily}
  {\color{secnum}\contentslabel{2.3em}\normalcolor}{}
  {\titlerule*[1000pc]{.}\contentspage\\\hspace*{-3em}\vspace*{2pt}%
    \color{ptctitle}\rule{\dimexpr\textwidth-20pt\relax}{1pt}}

\titlecontents{lsection}
  [5.8em]{\sffamily}
  {\color{secnum}\contentslabel{2.3em}\normalcolor}{}
  {\titlerule*[1000pc]{.}\contentspage\\\hspace*{-5.8em}\vspace*{2pt}%
    \color{ptctitle}\rule{\dimexpr\textwidth-15.5pt\relax}{1pt}}

\makeatletter
%%%%%%%%%%%%%%%%%%%%%%%%%%%%%%%%%
\newcommand\HUGE{\@setfontsize\Huge{38}{47}}
\newcommand\HHUGE{\@setfontsize\HHUGE{58}{67}}
\newcommand\peque{\@setfontsize\peque{8}{9}}
%%%%%%%%%%%%%%%%%%%%%%%%%%%%%%
\renewcommand*\l@chapter[2]{%
\thispagestyle{empty}
  \ifnum \c@tocdepth >\m@ne
    \addpenalty{-\@highpenalty}%
    \vskip 1.0em \@plus\p@
    \setlength\@tempdima{1.5em}%
    \begingroup
      \parindent \z@ \rightskip \@pnumwidth
      \parfillskip -\@pnumwidth
      \leavevmode
      \advance\leftskip\@tempdima
      \hskip -\leftskip
      \colorbox{ptctitle}{\strut%
        \makebox[\dimexpr\textwidth-2\fboxsep-7pt\relax][l]{%
          \color{white}\bfseries\sffamily\protect#1%
          \nobreak\hfill\nobreak\hb@xt@\@pnumwidth{\hss #2}}}\par\smallskip
      \penalty\@highpenalty
    \endgroup
  \fi}
\makeatother
%%% crear toc por capitulo con etoc 
\newcommand*\chaptertoc{% 
  \setcounter{tocdepth}{2}% 
  \etocsettocstyle{\subsection*{\subtoc}}{}% 
  {\footnotesize \localtableofcontents }
} 
\def\subtoc{\colorbox{ptctitle}{
\renewcommand{\baselinestretch}{1}
     \parbox[t]{\dimexpr\textwidth-2\fboxsep\relax}{%
    \strut\color{white}\bfseries\sffamily \makebox[5em]{%
Contenido      }\hfill Cap\'{i}tulo~\thechapter\hfill P\'agina}
}}
%%% crear toc por capitulo con titlesec
\newcommand\PartialToC{%
\startcontents[chapters]%
\begin{mdframed}[backgroundcolor=ptcbackground,hidealllines=true]
\printcontents[chapters]{l}{1}{\colorbox{ptctitle}{%
  \parbox[t]{\dimexpr\textwidth-2\fboxsep\relax}{%
    \strut\color{white}\bfseries\sffamily \makebox[5em]{%
Contenido      }\hfill Cap\'{i}tulo~\thechapter\hfill P\'agina}}\vskip5pt}
\end{mdframed}%
}
% Define partial toc for part pages
%% Set the uniform width of the colour box
%% displaying the page number in footer
%% to the width of "99"
\newlength\pagenumwidth
\settowidth{\pagenumwidth}{99}

%% Define style of page number colour box
\tikzset{pagefooter/.style={
anchor=base,font=\sffamily\bfseries\small,
text=white,fill=ptctitle,text centered,
text depth=17mm,text width=\pagenumwidth}}

%% Concoct some colours of our own
\definecolor[named]{GreenTea}{HTML}{CAE8A2}
\definecolor[named]{MilkTea}{HTML}{C5A16F}
%%%%%%%%%%%%%%% Encabezado y pie de pagina
%%%%%%%%%%
%%% Re-define running headers on non-chapter pages
%%%%%%%%%%
\fancypagestyle{headings}{%
  \fancyhf{}   % Clear all headers and footers first
  %% Right headers on odd pages
  \fancyhead[RO]{%
    %% First draw the background rectangles
    \begin{tikzpicture}[remember picture,overlay]
    \fill[ptcbackground] (current page.north east) rectangle (current page.south west);
    \fill[white, rounded corners] ([xshift=-10mm,yshift=-20mm]current page.north east) rectangle ([xshift=15mm,yshift=17mm]current page.south west);
    \begin{pgfonlayer}{background}
    %          \path (-1.4cm,2.8cm) node (tl) {};
    %          \path (2.3cm, -8.4cm) node (br) {};
              \path[fill=brown!20] (current page.north west) rectangle (current page.south east);
          \end{pgfonlayer}
    \end{tikzpicture}
    %% Then the decorative line and the right mark
    \begin{tikzpicture}[xshift=-.75\baselineskip,yshift=.25\baselineskip,remember picture,    overlay,fill=ptctitle,draw=ptctitle]\fill circle(3pt);
    \draw[semithick](0,0) -- (current page.west |- 0,0);
        \end{tikzpicture} \sffamily\itshape\small\protect\nouppercase{\rightmark}
  }

  %% Left headers on even pages
  \fancyhead[LE]{%
    %% Background rectangles first
    \begin{tikzpicture}[remember picture,overlay]
     \fill[brown!20] (current page.north east) rectangle (current page.south west);
    \fill[ptcbackground] (current page.north east) rectangle (current page.south west);
    \fill[white, rounded corners] ([xshift=-15mm,yshift=-20mm]current page.north east) rectangle ([xshift=10mm,yshift=17mm]current page.south west);
           \end{tikzpicture}
    %% Then the right mark and the decorative line
    \sffamily\itshape\small\protect\nouppercase{\leftmark}\ 
    \begin{tikzpicture}[xshift=.5\baselineskip,yshift=.25\baselineskip,remember picture, overlay,fill=ptctitle,draw=ptctitle]
    \fill (0,0) circle (3pt); \draw[semithick](0,0) -- (current page.east |- 0,0 );
       \end{tikzpicture}
  }

  %% Right footers on odd pages and left footers on even pages,
  %% display the page number in a colour box
  \fancyfoot[RO,LE]{\tikz[baseline]\node[pagefooter]{\thepage};}
   \fancyfoot[CO,CE]{\tikz\node{\color{ptctitle}Antalcides Olivo};}
  \renewcommand{\headrulewidth}{0pt}
  \renewcommand{\footrulewidth}{0pt}
}

%%%%%%%%%%
%%% Re-define running headers on chapter pages
%%%%%%%%%%
\fancypagestyle{plain}{%
  %% Clear all headers and footers
  \fancyhf{}
  %% Right footers on odd pages and left footers on even pages,
  %% display the page number in a colour box
  \fancyfoot[RO,LE]{\tikz[baseline]\node[pagefooter]{\thepage};}
 
  
    %% First draw the background rectangles
     \fancyhead[LE]{\begin{tikzpicture}[remember picture,overlay]
    \fill[ptcbackground] (current page.north east) rectangle (current page.south west);
    \fill[white, rounded corners] ([xshift=-10mm,yshift=-20mm]current page.north east) rectangle ([xshift=15mm,yshift=17mm]current page.south west);
    \begin{pgfonlayer}{background}
    %          \path (-1.4cm,2.8cm) node (tl) {};
    %          \path (2.3cm, -8.4cm) node (br) {};
              \path[fill=ptcbackground] (current page.north west) rectangle (current page.south east);
          \end{pgfonlayer}
    \end{tikzpicture}
    \sffamily\itshape\small\protect\nouppercase{\rightmark}
  }
    \fancyhead[RO]{%
    %% First draw the background rectangles
    \begin{tikzpicture}[remember picture,overlay]
    \fill[ptcbackground] (current page.north east) rectangle (current page.south west);
    \fill[white, rounded corners] ([xshift=-10mm,yshift=-20mm]current page.north east) rectangle ([xshift=15mm,yshift=17mm]current page.south west);
    \begin{pgfonlayer}{background}
    %          \path (-1.4cm,2.8cm) node (tl) {};
    %          \path (2.3cm, -8.4cm) node (br) {};
              \path[fill=ptcbackground] (current page.north west) rectangle (current page.south east);
          \end{pgfonlayer}
    \end{tikzpicture}
    \sffamily\itshape\small\protect\nouppercase{\rightmark}
  }
  \renewcommand{\headrulewidth}{0pt}
  \renewcommand{\footrulewidth}{0pt}
}
%%%%% def de seccion
 \usetikzlibrary{shapes.symbols,shadows,calc}
% the tikz picture that will be used for the title formatting
% \SecTitle{<signal direction>}{<node anchor>}{<node horiz, shift>}{<node x position>}{#5}
% the fifth argument will be used by \titleformat to write the section title using #1
\newcommand\SecTitle[5]{%
\begin{tikzpicture}[overlay,every node/.style={signal, draw, text=white, signal to=nowhere}]
  \node[ptctitle,fill, signal to=#1, inner sep=1em, drop shadow,
    text=white,font=\huge\sffamily,anchor=#2,
    xshift=\the\dimexpr-\marginparwidth-\marginparsep-#3\relax] 
    at (#4,0) {#5};
\end{tikzpicture}%
}

\titleformat{name=\section,page=even}
{\normalfont}{}{20pt}
{\SecTitle{east}{west}{16pt}{5cm}{\thesection\ #1}}[\addvspace{20pt}]

\titleformat{name=\section,page=odd}
{\normalfont\sffamily}{}{0em}
{\SecTitle{west}{east}{16pt}{\paperwidth}{#1\  \thesection}}[\addvspace{20pt}]
  %%%%%8============ Final de tabla de contenido ===========================)
% % % % % % % % % % % %bibname url
\usepackage{url}

%% Define a new 'leo' style for the package that will use a smaller font.
\makeatletter
\AtBeginDocument{%
\let\ref\autoref
\renewcommand\equationautorefname{\@gobble}
}
\def\url@leostyle{%
  \@ifundefined{selectfont}{\def\UrlFont{\sf}}{\def\UrlFont{\small\ttfamily}}}
\makeatother
%% Now actually use the newly defined style.
\urlstyle{leo}
% % % % % % % % % % % % % % % %
\usepackage{bodegraph}

\usetikzlibrary{intersections}
\usetikzlibrary{calc}
\usetikzlibrary{positioning}
% Define the layers to draw the diagram
\pgfdeclarelayer{background}
\pgfdeclarelayer{foreground}
\pgfsetlayers{background,main,foreground}


% % % % % % % % % % % % % % % % % %
\newenvironment{lista}{
\begin{itemize}
 \renewcommand{\labelitemi}{{
 \colorbox{wood!70!black}{\color{white}{\ding{42}}}
 }}
}{\end{itemize}}
\newenvironment{figura}[3]{\begin{figure}[H]
\centering
                               #1
                              \caption{#2}
                              \label{#3}
                              \end{figure}
}{ \vskip 5pt }
\newcommand{\nota}{\colorbox{teal!20!white}{\color{black}{Nota:}}\ }
\newcommand{\prop}{\colorbox{teal!20!white}{\color{black}{Proposición:}}\ }
\newcommand{\dem}{\colorbox{teal!20!white}{\color{black}{Demostraci\'on:}}\ }
\newcommand{\notacion}{\colorbox{teal!20!white}{\color{black}{Notaci\'on:}}\ }
\newcommand{\solucion}{\colorbox{teal!20!white}{\color{black}{Soluci\'on:}}\ }
\newcommand{\resp}{\colorbox{teal!20!white}{\color{black}{Respuesta:}}\ }
\def\texto{Sean $f$ y $g$ dos funciones y sean $\alpha$ y $\beta$ dos n\'umeros reales. 
Entonces se verifican las siguientes propiedades:

 \[1.\quad \int (f(x)+g(x))\,dx = \int f(x)\,dx + \int g(x)\,dx  \]
 \[2.\quad \int \alpha f(x)\, dx =\alpha \int f(x)\,dx \]

 Estas dos propiedades se pueden englobar en una:
 \[ \int (\alpha f(x)+\beta g(x)) \, dx = \alpha\int f(x)\,dx+\beta\int
   g(x)\,dx \]
{\bf Ejemplo}:

 \[ \int (2x-3x^2)\, dx = 2\int x\,dx -3\int x^2\, dx \]}
 \def\Web#1{\href{#1}{%
     \tikz \node[fill=myBGcolor](0,0) {#1};%
   }}
   \def\Item{\colorbox{wood!70!black}{\color{white}{\ding{42}}}}
   \def\web#1{\Item\ \href{http://ctan.org/pkg/#1}{\textbf{#1.}}\par\vspace{10pt}}
   % % % % % %listing
   \lstset{%
   basicstyle=\small\ttfamily\bf,language={[LaTeX]TeX}, numbersep=5mm, numbers=left, numberstyle=\tiny, % number style
   breaklines=true,framexleftmargin=10mm, xleftmargin=10mm,
   backgroundcolor=\color{ptcbackground!60},frameround=fttt,escapeinside=??,
   rulecolor=\color{ptctitle},
   morekeywords={% Give key words here                                         % keywords
       maketitle},
   keywordstyle=\color[rgb]{0,0,1},                    % keywords
           commentstyle=\color[rgb]{0.133,0.545,0.133},    % comments
           stringstyle=\color[rgb]{0.627,0.126,0.941}  % strings
   %columns=fullflexible   
   }
   % % % % % %
%%%%%% Definicion de caja %%%%%%%%%%%%%%%%%%%%%
%  \newboxedtheorem[title= Teorema. \thesection.\thecaja ,labelbox= ,boxcolor=MilkTea,background = ptcbackground!60,titleboxcolor=black,titleboxcolor=MilkTea,titlebackground=ptctitle]{caja}{Teorema}
%  % % % % % % % %
%  \newboxedtheorem[title=Lemma.\ \thecaja ,labelbox= ,boxcolor=MilkTea,background = ptcbackground!60,titleboxcolor=black,titleboxcolor=MilkTea,titlebackground=ptctitle]{cajo}{Teorema}
  % % % % % % % % % % % % % %
  \nboxedtheorem[boxcolor=MilkTea,background = ptcbackground!60,titleboxcolor=black,titleboxcolor=MilkTea,titlebackground=ptctitle]{ncaja}{Postulado}
  % % % % % % % % % % % % % % %
 \tipptheorem[tipplogo=interrogacion,boxheadcol=MidnightBlue,boxcol=PowderBlue]{notas}{Nota}
 % % % % % % % % % % % %
 \notatheorem[tipplogo=pregunta,boxheadcol=MidnightBlue,boxcol=PowderBlue]{obs}{Observaci\'on}
 %%%%%%%%%%%%%%
 \frametheorem[]{ejemplo}{Ejemplo}
 %%%%%%%%%%%%%%
 \beamertheorem[]{beamercaja}{Estilo Beamer}
 %%%%%%%%%%%%%%
 \framedtheorem[]{frameth}{Fancy}
 %%%%%%%%%%%%%%%%%%%
 \xcolortheorem[background=mybrown!5 ,titlebackground=mybrown!40!black ,titleboxcolor=mybrown!40!black ,boxcolor=mybrown!40!black]{geo}{Ejemplo}
 %%%%%%%%%%%%%%
%  \warningtheorem[textcol=black, boxheadcol=gray!80, boxcol=ptctitle, tipplogo=icon-tipp, texttcolor=black ,labeltext=, size=0.8\textwidth, iconline=red ]{xcolorth}{Fancy}
  %%%%%%%%%%tcolorbox%%%
  \newcounter{postulado}
\newenvironment{postulado}[2]{\vskip 5pt
\refstepcounter{postulado}
    \begin{tcolorbox}[colback=mybrown!5,colframe=mybrown!40!black,title=Postulado.\thechapter.\thepostulado  \ \bf{#2}]
 #1
\end{tcolorbox}\index{Postulado!#2}            }{

                \vskip 5pt
 }
  %%%%%%%%%%%%%%%
  %%%<
\newcommand{\cdefault}[4][named]{\begin{tikzpicture}
\fill[#2,draw=negro] (0,0) rectangle ++(2,1);
\node[below] at (1,0) {#2};
\node[below=4mm] at (1,0) {\tiny #3 \{#4\}};
\node[below=6mm] at (1,0) {\tiny #1};
\end{tikzpicture}}
%%%%%%%%%%%%%>
%  \lstnewenvironment{javacode}[2]
%{\singlespacing\lstset{language=java, label=#1, caption=#2}}
%{}
%%%%%%%%%%%%%%%cambio de margen
\newenvironment{changemargin}[5]
{
\begin{list}{}
{
\global\setlength{\textheight}{#1}%
  \global\setlength{\textwidth}{#2}
\setlength{\topsep}{0pt}
\setlength{\evensidemargin}{0pt}%
\setlength{\oddsidemargin}{0pt}
\setlength{\leftmargin}{#3}%
\setlength{\rightmargin}{#4}%
\setlength{\listparindent}{\parindent}%
\setlength{\itemindent}{\parindent}%
\setlength{\parsep}{\parskip}%
\hoffset #5
}
\item[]
}
{\end{list}}
%%%%%%%%%cambiamargen%%%%%%%%%%%%%%%
\newenvironment{cambiamargen}[5]
{
\begin{list}{}
{
\global\setlength{\textheight}{#1}%
 \setlength{\topmargin}{#2}
\setlength{\evensidemargin}{0pt}%
\setlength{\oddsidemargin}{0pt}
\setlength{\leftmargin}{-}%
\setlength{\rightmargin}{#4}%
\setlength{\listparindent}{\parindent}%
\setlength{\itemindent}{\parindent}%
\setlength{\parsep}{\parskip}%
\hoffset #5
}
\item[]
}
{\end{list}}
\newenvironment{dems}[1]{ \dem
\it #1  }{\hfill$\square$\vspace*{5pt}}
\newcommand{\oper}[1]{\tikz\draw(0,0)  node[draw,circle,inner sep=0pt,minimum size=1pt]{#1} ;}

%%%%%%%%%%%%%%%%%%%%%%%%%%%%
%%%%%%%%%%%%%%%%%%%%%%%%%%%%%%%%%%%%%%%%%
%%%%%%%%%%%%%%%%%%%%%%%%%%%%%%%%%
%%%%%%%%%%%%%%%%%%%%%%%%%%%%%%%%%%%%%%%%%%
\tikzstyle{mybox} = [draw=red!40!black, fill=mybrown!10, very thick,
    rectangle, rounded corners, inner sep=10pt, inner ysep=20pt]
\tikzstyle{fancytitle} =[fill=red!40!black, text=white]
\NewEnviron{competencias}{\vskip 5pt
\begin{tikzpicture}

% First box
\node [mybox] (box1){%
    \begin{minipage}{0.92\textwidth}
      \BODY
    \end{minipage}
    };
    \node[fancytitle, rounded corners] at (box1.north) {\Large Objetivos};
\end{tikzpicture}
 }
%%%%%%%%%%%%%%%%%%%%%%
\NewEnviron{logros}{ \vskip 5pt
\begin{tikzpicture}

% First box
\node [mybox] (box1){%
    \begin{minipage}{0.92\textwidth}
      \BODY
    \end{minipage}
    };
    \node[fancytitle, rounded corners] at (box1.north) {\Large Indicadores de Logros
};
\end{tikzpicture}
 }
%%%%%%%%%%%%%%%%%%%%%%%%%%%%%%%%%%%%%%%%
%%%%%%%%%%%%%%%%%%%%%%%%%%%%%%%%
%\newcounter{ideas}
%%\newboxedtheorem[title=Definici\'on.\thesection . \thedefinicionn, labelbox=, boxcolor=caper!75!black  ,background=caper!5  ,titleboxcolor=caper!75!black  , titlebackground=caper!60 ]{definicionn}{Definici\'on}
%%\newboxedtheorem[title=T\'ermino.\thesection . \theideas, labelbox=, boxcolor=rhodo!75!black ,background=rhodo!5  ,titleboxcolor=rhodo!75!black, titlebackground=rhodo!60]{ideas}{T\'ermino}
%%%%%%%%%%%%%%%%%%%%%%%%%%%%%%%%
%%\newboxedtheorem[title=T\'ermino no defnido.\thesection . \thetndefinido, labelbox=, boxcolor=MilkTea ,background=ptcbackground!60  ,titleboxcolor=MilkTea , titlebackground=ptctitle]{tndefinido}{T\'ermno no definido}
%%%%%%%%%%%%%%%%%%%%%%%%%%%%%%
\newboxedtheorem[title=Definici\'on.\  \thedefinicionn , labelbox= ,boxcolor=caper!75!black,background =caper!5,titleboxcolor=caper!75!black,titlebackground=caper!60]{definicionn}%{Teorema}
  %%%%%%%%%%%%%%%
 \newboxedtheorem[title=T\'ermino.\ \thesection . \theideas, labelbox= ,boxcolor=doc!75!black,background =doc!5 ,titleboxcolor=caper!75!black,titlebackground=doc!60]{ideas}%{Teorema}
 %%%%%%%%%%%%%%
  % % % % % % % %
  \newboxedtheorem[title=T\'ermino no defnido. \thesection . \thetndefinido, labelbox= ,boxcolor=MilkTea,background = ptcbackground!60,titleboxcolor=MilkTea,titlebackground=ptctitle]{tndefinido}%{T\'erminos no definidos}
  % % % % % % % %
  \newboxedtheorem[title=Ejemplo. \thesection . \theejem, labelbox= ,boxcolor=ptctitle!75!black,background = ptcbackground!60,titleboxcolor=ptctitle!60!black,titlebackground=ptctitle]{ejem}%{T\'erminos no definidos}
  %%%%%%%%%%%%%%%%%%%
 
 \newboxedtheorem[title=Ejemplos. \thesection . \theejems, labelbox= ,boxcolor=ptctitle!75!black,background = ptcbackground!60,titleboxcolor=ptctitle!60!black,titlebackground=ptctitle]{ejems}%{T\'erminos no definidos}
 
%%%%%%%%%%%%%%%%%%%%%%%  
 
% \newcounter{axioma}
%\newenvironment{axioma}[2]{\vskip 5pt
%\refstepcounter{axioma}
%    \begin{tcolorbox}[colback=mybrown!5,colframe=mybrown!40!black,title=Axioma.\thechapter.\thepostulado  \ \bf{#2}]
% #1
%\end{tcolorbox}\index{Axioma!#2}            }{
%
%                \vskip 5pt
% }
 \newboxedtheorem[title=Axioma.\ \theaxioma, labelbox= ,boxcolor=mybrown!40!black,background =mybrown!5 ,titleboxcolor=mybrown!75!black,titlebackground=mybrown!60]{axioma}%{Teorema}
 \newboxedtheorem[title=Leyes de inferencia.\ \thesection . \theley, labelbox= ,boxcolor=doc!75!black,background =doc!5 ,titleboxcolor=caper!75!black,titlebackground=doc!60]{ley}%{Teorema}
 \newboxedtheorem[title=Lema.\ \thesection . \thelema, labelbox= ,boxcolor=doc!75!black,background =doc!5 ,titleboxcolor=caper!75!black,titlebackground=doc!60]{lema}%{Teorema}
 \newboxedtheorem[title=Teorema.\  \theteorema, labelbox= ,boxcolor=mybrown!40!black,background =mybrown!5 ,titleboxcolor=mybrown!75!black,titlebackground=mybrown!60]{teorema}%{Teorema}
  \newboxedtheorem[title=Operaci\'on\ \thesection . \theoperacion, labelbox= ,boxcolor=doc!75!black,background =doc!5 ,titleboxcolor=caper!75!black,titlebackground=doc!60]{operacion}%{Teorema}
  \newboxedtheorem[title=Proposición.\ \thesection . \theproposicion, labelbox= ,boxcolor=mybrown!40!black,background =mybrown!5 ,titleboxcolor=mybrown!75!black,titlebackground=mybrown!60]{proposicion}%{Teorema}
  %%%%%%%%%%%% conjuntos num\'ericos
  \newcommand{\CC}{\mathbb{C}}
%\newcommand{\NN}{\mathbb{N}}
\newcommand{\PP}{\mathbb{P}}
\newcommand{\HH}{\mathbb{H}}
\newcommand{\pp}{\mathbb{\overline{P}}}
\newcommand{\DD}{\mathbb{D}}
\newcommand{\QQ}{\mathbb{Q}}
\newcommand{\RR}{\mathbb{R}}
\newcommand{\ZZ}{\mathbb{Z}}
\newcommand{\EE}{\mathbb{E}}
\newcommand{\TT}{\mathbb{T}}
\newcommand{\XX}{\mathbb{X}}
\newcommand{\der}{\mathcal{D}}
\newcommand{\kk}{\mathcal{K}}
\newcommand{\mm}[1]{{\mathcal{M}\/}(#1)}
\newcommand{\nequiv}{{\equiv \hspace*{-3.7mm}/}}
\newcommand{\capa}[1]{\mbox{{\em Cap\/}}(#1)}
\newcommand{\gra}[1]{\mbox{{\em grad\/}}(#1)}
\newcommand{\sop}[1]{\mbox{{\em supp\/}}(#1)}
\newcommand{\esup}[1]{{\mbox{\rm ess\,sup\/}}#1}
\newcommand{\lqqd}{\hfill $\blacksquare$}
\newcommand{\II}{\mathbb{I}}
%\newcommand{\na}{I\! N}
%\newcommand{\re}{I\! R}
{\makeatletter   % dies ist lokal, damit man die Datei wahlweise
                 % mit \input oder mit \documentstyle[...] einlesen kann.
 
\gdef\re{\relax\ifmmode I\hskip -3\p@ R\else
    \hbox{$I\hskip -3\p@ R$}\fi} %Reales
\gdef\na{\relax\ifmmode I\hskip -2.7\p@ N\else
    \hbox{$I\hskip -2.7\p@ N$}\fi} % Naturales
\gdef\en{\relax\ifmmode Z\hskip -4.8\p@ Z\else
    \hbox{$Z\hskip -4.8\p@ Z$}\fi} %Enteros
\gdef\co{\relax\ifmmode C\hskip-4.8\p@\vrule \@height 5.8\p@ \@depth\z@
    \hskip 6.3\p@\else
    \hbox{$C\hskip-4.8\p@\vrule \@height 5.8\p@ \@depth\z@ \hskip 6.3\p@$}\fi}% Complejos
\gdef\ra{\relax\ifmmode Q\hskip-5.0\p@\vrule
          \@height 6.0\p@ \@depth \z@ \hskip 6\p@\else
    \hbox{$Q\hskip-5.0\p@\vrule \@height 6.0\p@ \@depth \z@ \hskip 6\p@$}\fi}% Racionales
\gdef\hz{\relax\ifmmode I\hskip -3\p@ H\else
    \hbox{$I\hskip -3\p@ H$}\fi} % Espacio de Hilbert
\gdef\kz{\relax\ifmmode I\hskip -3\p@ K\else
    \hbox{$I\hskip -3\p@ K$}\fi} % campo K
    \gdef\ir{\relax\ifmmode I\hskip -3\p@ I\else
    \hbox{$I\hskip -3\p@ K$}\fi} % Irracionales}
    %%%%%%%%%%%%%%%%%%%%%%%%%
   % \renewcommand\p@axioma{\thesection.\theaxioma}
   \makeatother
\usepackage{parskip}
%%%%%%%%%%%%%%%%%%%%%%%%%%%%%%%
% \makeatletter
%  \renewcommand\thesection{\arabic{chapter}.\arabic{section}}
%\renewcommand\thesubsection{\arabic{chapter}.\arabic{section}.\arabic{subsection}}
%\renewcommand\p@subsection{\thesection.}
%\renewcommand\p@section{\thechapter.}
%\gdef\label#1{\@bsphack
%  \protected@write\@auxout{}%
%         {\string\newlabel{sec@#1}{{\thechapter  .\@currentlabel}{\thepage}}}%
%  \@esphack}
   \newcommand*{\fullref}[1]{\ref{#1} en la p\'agina~\pageref{#1}}
   %%%%%%%%%%%%%%%%%%%%%%%%%%%%%%%\def\firstcircle{(0,0) circle (1.5cm)} 
\def\secondcircle{(45:2cm) circle (1.5cm)}
\def\thirdcircle{(0:2cm) circle (1.5cm)} 
\def\rectangulo{(-2,-2) rectangle (4,3.6)}
\def\fourcircle{(1.1,1) circle (1.5cm)}
\def\fivecircle{(-2,1) circle (1.5cm)}
\def\rect{(-4,-2) rectangle (4,3.6)} 
\def\sixcircle{(-2.3,1) circle (0.8cm)} 
\def\sevencircle{(1.1,1) circle (0.8cm)}
\def\myref#1{\thesection .\ref{#1}}
 \renewcommand\theaxioma{\arabic{chapter}.\arabic{section}.\arabic{axioma}}
\renewcommand\thedefinicionn{\arabic{chapter}.\arabic{section}.\arabic{definicionn}}
\renewcommand\theteorema{\arabic{chapter}.\arabic{section}.\arabic{teorema}}
  %%%%%%%%%%%%%%%%%%%%%%%%%%%%% Empieza el documento
 \makeatletter
%%%%%%%%%%%%%%%%%%%%%%%%
%original de ldesc2e.sty 
\newcommand{\manual}{Manual de \emph{\LaTeX{}}~\cite{manual}} 
\newcommand{\companion}{\emph{The \LaTeX{} Companion}~\cite{companion}} 
\newcommand{\guia}{\emph{Gu\'{\i}a Local}~\cite{local}}
\newcommand{\contrib}[3]{#1\quad$<$\texttt{#2}$>$%
{\small\\\quad\textit{#3}}\\[1ex]}
%
% Algunas instrucciones para ayudar a la creaci'on del 'indice de
% materias.
%
%\newcommand{\bs}{\symbol{'134}}%Print backslash
\ifx\bs\undefined
  \newcommand{\bs}{\symbol{92}}%Print backslash
\else
  \renewcommand{\bs}{\symbol{92}}%Print backslash
\fi
%\newcommand{\bs}{\ensuremath{\mathtt{\backslash}}}%Imprime barra invertida
% Entrada en el 'indice para una orden
%
% Entrada de s'imbolo para la tabla de s'imbolos matem'aticos
%
\newcommand{\X}[1]{$#1$&\texttt{\string#1}\hspace*{1ex}}
% Text normal.... 
\newcommand{\SC}[1]{#1&\texttt{\string#1}\hspace*{1ex}}
% para los acentos en modo texto
\newcommand{\A}[1]{#1&\texttt{\string#1}\hspace*{1ex}}
\newcommand{\B}[2]{#1#2&\texttt{\string#1{} #2}\hspace*{1ex}}

\newcommand{\W}[2]{$#1{#2}$&
  \texttt{\string#1}\texttt{\string{\string#2\string}}\hspace*{1ex}}
\newcommand{\Y}[1]{$\big#1$ &\texttt{\string#1}}  %
% Tabla de s'imbolos matem'aticos
\newsavebox{\symbbox}
\newenvironment{symbols}[1]%
{\par\vspace*{2ex}
\renewcommand{\arraystretch}{1.1}
\begin{lrbox}{\symbbox}
\hspace*{4ex}\begin{tabular}{@{}#1@{}}}%
{\end{tabular}\end{lrbox}\makebox[\textwidth]{\usebox{\symbbox}}\par\medskip}
%
% Preparaci'on especial para imprimir los s'imblos de la AMS
% Si no se encuentra AMS, deber'ia funcionar.
%

% No se tienen versiones PS de los tipos rsfs.
% Por ello, esto no no se puede hacer para pdf
\ifx\pdfoutput\undefined % No estamos corriendo pdftex
\IfFileExists{mathrsfs.sty}
  {\RequirePackage{mathrsfs}\let\MathRSFS\mathscr\let\mathscr\relax}{}
\fi
\IfFileExists{amssymb.sty}
  {\let\noAMS\relax \RequirePackage{amssymb}}
  {\def\noAMS{\endinput}\RequirePackage{latexsym}}

\IfFileExists{euscript.sty}
  {\RequirePackage{euscript}}{}
%\IfFileExists{eufrak.sty}
%  {\RequirePackage{eufrak}}{}


%
% Imprimir |--| para mostrar distancia
%
\newcommand{\demowidth}[1]{\rule{0.3pt}{1.3ex}\rule{#1}{0.3pt}\rule{0.3pt}{1.3ex}}
\renewcommand{\cleardoublepage}
    {\clearpage\if@twoside \ifodd\c@page\else
    \hbox{}\thispagestyle{empty}\newpage\if@twocolumn\hbox{}\newpage\fi\fi\fi}

\newcommand{\BibTeX}
     {\textsc{Bib}\TeX}
%%%%% hasta aqui original de ldesc2e.sty
%%%%%%%%%%%%%%%%%%%%%%%%%%%%%%%%%%%%%%%%%%%%%%
\xcolortheorem[background=Tan!5 ,titlebackground=thered!40!black ,titleboxcolor=mybrown!40!black ,boxcolor=mybrown!40!black]{out1}{Salida}
\DeclareCaptionFormat{ejer}{\colorbox{thered!40!black}{\parbox{\dimexpr\textwidth-2\fboxsep\relax}{\color{white}Ejercicio \thesection .\ 
\theejercicio\hfill #3}}}
\captionsetup[ejer]{format=ejer,labelfont=white,textfont=white, singlelinecheck=false, margin=0pt, font={bf,footnotesize}}
\newcounter{ejercicio}[section]
\newwrite\ejercicio@out
\newenvironment{ejercicio}%
 {\begingroup% Lets Keep the Changes Local
  \@bsphack
  \immediate\openout \ejercicio@out \jobname.exa
  \let\do\@makeother\dospecials\catcode`\^^M\active
  \def\verbatim@processline{%
    \immediate\write\ejercicio@out{\the\verbatim@line}}%
  \verbatim@start}%
 {\immediate\closeout\ejercicio@out\@esphack\endgroup%
%
% Y aqu'i lo que se ha a~nadido

 \setlength{\parindent}{0pt}
    %\setlength{\parskip}{1ex plus 0.5ex minus 0.7ex}
     \noindent
  %  \hspace*{+2ex}
%  \setlength{\parindent}{0pt}
    \setlength{\parskip}{1ex plus 0.3ex minus 0.7ex}
            \makebox[0.45\linewidth][t]{
            \hspace{+2ex}
            \setlength{\parindent}{0pt}
              \begin{minipage}[t]{0.45\linewidth}
              \vspace{-3.5ex}
      \refstepcounter{ejercicio}
   \captionsetup{options=ejer}
   \lstset{%
   caption=Entrada,basicstyle=\tiny\ttfamily\bf,language={[LaTeX]TeX}, numbersep=5mm, numbers=left, numberstyle=\tiny, % number style
   breaklines=true,framexleftmargin=10mm, xleftmargin=10mm,
   backgroundcolor=\color{Tan!5},frameround=fttt,escapeinside=??,
   rulecolor=\color{thered!40!black},
   morekeywords={% Give key words here                                         % keywords
       maketitle},
   keywordstyle=\color[rgb]{0,0,1},                    % keywords
           commentstyle=\color[rgb]{0.133,0.545,0.133},    % comments
           stringstyle=\color[rgb]{0.627,0.126,0.941}  % strings
   %columns=fullflexible   
   }
    \lstinputlisting[]{\jobname.exa}
         \end{minipage}}%
      \hspace{30pt} 
           \hfill
  %
   \setlength{\parindent}{0pt}
   \setlength{\parskip}{1ex plus 0.5ex minus 0.7ex}
%   
  \makebox[0.5\linewidth][t]{%
%  %\colorbox{ptcbackground!60}{
  \setlength{\parindent}{0pt}
    \setlength{\parskip}{1ex plus 0.3ex minus 0.7ex}
      \begin{minipage}[t]{0.40\linewidth}
         \setlength{\parindent}{0pt}
  \setlength{\parskip}{1ex plus 0.4ex minus 0.7ex}
  \vspace*{-3.5ex} 
        \begin{trivlist}
     \scriptsize\bf\ttfamily \item\begin{out1}{Pdf}\input{\jobname.exa}\end{out1}
    \end{trivlist}
      \end{minipage}%}
      }
    
      \par\addvspace{3ex plus 1ex}\vskip -\parskip
} 
%%%%%%%%%%%%%%%%%%%%%%%%%%%%%%%%%%%%%%%%%%%%%%%%%%%%
\DeclareCaptionFont{white}{\color{white}}
\newcounter{example}[section]
\DeclareCaptionFormat{out}{\colorbox{mybrown!40!black}{\parbox{\dimexpr\textwidth-2\fboxsep\relax}{\color{white}C\'odigo del ejemplo\, \thesection .\ 
\theexample\hfill #3}}}
\captionsetup[out]{format=out,labelfont=white,textfont=white, singlelinecheck=false, margin=0pt, font={bf,footnotesize}}
\xcolortheorem[background=mybrown!5 ,titlebackground=mybrown!40!black ,titleboxcolor=mybrown!40!black ,boxcolor=mybrown!40!black]{out}{Salida}
\newwrite\example@out
\newenvironment{example}%
 {\begingroup% Lets Keep the Changes Local
  \@bsphack
  \immediate\openout \example@out \jobname_ex.exa
  \let\do\@makeother\dospecials\catcode`\^^M\active
  \def\verbatim@processline{%
    \immediate\write\example@out{\the\verbatim@line}}%
  \verbatim@start}%
 {\immediate\closeout\example@out\@esphack\endgroup%
 % \par\small\addvspace{3ex plus 1ex}\vskip -\parskip
 \setlength{\parindent}{0pt}
    \setlength{\parskip}{1ex plus 0.5ex minus 0.7ex}
  \noindent
  \makebox[0.45\linewidth][t]{%
  \hspace*{+2ex}
  \setlength{\parindent}{0pt}
    \setlength{\parskip}{1ex plus 0.3ex minus 0.7ex}
  \begin{minipage}[t]{0.45\linewidth}
   % \vspace*{-1ex}
   \vspace*{-3.5ex}
   \refstepcounter{example}
   \captionsetup{options=out}
   \lstset{%
   caption=Entrada,basicstyle=\tiny\ttfamily\bf,language={[LaTeX]TeX}, numbersep=5mm, numbers=left, numberstyle=\tiny, % number style
   breaklines=true,framexleftmargin=10mm, xleftmargin=10mm,
   backgroundcolor=\color{mybrown!5},frameround=fttt,escapeinside=??,
   rulecolor=\color{mybrown!40!black},
   morekeywords={% Give key words here                                         % keywords
       maketitle},
   keywordstyle=\color[rgb]{0,0,1},                    % keywords
           commentstyle=\color[rgb]{0.133,0.545,0.133},    % comments
           stringstyle=\color[rgb]{0.627,0.126,0.941}  % strings
   %columns=fullflexible   
   }
    \lstinputlisting[]{\jobname_ex.exa}
         \end{minipage}}
  \hfill
  \hspace{30pt}%
   \setlength{\parindent}{0pt}
    \setlength{\parskip}{1ex plus 0.5ex minus 0.7ex}
  \makebox[0.5\linewidth][t]{%
  %\colorbox{ptcbackground!60}{
  \setlength{\parindent}{0pt}
    \setlength{\parskip}{1ex plus 0.3ex minus 0.7ex}
      \begin{minipage}[t]{0.4\linewidth}
   % \vspace*{0.5ex}
    \setlength{\parindent}{0pt}
      \noindent
  \setlength{\parskip}{1ex plus 0.4ex minus 0.7ex}
  \vspace*{-3.5ex} 
        \begin{trivlist}
             \scriptsize\bf\ttfamily \item \begin{out}
             {Pdf}\input{\jobname_ex.exa}\end{out}
    \end{trivlist}
      \end{minipage}%}
      }
      \par\addvspace{3ex plus 1ex}\vskip -\parskip
}
%%%%%%%%%%%%%%%%%%%%%%%%
\DeclareCaptionFormat{salid}{\colorbox{thered!40!black}{\parbox{\dimexpr\textwidth-2\fboxsep\relax}{\color{white}Ejercicio \thesection .\ 
\thesalida\hfill #3}}}
\captionsetup[salid]{format=salid,labelfont=white,textfont=white, singlelinecheck=false, margin=0pt, font={bf,footnotesize}}
\xcolortheorem[background=mybrown!5 ,titlebackground=mybrown!40!black ,titleboxcolor=mybrown!40!black ,boxcolor=mybrown!40!black]{salid}{Salida}
\newcounter{salida}[section]
\newwrite\salida@out
\newenvironment{salida}%
  {\begingroup% Lets Keep the Changes Local
  \@bsphack
  \immediate\openout \salida@out \jobname_eps.tex
  \let\do\@makeother\dospecials\catcode`\^^M\active
  \def\verbatim@processline{%
    \immediate\write\salida@out{\the\verbatim@line}}%
  \verbatim@start}%
 {\immediate\closeout\salida@out\@esphack\endgroup%
 \immediate\write18{pdflatex \jobname_eps.tex \jobname_eps.pdf }
 \immediate\write18{pdf2ps \jobname_eps.pdf \jobname_eps.eps }
    
  \setlength{\parindent}{0pt}
    \setlength{\parskip}{1ex plus 0.3ex minus 0.7ex}
     \begin{minipage}[t]{0.45\linewidth}
   % \vspace*{-1ex}
   %\vspace*{-3.5ex}
   \refstepcounter{salida}
   \captionsetup{options=salid}
   \lstset{%
   caption=Entrada,basicstyle=\tiny\ttfamily\bf,language={[LaTeX]TeX}, numbersep=5mm, numbers=left, numberstyle=\tiny, % number style
   breaklines=true,framexleftmargin=10mm, xleftmargin=10mm,
   backgroundcolor=\color{Tan!5},frameround=fttt,escapeinside=??,
   rulecolor=\color{thered!40!black},
   morekeywords={% Give key words here                                         % keywords
       maketitle},
   keywordstyle=\color[rgb]{0,0,1},                    % keywords
           commentstyle=\color[rgb]{0.133,0.545,0.133},    % comments
           stringstyle=\color[rgb]{0.627,0.126,0.941}  % strings
   %columns=fullflexible   
   }
    \lstinputlisting[]{\jobname_eps.tex}
         \end{minipage}
  %\hspace{30pt}%
   % \setlength{\parindent}{0pt}
    %\setlength{\parskip}{1ex plus 0.4ex minus 0.2ex}
    \hspace{20pt}
     \begin{minipage}[t]{0.40\linewidth}
   \begin{figure}[H]
\fcolorbox{thered!40!black}{thered!5}{\includegraphics[scale=0.4]{\jobname_eps.eps}}\end{figure}\end{minipage}
}
% }

  
%}
\DeclareCaptionFormat{prob}{\colorbox{thered!40!black}{\parbox{\dimexpr\textwidth-2\fboxsep\relax}{\color{white}Ejercicio \thesection .\ 
\theprob\hfill #3}}}
\captionsetup[prob]{format=prob,labelfont=white,textfont=white, singlelinecheck=false, margin=0pt, font={bf,footnotesize}}
\xcolortheorem[background=mybrown!5 ,titlebackground=mybrown!40!black ,titleboxcolor=mybrown!40!black ,boxcolor=mybrown!40!black]{prob}{Salida}
\newcounter{problema}[section]
\newwrite\problema@out
\newenvironment{problema}%
 {\begingroup% Lets Keep the Changes Local
  \@bsphack
  \immediate\openout \problema@out \jobname_pdf.tex
  \let\do\@makeother\dospecials\catcode`\^^M\active
  \def\verbatim@processline{%
    \immediate\write\problema@out{\the\verbatim@line}}%
  \verbatim@start}%
 {\immediate\closeout\problema@out\@esphack\endgroup%
 \immediate\write18{pdflatex \jobname_pdf.tex \jobname_pdf.pdf }
   \par\small\addvspace{3ex plus 1ex}\vskip -\parskip
\noindent
    \vspace*{-2ex}%
  \makebox[0.4\linewidth][l]{%
  \begin{minipage}[t]{0.9\linewidth}
  \captionsetup{options=prob}
   \lstset{%
   caption=Entrada,basicstyle=\tiny\ttfamily\bf,language={[LaTeX]TeX}, numbersep=5mm, numbers=left, numberstyle=\tiny, % number style
   breaklines=true,framexleftmargin=10mm, xleftmargin=10mm,
   backgroundcolor=\color{mybrown!5},frameround=fttt,escapeinside=??,
   rulecolor=\color{mybrown!40!black},
   morekeywords={% Give key words here                                         % keywords
       maketitle},
   keywordstyle=\color[rgb]{0,0,1},                    % keywords
           commentstyle=\color[rgb]{0.133,0.545,0.133},    % comments
           stringstyle=\color[rgb]{0.627,0.126,0.941}  % strings
   %columns=fullflexible   
   }
  
        \lstinputlisting[]{\jobname_pdf.tex}
        \end{minipage}
       }
 \vspace{20pt}

 \IfFileExists{\jobname_pdf.pdf}{% Si el fichero existe
 \begin{minipage}[t]{0.8\linewidth}
      \setlength{\parindent}{0pt}
    \setlength{\parskip}{1ex plus 0.4ex minus 0.2ex}
    \begin{figure}[H]
    \centering
    \caption{Salida}
  \fcolorbox{mybrown!40!black}{mybrown!5}{ \includegraphics[scale=0.4]{\jobname_pdf.pdf}}
\end{figure}\end{minipage}
}


  \par\addvspace{3ex plus 1ex}\vskip -\parskip
}

\newcommand{\figcaption}[1]{\def\@captype{figure}\caption{#1}}
 \renewcommand\@seccntformat[1]%
{\color{green}\csname the#1\endcsname.\quad}
\newcommand{\helv}{\fontfamily{phv}\fontsize{9}{11}\selectfont}
\newcommand{\helvi}{\fontfamily{phv}\fontseries{b}\fontsize{9}{11}\selectfont}
%\newcounter{notas}
% \newcommand{\notas }{\stepcounter{notas}\vskip 6pt \colorbox{red}{\thechapter.\thenotas.\color{blue}{Nota:}\vskip 6pt}
%  }
\newcommand{\margen }[1]{\marginpar{\parbox{4cm}{\small\emph{#1}}}}
\newenvironment{marnota}[1]{\begin{minipage}{4cm}\small\emph{#1}
}
{\end{minipage}
}
\newcommand{\pie}[1]{\begin{changemargin}{6cm}{5cm}{-0.1cm}{5cm}{2cm}#1\end{changemargin}}
%\renewenvironment{code}{\begin{quote}}{\end{quote}}
\newcommand{\cih}[1]{%
\index{instrucciones!#1@\texttt{\bs#1}}%
\index{#1@\texttt{\hspace*{-1.2ex}\bs #1}}}
\newcommand{\ci}[1]{\cih{#1}\texttt{\bs#1}}
%Package
\newcommand{\pai}[1]{%
\index{paquetes!#1@\textsf{#1}}%
\index{#1@\textsf{#1}}%
\textsf{#1}}
% Entrada en el 'indice de entorno
\newcommand{\ei}[1]{%
\index{entornos!\texttt{#1}}%
\index{#1@\texttt{#1}}%
\texttt{#1}}
% Entrada en el 'indice para mensajes
\newcommand{\wni}[1]{%
\index{mensaje!\texttt{#1}}%
\texttt{#1}}
% Entrada en el 'indice de una palabra
\newcommand{\wi}[1]{\index{#1}#1}
%
% Instrucciones de composici'on
%
\newenvironment{command}%
    {\nopagebreak\par\small\addvspace{3.2ex plus 0.8ex minus 0.2ex}%
     \vskip -\parskip
     \noindent%
      \setlength{\arrayrulewidth}{1mm}
          \arrayrulecolor{mybrown!40!black}
     \begin{tabular}{|l|}\rowcolor{mybrown!5}\hline\rule{0pt}{1em}\ignorespaces}%
    {\\\hline\end{tabular}\par\nopagebreak\addvspace{3.2ex plus 0.8ex
        minus 0.2ex}%
     \vskip -\parskip}
%
% Composici'on de fragmentos de c'odigo
%
\newenvironment{code}%
    {\nopagebreak\par\small\addvspace{3.2ex plus 0.8ex minus 0.2ex}%
     \vskip -\parskip
     \noindent%
      \setlength{\arrayrulewidth}{1mm}
          \arrayrulecolor{thered!40!black}
     \begin{tabular}{|l|}\rowcolor{Tan!5}\hline\rule{0pt}{1pt}\ignorespaces}%
    {\\\hline\end{tabular}\par\nopagebreak\addvspace{3.2ex plus 0.8ex
        minus 0.2ex}%
     \vskip -\parskip}
% \cvhrulefill{<color>}{<thickness>}
\newcommand*\cvhrulefill[2]{%
  \leavevmode\color{#1}\leaders\hrule\@height#2\hfill \kern\z@\normalcolor}
% \crule{<color>}{<width>}{<thickness>}
\newcommand\crule[3]{%
  \color{#1}\rule{#2}{#3}\normalcolor}
\newcommand{\codigo}[1]{\lstinline!\\#1!}
% Entorno Intro
\newenvironment{intro}{\sffamily}{\vspace*{2ex minus 1.5ex}}
%%%%%lined
\NewEnviron{lined}[1]%
 {\begin{center}
  \begin{minipage}{#1}\crule{red!40!black}{#1 +0.1\linewidth}{2pt}\vspace{2ex}
   \BODY 
   \crule{red!40!black}{#1 +0.1\linewidth}{2pt}
   \end{minipage}\vspace{2ex}
 \end{center}\vspace{2ex}}
%%%%%%
\protected\def\PdfLaTeX{P\kern -.15em\raisebox{-0.21em}{D}\kern -.05em F\LaTeX}
\protected\def\PdfTeX{P\kern -.15em\raisebox{-0.21em}{D}\kern -.05em F\TeX}
\newenvironment{lcpar}{%
\begingroup
\setlength{\leftskip}{0pt plus 1fil}%
\setlength{\rightskip}{-\leftskip}%
\setlength{\parfillskip}{0pt plus 2fil}
}{%
\par\endgroup
}
\renewcommand*{\LettrineFontHook}
{\bfseries}
\renewcommand*{\LettrineTextFont}
{\bfseries}


 \makeatother
\usepackage{fancyvrb}
\usepackage{cancel}
\usetikzlibrary{fit,shapes}
\usepackage{mivenndiagram}
\usepackage{tikz-cd}% diagramas commutativos
\usetikzlibrary{matrix}
\usepackage{mathtools}
\usepackage[unicode=true,pdfusetitle,
 bookmarks=true,bookmarksnumbered=true,bookmarksopen=true,bookmarksopenlevel=1,
 breaklinks=false,pdfborder={0 0 0},backref=false,colorlinks=false]
 {hyperref}
\hypersetup{
 pdfpagelayout=OneColumn, pdfnewwindow=true, pdfstartview=XYZ, plainpages=false}
 \newsavebox{\LstBox}
 %\graphicspath{ {./img/} }
 \makeatletter
\def\input@path{{./}{./build/}{./styles/}}
\makeatother
 \graphicspath{{./}{./img/}{./build/}}
 \renewcommand{\baselinestretch}{1.5}
\raggedbottom %para que no distibuya los espacios verticales en una hoja
\setcounter{problema}{1}
%\usepackage[spanish]{babel}
 

%}
\newcommand{\triangulo}{\tikz \filldraw[scale=0.5,fill=teal!20] (0,0) -- ++(60:1) -- ++(-60:1) -- cycle ;
                 }
                 \newcommand{\linea}[1]{\overleftrightarrow{#1}}
\newcommand{\degre}{\ensuremath{^\circ}}
\newenvironment{prueba}[2]{\renewcommand{\arraystretch}{1.06}
\vskip 5pt  {
 \colorbox{teal!30}{\color{white}{Prueba:}}
 }\vskip 5pt
$ %
\begin{tabular}
[t]{ll||l}\hline
\multicolumn{2}{c||}{\mbox{Afirmaciones}} & \mbox{Razones}\\\hline\hline
1. & 
#1 
& \multicolumn{1}{|l}{\mbox{Dado}}\\
#2
\end{tabular}
\ \ \ $ 
\vskip 5pt}{\hfill  \raggedright{ \rule{0.5em}{0.5em} }\vskip 5pt }
\newcommand{\problemas}[1]
{
\section{Problemas}
\peque
\begin{changemargin}{23.2cm}{18cm}{0cm}{0cm}{0cm}
\begin{multicols}{2}
 \noindent #1
\end{multicols}
\end{changemargin}
{\setlength{\parindent}{0mm}\color{ptctitle}\rule{\linewidth}{1mm}}
}
\setlength\columnsep{40pt} 
%\renewcommand\columnseprulecolor{\color{ptctitle}}
\def\spanishtablename{Tabla}
\setlength{\tabcolsep}{8pt}
%\setlength\columnseprule{2pt} %linea entrecolumnas por defecto 0pt

\makeatother

\usepackage[english,spanish]{babel}
\addto\shorthandsspanish{\spanishdeactivate{~<>}}

\begin{document}

%\pagecolor{white} \mainmatter 
\pagestyle{headings} 


\chapter{Conjuntos Numéricos.}

 \chaptertoc  

\begin{competencias}  \begin{lista}  

\item Interpreta correctamente las propiedades de las operaciones
definidas en los conjuntos numéricos. 

\item Clasifica correctamente los diferentes conjuntos numéricos. 

\item Argumenta sobre las diferencias entre las propiedades de los
conjuntos numéricos. 

\end{lista} \end{competencias}

\begin{logros}  \begin{lista}  

\item Identifica los diferentes números. 

\item Diferencia las propiedades válidas en cada conjunto numérico. 

\item Argumenta sobre las diferencias entre los conjuntos numéricos
y sus construcciones.. 

\end{lista} \end{logros}


\section{Introducción}

En este capítulo realizaremos la construcción de los conjuntos numéricos
empezando con una construcción axiomática de los números Naturales
según los axiomas de Peano , luego a partir de los números Naturales
haremos una construcción de los números Enteros construyendo primero
un conjunto equicomparable con los Naturales bajo una aplicación,
llamado el conjunto de los Enteros números, también haremos una construcción
alternativa donde a partir de la definición de la diferencia entre
dos números Naturales construiremos conjuntos que representaran cada
uno un número, esa misma idea la utilizaremos para construir los números
racionales a partir de la definición de división de números Enteros,
luego demostraremos que existen números que no se pueden representar
como el cociente de dos Enteros a ese conjunto lo llamaremos Irracionales.
Y para terminar construiremos de manera axiomática los números Reales.
Para terminar veremos que existen números que al elevarlos al cuadrado
nos da un número negativo, ese conjunto es el de los números Complejos.


\section{Números Naturales}


\subsection{Introducción}

En la antigüedad, el concepto de número surgió como consecuencia de
la necesidad práctica de contar objetos. Para ello, al principio el
hombre se valió de los elementos de que disponía a su alrededor: dedos,
piedras... Basta recordar, por ejemplo, que la palabra cálculo deriva
de la palabra latina calculus, que significa 'contar con piedras'.
La serie de números naturales era, obviamente, limitada; pero la conciencia
sobre la necesidad de ampliar el conjunto de los números naturales,
representaba ya una importante etapa en el camino hacia la matemática
moderna. Paralelamente a la ampliación de los conjuntos numéricos,
se desarrollaron su simbología y los sistemas de numeración diferentes
para cada civilización.

Fue en la India entre los siglos V y XII d.C. donde se empezaron a
usar correctamente los números negativos, se introdujo el cero e incluso
se llegó a aceptar como válidos los números irracionales. Es indiscutible
la procedencia hindú del sistema de numeración decimal y de las reglas
de cálculo. 

\textbf{El cero ¿es o no un número natural? }

Este es uno de los temas de más frecuente discusión entre quienes
se dedican a las matemáticas. 

Cuando Peano introdujo los axiomas para definir el conjunto de los
números naturales, inició este conjunto por el número uno. Pero cuando
Cantor estudió la teoría de conjuntos, encontró que debía empezar
por el cero, dada la necesidad de asignarle un cardinal al conjunto
vacío. Quizá fue esto lo que hizo que, diez años más tarde, Peano
empezara los números naturales con el cero.

En las últimas décadas ha sido muy popular la teoría de conjuntos,
lo cual justifica que muchos profesores prefieran comenzar el conjunto
de los números naturales por el cero. En este capítulo se elige iniciar
el conjunto de los números naturales por el uno, aunque el cero es
necesario para el cardinal del conjunto vacío, para el neutro de la
suma y para tantas otras aplicaciones. Pero en algunos temas, como
el de las sucesiones, es mejor iniciar los números naturales sin el
cero, pues es normal que se relacione el primer término con el número
uno, el segundo término con el número dos y así sucesivamente, recordando
que no hay un ordinal para el cardinal cero. 

\textbf{Axiomas de Peano}

La más conocida axiomatización de los números naturales, contenida
en el escrito \textbf{\textsl{Arithmetices Principia Nova Methodo
Exposita}} del italiano \textbf{\textsl{Giuseppe Peano}}, se presenta
en esta sección en forma detallada, al igual que la forma mas moderna
de la axiomatización, la definición de las operaciones y sus propiedades
debidamente demostradas. 

Durante el siglo XlX, sin duda con el impulso de la aparición de las
geometrías no euclidianas, se multiplicaron los esfuerzos por axiomatizar
la geometría empeño culminado finalmente en 1899 con la publicación
de Fundamentos de la Geometría, de David Hilbert , también se presentaron
varias axiomatizaciones de la aritmética o, con más precisión, de
los números naturales. 

Sin lugar a dudas, la más conocida es la que presentó el matemático
italiano Giuseppe Peano (1858-1932) por primera vez en 1889 en un
pequeño libro publicado en Turín titulado \textbf{\textsl{Arithmetices
Principia Nova Methodo Exposita}}. Este texto incluye sus famosos
axiomas, pero más que un texto a de aritmética, este documento contiene
una introducción a la lógica en la cual se presentan por primera vez
los símbolos actuales para representar la pertenencia, la existencia,
la contenencia (en la actualidad es invertido, acorde con el de los
números) y para la unión y la intersección. 

Peano reconoce hacer uso de estudios de otros autores: en 1888 después
de estudiar a G. Boole, E. Schröder, C. S. Peirce y otros, estableció
una o o analogía entre operaciones geométricas y algebraicas con las
operaciones de la lógica; en aritmética menciona el trabajo de Dedekind
publicado el año anterior reconocido de manera generalizada como la
primera axiomatización de la aritmética, aunque salió a la luz 7 años
después del artículo de Peirce y un texto de Grassmann de 1861. Este
ultimo libro posiblemente fué fuente de inspiración tanto para Peano
y Dedekind como para Peirce. 

Arithmetices Principia, escrito en latín es el primer intento de Peano
, para lograr una axiomatización de las matemáticas en un lenguaje
simbólico. 

Consiste en un prefacio y 10 secciones:

1. Números y Adición.

2. Sustracción.

3. Máximos y Mínimos. 

4. Multiplicación.

5. Potenciación. 

6. División.

7. Teoremas varios.

8. Razones de Números.

9. Sistemas de Racionales e Irracionales.

10. Sistemas de Cantidades.

En libro establece los siguientes axiomas:
\begin{enumerate}
\item $1\in\na$.
\item Si $a\in\na$entonces $a=a.$
\item Si $a\in\na$ entonces $a=b$ si y sólo si $b=a.$
\item Si $a,b,c\in\na$ entonces $a=b,$ $b=c$ implica $a=c.$
\item Si $a=b$ y $b\in\na$ entonces $a\in\na.$
\item Si $a\in\na$ entonces $a+1\in\na.$
\item Si $a\in\na$ entonces $a=b$ si y sólo si $a+1=b+1$.
\item Si $a\in\na$ entonces $a+1\neq1.$
\item Si $k$ es una clase, $1\in k,$ y si para $x\in\na:$ $x\in k$ implica
$x+1$.
\end{enumerate}

\subsection{Axiomas de los números naturales}

Empezaremos ahora la construcción de los números naturales siguiendo
la idea de Peano.

\begin{axioma}{Existencia de los números naturales}\label{ap1}

Existe un conjunto $\na\neq\emptyset$ llamado conjunto de los números
Naturales.

\end{axioma}

Podríamos decir que este axioma establece uno de los términos no definidos
de la teoría el cual es el número natural.

\begin{axioma}{Existencia del número mínimo}\label{ap4}

Existe un número llamado $1\in\na.$

\end{axioma}

Del axioma \ref{ap1} tenemos que como $\na\neq\emptyset$ entonces
debe tener por lo menos un elemento, el axioma \ref{ap4} lo define
como el número $1.$

\begin{axioma}{Relación de igualdad}\label{ap3}

Existe una relación $"="$ tal que $a=b$ significa que $a$ y $b$
representan el mismo número natural, que cumpla las siguientes propiedades

$\forall a,b,c\in\na$ se tiene que:
\begin{description}
\item [{i)}] Si $a\in\na$entonces $a=a.$
\item [{ii)}] Si $a\in\na$ entonces $a=b$ si y sólo si $b=a.$
\item [{iii)}] Si $a,b,c\in\na$ entonces $a=b,$ $b=c$ implica $a=c.$
\item [{iv)}] Si $a=b$ y $b\in\na$ entonces $a\in\na.$
\end{description}
\end{axioma}

En el capítulo \ref{chap:chap2} página \pageref{tii} presentamos
el concepto de igualdad como un término no definido, pero Peano lo
presentó como un axioma y además establece que es una relación, algo
que no planteamos en la teoría de conjuntos al presentar el término
, pero si cuando se estableció el axioma de extensión.

Observamos también que el axioma \ref{ax: igualdad} establece las
tres primeras propiedades, por tanto este axioma establece sólo la
propiedad iv), la cual nos asegura que un número natural sólo puede
ser representado por un número natural. Es decir que si existen otros
números estos no pueden representar a un número natural.

\begin{axioma}{Sucesor}\label{ap2}

Existe una aplicación %
\begin{tabular}{cccc}
$\varphi$ : & $\na$ & $\rightarrow$ & %
$\na$%
\tabularnewline
\end{tabular} que cumple las siguientes propiedades
\begin{description}
\item [{i)}] $\forall n\in\na,\;\varphi\left(n\right)\neq1$
\item [{ii)}] $\forall n,m\in\na\quad\varphi\left(n\right)=\varphi\left(m\right)\Rightarrow n=m.$
\end{description}
\end{axioma}

La propiedad i) que establece el axioma \ref{ap2} nos dice que el
número uno no tiene preimagen, es decir no es sucesor de ningún natural,
es el primero.

La segunda propiedad del axioma \ref{ap2} establece la relación de
igualdad, en función de la función sucesor, pero también nos aclara
que el sucesor de un número natural es único. 

\begin{axioma}{Adición entre naturales}\label{ap5}

Existe una operación %
\begin{tabular}{cccc}
$+$ : & $\na\times\na$ & $\rightarrow$ & %
$\na$%
\tabularnewline
 & $\left(a,b\right)$ & $\mapsto$ & $c=+\left(a,b\right)=a+b$\tabularnewline
\end{tabular} que cumple las siguientes condiciones .

$\forall n,n\in\na$ se tiene que 
\begin{description}
\item [{i)}] $+\left(n,1\right)=\varphi\left(n\right)$
\item [{ii)}] $+\left(\varphi\left(n\right),m\right)=+\left(n,\varphi\left(m\right)\right)=\varphi\left(+\left(n,m\right)\right)$
\end{description}
\end{axioma}

Hubiéramos podido establecer los axiomas \ref{ap1} y \ref{ap5} como
definiciones, pero decidimos establecerlos como axiomas, porque no
conocemos todavía la estructura de $\na$.

\begin{definicionn}{Sucesor}\label{def:sucesor}

La aplicación que garantiza el axioma \ref{ap2} la definiremos %
\begin{tabular}{cccc}
$\varphi$ : & $\na$ & $\rightarrow$ & %
$\na$%
\tabularnewline
 & $n$ & $\mapsto$ & $\varphi\left(n\right)=n+1$\tabularnewline
\end{tabular} de acuerdo con la propiedad i) del axioma \ref{ap5}

\end{definicionn}

Esta definición la podemos expresar de la siguiente manera:
\[
\mbox{Si \ensuremath{n\in\na\Rightarrow n+1\in\na.}}
\]


Y con el podemos construir los números naturales de la siguiente forma
\begin{description}
\item [{i)}] $1\in\na$ entonces existe $\varphi\left(1\right)=1+1\in\na$.
Al número $1+1$ lo llamaremos dos, es decir $1+1:=2.$
\item [{ii)}] $2\in\na$ entonces existe $\varphi\left(2\right)=2+1\in\na.$
al números $2+1$ lo llamaremos tres, es decir $2+1:=3.$
\item [{iii)}] $3\in\na$ entonces existe $\varphi\left(3\right)=3+1\in\na.$
al números $3+1$ lo llamaremos cuatro, es decir $3+1:=4.$
\end{description}
Podemos seguir construyendo los números naturales de manera indefinida,
es decir el conjunto de los números naturales es infinito y de acuerdo
con el axioma del conjunto infinito ( axioma \fullref{ainf}) este
conjunto es posible ya que sigue la misma construcción que determina
este axioma. 

En estos momentos como ya conocemos la estructura del conjunto de
los números naturales podemos definir la adición entre naturales de
acuerdo con la definición \ref{def:sucesor}.

Sean $m$ y $n$ números naturales entonces $m+n=\underset{m\mbox{ veces}}{(1+1+1+\cdots+1)}+\underset{n\mbox{ veces}}{\left(1+1+1+\cdots+1\right)}$
es decir:

\begin{table}[H]
\centering

\caption{Tabla de la adición entre números naturales.}
\resizebox{0.3\hsize}{!}{

\setlength\arrayrulewidth{1pt}\arrayrulecolor{ptctitle} 

\begin{tabular}{c|ccccc}
\arrayrulecolor{ptctitle}\hline\cellcolor{ptctitle!50}$+$ & \cellcolor{ptctitle!50}$1$ & \cellcolor{ptctitle!50}$2$ & \cellcolor{ptctitle!50}$3$ & \cellcolor{ptctitle!50}$4$ & \cellcolor{ptctitle!50}$\cdots$\tabularnewline
\hline 
\hline\cellcolor{ptctitle!50}$1$ & \cellcolor{ptcbackground} $2$ & \cellcolor{ptcbackground} $3$ & \cellcolor{ptcbackground}$4$ & \cellcolor{ptcbackground}$5$ & \cellcolor{ptcbackground}$\cdots$\tabularnewline
\hline\cellcolor{ptctitle!50}$2$ & \cellcolor{gray!50} $3$ & \cellcolor{gray!50} $4$ & \cellcolor{gray!50}$5$ & \cellcolor{gray!50}$6$ & \cellcolor{gray!50}$\cdots$\tabularnewline
\hline\cellcolor{ptctitle!50}$3$ & \cellcolor{ptcbackground} $4$ & \cellcolor{ptcbackground} $5$ & \cellcolor{ptcbackground} $6$ & \cellcolor{ptcbackground} $7$ & \cellcolor{ptcbackground} $\cdots$\tabularnewline
\hline\cellcolor{ptctitle!50}$4$ & \cellcolor{gray!50} $5$ & \cellcolor{gray!50} $6$ & \cellcolor{gray!50} $7$ & \cellcolor{gray!50} $8$ & \cellcolor{gray!50} $\cdots$\tabularnewline
\hline\cellcolor{ptctitle!50}$\vdots$ & \cellcolor{ptcbackground} $\vdots$ & \cellcolor{ptcbackground} $\vdots$ & \cellcolor{ptcbackground} $\vdots$ & \cellcolor{ptcbackground} $\vdots$ & \cellcolor{ptcbackground} $\vdots$\tabularnewline
\end{tabular}

}\label{tun}
\end{table}
Observemos en ta tabla que $1+1=2$ o que $3+1=4$ , que es lo que
conocemos desde la básica primaria.

\begin{axioma}{ Inducción }\label{induccion}

Si $A\subseteq\na,$ $A\neq\emptyset,$ tal que 
\begin{description}
\item [{i)}] $1\in A.$
\item [{ii)}] Si $\left(n\in A\Rightarrow n+1\in A\right)\Rightarrow A=\na.$
\end{description}
\end{axioma}

El axioma de inducción se puede presentar también de la siguiente
forma :

Sea $P$ una propiedad cualesquiera. entonces todo número satisface
la propiedad $P$ si se tiene que:
\begin{description}
\item [{i)}] $1$ satisface la propiedad $P$, es decir $P\left(1\right)$es
cierto.
\item [{ii)}] Si $n$ satisface la propiedad $P$, entonces $n+1$ también
satisface la propiedad $P$, es decir 
\[
\mbox{Si \ensuremath{P\left(n\right)\Rightarrow P\left(n+1\right).}}
\]

\item [{iii)}] Se concluye $P$ se satisface para cualquier natural, es
decir $P\left(m\right)$ es cierta $\forall m\in\na.$ 
\item [{\nota }] Cuando el axioma se presenta de esta forma se le llama
Principio de Inducción. Queda da tarea su demostración.
\end{description}
Resumiendo lo que afirman estos postulados o axiomas, podemos entender
que se trata de un conjunto que tiene un elemento mínimo, el uno (Axioma
\ref{ap4}), es decir no es siguiente de ningún otro (Axioma \ref{ap2}.i)),
es decir, se trata del primer elemento del conjunto, y todos los demás
elementos tienen cada uno un elemento siguiente (Axioma \ref{ap4}.ii)
), de modo que dos elementos distintos tienen siguientes distintos.
El \ref{induccion} postulado es de suma importancia por dotarnos
de un método de demostración de propiedades, ya que nos indica que
todo conjunto $A$ al que pertenezca el uno, y tal que todo elemento
de $A$ tiene siguiente en $A$, necesariamente ha de coincidir con
el conjunto $\na$ de los números naturales. Es lo que se acostumbra
a denominar método simple de inducción completa. 

A partir de estos cinco axiomas, y usando sistemáticamente el \ref{induccion}
axioma, de la inducción completa, podemos probar todas las propiedades
del conjunto $\na$. 

\begin{teorema}{}

Ningún número natural coincide con su sucesor, es decir $\forall n\in\na,\: n\neq\varphi\left(n\right).$

\end{teorema}

\begin{dems}

Sea $A=\left\{ n\in\na\::\: n\neq\varphi\left(n\right)\right\} $,
ahora apliquemos el axioma de inducción (\ref{induccion}) 
\begin{enumerate}
\item $1\in A,$ por el axioma del mínimo (\ref{ap4}) y el axioma del sucesor
(\ref{ap2}.i)) $\varphi\left(1\right)\neq1$.
\item $\forall n\in A,$ $n\neq\varphi\left(n\right)\Rightarrow\varphi\left(n\right)\neq\varphi\left(\varphi\left(n\right)\right)$
por el recíproco del axioma \ref{ap2}.ii), de lo que se deduce que
$\varphi\left(n\right)\in A.$
\item Por tanto vemos que de 1) $1\in A$ y de 2) si $n\in\na\Rightarrow\varphi\left(n\right)\in A$
se deduce que $A=\na$ por el axioma de inducción (\ref{induccion}),
luego $\forall n\in\na$, $n\neq\varphi\left(n\right).$ 
\end{enumerate}
\end{dems}

\begin{teorema}{}

Si dos aplicaciones definidas en $\na\rightarrow\na$ conmutan con
la aplicación sucesor ($\varphi\left(n\right))$ y tiene la misma
imagen para el 1, entonces ambas coinciden.

En el lenguaje $L_{\in}$ lo podemos expresar :

Si \foreignlanguage{english}{$f\;:\;\na\rightarrow\na$ y $g\;:\;\na\rightarrow\na$
son }aplicaciones tal que\foreignlanguage{english}{ 
\[
f\circ\varphi=\varphi\circ f\;\wedge\; g\circ\varphi=\varphi\circ g\;\wedge\; f\left(1\right)=g\left(1\right),\;\forall n\in\na\Rightarrow f\left(n\right)=g\left(n\right).
\]
}

\end{teorema}

\begin{dems}

Sea $A=\left\{ n\in\na\;:\: f\left(n\right)=g\left(n\right)\right\} $,
ahora apliquemos el axioma de inducción.
\begin{description}
\item [{i)}] $1\in A,$ pues por hipótesis $f\left(1\right)=g\left(1\right).$
\item [{ii)$\forall\in\na,$}] 
\begin{eqnarray*}
f\left(n\right)=g\left(n\right) & \Rightarrow & \varphi\left(f\left(n\right)\right)=\varphi\left(g\left(n\right)\right)\\
 & \Rightarrow & \left(\varphi\circ f\right)\left(n\right)=\left(g\circ\varphi\right)\left(n\right)\\
 & \Rightarrow & \left(f\circ\varphi\right)\left(n\right)=\left(g\circ\varphi\right)\left(n\right)\\
 & \Rightarrow & f\left[\varphi\left(n\right)\right]=g\left[\varphi\left(n\right)\right]\\
 & \Rightarrow & \varphi\left(n\right)\in A.
\end{eqnarray*}

\item [{iii)}] En definitiva, vemos de i) $1\in A$ , de ii) si $n\in A\Rightarrow\varphi\left(n\right)\in A$,
de lo que se deduce por el axioma \ref{induccion} que $A=\na.$
\end{description}
\end{dems}

\begin{teorema}{}\label{teo:fidg}

Dadas dos aplicaciones \foreignlanguage{english}{$f\;:\;\na\rightarrow\na$
y $g\;:\;\na\rightarrow\na$} si existe alguna tercera aplicación
\foreignlanguage{english}{$\psi\;:\;\na\rightarrow\na$} tal que,\foreignlanguage{english}{
\[
f\circ\varphi=\psi\circ f\;\wedge\; g\circ\varphi=\psi\circ g\;\wedge\; f\left(1\right)=g\left(1\right),\;\forall n\in\na\Rightarrow f\left(n\right)=g\left(n\right).
\]
}

\end{teorema}

\begin{dems}

Sea $A=\left\{ n\in\na\;:\: f\left(n\right)=g\left(n\right)\right\} $,
ahora apliquemos el axioma de inducción.
\begin{description}
\item [{i)}] $1\in A,$ pues por hipótesis $f\left(1\right)=g\left(1\right).$
\item [{ii)$\forall n\in\na,$}] 
\begin{eqnarray*}
\left(f\circ\varphi\right)\left(n\right)=\left(\psi\circ f\right)\left(n\right) & \Rightarrow & f\left(\varphi\left(n\right)\right)=\psi\left(f\left(n\right)\right)=\psi\left(g\left(n\right)\right)=g\left(\varphi\left(n\right)\right)\\
 & \Rightarrow & \varphi\left(n\right)\in A.
\end{eqnarray*}

\item [{iii)}] En definitiva, vemos de i) $1\in A$ , de ii) si $n\in A\Rightarrow\varphi\left(n\right)\in A$,
de lo que se deduce por el axioma \ref{induccion} que $A=\na.$
\end{description}
\end{dems}


\subsection{Propiedades de la adición entre números naturales}

\begin{teorema}{unicidad de la suma}

La aplicación $+$ es única, es decir si $+_{1}$ y $+_{2}$ son sumas
entonces $+_{1}=+_{2}.$

\end{teorema}

\begin{dems}

Definamos dos aplicaciones, $f$ y $g$, mediante $+_{1}$ y $+_{2}$
de la siguiente manera 

Sea \foreignlanguage{english}{$f\;:\;\na\rightarrow\na$ }definida\foreignlanguage{english}{
$\forall n\in\na\: f\left(n\right)=+_{1}\left(n,m\right),\quad\mbox{para algún \ensuremath{m\in\na}}$
.}

Sea \foreignlanguage{english}{$g\;:\;\na\rightarrow\na$ }definida\foreignlanguage{english}{
$\forall n,m\in\na\: g\left(n\right)=+_{2}\left(n,m\right),\quad\mbox{para algún \ensuremath{m\in\na}}$
.}

Entonces : 
\begin{description}
\item [{i)}] $1\in\na$
\[
\left|\begin{array}{c}
f(1)=+_{1}(1,m)=1+m\\
g(1)=+_{2}(1,m)=1+m
\end{array}\right\} \Rightarrow f(1)=g(1).
\]

\item [{ii)}] $\forall n\in\na$
\[
\left(f\circ\varphi\right)\left(n\right)=+_{1}(\varphi\left(n\right),m)=\varphi\left(+_{1}\left(n,m\right)\right)=\varphi\left[f\left(n\right)\right]=\left(\varphi\circ f\right)\left(n\right)
\]
entonces $\forall n\in\na,$ $\left(f\circ\varphi\right)\left(n\right)=\left(\varphi\circ f\right)\left(n\right)\Rightarrow f\circ\varphi=\varphi\circ f.$
\item [{iii)}] $\forall n\in\na$
\[
\left(g\circ\varphi\right)\left(n\right)=+_{2}(\varphi\left(n\right),m)=\varphi\left(+_{2}\left(n,m\right)\right)=\varphi\left[g\left(n\right)\right]=\left(\varphi\circ g\right)\left(n\right)
\]
entonces $\forall n\in\na,$ $\left(g\circ\varphi\right)\left(n\right)=\left(\varphi\circ g\right)\left(n\right)\Rightarrow g\circ\varphi=\varphi\circ g.$
\end{description}
Es decir las aplicaciones $f$ y $g$ con la aplicación sucesor $\varphi$
de acuerdo con el teorema \ref{teo:fidg} , entonces $f\left(n\right)=g\left(n\right),\quad\forall n\in\na.$
p $+_{1}\left(n,m\right)=+_{2}\left(n,m\right).$

\end{dems}

\notacion  representaremos de aquí en adelante la adición entre dos
números naturales $n$ y $m$ como $n+m$ en vez de $+\left(n,m.\right)$
por tanto las condiciones que cumple la operación se denotará
\begin{description}
\item [{i)}] $\varphi\left(n\right)=n+1$
\item [{ii)}] $\varphi\left(n\right)+m=\varphi\left(n+m\right).$
\end{description}
\begin{definicionn}{Menor que}

Se dice que $a\in\na$ es menor o igual que $b\in\na$ si existe un
$c\in\na$ tal que $b=a+c.$ 

\end{definicionn}

\notacion  Si un número $a$ es menor o igual a un número $b$ se
denota $a\leq b.$

\begin{definicionn}{Meyor que}

Se dice que $a\in\na$ es mayor o igual que $b\in\na$ si existe un
$c\in\na$ tal que $a=b+c.$ 

\end{definicionn}

\notacion  Si un número $a$ es mayor o igual a un número $b$ se
denota $a\geq b.$

\begin{axioma}{Principio de la buena ordenación}

Si $A\neq\emptyset$ y $A\subseteq\na$ existe un único número $m\in A$
tal que $m\leq a,$ $\forall a\in A$, $m$ recibe de mínimo elemento
de $A.$

\end{axioma}

\begin{definicionn}{Cero} 

Existe un número llamado neutro aditivo o cero que representa con
el símbolo $"0$'', tal que si $a\in\na\Rightarrow a+0=0+a=a$, además
definimos la aplicación 
\[
\varphi^{*}\left(n\right)=\begin{cases}
\varphi\left(n\right) & \mbox{si \ensuremath{n\in\na,}}\\
1 & \mbox{si \ensuremath{n=0.}}
\end{cases}
\]


\end{definicionn}\nota  Podemos ahora construir un nuevo conjunto
$\na_{0}:=\na\cup\left\{ 0\right\} $ y podemos redefinir la adición 

\begin{center}
\begin{tabular}{cccc}
$+$ : & $\na_{0}\times\na_{0}$ & $\rightarrow$ & %
$\na_{0}$%
\tabularnewline
 & $\left(a,b\right)$ & $\mapsto$ & $c=a+b$\tabularnewline
\end{tabular}
\par\end{center}

Si definimos la siguiente proposición $0:=0+0$. Con esta definición
acabamos con la discusión si el cero es o no un número natural, para
nosotros no, pero cuando lo necesitemos trabajamos con el conjunto
$\na_{0}$ 

El axioma de inducción matemática lo podemos extender a $\na_{0}$
cambiando la primera condición a $0\in A.$ 

En general el axioma es válido para cualquier subconjunto de $\text{\na\ }$que
le falten un numero finito de los primeros números, es decir si $B_{0}\subset\na$
tal que $B_{0}=\{a_{0},a_{1},\cdots,\},\: a_{i}\in\na,\;\forall i=0,1,2,\cdots$,
entonces el axioma de inducción se cumple cambiando la primara condición
por $a_{0}\in A.$

\begin{teorema}{Propiedades de la adición}\label{teo: ps}

$\forall a,b,c\in\na_{0}$ se cumplen las siguientes propiedades:
\begin{description}
\item [{i)}] Propiedad conmutativa: $a+b=b+a$.
\item [{ii)}] Propiedad asociativa: $\left(a+b\right)+c=a+\left(b+c\right).$
\item [{iii)}] Propiedad cancelativa: Si $a+c=b+c\Rightarrow a=b.$
\end{description}
\end{teorema}

\begin{dems}
\begin{description}
\item [{i)}]~

\begin{description}
\item [{1}] Veamos primero que $\forall a,b\in\na,\quad0+b=b+0$
\[
\mbox{Sea }A=\left\{ b\in\na\,:\:0+b=b+0\right\} 
\]
\[
0\in A,\mbox{ ya que \ensuremath{0+0=0+0.}}
\]
\[
\forall b\in A,\;\varphi^{*}\left(b\right)+0=\varphi^{*}\left(b+0\right)=\varphi^{*}\left(b\right)=0+\varphi^{*}\left(b\right)\Rightarrow\varphi^{*}\left(b\right)\in A.
\]
Luego por el axioma de inducción para $\na_{0}$, tenemos que $A=\na_{0},$
y se verifica que $\forall b\in\na$$_{0},\;0+b=b+0.$ 
\item [{2}] Veamos ahora que $\forall b\in\na_{0},$ $\varphi^{*}\left(0\right)+b=b+\varphi^{*}\left(0\right)$
\[
\mbox{Sea \ensuremath{A=\left\{ b\in\na_{0}\,:\;\varphi^{*}\left(0\right)+b=b+\varphi^{*}\left(0\right)\right\} }}
\]
\[
0\in A,\mbox{ pues \ensuremath{\varphi^{*}\left(0\right)+0=\varphi^{*}\left(0+0\right)=0+\varphi^{*}\left(0\right)}}
\]
\begin{eqnarray*}
\forall b\in A,\;\varphi^{*}\left(b\right)+\varphi^{*}\left(0\right) & = & \varphi^{*}\left(b+\varphi^{*}\left(0\right)\right)\\
 & = & \varphi^{*}\left(\varphi^{*}\left(0\right)+b\right)\\
 & = & \varphi^{*}\left(\varphi^{*}\left(0+b\right)\right)\\
 & = & \varphi^{*}\left(b\right)\\
 & = & \varphi^{*}\left(0++\varphi^{*}\left(b\right)\right)\\
 & = & \varphi^{*}\left(0\right)+\varphi^{*}\left(b\right)\Rightarrow\varphi^{*}\left(b\right)\in A.
\end{eqnarray*}

\item [{3}] Veamos finalmente que $\forall a,b\in\na_{0},\quad a+b=b+a$
\[
\mbox{Sea }A=\left\{ b\in\na_{0}\,:\; a+b=b+a,\quad\mbox{ para cada }a\in\na_{0}\right\} 
\]


\[
0\in A,\quad a+0=0+a,\forall a\in\na_{0}
\]
 
\begin{eqnarray*}
\forall b\in A,\; a+\varphi^{*}\left(b\right) & = & a+\varphi^{*}\left(0+b\right)\\
 & = & a+\varphi\left(0\right)+b\\
 & = & \varphi^{*}\left(a\right)+0+b\\
 & = & 0+\varphi^{*}\left(a\right)+b\\
 & = & \varphi^{*}\left(0\right)+a+b\\
 & = & \varphi^{*}\left[0+(a+b)\right]\\
 & = & \varphi^{*}\left(a+b\right)\\
 & = & \varphi^{*}\left(b+a\right)\\
 & = & \varphi^{*}\left(b\right)+a\Rightarrow\varphi^{*}\left(b\right)\in A
\end{eqnarray*}

\end{description}

Luego , por el axioma de inducción para $\na_{0}$, $A=\na_{0},$
y se verifica que $\forall a,b\in\na_{0},\; a+b=b+a.$

\end{description}
\end{dems}

Las demostraciones de las otras propiedades quedan como ejercicio
para el lector.


\subsubsection{Multiplicación entre números naturales}

\begin{definicionn}{Multiplicación entre números naturales}\label{def:produt}

Definimos la multiplicación entre números naturales como una aplicación
\begin{tabular}{cccc}
 $\times$: & $\na\times\na$ & $\rightarrow$ & %
$\na$%
\tabularnewline
 & $\left(n,m\right)$ & $\mapsto$ & $l=\times\left(n,m\right)$$=nm$\tabularnewline
\end{tabular}, de modo que $\forall a,b\in\na$ se cumple 
\begin{description}
\item [{i)}] $\times\left(1,m\right)=\times\left(m,1\right)=m$
\item [{ii)}] $\times\left(\varphi\left(n\right),m\right)=\times\left(n,m\right)+m.$
\end{description}
\end{definicionn}


\section{Números Enteros}


\section{Números Racionales}


\section{Números Reales}


\section{Números Complejos.}

\label{sec:problema3} \problemas{ 
\begin{enumerate}
\item En los incisos del \textbf{a)} al \textbf{e)}, escriba las proposiciones
como implicaciones, luego decida si es falso o verdadero 

\begin{enumerate}
\item \textbf{Hipótesis} $p:$ Un hombre vive en Barranquilla, \textbf{Conclusión}
$q:$ Vive en Antioquía. 
\item \textbf{Hipótesis} $p:$ Algunas manzanas son rojas, \textbf{Conclusión}
$q:$ Los caballos tienen cuatro patas. 
\item \textbf{Hipótesis} $p:$ Dos rectas se intersecan, \textbf{Conclusión}
$q:$ Las dos rectas no son paralelas. 
\item \textbf{Conclusión} $q:$ Dos rectas son perpendiculares, \textbf{Hipótesis}
$p:$ Las rectas se intersecan. 
\item \textbf{Hipótesis} $p:$ Dos ángulos tienen la misma medida, \textbf{Conclusión}
$q:$ Los ángulos son congruentes. 
\end{enumerate}
\item \textbf{Analice la siguiente conjetura}: Si un triángulo tiene un
ángulo recto, tiene dos lados congruentes.\textbf{ Comentario:} Para
demostrar que la conjetura es falsa, debe presentar un contraejemplo,
para explicar que es verdadera use las definiciones. 
\end{enumerate}
				 

}

\end{document}
